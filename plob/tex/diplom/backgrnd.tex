%
\chapter{Persistente Systeme}%
\label{chap:back}%
%
Dieser Abschnitt erl\"{a}utert verschiedene bisherige Ans\"{a}tze f\"{u}r
persistente Systeme und ihre Integration in Programmiersprachen.
%
\section{Relationale Datenbanken}%
\label{sec:rdb}%
%
Relationale Datenbanken wurden von Codd Anfang 1970 in erster Linie
definiert, damit deren BenutzerInnen besser von Details der
Datenrepr\"{a}sentation abstrahieren k\"{o}nnen. Eine Datenbank ist genau
dann relational, wenn sie mehrere von Codd aufgestellte Kriterien
erf\"{u}llt \cite[\citepage{398}]{bib:co79}; an dieser Stelle gehe ich
nur auf die in Zusammenhang mit persistenten Objekten wichtigen
Aspekte ein. Die hier wiedergegebenen Definitionen stammen aus
\cite[\citepage{399}]{bib:co79} und \cite[\citepage{81}]{bib:schl83}.
%
\par{}Ein \gande{{\em Bereich\/}}{domain}\ umfa\ss{}t alle Werte mit
\"{a}hnlichem Typ; ein Bereich ist {\em einfach}, wenn seine Werte atomar
sind, d.h.\ von einer Datenbank nicht weiter zerlegt werden
k\"{o}nnen. Einfache Bereiche sind im allgemeinen Zahlen, Datumsangaben,
Zeichen, Zeichenketten usw.  {\it D}$_1$,\,""{\it
D}$_2$,\,""\ldots,\,""{\it D}$_n$ mit {\it n}\,$>$\,0 seien nicht
notwendigerweise unterschiedliche Bereiche; das kartesische Produkt
{\it D}$_1$\,$\times$\,""{\it
D}$_2$\,$\times$\,""\ldots\,$\times$\,""{\it D}$_n$ ist die Menge
aller {\it n\/}-Tupel $\langle$\,{\it t}$_1$,\,""{\it
t}$_2$,\,""\ldots,\,""{\it t}$_n$\,$\rangle$ mit {\it
t}$_i$\,$\in$\,{\it D}$_i$ f\"{u}r 1\,$\leq$\,{\it i}\,$\leq$\,{\it
n}. Eine Teilmenge {\it R} dieses kartesischen Produktes ist eine {\it
n\/}-stellige Relation. Ein {\it n\/}-Tupel {\it
x}\,=\,$\langle$\,{\it x}$_1$,\,""{\it x}$_2$,\,\ldots,\,""{\it
x}$_n$\,$\rangle$ mit {\it x}\,$\in$\,{\it R}, {\it
x}$_i$\,$\in$\,{\it D}$_i$, 1\,$\leq$\,{\it i}\,$\leq$\,{\it n} wird
als {\em Tupel der Relation R\/} bezeichnet.
%
\par{}Anstatt der Indexmenge
\{1,\,""2,\,""\ldots,\,""{\it n}\}\ kann auch eine andere ungeordnete
Menge verwendet werden, vorausgesetzt, da\ss{} zu jeder Tupelkomponente
nicht nur sein Bereich, sondern auch sein jeweiliger Index assoziiert
wird; die Elemente dieser Menge hei\ss{}en {\em Attribute}.
%
\par{}Repr\"{a}sentiert werden Relationen durch Tabellen; jede Spalte der
Tabelle entspricht einem Attribut und jede Zeile einem Tupel
(\figurename~\ref{fig:exrdb}).
%
\begin{figure}[htbp]\centering%
\begin{\figurefontsize}%
\begin{tabular}{r|c|c|c|l|}%
\multicolumn{1}{l}{Relation}% \hspace*{1.333em}}
        &\multicolumn{2}{l}{}
                &\multicolumn{1}{c}{Attribut}
                        &\multicolumn{1}{l}{}\\
\multicolumn{1}{l}{\class{Angestellter}\ $\searrow$}
        &\multicolumn{2}{l}{}
                &\multicolumn{1}{c}{$\downarrow$}
                        &\multicolumn{1}{l}{}\\[\smallskipamount]
\cline{2-5}
&\tabularheader{ANG-NR}
 &\tabularheader{NAME}
  &\tabularheader{GEHALT}
   &\tabularheader{ABTEILUNG}\\[0.1\smallskipamount]
\cline{2-5}%
Tupel $\rightarrow$
        & 4711 &\namei & 3000
        & Marzipankartoffeln\\[0.1\smallskipamount]
\cline{2-5}
        & 4712 &\nameii & 6000
        & Gesch\"{a}ftsf\"{u}hrung\\[0.1\smallskipamount]
\cline{2-5}
\end{tabular}%
\end{\figurefontsize}%
\caption%
 [Beispiel Relation \protect\rglq{}Angestellter\protect\rgrq]%
 {Beispiel Relation \protect\class{Angestellter}}%
\label{fig:exrdb}%
\end{figure}%
%
Relationen sind Mengen; daher m\"{u}ssen die Tupel einer Relation
eindeutig sein. Ebenso ist durch den Mengencharakter die Reihenfolge
der Tupel und Attribute als beliebig anzunehmen; die in einer real
existierenden relationalen Datenbank etablierte Ordnung der Tupel wird
ausschlie\ss{}lich aus Effizienzgr\"{u}nden vorgenommen.
%
\par{}F\"{u}r relationale Datenbanken wird gefordert, da\ss{} die Bereiche
der Attribute immer einfach sein m\"{u}ssen; nur dann lassen sich die
\rglq{}angenehmen\rgrq\ Eigenschaften von relationalen Datenbanken, wie
die Existenz von Normalformen zur Vermeidung von Redundanzen und
Anomalien, beweisen. Nicht-einfache Bereiche werden auf Relationen
abgebildet. Die Relationenalgebra er\-m\"{o}g\-licht es, auf das Konzept
einer (expliziten) Referenz, so wie sie in allgemeinen
Programmiersprachen zur Verf\"{u}gung steht, zu verzichten; zwei Tupel
werden durch den jeweils in einem Attribut enthaltenen gleichen Wert
kommutativ in Beziehung gesetzt.
%
\par{}Die grundlegende Eigenschaft einer relationalen Datenbank ist
Persistenz aller in ihr enthaltenen Daten; Persistenz ist damit
orthogonal als auch transparent. Die Relationenalgebra kann gewisse
Eigenschaften einer relationalen Datenbank garantieren.  Damit
f\"{u}hrten \"{U}berlegungen \"{u}ber persistente \clos-Objekte bei den
AutorInnen einiger persistenter Systeme zu dem Ansatz, das relationale
Modell auch f\"{u}r Persistenz von \clos-Objekten zu benutzen; inwieweit
dieser Ansatz sinnvoll ist, wird \ua\ in
Kapitel~\ref{chap:arch} (\citepage{\pageref{chap:arch}}) analysiert.
%
\subsection{Eingebettete Datenbank-Sprachen}
%
Da relationale Datenbanken normalerweise \"{u}ber keine
\gande{Datenmanipulationssprache}{DML}\ ver\-f\"{u}\-gen, die m\"{a}chtig
genug f\"{u}r eine Erstellung von Applikationen ist, bieten sie eine
Schnittstelle zu einer allgemeinen Programmiersprache an. Die Leistung
dieser Schnittstellen ersch\"{o}pft sich meist in einer syntaktischen
Umsetzung der im Programm eingef\"{u}gten Datenbank-Anfragen in die
Datenmanipulationssprache, ohne da\ss{} die Konzepte der
Programmiersprache ber\"{u}cksichtigt werden; ebensowenig finden sich in
der Programmiersprache geeignete Konzepte f\"{u}r die Verarbeitung von
relationalen Daten. Der bei Verwendung einer solchen Schnittstelle
entstehende Code besteht damit meist aus einem Gemisch zweier
Sprachen, der Datenmanipulationssprache und der allgemeinen
Programmiersprache (siehe z.B.\ {\em Embedded SQL\/}
\cite[\citepage{201--283}]{bib:on94} oder die
{\em LIBPQ\/}-Schnittstelle von
\postgres\ \cite[\citepage{112--126}]{bib:we93}). Persistenz \"{u}ber
eingebettete Datenbank-Sprachen ist weder orthogonal noch transparent.
%
\subsection{\protect\pascalr}
%
In \pascalr\ \cite{bib:schm80} \cite{bib:schm77}
\cite[\citepage{5--6}]{bib:cl91} wurde 
eine relationale Datenbank in \pascal\ integriert. Die Erg\"{a}nzung
besteht zum einen in der Erweiterung um die Typen
\class{relation}\ und \class{database}, deren Instanzen Relationen
bzw.\ relationale Datenbanken sind, und zum anderen in der
Bereitstellung von Konstrukten zum Traversieren und Manipulieren von
Relationen, die vollst\"{a}ndig in die Sprache integriert wurden. Da in
\pascalr\ ausschlie\ss{}lich Tupel persistent gehalten werden k\"{o}nnen,
ist Persistenz nicht orthogonal.  Durch die Auswahl und die erfolgte
Integration der Manipulationsm\"{o}glichkeiten in \pascal\ ist die f\"{u}r
Tupel angebotene Persistenz weitgehend transparent. Die
Weiterentwicklung von \pascalr\ f\"{u}hrte zu der auf Modula-2
basierenden relationalen Programmiersprache
\mbox{DBPL}\ \cite{bib:ma92a}\ \cite{bib:ma92b}.
%
\section{Nicht-relationale Ans\"{a}tze}
%
Da die Kopplung einer relationalen Datenbank mit einer allgemeinen
Programmiersprache wegen der unterschiedlichen Datenmodellierung
\ia\ dazu f\"{u}hrt, da\ss{} die M\"{o}glichkeiten beider Systeme nicht
zufriedenstellend genutzt werden k\"{o}nnen, begann ab 1980 eine
Entwicklung, bei der nicht das relationale Modell in
eine allgemeine Programmiersprache integriert wurde, sondern umgekehrt
eine Programmiersprache um Persistenz erweitert wurde.
%
\subsection{PS/Algol}
%
In PS/Algol \cite[\citepage{16--22}]{bib:cl91}
\cite[\citepage{55}]{bib:ni88} wird Persistenz erstmals als
grundlegende Eigenschaft eines Datums unabh\"{a}ngig vom Typ angesehen;
es handelt sich damit um die erste Programmiersprache mit orthogonaler
Persistenz. Anders als in \pascalr\ wurden persistente Objekte nicht
nach dem relationalen Modell realisiert, sondern ihre Zust\"{a}nde werden
in einem persistenten \heap\ abgelegt; innerhalb dieses
\heap[s]\ werden Referenzen explizit repr\"{a}sentiert. F\"{u}r die
Lokalisierung von Objekten wird die M\"{o}glichkeit geboten, sie unter
einem Namen ablegen und adressieren zu k\"{o}nnen. Der Nachfolger von
PS/Algol ist die persistente Programmiersprache Napier
\cite{bib:mo89b} \cite{bib:mo89a}; dort werden
zus\"{a}tzlich Prozeduren als persistente Objekte aufgefa\ss{}t.
%
%\section{PQuest}
%
\section{Tycoon}
%
Tycoon (Typed Communicating Objects in Open Environments)
\cite{bib:ma93} \cite{bib:ma92c}\ baut auf den mit \pascalr\
und \mbox{DBPL}\ gemachten Erfahrungen auf.  Die Tycoon Umgebung sieht
Persistenz wie PS/Algol als grundlegende Eigenschaft eines Datums an.
Sie ist in Tycoon vollst\"{a}ndig und transparent von vornherein in die
Sprache integriert worden. Durch die orthogonale Sichtweise von
Persistenz wird auf eine Modellierung des verwendeten persistenten
Speichers und einem direkten Zugriff auf die in ihm realisierten
Manipulationsm\"{o}glichkeiten innerhalb der Sprache verzichtet; statt
dessen werden in Tycoon die gew\"{u}nschten Datenbankkonstrukte mittels
der Sprache selbst realisiert.
%
\section{Fazit}
%
Das relationale Modell ist konzeptionell sehr elegant und einfach; es
bietet m\"{a}chtige Manipulationsm\"{o}glichkeiten und mathematisch
beweisbare \rglq{}angenehme\rgrq\ Eigenschaften. Wie jedoch in
Abschnitt~\ref{sec:plobjs} (\citepage{\pageref{sec:plobjs}}) gezeigt
wird, f\"{u}hren die durch das Modell vorgegebenen Kriterien zu
Einschr\"{a}nkungen, die gegen die Verwendung des relationalen Modells
f\"{u}r persistente \clos-Objekte im Rahmen \ifbericht dieses
Berichtes \else\ifbuch dieses Buches \else dieser Arbeit \fi\fi
sprechen. Statt dessen vertrete ich den Ansatz, f\"{u}r Persistenz den
durch PS/Algol vorgezeigten L\"{o}sungsweg weiter zu verfolgen.
%%% Local Variables: 
%%% mode: latex
%%% TeX-master: "main"
%%% buffer-file-coding-system: raw-text-unix
%%% End: 
