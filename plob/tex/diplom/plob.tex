%
\chapter{Persistente LISP Objekte}%
\label{chap:plob}%
%
Die Zielsetzung f\"{u}r die Entwicklung von \plob\ ({\em P\/}ersistente
{\em L\/}ISP {\em Ob\/}jekte) besteht in einem Einsatz des Systems im
Rahmen des Projekts \oscar\ zur langlebigen Speicherung von LISP
Objekten, die unter Umst\"{a}nden sehr gro\ss{}e Daten referenzieren, wie
z.B.\ Satellitenbilder hoher Aufl\"{o}sung; ebenso kann die Anzahl der zu
speichernden Objekte sehr gro\ss{} werden. Aus diesem Grund soll das
persistente System auch eine Speicherr\"{u}ckgewinnung haben. Da ein Teil
der persistenten Instanzen geografische Objekte repr\"{a}sentiert, sollen
persistente Objekte assoziativ m\"{o}glichst mit mehrdimensionalen
Schl\"{u}sseln gesucht werden k\"{o}n\-nen.
%
\par{}Die Anforderungen an ein persistentes System im Rahmen
\ifbericht dieses Berichtes \else\ifbuch dieses Buches \else dieser
Arbeit \fi\fi werden von den von mir betrachteten vorhandenen Systemen
{\em Shared Object Hierarchy\/} (\soh\ifbericht\ 
\cite[\citepage{67}]{bib:ki94a}\ \else, Kapitel~\ref{chap:soh},
\citepage{\pageref{chap:soh}}\fi) und {\em Persistent \clos-Objects\/}
(\pclos\ \cite{bib:pa91a}) nicht erf\"{u}llt.
%; statt dessen wurden die in diesen Systemen vorgefundenen und
%f\"{u}r positiv gehaltenen Konzepte in das von mir realisierte System
%integriert.
%
\section{\"{U}berblick}%
%
Das System \plob\ bietet orthogonale Persistenz f\"{u}r den Zustand fast
aller in einem
\cl-System vorkommenden Daten mit Ausnahme von Funktionscode. Durch
Orthogonalit\"{a}t wird eine hohe Transparenz erreicht. Das System nimmt
zun\"{a}chst an, da\ss{} beim Speichern eines Objektes grunds\"{a}tzlich alle
davon referenzierten Instanzen ebenfalls persistent gehalten werden
sollen. Es werden hierarchische, auf Modularisierungsebene geordnete
Deklarationen benutzt, um festzustellen, welche referenzierten
Instanzen tats\"{a}chlich persistent gehalten werden sollen; damit kann
die Anzahl der persistent gehaltenen Objekte begrenzt werden. Die
angebotenen Datenbankfunktionalit\"{a}ten k\"{o}n\-nen ebenfalls deklarativ
eingebunden werden.
%
\par{}F\"{u}r Instanzen der Basistypen wird Persistenz \"{u}ber Funktionen
zur Verf\"{u}gung gestellt; die  Namen der Funktionen wurden aus dem in
\cite{bib:st90} spezifizierten \cl\ \std\ f\"{u}r transiente Objekte
abgeleitet.
%
\par{}Die Struktur eines persistenten Objektes wird ebenfalls als
persistentes Objekt repr\"{a}sentiert; \plob\ sorgt daf\"{u}r, da\ss{} sich
beim Abspeichern eines transienten Objektes seine aus der \cls\ des
Objektes abgeleitete Strukturrepr\"{a}sentation ebenfalls im persistenten
Speicher befindet.
%
\par{}Au\ss{}er Persistenz wurden noch zu\-s\"{a}tz\-li\-che als n\"{u}tzlich
angesehene Funktionalit\"{a}ten in \plob\ integriert:
\begin{description}
%
\item[Sitzungen]
Auf persistente Objekte kann gleichzeitig durch mehrere Sitzungen
zugegriffen werden.\footnote{F\"{u}r Mehr-BenutzerInnen-Zugriff siehe
Abschnitt~\ref{sec:plsac}, \citepage{\pageref{sec:plsac}}.}
%
\item[Transaktionen]
Alle Zugriffe auf persistente Objekte laufen innerhalb einer an ihre
Sitzung gebundenen Transaktion ab.
%
\item[{\gande{Objektsperren}{locking}}]
Ein persistentes Objekt wird transaktionsgebunden durch die lesende
bzw.\ modifizierende Sitzung gesperrt; Konflikte durch mehrere auf
ein gesperrtes Objekt zugreifende Sitzungen werden erkannt und
aufgel\"{o}st.
%
\item[B-B\"{a}ume]
Zum assoziativen Zugriff auf persistente Objekte wurden persistente
B-B\"{a}ume realisiert; diese k\"{o}n\-nen \"{a}hn\-lich wie \hashtbl{}n
verwendet werden.
%
\item[Indizes \"{u}ber {\Slt[s]}; assoziativer Zugriff]
F\"{u}r einen \Slt\/ von aktiv-persistenten Instanzen kann deklarativ
ein Index definiert werden, mit dem assoziativ auf aktiv-persistente
Instanzen zugegriffen werden kann.
%
\item[\gande{Modifikationen an Klassen-Metaobjekten}{schema
evolution}]
\"{A}nderungen an einer transienten \cls[n]definition werden in den
persistenten Speicher propagiert; die Struktur der betroffenen
persistenten Objekte wird nach den \"{u}blichen \clos\ Regeln auf den
Stand der neuen \cls[n]definition gebracht.
%
\end{description}
%
Der folgende Text erl\"{a}utert die oben aufgef\"{u}hrten Punkte im
Einzelnen.
%
\section{Entwurfsprinzipien}
%
Beim Entwurf des Gesamtsystems wurden die folgenden Prinzipien
ber\"{u}cksichtigt.
%
\subsection{\protect\cl-gem\"{a}\ss{}e Semantik}
%
Um eine m\"{o}glichst einfache Schnittstelle zu \cl\ zu erhalten, wurde
dessen Datenmodellierung \"{u}bernommen. An einigen Stellen ergaben sich
durch die Erweiterung der Lebensdauer von Objekten \"{u}ber den transienten
LISP Proze\ss{} hinaus Probleme mit der Semantik; dies betrifft in erster
Linie Klassen, deren Instanzen von LISP als eindeutig angesehen
werden, wie Symbole und Pakete. F\"{u}r diese Klassen wurde eine
entsprechende Semantik definiert.
%
\subsection{Schichtenmodell}
%
Das Gesamtsystem ist in aufeinander aufbauende Schichten gegliedert;
die verschiedenen Abstraktionsebenen sind damit klar getrennt. Ebenso
lassen sich untere Schichten austauschen, ohne die dar\"{u}ber
angesiedelten Schichten \"{a}ndern zu m\"{u}ssen.
%
\subsection{Portabilit\"{a}t}
%
Aus interner Sicht war Portabilit\"{a}t sowohl der System-Quelltexte als
auch der im persistenten Speicher abgelegten Objekte ein wichtiges
Grundprinzip.
%
\subsubsection{Portabler Code}
%
Der C Code wurde nach ANSI \std\ erstellt \cite{bib:ker88}. Der
\cl\ Code h\"{a}lt sich soweit m\"{o}glich an die in \cite{bib:st90} und
\cite{bib:amop91} gegebenen Spezifikationen; systemspezifischer
LISP Code wurde, soweit sinnvoll, in einem einzigen Modul
zusammengefa\ss{}t. Da bisher keine system\"{u}bergreifende
Standardisierung einer Schnittstelle zwischen \cl\ und C erfolgte,
wurde darauf geachtet, zwischen \cl\ und C nur
einfache Daten, d.h.\ Instanzen einiger wenigen Basistypen,
auszutauschen. Ausnahmen erfolgten lediglich dann, wenn eine
Transformation in eine f\"{u}r den Austausch geeignete Instanz einen
unverh\"{a}ltnism\"{a}\ss{}ig hohen Effizienzverlust bedeutet h\"{a}tte.
%
\subsubsection{Portable persistente Objekte}
%
Um auf die im persistenten Speicher enthaltenen persistenten Objekte
mit verschieden realisierten Applikationsschichten zugreifen zu k\"{o}n\-nen,
wurde darauf Wert gelegt, bereits in den unteren Schichten eine
vollst\"{a}ndige Typisierung der persistenten Objekte einzuf\"{u}hren; jedes
persistente Objekt ist entweder Instanz eines Basistyps oder enth\"{a}lt
eine Referenz auf eine (persistente) Beschreibung seiner \cls.
\clsdo[e]\ selbst sind auch persistente Objekte.
Die im persistenten Speicher enthaltenen Objekte sind damit
selbstbeschreibend (Abschnitt~\ref{sec:ocls},
\citepage{\pageref{sec:ocls}}).
%
\subsection{Objekt-orientierte Programmierung}%
\label{sec:ploopr}%
%
Die Vorteile objekt-orientierter Programmierung wurden bereits in
anderen Arbeiten ausf\"{u}hrlich geschildert \cite{bib:go89}; aus
diesen Gr\"{u}nden wurde in der in C realisierten 2.~Schicht (siehe
nachfolgender Abschnitt~\ref{sec:arplob}) ein einfaches, nur in dieser
Schicht benutztes Objektsystem realisiert, das direkt auf den
\representations{} der persistenten Objekte arbeitet. Dieses
Objektsystem erg\"{a}nzt die C-typische statische Typisierung von
Datenobjekten zur \"{U}bersetzungszeit um eine dynamische \cls[n]bindung
der persistenten Objekte zur Laufzeit.
%
\par{}Durch dieses Objektsystem k\"{o}n\-nen einige erstellte Komponenten
mehrfach verwendet werden; so werden beispielsweise die in der
2.~Schicht realisierten persistenten B-B\"{a}ume sowohl f\"{u}r
verschiedene interne Tabellen genutzt als auch den BenutzerInnen
extern zur Verf\"{u}gung gestellt.
%
\par{}Ein anderer pragmatischer Aspekt war, da\ss{} durch die
\cl-C-Kopplung kein \gande{Programm zur
Fehlersuche}{debugger}[Programm zur Fehlersuche; kontrolliert den
Ablauf des zu untersuchenden Programms und f\"{a}ngt dessen
Laufzeitfehler ab.]\ in der 2.~Schicht verwendet werden konnte; durch
die dynamische \cls[n]bindung k\"{o}n\-nen beispielsweise Typpr\"{u}fungen von
Objekten vorgenommen werden, um bei einem unerwartetem Typ
kontrolliert einen Fehler signalisieren zu k\"{o}n\-nen.
%
%\subsection{Performanz; Effizienz}
%
% Argumentation in \cite[\citepage{30}]{bib:sto89}, da\ss{} \cl\ Code
% mindestens 2 mal langsamer ist als C Code.
%
\subsection{Objektspeicher als persistenter Speicher}%
\label{sec:plobjs}%
%
Die Verwendung eines Objektspeichers als persistenter Speicher
resultiert aus den bisher in \ifbericht diesem Bericht \else\ifbuch
diesem Buch \else dieser Arbeit \fi\fi dargelegten Einschr\"{a}nkungen
bei Verwendung einer relationalen Datenbank; sie ergeben sich aus der
unterschiedlichen Datenmodellierung von \cl\ und relationalen
Datenbanken:
\begin{itemize}
%
\item In \cl\ wird die Identit\"{a}t von Objekten repr\"{a}sentiert;
bei einer relationalen Datenbank mu\ss{} eine \representation{} der
Objektidentit\"{a}t durch das persistente System auf h\"{o}\-he\-ren
Schichten realisiert werden. Die Datenbank selbst hat keine Kenntnis
von dieser Identit\"{a}tsrepr\"{a}sentation und kann damit keine Referenzen
aufl\"{o}sen.
%
\item Die Typinformation eines Objektes ist in \cl\ an das Objekt
gebunden. In relationalen Datenbanken ist das Konzept eines Objektes
im Sinne der objekt-orientierten Programmierung unbekannt; bei der
in persistenten Systemen mit einer relationalen Datenbank benutzten
Abbildung zwischen den Objektsystem- und Datenbank-Strukturen
m\"{u}s\-sen \ia\ die die \Slt\/-Zust\"{a}nde repr\"{a}sentierenden
Attribute statisch typisiert werden.
%
\item Innerhalb einer relationalen Datenbank werden Tupel als Werte
und nicht als \rglq{}echte\rgrq\ Objekte aufgefa\ss{}t. Die Identit\"{a}t
eines Tupels ergibt sich damit aus seinem Zustand. In \cl\ hingegen
ist die Identit\"{a}t eines Objektes un\-ab\-h\"{a}n\-gig von seinem
Zustand; es kann sowohl Werte oder Objekte referenzieren als auch
selber referenziert werden. \soh\ und \pclos\ umgehen dieses Problem
durch Konvertierung von Referenzen zwischen \cl\ Daten in
Datenbankwerte, die zur Aufl\"{o}sung der Referenz wiederum evaluiert
oder konvertiert werden m\"{u}ssen; damit sind die Referenzen zwischen
Objekten in der relationalen Datenbank aber nur noch implizit
erkennbar.
%
\par{}Referenzen zwischen transienten \cl\ Objekten werden \"{u}ber ihre
transiente Iden\-ti\-t\"{a}ts\-re\-pr\"{a}\-sen\-ta\-tion (Adresse)
hergestellt; daraus folgte f\"{u}r meinen Entwurf, da\ss{} Referenzen
zwischen Objekten im persistenten Speicher auch \"{u}ber ihre dortige
Iden\-ti\-t\"{a}ts\-re\-pr\"{a}\-sen\-ta\-tion (\objid\/) hergestellt werden
sollen.
%
\item Die von \soh\ und \pclos\ vorgenommenen Konvertierungen zwischen
der transienten und der persistenten \representationform{} f\"{u}hren
meist dazu, da\ss{} die persistente Re\-pr\"{a}\-sen\-ta\-tions\-form
wesentlich ineffizienter als die transiente Form ist
\ifbericht\else(siehe z.B.\ \tablename~\ref{tab:bsplocal}, Zeile
\attr{local}{loid}\ CONS4711,
\citepage{\pageref{tab:bsplocal}})\fi\ und die Konvertierung relativ
viel Ressourcen verbraucht.
%
\par{}Aus diesem Grund pa\ss{}te ich die \representationform{} von
persistenten Objekten im Objektspeicher soweit m\"{o}glich an die im
transienten  LISP vorgefundenen transienten \representationforms\ an;
damit wird der Aufwand f\"{u}r Konvertierungen wesentlich geringer.
%
\end{itemize}
%
Um den Zeitaufwand f\"{u}r die Entwicklung des Systems zu begrenzen,
wurde festgelegt, f\"{u}r den Objektspeicher ein fertiges Subsystem zu
verwenden. Eine ausf\"{u}hrliche Analyse mehrerer in Frage kommender
Subsysteme findet sich in \cite[\citepage{15--83}]{bib:mu91}.
\ifbericht Im Rahmen dieses Berichtes \else\ifbuch Im Rahmen dieses
Buches \else In dieser Arbeit \fi\fi wurde auf einen erneuten
Vergleich verzichtet; als einzige M\"{o}glichkeit der in
\cite[\citepage{85--87}]{bib:mu91} getroffenen Auswahl des
Objektspeichers blieb auch \ifbericht im Rahmen dieses Berichtes
\else\ifbuch im Rahmen dieses Buches \else f\"{u}r diese Arbeit \fi\fi
das System \postore\ ({\em p\/}ersistent {\em o\/}bject {\em store\/})
\"{u}brig.
%
\section{Architektur}\label{sec:arplob}
%
\figurename~\ref{fig:ploblay} %
%
\begin{figure}[htbp]%
\ifbuch%
\centerline{\psfig{figure=twelvept/ploblay.eps}}%
\else%
\centerline{\psfig{figure=tenpt/ploblay.eps}}%
\fi%
\caption{\protect\plob\ Schichtenmodell}%
\label{fig:ploblay}%
\end{figure}%
%
zeigt ein Schichtenmodell des \plob\ Systems. Die 1.~Schicht wird aus
dem \"{u}ber\-nom\-me\-nen System \postore\ gebildet. Bei der Aufteilung
der \"{u}brigen Schichten wurden alle Basisfunktionalit\"{a}ten, wie
Transaktionen sowie Allozieren, Deallozieren oder Sperren von
persistenten Objekten, in der 2.~Schicht realisiert. Die Konzeption
erfolgte so, da\ss{} die Operationen \"{u}ber persistente Objekte oberhalb
der 2.~Schicht die in
\cite[\citepage{6, 166}]{bib:gr93} definierten \acid%
\footnote{Von der deutschen \"{U}bersetzung {\sc uksb} wird hier Abstand
genommen.}%
\addglossary{ACID}[{Abk\"{u}rzung f\"{u}r die vier Eigenschaften einer
Datenbank-Transaktion:
\protect\gande{\protect\see{Un\-teil\-bar\-keit}}%
{atomicity},
\protect\gande{\protect\see{Kon\-si\-stenz}}%
{consistency},
\protect\gande{\protect\see{Se\-ria\-li\-sier\-bar\-keit}}%
{isolation}\ und
\protect\gande{\protect\see{Be\-st\"{a}n\-dig\-keit}}%
{durability}.}]\ genannten
Eigenschaften f\"{u}r Transaktionen erf\"{u}llen:
\begin{quote}%
%
\def\acidprop#1#2#3#4{%
\item[{\gande{#1}{#2}[][{Die #3.\ Eigenschaft der \protect\acid-Eigenschaften
f\"{u}r Da\-ten\-bank-Trans\-ak\-tio\-nen: #4}]:}] {\let\see\relax #4}}
%
\begin{description}\label{ref:acid}%
%
%\item[Atomicity.] A transaction's changes to the state are
%atomic:\ either all happen or none happen. %
%% These changes include database changes, messages, and actions on
%% transducers.
%[\ldots]
\acidprop%
 {Unteilbarkeit}%
 {atomicity}%
 {1}%
 {Die von einer Transaktion vorgenommenen
  Zu\-stands\-\"{a}n\-de\-run\-gen sind unteilbar. 
  %Entweder alles oder nichts passiert.%
  }\ [\ldots]%
%
%\item[Consistency.] A transaction is a correct transformation of the
%state. The actions taken as a group do not violate any of the
%integrity constraints associated with the state.
%%This requires that the transaction be a correct program.
%[\ldots]
%
\acidprop%
 {Konsistenz}%
 {consistency}%
 {2}%
 {Eine Transaktion ist eine korrekte Zu\-stands\-\"{a}n\-de\-rung. Die zu
  einer Gruppe zusammengefa\ss{}ten Aktionen verletzen keinerlei
  In\-te\-gri\-t\"{a}ts\-be\-schr\"{a}n\-kun\-gen des Zustands.}\ [\ldots]%
%
%\item[Isolation.] Even though transactions execute concurrently, it
%appears to each transaction, {\it T}, that others executed
%either before {\it T} or after {\it T}, but not both.
%
\acidprop%
 {Serialisierbarkeit}%
 {isolation}%
 {3}%
 {Auch bei gleichzeitig laufenden Transaktionen sieht es f\"{u}r jede
  Transaktion {\it T} so aus, als ob andere Transaktionen entweder vor
  {\it T} oder nach {\it T} ausgef\"{u}hrt w\"{u}r\-den.}%
%
%\item[Durability.] Once a transaction completes successfully
%(commits), its changes to the state survive failures.
%
\acidprop%
 {Best\"{a}ndigkeit}%
 {durability}%
 {4}%
 {Nach \protect\gande{Ende einer Transaktion}%
  {\protect\see{commit}}\ \"{u}berstehen die
  Zu\-stands\-\"{a}n\-de\-run\-gen Systemausf\"{a}lle.}%
\addglossary{Ende einer Transaktion}[commit][Damit gilt der aus den
 w\"{a}hrend der Transaktion erfolgten Modifikationen hervorgegangene
 Gesamtzustand der Datenbank als konsistent.]%
%
\end{description}%
\end{quote}%
%
\par{}Die 3.~Schicht stellt unter Benutzung der
Basisfunktionalit\"{a}ten der 2.~Schicht die Verbindung zu den
transienten Objekten in den Sitzungen der 4.~Schicht her; so wird
in der 3.~Schicht z.B.\ beim Abspeichern eines transienten Objektes
aus der 4.~Schicht die transitiv referenzierte H\"{u}lle des Objektes
traversiert und mit abgespeichert.
%
\par{}In den nachfolgenden Abschnitten werden die extern sichtbaren
Konzepte der einzelnen Schichten erl\"{a}utert.
%
\section{Schicht 1: Die \postore\ Schicht}
%
\par{}Das System \postore\ \cite{bib:br92}
\cite[\citepage{63--83}]{bib:mu91} realisiert einen persistenten
\heap\ (Abschnitt~\ref{sec:objstore},
\citepage{\pageref{sec:objstore}}). Die Implementation erfolgte in C
nach Kernighan \& Ritchie- (d.h.\ nicht-\mbox{ANSI-)} \std\ auf SUN
Arbeitsplatzrechnern unter SunOS.
%
\par{}In \cite[\citepage{63--83}]{bib:mu91} wird das System
\postore\ sehr ausf\"{u}hrlich beschrieben und analysiert; die rein
funktionale Schnittstelle wird vom Autor des Systems selbst in
\cite{bib:br92} nur sehr knapp beschrieben. An dieser Stelle werden
lediglich die zum weiteren Verst\"{a}ndnis wichtigen Konzepte
geschildert.
%
\par{}Als persistenter \heap\ bietet \postore\ die M\"{o}glichkeit, einen
persistenten Speicherbereich zu allozieren und zu modifizieren; ein
solcher Speicherbereich wird im folgenden \postore-Vektor genannt.
Ein allozierter \postore-Vektor ist genau dann persistent, wenn er
erreichbar ist, d.h.\ wenn er transitiv vom Wurzelobjekt des
\postore\ referenziert wird. \figurename~\ref{fig:pvec} %
%
\begin{figure}[htbp]%
\centering%
\def\coliw{5em}%
\def\coliiw{9em}%
\begin{\figurefontsize}%
\begin{tabular}{|p{\coliw}|p{\coliw}||p{\coliiw}|p{\coliiw}|}%
\hline
\parbox{\coliw}{\centering\it m}
 &\parbox{\coliw}{\centering\it n}
  &\parbox{\coliiw}{\centering\smallskip%
                    {\it m} Worte Referenzen\smallskip}
   &\parbox{\coliiw}{\centering\smallskip%
                     ({\it n}\,--\,{\it m}\,--\,2) Worte
                     Werte\smallskip}\\
\hline
\multicolumn{1}{c}{\parbox{\coliw}{%
\centering\smallskip\small Anzahl\\ \makebox[0pt]{Referenzen}}}
 &\multicolumn{1}{c}{\parbox{\coliw}{%
\centering\smallskip\small Vektor-\\gr\"{o}\ss{}e}}
  &\multicolumn{1}{c}{\smallskip\small Referenzfeld}
   &\multicolumn{1}{c}{\smallskip\small Wertefeld}
\end{tabular}%
\end{\figurefontsize}%
\caption{\protect\postore-Vektor}\label{fig:pvec}%
\end{figure}%
%
zeigt die Struktur eines \postore-Vektors. \postore\ benutzt 2~Worte
zu je 32~Bit Kopf\mbox{}informationen. Danach folgen die frei zu
belegenden Worte, die sich in einen Bereich mit {\it m} Referenzen auf
andere \postore-Vektoren und einen Bereich mit ({\it n}\,--\,{\it
m}\,--\,2) Worten uninterpretierten Werten aufteilen. Referenziert wird
ein \postore-Vektor \"{u}ber seine 32~Bit \objid, die bei der
Allozierung des Vektors vergeben wird; die \objid\/ selbst enth\"{a}lt
keinerlei Informationen \"{u}ber die repr\"{a}sentierten Instanzen und ist
damit nach der in Abschnitt~\ref{sec:ovidrepr}
(\citepage{\pageref{sec:ovidrepr}}) gegebenen
\"{U}bersicht ein Surrogat. Die in dieser Schicht vergebene
\objid\/ wird zur Unterscheidung der in h\"{o}heren Schichten verwendeten
\sobjid\/ hier als \lobjid\/ bezeichnet.
%
\par{}Es werden verschiedene Funktionen zum Zugriff auf einen
\postore-Vektor angeboten. Die in \cite[\citepage{63}]{bib:mu91}
erw\"{a}hnte generische Architektur von \postore\ macht sich in dieser
Schicht auch dadurch bemerkbar, da\ss{} die Zugriffsfunktionen keinerlei
G\"{u}ltigkeitspr\"{u}fungen ihrer Argumente durchf\"{u}hren. Konkurrenter
Zugriff kann durch Setzen einer Schreibsperre %
%
\addglossary{Schreibsperre}[Markierung eines Objektes als
\protect\rglq{}gesperrt\protect\rgrq, um anderen zugreifenden Instanzen
zu signalisieren, da\ss{} der Objektzustand schreibend referenziert
wird.]%
%
synchronisiert werden. Die Auswertung, ob ein \postore-Vektor
schreibgesperrt ist oder nicht, mu\ss{} in einer h\"{o}heren Schicht
erfolgen; \postore\ f\"{u}hrt bei schreibenden Zugriffen keinerlei
Pr\"{u}fungen in dieser Hinsicht durch.
%
\par{}Als Hilfsmittel f\"{u}r Persistenz benutzt \postore\ die \sh-Datei,
in der die Zust\"{a}nde der persistenten Objekte abgelegt
werden.\footnote{Die \sh-Datei wird \"{u}ber den Aufruf der
Betriebssystemfunktion \lisp{mmap}\ als virtueller Speicher in
den Adre\ss{}raum des \unix\ Prozesses eingeblendet
\cite[\citepage{81}]{bib:mu91} \cite[\citepage{666}]{bib:gr93}
\cite{bib:mmap}; die \postore-Vektoren werden dann
direkt in diesem virtuellen Adre\ss{}raum alloziert und
modifiziert.} Modifikationen an den \postore-Vektoren wirken sich
nicht sofort auf die \sh-Datei aus; alle \"{A}nderungen der Vektoren
sind tempor\"{a}r, bis sie durch den Aufruf einer Stabilisierungsfunktion
in die \sh-Datei \"{u}bertragen werden.
%
\par{}Im \postore\ ist eine Speicherr\"{u}ckgewinnung realisiert, die
nicht vom Wurzelobjekt referenzierte \postore-Vektoren freigibt; die
R\"{u}ckgewinnung wird nicht automatisch ausgel\"{o}st, sondern mu\ss{} durch
einen expliziten Aufruf der entsprechenden Funktion veranla\ss{}t werden
\cite[\citepage{65}]{bib:mu91}.
%
\subsection{Zusammenfassung der 1.\ Schicht}
%
Da das System \postore\ als generisches System konzipiert wurde,
bietet es keine Transaktionsverarbeitung f\"{u}r einzelne
\postore-Vektoren; die Transaktionsverarbeitung wird statt dessen f\"{u}r
den persistenten Speicher als gesamte Einheit durchgef\"{u}hrt, d.h.\ eine
Transaktionsverarbeitung betrifft alle \postore-Vektoren. Ebenso ist
die Transaktionsverarbeitung nicht transparent, da die entsprechenden
Funktionen explizit aufgerufen werden m\"{u}ssen.  Damit k\"{o}n\-nen h\"{o}here
Schichten eine eigene Transaktionsverarbeitung realisieren.
\begin{description}
%
\item[Unteilbarkeit]
Die angebotenen Schreibsperren sind als Semaphoren
\cite[\citepage{143}]{bib:je87} realisiert; in Verbindung mit auf
h\"{o}heren Schichten emulierten Lesesperren
\cite[\citepage{73}]{bib:mu91} \cite[\citepage{2}]{bib:br92} kann
Unteilbarkeit erreicht werden. Die Sperrung eines \postore-Vektors
durch einen Semaphor mu\ss{} von den BenutzerInnen beachtet werden,
d.h.\ das \postore-be\-nut\-zen\-de Programm mu\ss{} selbst pr\"{u}fen, ob
der Semaphor gesetzt ist und gegebenenfalls darauf warten, da\ss{} er
wieder freigegeben wird.
%
\item[Konsistenz]
Ein konsistenter Zustand des Gesamtsystems wird \postore\ durch Aufruf
der Stabilisierungsfunktion signalisiert; das System \postore\ stellt
dann sicher, da\ss{} der tem\-po\-r\"{a}\-re Zustand aller modifizierter
\postore-Vektoren in die \sh-Datei \"{u}bertragen wird.
%
\par{}Bei Auftreten eines Fehlers wird der persistente Speicher beim
n\"{a}chsten \"{O}ffnen in den Zustand nach der letzten erfolgreichen
Stabilisierung zur\"{u}ckgesetzt; diese R\"{u}cksetzung kann simuliert, aber
nicht explizit veranla\ss{}t werden.
%
\item[Serialisierbarkeit]
Serialisierbarkeit kann erreicht werden, sofern der Zugriff auf die
\postore-Vektoren von den BenutzerInnen entsprechend organisiert
wird, indem z.B.\ alle Zugriffe innerhalb einer
\twophasetr\ stattfinden. \postore\ bietet alle Hilfsmittel
(Schreibsperren) zur Realisierung dieses Konzeptes; eine konkrete
Realisierung mu\ss{} aber auf einer h\"{o}heren Schicht erfolgen.
%
\par{}Nach der in \cite[\citepage{401}]{bib:gr93} angegebenen
\"{U}bersicht realisiert \postore\ eine Serialisierbarkeit von
0$^{\circ}$.
%
\item[Best\"{a}ndigkeit]
Die \postore-Vektoren werden in der \sh-Datei gesichert. Damit h\"{a}ngt
die Best\"{a}ndigkeit des Gesamtsystems von der Best\"{a}ndigkeit dieser
Datei ab; eine Er\-h\"{o}\-hung derselben lie\ss{}e sich auf
Betriebssystemebene erreichen (z.B.\ durch Sichern oder Spiegeln der
\sh-Datei).
%
\end{description}
%
Der folgende Abschnitt fa\ss{}t noch einmal kurz die Funktionalit\"{a}ten
der \postore\ Schicht aus der Sicht der n\"{a}chsth\"{o}heren 2.~Schicht
zusammen.
\begin{description}
%
\item[Allozieren eines \postore-Vektors]
Bei der Allozierung wird die Gesamtgr\"{o}\ss{}e {\it m} des Vektors
angegeben; zur\"{u}ckgegeben wird die zur Referenzierung des Vektors
n\"{o}tige \lobjid.
%
\item[Zugriff auf den Zustand von \postore-Vektoren]
Es werden Funktionen zum un\-ge\-pr\"{u}f\-ten Zugriff auf
einzelne oder mehrere Worte innerhalb eines \postore-Vektors
angeboten.
%
\item[Wurzelobjekt]
Die Struktur und der Zustand des \postore-Wurzelobjektes werden
innerhalb der 1.~Schicht definiert und verwaltet; es kann nicht
explizit gelesen oder \"{u}berschrieben werden au\ss{}er der frei
zugreifbaren Komponente, die die \lobjid\/ des Wurzelobjektes der
n\"{a}chsth\"{o}heren Schicht enth\"{a}lt (\figurename~\ref{fig:plroots},
\citepage{\pageref{fig:plroots}}).
%
\item[Speicherr\"{u}ckgewinnung]
Bei der Speicherr\"{u}ckgewinnung wird der Speicher der nicht transitiv
vom Wurzelobjekt referenzierten persistenten Objekte wieder
freigegeben. Die Speicherr\"{u}ckgewinnung mu\ss{} explizit veranla\ss{}t
werden.
%
\end{description}
%
\section{Schicht 2: Die C-\protect\plob\ Schicht}
%
Die Implementation der 2.~Schicht erfolgte in ANSI-C unter
zu\-s\"{a}tz\-li\-cher Benutzung eines ebenfalls von mir erstellten
minimalen Objektsystems. Diese Schicht realisiert
transaktionsgesicherten, durch das Sperrprotokoll unteilbaren Zugriff
auf dynamisch typisierte, selbstbeschreibende Objekte und stellt die
f\"{u}r eine Indexverwaltung notwendigen Hilfsmittel zur Verf\"{u}gung.
%
\subsection{\"{U}berlegungen zur Performanz und Datenabstraktion}%
\label{sec:plpeda}%
%
Eine hohe Performanz kann erreicht werden, wenn wie in
\wood\ \cite{bib:wo93} die transienten LISP Daten auf ihrer
niedrigsten Abstraktionsebene verarbeitet werden, d.h.\ wenn die
\"{U}ber\-tra\-gung eines Objektes zwischen dem transienten und dem
persistenten Speicher innerhalb der 2.~Schicht erfolgen w\"{u}rde und
dazu die transiente \representation{} des Objektes ausgenutzt werden
w\"{u}rde. Dies w\"{a}re mit folgenden Schwierigkeiten verbunden:
\begin{itemize}
%
\item Die dazu notwendigen Informationen \"{u}ber die Struktur der
transienten Objekte auf dieser niedrigen Ebene sind nur schwierig zu
erhalten, da sie von den AutorInnen des verwendeten LISP-Systems
festgelegt wurden und \ia\ von diesen nicht \"{o}ffentlich zug\"{a}nglich
gemacht werden (siehe z.B.\ \cite{bib:sn93}); weiterhin w\"{a}re das
persistente System bei einer derartigen Festlegung auf ein transientes
System nicht portabel, d.h.\ es ist nicht ohne weiteres m\"{o}glich, die
h\"{o}heren Schichten des Systems gegen ein anderes Objektsystem
auszutauschen. In \wood\ stellten sich diese Probleme nicht, da zum
einen sein Autor an der Entwicklung des zur Realisierung von
\wood\ verwendeten Macintosh \cl\ (MCL) direkt beteiligt ist und zum
anderen das Format des persistenten Speichers nach seinen Worten mit
der folgenden Absicht erstellt wurde \cite{bib:wo93}:
\begin{quote}
\rglqq It is not intended to be compatible with anything.\rgrqq
\end{quote}
%
\item Ein transientes Datenobjekt enth\"{a}lt au\ss{}er dem Objektzustand
noch unbekannte systeminterne Informationen zu seiner Verwaltung.
F\"{u}r die \representation{} eines persistenten Objektes im transienten
Speicher mu\ss{} gegebenenfalls eine transiente \representation{} alloziert
und mit den systeminternen Informationen initialisiert werden;
ferner kann die Allozierung auch zu ber\"{u}cksichtigende Seiteneffekte
nach sich ziehen.
%
\item \cl\ Daten sind oft sehr redundant (beispielsweise enth\"{a}lt ein
\clsmo\ eine Liste mit den \clsmo[en]\ der Subklassen). Auf
der LISP Ebene der 3.~Schicht ist die Bedeutung eines Datums oft
bekannt; damit k\"{o}n\-nen die Redundanzen eines transienten Objektes bei
seiner \"{U}bertragung in den persistenten Speicher weggelassen und bei
der R\"{u}ck\"{u}bertragung wieder hinzugef\"{u}gt werden. Dieses Wissen um die
Bedeutung fehlt in der 2.~Schicht.
%
\item Bei der \"{U}bertragung eines Objektzustands auf einer niedrigen
Abstraktionsebene kann die Bedeutung des Objektes f\"{u}r die h\"{o}heren
Ebenen verloren gehen. F\"{u}r die Rekonstruktion eines solchen Objektes
reicht eine blo\ss{}e Erzeugung der transienten \representation{} und die
\"{U}bertragung des Zustands nicht aus, sondern es mu\ss{} zu\-s\"{a}tz\-lich
die Bedeutung des Objektes mit ber\"{u}cksichtigt werden.  In \lw\ werden
beispielsweise \gande{Pakete}{packages}\ durch Instanzen einer
Strukturklasse repr\"{a}sentiert; eine Rekonstruktion mu\ss{} das
wiederhergestellte Paket dem LISP-System auch als Paket bekannt
machen.
%
\end{itemize}
%
\"{A}hnliche Feststellungen wurden auch bei der Realisierung des
persistenten Objektsystems \zeitgeist\ \cite[\citepage{35}]{bib:fo88}
gemacht. Aus den genannten Gr\"{u}nden wird die \"{U}bertragung von
Objektzust\"{a}nden von der 3.~Schicht aus durchgef\"{u}hrt.
%
\subsection{Typisierung der \protect\lobjid[s]\ f\"{u}r
\protect\immval[s]}%
%
Das System \postore\ garantiert, da\ss{} die vergebenen \lobjid[s]\ immer
ganzzahlige Vielfache von 8 sind, d.h.\ die untersten 3~Bit sind immer
0. Zus\"{a}tzlich werden die im Referenzfeld eines \postore-Vektors
abgelegten \lobjid[s]\ von \postore\ nur als Referenz angesehen, wenn
sie ganzzahlige Vielfache von 8 sind; andere Werte werden
niemals dereferenziert. Dies wurde benutzt, um \lobjid[s]\ zu
typisieren: die untersten 3~Bit der \lobjid[s]\ werden als
\gande{Typkennung}{type tag}[][Numerischer Wert, der den Typ
eines Objektes eineindeutig identifiziert.]\ benutzt. Die
Belegung ist in \tablename~\ref{tab:typetags} %
%
\begin{figure}[htbp]%
\centering%
\begin{\figurefontsize}%
\begin{tabular}{|r@{\ldots}l|c|c|c|c|r@{\ldots}l|l|}%
\cline{1-6}
\multicolumn{5}{|c|}{\tabularheader{Bit-Position}}
        &\tabularheader{Ken-}\\
\cline{7-9}
31 & 3 & 2 & 1 & 0  & \tabularheader{nung}
        & \multicolumn{2}{|c|}{\tabularheader{Zustand in}}
                &\tabularheader{Beschreibung}\\
\hline\hline
&& 0 & 0 & 0 & 0    & Bit 31 & 3 & \lobjid\/ als Referenz auf Objekt\\
&& . & 0 & 1 & 1, 5 & Bit 31 & 2 & Instanz der \cls\ \class{fixnum}\\
&& 0 & 1 & 0 & 2    & Bit 31 & 3 & Instanz der \cls\ \class{short-float}\\
&& 0 & 1 & 1 & 3    & Bit 31 & 3 & Instanz der \cls\ \class{character}\\
&& 1 & 0 & 0 & 4    & Bit 31 & 3 & [Instanz der \cls] Markierung\\
\multicolumn{2}{|c|}{}
 &   &   &   & 6, 7 & \multicolumn{2}{c|}{}
                                 & Zur Zeit unbenutzte Typkennungen\\
\hline
\end{tabular}%
\end{\figurefontsize}%
%
\let\figurename\tablename%
\caption{Typkennungen der typisierten \protect\lobjid[s]}%
\label{tab:typetags}%
\end{figure}%
%
wiedergegeben. Eine \lobjid\/ ist damit entweder eine Referenz auf
einen \postore-Vektor (Kennung~0 in \tablename~\ref{tab:typetags})
oder enth\"{a}lt den Zustand einer Instanz der in den Zeilen~2--5 in
\tablename~\ref{tab:typetags} aufgef\"{u}hrten Basisklassen.
F\"{u}r \immval[s]\/ enth\"{a}lt eine \lobjid\/ damit den Zustand des
repr\"{a}sentierten Objektes; f\"{u}r alle anderen Objekte ist eine
\lobjid\/ ein \rglq{}echtes\rgrq\ Surrogat ohne Informationen \"{u}ber das
repr\"{a}sentierte Objekt. Die hier realisierten typisierten
\lobjid[s]\ sind Surrogate mit nicht vollst\"{a}ndig kanonischer Form
(Abschnitt~\ref{sec:idsur}, \citepage{\pageref{sec:idsur}}).
%
\subsubsection{Markierungen}
%
In \cl\ werden \"{u}blicherweise ausgezeichnete Objekte als Markierungen
f\"{u}r bestimmte singul\"{a}re Eigenschaften oder Bedingungen verwendet,
wie z.B.\ in der \closette\ Implementation das in der Variablen
\lisp{secret-unbound-value}\ enthaltene Objekt
als \representation{} der Eigenschaft \rglq{}ungebundener \Slt\/\rgrq\
verwendet wird \amopcite{28}.  In \plob\ gibt es f\"{u}r Markierungen eine
eigene \rglq\cls\rgrq\ (Kennung~4 in \tablename~\ref{tab:typetags});
sie sind damit disjunkt zu allen anderen Daten. Markierungen werden
durch Instanzen der \rglq\cls\rgrq\ Markierung repr\"{a}sentiert.
%
\subsubsection{\protect\representation\ von \protect\objid[s]}
%
Aus Effizienzgr\"{u}nden sollen die in den nachfolgenden Schichten
verwendeten \objid[s]\ f\"{u}r ihre \representation{} m\"{o}glichst wenig
Speicherplatz ben\"{o}tigen; deswegen werden in den in \cl\ realisierten
h\"{o}heren Schichten zu ihrer \representation{} Instanzen des Typs
\class{fixnum}\ verwendet. Diese \representation{} be\"{o}tigt damit
wesentlich weniger Speicherplatz als beispielsweise die in
\soh\ \ifbericht\cite[\citepage{67}]{bib:ki94a}\fi\ realisierte
\representation{} von \objid[s]\ als typstrukturierte Bezeichner
\ifbericht%
\cite[\citepage{72}]{bib:ki94a}%
\else%
(Abschnitt~\ref{sec:sohapi}, \citepage{\pageref{sec:sohapi}})\fi.
%
\par{}Da drei der 32~Bit einer \lobjid\/ f\"{u}r die
Typisierung verwendet werden, kann eine \lobjid\/ 2$^{(32-3)}$ =
512~{\it M} Objekte adressieren. Instanzen des Typs
\class{fixnum}\ werden im 
verwendeten \lwcl\ durch 30~Bit repr\"{a}sentiert. Eine \lobjid\/ in einer
der h\"{o}heren Schichten h\"{a}tte dann 2$^{(30-3)}$ = 128~{\it M} Objekte
adressieren k\"{o}n\-nen. Um die Anzahl der adressierbaren Objekte
m\"{o}glichst hoch zu halten, werden daher die \lobjid[s]\ beim Austausch
mit der 3.~Schicht durch Weglassen der Typkennung in ein k\"{u}rzeres
29~Bit Format, die \sobjid[s], konvertiert. Die Typkennung
wird, falls n\"{o}tig, von der 3. in die 2.~Schicht mit
einem zu\-s\"{a}tz\-li\-chen Funktionsparameter \"{u}bergeben und von der
2. in die 3.~Schicht als zus\"{a}tzlicher R\"{u}ckgabewert eines
Funktionsaufrufs zur\"{u}ckgegeben.
%
\subsubsection{Allozierung von persistenten Objekten}
%
Jeder Basistyp wird im Code durch einen Datensatz beschrieben, der
die Gr\"{o}\ss{}e und Struktur seiner Instanzen festlegt; alle anderen
Klassen werden explizit durch \clsdo[e]\ beschrieben.
F\"{u}r die Allozierung eines persistenten Objektes wird nicht wie
in der 1.~Schicht seine Speichergr\"{o}\ss{}e, sondern seine Typkennung
angegeben.
%
\subsection{\protect\representation\ von persistenten Objekten}%
\label{sec:plrpr}%
%
In Gegensatz zu relationalen Datenbanken definiert der Objektspeicher
\postore\ keine feste Menge von Basistypen und M\"{o}glichkeiten zur
Typbildung aus diesen Basistypen, sondern bietet ein
generisches Objektformat zur Festlegung benutzerInnen-definierter
Datentypen an. Die von mir gew\"{a}hlte \representationform{} f\"{u}r
persistente Objekte orientiert sich daher an der in \lw\ vorgefundenen
\representationform{} f\"{u}r transiente Objekte, erweitert um
datenbank-spezifische Komponenten.
%
\begin{figure}[htbp]%
\centering%
\def\coliw{4em}%
\def\coliiw{5em}%
\begin{\figurefontsize}%
\begin{tabular}{%
|p{\coliw}|p{\coliw}|p{\coliw}|p{\coliw}||p{\coliiw}|p{\coliiw}|}%
\hline
\parbox{\coliw}{\centering {\it m}}
 &\parbox{\coliw}{\centering {\it n}}
  &\parbox{\coliw}{\centering\smallskip Kennung, {\em [Flags]}\smallskip}
   &\parbox{\coliw}{\centering\smallskip Lock-Info\smallskip}
    &\parbox{\coliiw}{\centering\smallskip%
                      ({\it m}\,--\,2) Referenzen\smallskip}
     &\parbox{\coliiw}{\centering\smallskip%
                       ({\it n}\,--\,{\it m}\,--\,2) Werte\smallskip}\\
\hline
\multicolumn{1}{c}{\parbox{\coliw}{%
\centering\smallskip\small Anzahl\\ \makebox[0pt]{Referenzen}}}
 &\multicolumn{1}{c}{\parbox{\coliw}{%
\centering\smallskip\small Vektor-\\gr\"{o}\ss{}e}}
  &\multicolumn{2}{}{}
   &\multicolumn{1}{c}{\smallskip\small Referenz\-feld}
    &\multicolumn{1}{c}{\smallskip\small Werte\-feld}
\end{tabular}%
\end{\figurefontsize}%
\caption{\protect\representation\ eines persistenten Objektes}%
\label{fig:povec}%
\end{figure}%
%
\par{}In der 2.~Schicht werden direkt hinter die von
\postore\ benutzten Kopf\mbox{}informationen weitere 2~Worte mit
Kopf\mbox{}informationen angef\"{u}gt (\figurename~\ref{fig:povec}). Die
einzelnen Objektkomponenten werden folgenderma\ss{}en genutzt:
\begin{description}
%
\item[Kennung]
Das Feld Kennung enth\"{a}lt eine zu den bereits in
\tablename~\ref{tab:typetags} (\citepage{\pageref{tab:typetags}})
vergebenen Typkennungen disjunkte numerische
Typkennung in Form einer auf die \cls\ \class{fixnum}\ typisierten
\lobjid, die die Basisklasse des persistenten Objektes angibt.
F\"{u}r jeden \cl\ Basistyp wurde ein entsprechender Basistyp mit
\"{a}hn\-li\-cher Struktur in \plob\ definiert. Wie auch in transienten
LISP-Systemen werden die Basistypen als
\rglq{}eingebaut\rgrq\ angesehen; sie unterliegen damit den gleichen
Einschr\"{a}nkungen.
%
\item[{\em Flags}]
Die angedeuteten \gande{Statusvariablen}{flags}\ sind zwar
realisiert, werden aber im Moment nicht genutzt.
%
\item[Lock-Info]
Das Feld Lock-Info enth\"{a}lt eine Referenz auf die Objekte, die zur
\representation{} des aktuellen Sperrzustands benutzt werden.
%
\item[Referenzfeld]
Das Referenzfeld umfa\ss{}t ({\it m}\,--\,2) typisierte \lobjid[s], die,
wie im vorigen Abschnitt erl\"{a}utert, entweder Verweise auf persistente
Objekte oder den Zustand von Instanzen der in
\tablename~\ref{tab:typetags} aufgef\"{u}hrten Basistypen enthalten.
%
\item[Wertefeld]
Das Wertefeld mit ({\it n}\,--\,{\it m}\,--\,2) Worten wird f\"{u}r
persistente Werte benutzt, wie z.B.\ Zeichenketten oder Bit-Vektoren.
%
\end{description}
%
\subsubsection{\protect\representation\ eines persistenten Feldes}%
%
\figurename~\ref{fig:plexarr} %
%
\begin{figure}[htbp]%
\ifbuch%
\centerline{\psfig{figure=twelvept/plexarr.eps}}%
\else%
\centerline{\psfig{figure=tenpt/plexarr.eps}}%
\fi%
\caption{Beispiel f\"{u}r die \protect\representation\ eines persistenten Feldes}%
\label{fig:plexarr}%
\end{figure}%
%
zeigt beispielhaft die \representation{} eines persistenten
\gande{Feldes}{array}\ mit den Dimensionen \lisp{(3 4)} (im Vergleich
zur \figurename~\ref{fig:povec} ist die Darstellung der
Objektkomponenten um 270$^{\circ}$ gedreht).
%
\par{}Das Referenzfeld teilt sich auf in einem
Bereich mit konstanter Gr\"{o}\ss{}e (Komponenten {\em data-vector\/} bis
{\em array-rank\/}) und einen Bereich mit variabler Gr\"{o}\ss{}e
(Komponenten {\em 0.\ dimension\/} bis {\em 1.\ dimension\/}).  Die
Gr\"{o}\ss{}e des variablen Bereichs (in diesem Beispiel 2) ist die Anzahl
der Dimensionen des persistenten Feldes und wird bei der Allozierung
angegeben. Das Wertefeld ist f\"{u}r persistente Felder immer leer.
%
\par{}Die Bedeutungen der Komponenten {\em fill-pointer\/}
\stcite{455}, {\em adjustable-array\/} \stcite{452}\ und
{\em array-rank\/} \stcite{448}\ entsprechen dem in \cite{bib:st90}
definierten \std\ f\"{u}r LISP Felder. Die Komponenten
{\em 0.\ dimension\/} und {\em 1.\ dimension\/} enthalten die
\gande{Dimensionen des Feldes}{array dimensions}\ \stcite{449}.
%
\par Die Komponente {\em data-vector\/} enth\"{a}lt einen persistenten
Vektor mit den Feldelementen. Die Anzahl der Elemente des Vektors ist
gleich der Anzahl der Feldelemente. Der Typ des Vektors {\em
data-vector\/} ergibt sich aus dem Elementtyp, auf den das Feld
spezialisiert ist:
\begin{description}
%
\item[Elementtyp \lispt] Der persistente Vektor {\em
data-vector\/} ist eine Instanz des \rglq{}normalen\rgrq\ persistenten
Vektortyps, d.h.\ des Vektortyps, dessen Instanzen Elemente des Typs
\lispt\ enthalten.
%
\item[Elementtyp \lisp{(unsigned-byte 1)}] Der Vektor {\em
data-vector\/}  ist ein persistenter Bitvektor.
%
\item[Elementtyp \lisp{float}, \lisp{(unsigned-byte {\rm\it n\/})},
{\rm\it n}\,$>$\,1] In diesem Fall ist der Vektor {\em data-vector\/}
ein persistenter {\em ivector\/} ({\em i\/}mmediate {\em vector\/}),
der nur Werte genau eines Basistyps enthalten kann. Der {\em ivector\/}
wird auf den jeweiligen Elementtyp spezialisiert.
%
\end{description}
Der Elementtyp, auf den das persistente Feld spezialisiert ist,
l\"{a}\ss{}t sich damit aus dem Typ der Komponente {\em data-vector\/}
bestimmen.
%
\par{}Die Komponente {\em displaced-offset\/} enth\"{a}lt 
f\"{u}r ein \gande{Feld, das nicht Elemente eines anderen Feldes
referenziert}{non-displaced array}\ \stcite{444},
das persistente Symbol \lispnil. Sofern das \gande{Feld
Elemente eines anderen Feldes referenziert}{displaced array}, enth\"{a}lt
die Komponente {\em dis\-pla\-ced-""off\-set\/} den Index, ab dem das
andere Feld referenziert wird. In diesem Fall verweist die Komponente
{\em data-vector\/} f\"{u}r ein referenzierendes \gande{Feld variabler
Gr\"{o}\ss{}e}{adjustable array}\ nicht auf einen persistenten Vektor,
sondern auf das referenzierte persistente Feld; bei einem
referenzierendem \gande{Feld mit
konstanter Gr\"{o}\ss{}e}{non-adjustable array}\ enth\"{a}lt die Komponente
{\em data-vector\/} des referenzierenden Feldes direkt die Komponente
{\em data-vector\/} des referenzierten Feldes.
%
\par{}Persistente Felder entstehen \ia\ aus der Speicherung
eines transienten Feldes; aus dem transienten Feld werden mit den
\gande{Feld-Informationsfunktionen}{array information functions}\ die
entsprechenden Parameter gelesen und in das persistente Feld
eingetragen.  Der Elementtyp des persistenten Feldes
wird beim Speichern eines transienten Feldes aus dem bei der Erzeugung
des transienten Feldes mit dem Schl\"{u}sselwort-Parameter
\lisp{:element-type}\ \stcite{443}\ \"{u}bergebenen Elementtyp bestimmt,
f\"{u}r den das transiente Feld spezialisiert wurde. F\"{u}r transiente
Felder, deren Elementtyp bei der Erzeugung nicht spezialisiert wurde,
wird der Elementtyp \lispt\ verwendet.
%
\subsubsection{\protect\representation\ von Struktur- und
\protect\clos-Objekten}%
%
F\"{u}r persistente Struktur- und \clos-Instanzen gibt es jeweils
noch eine gesonderte Typkennung; die Kopf\mbox{}information wird um ein
weiteres Wort erweitert, das eine Referenz auf eine persistente
Struktur- bzw.\ Klassen-Beschreibung enth\"{a}lt
(Abbildungen~\ref{fig:pivec} und \ref{fig:pcivec}). Das Wertefeld ist
f\"{u}r persistente Struktur- und \clos-Instanzen immer leer.
%
\begin{figure}[htbp]%
\centering%
\def\coliw{4em}%
\def\coliiw{5em}%
\begin{\figurefontsize}%
\begin{tabular}{%
|p{\coliw}|p{\coliw}|p{\coliw}|p{\coliw}|p{\coliiw}||p{\coliiw}|}%
\hline
\parbox{\coliw}{\centering {\it m}}
 &\parbox{\coliw}{\centering ({\it m}\,+\,2)}
  &\parbox{\coliw}{\centering\smallskip \lisp{\#x38}\\
                   {\em [Flags]}\smallskip}
   &\parbox{\coliw}{\centering\smallskip Lock-Info\smallskip}
    &\parbox{\coliiw}{\centering\smallskip Strukturbe-\\schreibung\smallskip}
     &\parbox{\coliiw}{\centering\smallskip%
                       ({\it m}\,--\,3) Referenzen\smallskip}\\
\hline
\multicolumn{1}{c}{\parbox{\coliw}{%
\centering\smallskip\small Anzahl\\ \makebox[0pt]{Referenzen}}}
 &\multicolumn{1}{c}{\parbox{\coliw}{%
\centering\smallskip\small Vektor-\\gr\"{o}\ss{}e}}
  &\multicolumn{3}{}{}
   &\multicolumn{1}{c}{\smallskip\small Referenz\-feld}
\end{tabular}%
\end{\figurefontsize}%
\caption{\protect\representation\ eines persistenten Strukturobjektes}%
\label{fig:pivec}%
\end{figure}%
%
\paragraph{Persistente Strukturobjekte}%
%
Alle Strukturobjekte haben die einheitliche Typkennung \lisp{\#x38};
die Klasse des Strukturobjektes geht aus der referenzierten
Strukturbeschreibung hervor. Die Strukturbeschreibung selbst ist
ebenfalls ein Strukturobjekt.  Die Gr\"{o}\ss{}e des Referenzfeldes
entspricht der Anzahl der Struktur-\Slt[s]
(\figurename~\ref{fig:pivec}).
%
\paragraph{Persistente \protect\clos-Instanzen}%
%
Da die Klasse einer \clos-Instanz  bei Erhalt ihrer Identit\"{a}t
ge\-\"{a}n\-dert werden kann, werden \clos-Instanzen anders als
Strukturobjekte repr\"{a}sentiert. Die \Slt\/-Zust\"{a}nde werden
nicht direkt in einer persistenten \clos-Instanz repr\"{a}sentiert,
sondern durch einen referenzierten persistenten Zustandsvektor; die
Anzahl der Elemente des Zustandsvektors ergibt sich aus der Anzahl der
\Slt[s]. Der Zustandsvektor ist eine Instanz des Basistyps
persistenter Vektor mit der in \figurename~\ref{fig:povec}
(\citepage{\pageref{fig:povec}}) gezeigten \representation.
Persistente \clos-Instanzen haben immer die Typkennung
\lisp{\#x48}; die Klasse der \clos-Instanz geht aus der
referenzierten Klassenbeschreibung hervor. Die Klassenbeschreibung
selbst ist auch eine persistente \clos-Instanz
(\figurename~\ref{fig:pcivec}).%
%
\begin{figure}[htbp]%
\centering%
\def\coliw{4em}%
\def\coliiw{5em}%
\begin{\figurefontsize}%
\begin{tabular}{%
|p{\coliw}|p{\coliw}|p{\coliw}|p{\coliw}|p{\coliiw}|p{\coliiw}|}%
\hline
\parbox{\coliw}{\centering 4}
 &\parbox{\coliw}{\centering 6}
  &\parbox{\coliw}{\centering\smallskip \lisp{\#x48}\\
                   {\em [Flags]}\smallskip}
   &\parbox{\coliw}{\centering\smallskip Lock-Info\smallskip}
    &\parbox{\coliiw}{\centering\smallskip \cls[n]be-\\schreibung\smallskip}
     &\parbox{\coliiw}{\centering\smallskip Zustands-\\vektor\smallskip}\\
\hline
\multicolumn{1}{c}{\parbox{\coliw}{%
\centering\smallskip\small Anzahl\\ \makebox[0pt]{Referenzen}}}
 &\multicolumn{1}{c}{\parbox{\coliw}{%
\centering\smallskip\small Vektor-\\gr\"{o}\ss{}e}}
  &\multicolumn{3}{}{}
   &\multicolumn{1}{c}{\smallskip\small Referenz\-feld}
\end{tabular}%
\end{\figurefontsize}%
\caption{\protect\representation\ einer persistenten \protect\clos-Instanz}%
\label{fig:pcivec}%
\end{figure}%
%
\subsection{Bedeutung von Objekten und Portabilit\"{a}t}%
\label{sec:plmap}%
%
In diesem Abschnitt wird die Bedeutung von Objekten in Bezug auf
Portabilit\"{a}t untersucht. Portabilit\"{a}t meint den Austausch der
3.~Schicht gegen ein anderes, nicht auf \clos\ basierendes
Objektsystem.
%
\subsubsection{Bedeutung der Objekte von Basisklassen}
%
Basisklassen sind grundlegende Klassen eines Programmsystems mit einer
von den AutorInnen in das System fest
\gande{\rglq{}eingebauten\rgrq}{built in}\ Struktur und einem fest 
eingebautem Verhalten. Da die Bedeutung ihrer Instanzen allgemein
bekannt ist, bereitet eine Nutzung keine Schwierigkeiten.  Ein
Beispiel ist die Bedeutung von {\em cons\/}-Zellen, f\"{u}r die
festgelegt wurde, das sie ausschlie\ss{}lich Paare repr\"{a}sentieren. Aus
dieser festgelegten Bedeutung erschlie\ss{}t sich auch das Verhalten
einer {\em cons\/}-Zelle; so k\"{o}nnen die beiden Elemente manipuliert
werden.
%
\par{}Da \ua\ die Struktur der Instanzen von Basisklassen
festgelegt ist und damit ihre \representationform{} konstant bleibt,
kann intern der Zugriff auf den Zustand einer Basisklasse
auf einer niedrigen Abstraktionsebene realisiert werden, bei der die
Bedeutung des \Slt[s], auf den zugegriffen wird, klar definiert ist.
%
\subsubsection{Bedeutung von \protect\clos-Instanzen und
\protect\mo[en]}
%
Dieser Abschnitt betrachtet (transiente) \clos-Instanzen mit Struktur
und Verhalten.  F\"{u}r eine \clos-Instanz ergibt sich ihre Bedeutung
aus ihrer Klasse, die alle Informationen \"{u}ber Struktur und Verhalten
enth\"{a}lt. Anders als bei Basisklassen ist eine \clos\ Klasse nicht als
\rglq{}eingebaut\rgrq\ festgelegt, sondern kann dynamisch erzeugt und
ver\"{a}ndert werden. Nur f\"{u}r die AutorInnen einer Klasse erschlie\ss{}t
sich die vollst\"{a}ndige Bedeutung der Objekte selbst, d.h.\ welche
realen Dinge oder Sachverhalten zu Objekten abstrahiert wurden.
%
\par{}Klassen werden wiederum durch Objekte, genauer \clsmo[e],
repr\"{a}sentiert. F\"{u}r \clos\ selbst mu\ss{} die Bedeutung der
\clsmo[e]\ und der von ihnen referenzierten weiteren \mo[e]\ bekannt
sein, um sie nutzen zu k\"{o}n\-nen. Zu diesem Zweck wird die Bedeutung der
\mo[e]\ in \cite{bib:amop91} ausschlie\ss{}lich \"{u}ber ihr Verhalten 
festgelegt; eine Festlegung f\"{u}r die Strukturen der
\mo[e]\ ist im Rahmen der Normierung unn\"{o}tig und l\"{a}\ss{}t
den AutorInnen eines \clos-Systems Freiheit in der Auswahl der
Strukturen.
%
\par{}Da die Struktur von \clos-Objekten nicht von vornherein
festgelegt ist und sich ihre \representationform{} aus den im
\clsmo\ abgelegten Informationen ergibt, erfolgt der Zugriff auf den
Zustand eines \clos-Objektes intern auf einer relativ hohen
Abstraktionsebene \"{u}ber das Verhalten des \clsmo[es].
%
\subsubsection{Bedeutung von persistenten \protect\mo[en]}
%
Da die Bedeutung von \mo[en]\ wie im letzten Abschnitt
erl\"{a}utert ausschlie\ss{}lich \"{u}ber ihr Verhalten definiert wird und das
persistente System nur Struktur, aber kein Verhalten speichern kann,
ist es in der 2.~Schicht nicht m\"{o}glich, die Bedeutung von
persistenten \mo[en]\ zu kennen.
%
\subsubsection{Folgerungen f\"{u}r die Bedeutung von Objekten und
Portabilit\"{a}t}%
%
\begin{fortune}[6cm]%
Wir k\"{o}n\-nen auch anders!
\from{Detlev Buck}
\end{fortune}%
\nopagebreak[4]%
%
Bei einem Austausch der 3.~Schicht gegen ein nicht auf
\clos\ basierendes Objektsystem w\"{a}re es vorteilhaft, \"{u}ber die im
persistenten Speicher befindlichen Objekte Aussagen auf Klassenebene
treffen zu k\"{o}n\-nen, z.B.\ wie der Name der Klasse ist oder welche
\Slt[s]\/ eine persistente \clos-Instanz hat. F\"{u}r persistente 
Instanzen von Basistypen stellt dies kein Problem dar, da ihre
Bedeutung \"{a}hn\-lich wie in einem \cl-System in das persistente
System \rglq{}eingebaut\rgrq\ wurde. Bei persistenten \clos-Instanzen
mu\ss{} f\"{u}r Aussagen auf Klassenebene die Bedeutung der
Klassenbeschreibung (\figurename~\ref{fig:pcivec},
\citepage{\pageref{fig:pcivec}}) bekannt sein; wie im letzten
Abschnitt gezeigt, k\"{o}n\-nen persistente \mo[e]\ diese Bedingung nicht
erf\"{u}llen.
%
\paragraph{Verzicht auf Portabilit\"{a}t}%
%
Sofern Portabilit\"{a}t nicht ben\"{o}tigt wird, besteht eine L\"{o}sung darin,
die Klassenbeschreibungen ausschlie\ss{}lich den h\"{o}heren Schichten zu
\"{u}berlassen; deren Bedeutung ist dann auch nur dort
bekannt. Vorteilhaft ist, da\ss{} v\"{o}llige Freiheit in der Wahl der
Klassenbeschreibung besteht und damit keine bestimmte
Datenmodellierung f\"{u}r persistente Objekte von der 2.~Schicht
vorgeschrieben wird. Der Nachteil ist, da\ss{} Portabilit\"{a}t nicht
gew\"{a}hrleistet werden kann, da die Bedeutung einer Klassenbeschreibung
auf eine konkrete h\"{o}here Schicht spezialisiert ist.
%
\paragraph{Integration von \protect\clos}%
%
Eine radikale L\"{o}sung best\"{a}nde in der Integration von \clos\ in
die 2.~Schicht, damit die Bedeutung der Klassenbeschreibung wieder
\"{u}ber ihr Verhalten definiert werden k\"{o}nnte; dies w\"{u}rde
allerdings den in Abschnitt~\ref{sec:arcis}
(\citepage{\pageref{sec:arcis}}) erl\"{a}uterten Architekturansatz
realisieren, der aus den dort genannten Gr\"{u}nden im Rahmen
\ifbericht dieses Berichtes \else\ifbuch dieses Buches \else dieser
Arbeit \fi\fi ausscheidet.
%
\paragraph{Festlegung der Klassenbeschreibungen in der 2.\ Schicht}%
%
Ein moderater Ansatz besteht darin, Klassenbeschreibungen
nicht durch \clsmo[e]\ und die von ihnen referenzierten \mo[e]\ zu
repr\"{a}sentieren, deren Bedeutung \"{u}ber ihr Verhalten definiert
wird, sondern stattdessen \clsdo[e]\ und von ihnen referenzierte
\deo[e]\ zu benutzen, deren Bedeutung ausschlie\ss{}lich \"{u}ber ihre
Struktur festgelegt wird. Die Struktur der \deo[e]\ wird durch das in
den h\"{o}heren Schichten 
benutzte Objektsystem (in diesem Fall \clos) bestimmt und 
anschlie\ss{}end in der 2.~Schicht festgelegt; f\"{u}r die
h\"{o}heren Schichten wird die Konvention eingef\"{u}hrt, beim Speichern einer
persistenten \clos-Instanz sicherzustellen, da\ss{} seine
Klassenbeschreibung g\"{u}ltig ist.
Die Klassen der \deo[e]\ k\"{o}n\-nen sinnvollerweise aus den \mc[n]\ von
\clos\ abgeleitet werden (siehe Abschnitt~\ref{sec:mopdescr},
\citepage{\pageref{sec:mopdescr}}); dabei k\"{o}n\-nen durch Auslassen der
f\"{u}r ein \deo\ nicht ben\"{o}tigten \Slt[s]\/ der \mo[e]\ die in
Abschnitt~\ref{sec:plpeda} (\citepage{\pageref{sec:plpeda}})
erw\"{a}hnten Redundanzen vermieden werden, beispielsweise indem in
\clsdo[en]\ die Liste der Subklassen oder der direkten \mtd[n]\ nicht
repr\"{a}sentiert werden.
%
\par{}Nachteilig ist, da\ss{} durch die Festlegung der \dec[n]\ \"{u}ber
ihre Struktur Abstraktion verloren geht; die Struktur jeder \dec\ mu\ss{}
in der 2.~Schicht verbindlich festgelegt werden. Damit werden
\dec[n]\ zu Basisklassen des persistenten Systems. Ebenso m\"{u}ssen bei
Einhaltung der \og[en]\ Konvention die \deo[e]\ von der 3.~Schicht aus
aufgebaut werden, da die vorhandenen \mo[e]\ selbst nicht verwendet
werden d\"{u}rfen. Implizit wird durch diese Vorgehensweise f\"{u}r
persistente Objekte ihre Datenmodellierung auf die von
\clos\ festgelegt; f\"{u}r h\"{o}here Schichten folgt daraus, f\"{u}r
persistente Objekte ebenfalls die Datenmodellierung von
\clos\ \"{u}bernehmen zu  m\"{u}ssen.  Dies kann insbesondere f\"{u}r h\"{o}here
Schichten mit Objektsystemen ohne ein in die Sprache integriertes
\mo-Protokoll (wie z.B.\ \cpp) schwierig werden, da in diesem Fall ein
\mo-Protokoll (wenn auch nicht so umfangreich wie das \mop\ von \clos)
zur h\"{o}heren Schicht hinzugef\"{u}gt werden mu\ss{}.%
\footnote{Die in Abschnitt~\ref{sec:ocls}
(\citepage{\pageref{sec:ocls}}) erw\"{a}hnte angenehme Eigenschaft der
Selbstbeschreibung eines Systems bedingt zu\-s\"{a}tz\-lich ein
\mo-Protokoll \cite[\citepage{12}]{bib:we90}.}
%
\paragraph{Realisierte L\"{o}sung}%
\label{sec:plpsol}%
%
Nach Abw\"{a}gung der Vor- und Nachteile entschied ich mich f\"{u}r
den letztgenannten Ansatz, \dec[n]\ zu realisieren und in der
3.~Schicht die \og[e]\ Konvention zu befolgen, um Portabilit\"{a}t
sicherzustellen; eine genaue Erl\"{a}uterung der \dec[n]\ folgt in
Abschnitt~\ref{sec:mopdescr} (\citepage{\pageref{sec:mopdescr}}). Ein
weiterer Vorteil dieses Vorgehens ist, da\ss{} bereits
die 2.~Schicht mit Kenntnis der Bedeutung umfangreiche
Pr\"{u}fungen zur Fehlersuche vornimmt; bei auftretenden
Laufzeitfehlern in der 2.~Schicht werden die Fehlermeldungen
aussagekr\"{a}ftiger gestaltet.  Die hier getroffenen Aussagen und
Schlu\ss{}folgerungen f\"{u}r \clos-Instanzen gelten sinngem\"{a}\ss{} auch f\"{u}r
Strukturobjekte.
%
\subsection{Sitzungen}%
%
Eine \gande{Sitzung}{session}[][Eine Folge von zeitlich
disjunkten Da\-ten\-bank-Trans\-ak\-tio\-nen sowie einem Proze\ss{}, der
innerhalb einer Transaktion auf persistente Objekte zugreifen
kann.]\ besteht aus  einer Folge von zeitlich disjunkten Transaktionen
sowie einem Proze\ss{}, der innerhalb einer Transaktion auf persistente
Objekte zugreifen kann. Ein an eine Sitzung gebundener Proze\ss{} ist
kein \unix\ Proze\ss{}, sondern ein \lw\ {\em lightweight\/}-Proze\ss{}.
%
\par{}Vor Beginn einer Transaktion wird der Gesamtzustand aller
persistenten Objekte als konsistent angenommen; innerhalb einer
Transaktion d\"{u}rfen Objekte modifiziert werden, bis die Transaktion
entweder beendet oder abgebrochen wird. Bei einem Ende gilt der aus
den \"{A}n\-de\-run\-gen hervorgegangene Zustand als neuer konsistenter
Gesamtzustand; bei einem \gande{Abbruch}{abort}[Abbbruch einer
Transaktion][Damit gilt der aus den w\"{a}hrend der Transaktion
erfolgten Modifikationen hervorgegangene Gesamtzustand der Datenbank
als inkonsistent; der Gesamtzustand wird daher \protect\gande{auf den
letzten konsistenten Stand bei Beginn der Transaktion
zur\"{u}ckgebracht}{\protect\see{rollback}}.]\ wird angenommen, da\ss{} der
Gesamtzustand durch die erfolgten \"{A}n\-de\-run\-gen inkonsistent ist
und der \gande{letzte konsistente Gesamtzustand (bei Beginn der
Transaktion) wiederhergestellt werden soll}{rollback}[Zur\"{u}cksetzen
der Datenbank auf den Zustand bei Beginn der Transaktion nach Abbruch
einer Datenbank-Transaktion.]\ (\figurename~\ref{fig:sttrans}). %
%
\begin{figure}[htbp]%
\ifbuch%
\centerline{\psfig{figure=twelvept/sttrans.eps}}%
\else%
\centerline{\psfig{figure=tenpt/sttrans.eps}}%
\fi%
\captionfrom%
 {S/T-Diagramm einer Transaktion aus Applikationssicht}%
 {\cite[\citepage{181}]{bib:gr93}}%
\label{fig:sttrans}%
\end{figure}%
%
Eine Transaktion ist damit die Realisierung eines Konzepts f\"{u}r den
\"{U}ber\-gang des Gesamtsystems von einem konsistentem Zustand in den
n\"{a}chsten konsistenten Zustand. Die Aufgabe des persistenten Systems
dabei ist, diesen \"{U}bergang als unteilbar erscheinen zu lassen.
%
\par{}Die Zustands\"{a}nderungen werden von dem an die Sitzung gebundenen
Proze\ss{} durchgef\"{u}hrt. Bei mehreren beteiligten Sitzungen und damit
Prozessen mu\ss{} jede Sitzung f\"{u}r schreibenden Zugriff die persistenten
Objekte alleine im Zugriff haben; bei lesendem Zugriff sollten
schreibende Zugriffe durch andere Sitzungen gesperrt werden.  Das
persistente System mu\ss{} also daf\"{u}r sorgen, da\ss{} Zugriffe auf die
Objekte ausschlie\ss{}lich durch Sperren derselben erm\"{o}glicht
werden.
%
\par{}Sitzungen werden durch Instanzen des Basistyps
\class{persistent-heap}\ repr\"{a}sentiert; ihre Verwendung ist
schicht-abh\"{a}ngig:
\begin{description}
%
\item[Oberhalb der 2.~Schicht]
In den oberen Schichten wird eine Sitzung f\"{u}r den
trans\-ak\-tions\-ge\-steu\-er\-ten Zugriff auf alle im persistenten
Speicher enthaltenen persistenten Objekte benutzt. Ihre Aufgabe
besteht darin, die f\"{u}r das Sperrprotokoll notwendigen Transaktionen
zur Verf\"{u}gung zu stellen.
%
\par{}Sitzungen werden damit \"{a}hn\-lich benutzt wie die Instanzen der
in \pclos\ definierten
\cls\ \class{protector}\ \cite[\citepage{9}]{bib:pa91a}.
%
\item[Innerhalb der 2.~Schicht]
Hier werden Sitzungen als \representation{} des (zur Zeit noch) implizit
an die Sitzung gebundenen Prozesses angesehen (zur expliziten
\representation{} von Prozessen siehe Abschnitt~\ref{sec:expiil},
\citepage{\pageref{sec:expiil}}). Beispielsweise werden 
sie in diesem Sinne beim Sperrprotokoll als sperrendes Objekt
benutzt; Sperrkonflikte werden der Sitzung signalisiert, um
den daran gebundenen Proze\ss{} zum passiven Warten zu veranlassen.
%
\end{description}
%
\iffalse
\par{}Die Wahl des Namens \class{persistent-heap}\ f\"{u}r die
sitzungs-repr\"{a}sentierende \cls\ hat historische Gr\"{u}nde. Bei Beginn
der Systementwicklung hatte genau eine Instanz der
\cls\ \class{persistent-heap}\ in der 3.~und 4.~Schicht die Aufgabe,
den gesamten persistenten Speicher zu repr\"{a}sentieren. Sp\"{a}ter wurde
dann diese \cls\ im Hinblick auf den Zugriff durch mehrere Sitzungen
erweitert, ohne allerdings den Namen der \cls\ zu \"{a}ndern.
\fi
%
\par{}Das hier realisierte Sitzungskonzept lehnt sich an das in
\cite[\citepage{269}]{bib:ki89} \cite[\citepage{20}]{bib:it91}
\cite[\citepage{29}]{bib:ah91} f\"{u}r die Datenbank \orion\ (jetzt
\itasca) vorgestellte Konzept an. Transaktionen wurden in der
2.~Schicht als \gande{{\protect\twophasetrs[]}}{two-pha\-sed
trans\-ac\-tions}\ realisiert. F\"{u}r
\gande{nicht-wohlgeformte Transaktionen}{non-well-formed
transactions}%
\addglossary{Wohlgeformte Transaktion}[well-formed transaction]%
[{In einer wohlgeformten Transaktion sind die Objekte f\"{u}r den Zugriff
auf ihren Zustand durch die dereferenzierende Instanz
(\protect\ia\ eine \protect\see{Sit\-zung}) gesperrt. Damit in einer
Transaktion \protect\see{Se\-ria\-li\-sier\-bar\-keit}\ gew\"{a}hrleistet
wird, mu\ss{} die Transaktion wohlgeformt
sein.}]\ \cite[\citepage{385}]{bib:gr93}\ wird
ein Fehler signalisiert. Wie bei der Datenbank \orion\ wurde kein
vollst\"{a}ndiges DO-UNDO-REDO-Protokoll \cite[\citepage{538}]{bib:gr93}
realisiert; es wird lediglich eine UNDO-{\em Log\/}-Datei pro Sitzung
gef\"{u}hrt, in der die Zust\"{a}nde der modifizierten Objekte vor der
Modifikation abgelegt werden \cite[\citepage{272}]{bib:ki89}
\cite[\citepage{21}]{bib:it91}. Bei Ende einer Transaktion wird der
Gesamtzustand des persistenten Speichers mittels eines Aufrufs der
Stabilisierungsfunktion aus der 1.~Schicht gesichert; damit ist eine
Mitf\"{u}hrung der neuen Objektzust\"{a}nde in einer REDO-{\em Log\/}-Datei
unn\"{o}tig.  Bei einem Abbruch werden die Objektzust\"{a}nde aus der
UNDO-{\em Log\/}-Datei restauriert.
%
\par{}Durch den Fortfall der REDO-{\em Log\/}-Datei halbiert sich
ferner die Datenmenge, die sonst pro Schreibzugriff auf ein Objekt in
den {\em Log\/}-Dateien vermerkt werden m\"{u}\ss{}te.
%
\par{}In \orion\ kann eine Sitzung von mehreren Prozessen
(dort {\em windows\/} genannt) genutzt werden
\cite[\citepage{269}]{bib:ki89} \cite[\citepage{20}]{bib:it91}. Dies ist
in \plob\ nicht m\"{o}glich; eine Sitzung entspricht genau einem Proze\ss{}.
Die in \cite[\citepage{270}]{bib:ki89} \cite[\citepage{29}]{bib:ah91}
erl\"{a}uterten hypothetischen Transaktionen,%
\addglossary{Hypothetische Transaktion}%
[Eine hypothetische Transaktion ist eine Transaktion, die bei Ende
immer abgebrochen wird; sie kann z.B.\ f\"{u}r Entwicklungsarbeiten
genutzt werden, deren Ergebnisse niemals persistent gehalten werden
sollen.]%
\ d.h.\ Transaktionen, die
bei Ende immer abgebrochen werden, wurden von mir nicht
realisiert.\footnote{Offensichtlich wurden hypothetische
Transaktionen im \orion-Nachfolger \itasca\ ebenfalls nicht mehr
realisisert, da sie in \cite{bib:it91} nicht mehr erw\"{a}hnt werden.}
%
\subsection{Objektsperren}%
\addglossary{Objektsperre}[locking][{Vor der Dereferenzierung des
Objektzustands wird das Objekt
mit einem der \protect\gande{\protect\see{nicht-ex\-klu\-si\-ven
Sperrmodi}}{shared
mode}\ \protect\gande{Nur-Lesen}{read-only},
\protect\gande{Lesen}{read}\ oder dem
\protect\gande{\protect\see{exklusivem Sperrmodus}}{exclusive
mode}\ \protect\gande{Schreiben}{write}\ durch ein
sperrendes Objekt (\protect\ia\ eine \protect\see{Sitzung})
f\"{u}r den Zugriff durch andere Instanzen gesperrt.}]%
%
Die in der 1.~Schicht angebotenen Schreibsperren werden in der
2.~Schicht lediglich benutzt, um den konkurrenten Zugriff auf das Feld
mit der Sperrinformation zu synchronisieren. In diesem Feld befindet
sich, sofern das persistente Objekt gesperrt ist, eine Referenz auf
ein persistentes Objekt mit dem aktuellen Sperrzustand.
%
\subsubsection{Hierarchisches Sperrprotokoll}
%
Realisiert wurde ein \gande{hierarchisches pessimistisches
Sperrprotokoll}{tree locking}[Hierarchisches
Sperrprotokoll][{Das hierarchische Sperrprotokoll impliziert mit dem
\protect\see{Sperren}\ eines Objektes auf einer Ebene mit einem Modus
denselben Sperrmodus f\"{u}r alle Unterebenen. Wenn beispielsweise die
Sperrebenen \protect\rglq{}Speicher\protect\rgrq\ --
\protect\rglq{}Vektor\protect\rgrq\ -- \protect\rglq{}Komponente eines
Vektors\protect\rgrq\ gegeben sind, impliziert eine Lesesperre auf der
Ebene \protect\rglq{}Speicher\protect\rgrq\ Lesesperren f\"{u}r alle
weiteren unteren Ebenen \protect\rglq{}Vektor\protect\rgrq\ (d.h.\ f\"{u}r
{\protect\em alle\/} Vektoren) und \protect\rglq{}Komponente eines
Vektors\protect\rgrq.}]\ nach \cite[\citepage{406}]{bib:gr93} f\"{u}r die
hierarchisch geordneten Ebenen {\em Speicher\/} (gesamter persistenter
Speicher mit allen persistenten Objekten), {\em Vektor\/} (ein
persistentes Objekt) und {\em Element\/} (ein Element, d.h.\ ein Wort
innerhalb eines persistenten Objektes) mit den extern verf\"{u}gbaren,
\gande{nicht-exklusiven Modi}{shared modes}%
\addglossary{Nicht-exklusiver Sperrmodus}[shared mode]%
[Ein Modus, der mehreren sperrenden Objekten Zugriff auf eine
gesperrte Instanz erlaubt.]\ \gande{Nur-Lesen}{read-only},
\gande{Lesen}{read}\ und dem
\gande{exklusiven Modus}{exclusive mode}[Exklusiver
Sperrmodus][Ein Modus, der nur einem sperrenden
Objekt Zugriff auf eine gesperrte Instanz
erlaubt.]\ \gande{Schreiben}{write}\ sowie den intern benutzten Modi
\gande{Nur-Lese-Absicht}{read-only-intent},
\gande{Leseabsicht}{read-intent}\ und
\gande{Schreibabsicht}{write-intent}.
Beim hierarchischen Sperrprotokoll impliziert das Sperren eines
Objektes auf einer h\"{o}heren Ebene mit einem Modus den gleichen
Sperrmodus f\"{u}r alle unteren Ebenen. Die intern benutzten
\gande{Absichtssperren}{intent locks}%
\addglossary{Absichtssperre}[intent lock]%
[{Eine Absichtssperre repr\"{a}sentiert beim
\protect\see{hierar\-chi\-schem Sperrprotokoll}\ die Absicht eines
sperrenden Objektes, auf einer tieferen Ebene eine entsprechende
Sperre vornehmen zu wollen. Wenn beispielsweise die
Sperrebenen \protect\rglq{}Speicher\protect\rgrq\ --
\protect\rglq{}Vektor\protect\rgrq\ -- \protect\rglq{}Komponente eines
Vektors\protect\rgrq\ gegeben sind, wird mit einer
Leseabsichtssperre auf der Ebene
\protect\rglq{}Speicher\protect\rgrq\ die Absicht angezeigt, auf einer
der Ebenen \protect\rglq{}Vektor\protect\rgrq\ oder
\protect\rglq{}Komponente eines Vektors\protect\rgrq\ eine
Lesesperre setzen zu wollen.}]\ repr\"{a}sentieren die
Absicht eines sperrenden Objektes, auf einer tieferen Ebene eine
entsprechende Sperre vornehmen zu wollen.
%
\par{}An einem Sperrvorgang sind immer zwei persistente
Objekte beteiligt: eine sperrende und eine zu sperrende Instanz.
F\"{u}r die Modi Leseabsicht, Lesen, Schreiben und Schreibabsicht
re\-pr\"{a}\-sen\-tiert das sperrende Objekt als Instanz der
\cls\ \class{persistent-heap}\ die benutzende Sitzung. An das
sperrende Objekt ist eine Transaktion gebunden, in der die
Sperre vermerkt wird. Bei Transaktionsende sorgt eine auf die
\cls\ \class{persistent-heap}\ spezialisierte Methode daf\"{u}r, da\ss{}
alle innerhalb der Transaktion vorgenommenen Sperren aufgehoben
werden.  Bei den Modi Nur-Lesen und Nur-Lese-Absicht sind das
sperrende und das zu sperrende Objekt identisch; sie k\"{o}n\-nen daher
nur explizit aufgehoben werden. \tablename~\ref{tab:lockconfl} %
%
\begin{figure}[htbp]%
\centering%
\def\coliw{3em}%
\def\pbox#1{\parbox{\coliw}{\centering#1}}%
\def\vvbox{\pbox{\checked}}%
\def\XXbox{\pbox{\crossed}}%
\begin{\figurefontsize}%
\begin{tabular}{%
|c|p{\coliw}|p{\coliw}|p{\coliw}|p{\coliw}|p{\coliw}|p{\coliw}|}
\cline{2-7}
\multicolumn{1}{c|}{\parbox{8em}{%
 \hspace*{\fill}Gesetzt ist$\rightarrow$\\[\smallskipamount]%
 Angefordert wird$\downarrow$\hspace*{\fill}}}
 &\pbox{\tabularheader{RO-Intent}}
  &\pbox{\tabularheader{Read-Only}}
   &\pbox{\tabularheader{Read-Intent}}
    &\pbox{\tabularheader{Read}}
     &\pbox{\tabularheader{Write-Intent}}
      &\pbox{\tabularheader{Write}}\\ \hline
%
%                  ROI     RO      RI      R       WI      W
\tabularheader{Read-Only-Intent}
        &\vvbox &\vvbox &\vvbox &\vvbox &\vvbox &\vvbox\\ \hline
\tabularheader{Read-Only}
        &\vvbox &\vvbox &\vvbox &\vvbox &\vvbox &\vvbox\\ \hline
\tabularheader{Read-Intent}
        &\vvbox &\vvbox &\vvbox &\vvbox &\vvbox &\XXbox\\ \hline
\tabularheader{Read}
        &\vvbox &\vvbox &\vvbox &\vvbox &\XXbox &\XXbox\\ \hline
\tabularheader{Write-Intent}
        &\vvbox &\vvbox &\vvbox &\XXbox &\vvbox &\XXbox\\ \hline
\tabularheader{Write}
        &\vvbox &\XXbox &\XXbox &\XXbox &\XXbox &\XXbox\\ \hline
%
\multicolumn{7}{r}{%
\checked\ $\equiv$\ Kompatible Sperrmodi\quad
\crossed\ $\equiv$\ Inkompatible Sperrmodi}
\end{tabular}%
\end{\figurefontsize}%
%
\let\figurename\tablename%
\caption{Kompatibilit\"{a}tsmatrix f\"{u}r Sperrmodi}%
\label{tab:lockconfl}%
\end{figure}%
%
zeigt die Kompatibilit\"{a}tsmatrix der realisierten Sperrmodi. Eine
Sperre wird zugelassen, wenn der angeforderte Sperrmodus
kompatibel zu allen bisher auf das Objekt gesetzten Sperrmodi
ist. Zus\"{a}tzlich zu den in
\tablename~\ref{tab:lockconfl} gezeigten F\"{a}llen ist eine
Sperre auch dann kompatibel, wenn das sperrende Objekt bereits eine
Sperre mit einem beliebigem anderen Modus erhielt und kein
inkompatibler Sperrmodus zur Sperre eines anderen sperrenden Objektes
besteht; alle Sperren eines einzigen sperrenden Objektes sind
un\-ab\-h\"{a}n\-gig vom Modus kompatibel zueinander.
%
\subsubsection{Sperrkonflikte}
%
Kann eine Sperre nicht erteilt werden, mu\ss{} diejenige Instanz
(d.h.\ Sitzung), die die Sperre zuletzt anforderte, passiv warten, bis
die bestehenden Sperren, deren Modi inkompatibel zum Modus der
angeforderten Sperre sind, von den anderen sperrenden Objekten
(d.h.\ Sitzungen) aufgehoben werden. F\"{u}r eine m\"{o}glichst schnelle
Aktualisierung der Daten werden wartende Instanzen mit Anforderungen
f\"{u}r Schreibsperren vor wartenden Instanzen mit Anforderungen f\"{u}r
Lesesperren aktiviert \cite[\citepage{189}]{bib:je87}.
%
\subsubsection{Verklemmungen}
%
Da \gande{Verklemmungen}{deadlocks}%
\addglossary{deadlock}[Verklemmung.]\ relativ selten auftreten, wird
auf eine explizite Detektion verzichtet; statt dessen wird die in
\cite[\citepage{424}]{bib:gr93} vorgeschlagene Technik benutzt,
nach \gande{Ablauf einer bestimmten Wartezeit}{time out}[]\ f\"{u}r ein
sperrendes Objekt anzunehmen, da\ss{} seine Sperranforderung
eine Verklemmung verursachte. Die Anforderung wird in diesem Fall mit
einer entsprechenden Fehlermeldung abgebrochen.
%
\subsubsection{Zugriff auf den Zustand von persistenten Objekten}
%
Sowohl lesender als auch schreibender Zugriff aus den h\"{o}heren
Schichten auf ein persistentes Objekt bedingen eine
Sperre. Bei jedem Aufruf einer Zugriffsfunktion f\"{u}r ein
persistentes Objekt wird von der h\"{o}heren Schicht die Sitzung mit
\"{u}bergeben;  sofern eine Sperre noch nicht erfolgte, wird sie
vor dem eigentlichen Zugriff von der 2.~Schicht f\"{u}r die angegebene
Sitzung angefordert.
%
\subsection{Persistente B-B\"{a}ume}%
\label{sec:plbtr}%
%
Persistente B-B\"{a}ume \cite[\citepage{851--858}]{bib:gr93} werden
benutzt, um Eintr\"{a}ge bestehend aus einem persistentem Datenobjekt
assoziiert zu einem persistentem Schl\"{u}sselobjekt zu speichern. Sie
werden \"{a}hn\-lich wie \hashtbl{}n \stcite{435--441}\ verwendet. Von
Vorteil ist zum einen, da\ss{} die Verwaltung sehr gro\ss{}er Datenmengen
effizienter als bei \hashtbl{}n ist und zum anderen die Schl\"{u}ssel
sortiert abgelegt werden; damit k\"{o}n\-nen die zu einem Schl\"{u}sselbereich
assoziierten Daten effizient gesucht werden.
%
\par{}Repr\"{a}sentiert werden sie durch Instanzen des Basistyps
\class{persistent-btree}. Die realisierten
Ordnungskriterien f\"{u}r Schl\"{u}sselobjekte sind:
\begin{description}
%
\item[Ordnungskriterium \lisp{eq}\ oder \lisp{eql}]
Als Ordnungskriterium werden die Identit\"{a}ten, d.h.\ die numerischen
Werte der \lobjid[s], der Schl\"{u}sselobjekte benutzt.
%
\item[Ordnungskriterium \lisp{equal}]
Bei diesem Ordnungskriterium werden die Eintr\"{a}ge nach dem Zustand
ihrer Schl\"{u}sselobjekte aufsteigend sortiert. Die Festlegung eines
B-Baumes auf einen bestimmten Schl\"{u}sseltyp erfolgt implizit mit dem
Einf\"{u}gen des ersten Schl\"{u}ssels. Als nachfolgende Objekte k\"{o}n\-nen nur
Schl\"{u}ssel eingef\"{u}gt werden, f\"{u}r deren Typ eine Vergleichsmethode
mit den Typen aller bereits im B-Baum befindlichen Schl\"{u}ssel
exisitiert; andernfalls wird ein Fehler signalisiert.
%
\end{description}
%
\par{}Der in \cite[\citepage{283--287}]{bib:wi83} angegebene
Code wurde \"{u}bernommen und auf das in der 2.\ Schicht verwendete
Objektsystem angepa\ss{}t; ferner werden alle Operationen in
das Sperrprotokoll eingebettet.
%
\subsection{Das C-\protect\plob\ Wurzelobjekt}
%
Die 1., 2.\ und 3.~Schicht haben jeweils eigene Wurzelobjekte; ein
Wurzelobjekt einer h\"{o}heren Schicht wird transitiv von allen
Wurzelobjekten der niedrigeren Schichten referenziert. Zur leichteren
Unterscheidung werden die Wurzelobjekte im folgenden Text mit der
Nummer ihrer Schicht-""Ebe\-ne versehen,
z.B.\ bezeichnet \rglq{}2.~Wurzelobjekt\rgrq\ das Wurzelobjekt der
2.\ Schicht (\figurename\ \ref{fig:plroots}).%
%
\begin{figure}[htbp]%
\ifbuch%
\centerline{\psfig{figure=twelvept/plroots.eps}}%
\else%
\centerline{\psfig{figure=tenpt/plroots.eps}}%
\fi%
\caption{Wurzelobjekte der Schichten 1--3}%
\label{fig:plroots}%
\end{figure}%
%
\par{}Das 2.~Wurzelobjekt wird als einziges Objekt vom 1.~Wurzelobjekt
referenziert. Es enth\"{a}lt folgende Komponenten:
\begin{description}
%
\item[Drittes Wurzelobjekt]
Innerhalb des 2.~Wurzelobjektes gibt es eine Referenz auf das
Wurzelobjekt der 3.~Schicht.
%
\item[Versionsnummer]
Sie gibt an, mit welcher Version das 2.~Wurzelobjekt in den
persistenten Speicher geschrieben wurde; \"{u}ber die Versionsnumer
k\"{o}n\-nen Inkompatibilit\"{a}ten zwischen der Struktur des im persistenten
Speicher befindlichen 2.~Wurzelobjektes und ihrer Strukturdefinition
im Code detektiert und aufgel\"{o}st werden.
%
\item[Transaktionsz\"{a}hler]
Jede Transaktion erh\"{a}lt zur Identifikation eine eindeutige Nummer;
innerhalb des 2.~Wurzelobjektes wird diese Nummer durch Inkrementieren
des Trans\-ak\-tions\-z\"{a}h\-lers generiert.
%
\item[Liste von freien Objekten]
Persistente Instanzen bestimmter \cls[n]\ werden bei expliziter
Freigabe in eine Freiliste eingeordnet; solange die Freiliste nicht
leer ist, werden dann Objekte nicht neu alloziert, sondern aus der
Freiliste entfernt und wiederverwendet. Damit entf\"{a}llt f\"{u}r diese
Objekte der Aufruf der Allozierungsfunktion aus der 1.~Schicht.
%
\item[Warteschlange mit passiv wartenden Sitzungen]
In der Warteschlange befinden sich nach Prio\-ri\-t\"{a}\-ten geordnete
Sitzungen, die eine inkompatible Sperranforderung stellten und
daher passiv warten m\"{u}ssen, bis entweder ihre Wartezeit abgelaufen
ist oder ihre Sperranforderung erf\"{u}llt werden kann.
%
\item[Sperrzustand des gesamten persistenten Speichers]
Der Sperrzustand f\"{u}r die Ebene {\em Speicher\/} wird im
2.~Wurzelobjekt repr\"{a}sentiert.
%
\item[Z\"{a}hler f\"{u}r die Anzahl der Aufrufe der Speicherr\"{u}ckgewinnung]
Nach einer Spei\-cher\-r\"{u}ck\-ge\-win\-nung haben sich durch die in
der letzten Phase durchgef\"{u}hrte Kompaktierung des persistenten
Speichers die \lobjid[s]\ der persistenten Objekte unter Umst\"{a}nden
verschoben. Der Z\"{a}h\-ler f\"{u}r die Anzahl der Aufrufe der
Spei\-cher\-r\"{u}ck\-ge\-win\-nung wird benutzt, um durch Vergleich mit
einem Referenzwert einen zwischenzeitlichen Aufruf der
Spei\-cher\-r\"{u}ck\-ge\-win\-nung und damit eine m\"{o}gliche Verschiebung
der \lobjid[s]\ zu detektieren. Diese Detektion wird f\"{u}r persistente
B-B\"{a}ume mit dem Ordnungskriterium \lisp{eq}\ genutzt, um nach einer
Spei\-cher\-r\"{u}ck\-ge\-win\-nung die B-B\"{a}ume neu aufzubauen.
%
\end{description}
%
Innerhalb der 2.~Schicht wird auch bereits die \representationform{}
des 3.~Wurzelobjektes festgelegt, um verschieden realisierten
3.~Schichten einen konsistenten Zugriff auf die persistenten Objekte
\"{u}ber die 2.~Schicht zu gew\"{a}hrleisten. Das 3.~Wurzelobjekt
hat \ua\ folgende Komponenten:
%
\begin{description}
%
\item[Liste der Sitzungen]
Alle Sitzungen befinden sich in einer Liste, damit sie gegebenenfalls
von der 2.~Schicht aus benachrichtigt werden k\"{o}n\-nen.
%
\item[Zeitpunkt der Formatierung]
Bei der Formatierung werden die \clsdo[e]\ aller \dec[n]\ in den
persistenten Speicher geschrieben; diese Komponente enth\"{a}lt den
Zeitpunkt der letzten Formatierung in \cl\ {\em Universal Time}.
%
\item[Pakettabelle]
Die \gande{Pakettabelle}{package table}\ wird durch einen persistenten
B-Baum re\-pr\"{a}\-sen\-tiert, der ein persistentes Paket zu einen
\gande{Paketnamen}{package name}\ in Form einer Zeichenkette
assoziiert. Innerhalb des persistenten Pakets befindet sich eine
Tabelle seiner internen persistenten Symbole, die durch einen
persistenten B-Baum repr\"{a}sentiert wird, der ein persistentes Symbol
zu seinem Namen in Form einer Zeichenkette assoziiert.
%
\item[Klassentabelle]
Die persistente Klassentabelle wird durch einen persistenten B-Baum
re\-pr\"{a}\-sen\-tiert, der ein persistentes \clsdo\ zu einem
Klassenamen in Form eines persistenten Symbols assoziiert.
%
\item[{\deo[e]}]
Schlie\ss{}lich enth\"{a}lt das 3.~Wurzelobjekt die \objid[s]\ der
\clsdo[e]\ aller \dec[n]\ (Abschnitt~\ref{sec:mopdescr},
\citepage{\pageref{sec:mopdescr}}).
%
\end{description}
%
\subsection{Fehlerbehandlung}
%
Wie bereits in 
Abschnitt~\ref{sec:ploopr} (\citepage{\pageref{sec:ploopr}}) erw\"{a}hnt,
konnte f\"{u}r die Fehlersuche in
der 2.~Schicht kein \debugger\/ verwendet werden; schlimmer noch war
der Umstand, da\ss{} die Laufzeitfehler im C~Code nicht vom LISP-System
abgefangen werden konnten und ohne weitere Warnung zur Terminierung
des LISP Prozesses f\"{u}hrten. Der C~Code wurde daher so erstellt, da\ss{}
m\"{o}glichst viele der Bedingungen, die zu einem Laufzeitfehler f\"{u}hren,
kontrolliert erkannt werden; bei einem derartig erkannten Fehler wird
mit einer entsprechenden Meldung eine Funktion der 3.~Schicht
aufgerufen, die den Fehler wiederum dem LISP \debugger\/ signalisiert.
Die Fehlerbehandlung ist die einzige Funktion der 2.~Schicht, die
direkt in die 3.~Schicht verzweigt.
%
\subsection{Zusammenfassung der 2.\ Schicht}
%
Die 2.~Schicht erf\"{u}llt die \acid\ Eigenschaften
(\citepage{\pageref{ref:acid}}) folgenderma\ss{}en:
\begin{description}%
%
\item[Unteilbarkeit]
Unteilbarkeit wird durch das realisierte hierarchische Sperrprotokoll
erreicht. Sperren sind an Transaktionen gebunden; bei Ende einer
Transaktion werden alle an sie gebundenen Sperren aufgehoben.
%
\item[Konsistenz]
Zust\"{a}nds\"{a}nderungen, die einen tempor\"{a}r inkonsistenten Gesamtzustand
zur Folge haben, m\"{u}ssen der 2.~Schicht durch Starten einer
Transaktion mitgeteilt werden. Bei Ende der Transaktion wird der neue
Gesamtzustand als konsistent angenommen; bei einem Abbruch wird der
letzte konsistente Gesamtzustand wiederhergestellt.
%
\item[Serialisierbarkeit]
Die in der 2.~Schicht realisierten \twophasetrs\ stellen
Serialisierbarkeit sicher \cite[\citepage{163}]{bib:je87}.
%
\par{}Nach der in \cite[\citepage{401}]{bib:gr93} angegebenen
\"{U}bersicht realisiert die 2.~Schicht eine Serialisierbarkeit von
3$^{\circ}$.
%
\item[Best\"{a}ndigkeit]
F\"{u}r die Best\"{a}ndigkeit gibt es keinen Unterschied zur 1.~Schicht.
%
\end{description}%
%
Der folgende Abschnitt fa\ss{}t noch einmal kurz die Funktionalit\"{a}ten
der 2.~Schicht aus der Sicht der n\"{a}chsth\"{o}heren 3.~Schicht zusammen.
\begin{description}
%
\item[Typisierung der persistenten Objekte]
Alle persistenten Objekte sind dynamisch typisiert.
%
\item[Allozieren eines persistenten Objektes]
Das Allozieren eines persistenten Objektes erfolgt anhand seiner
Typkennung; zur\"{u}ckgegeben wird die zur Referenzierung des
persistenten Objektes n\"{o}tige \sobjid. Das erzeugte Objekt ist
bereits typisiert.
%
\item[Zugriff auf den Zustand von persistenten Objekten]
Zugriffe auf persistente Objekte werden immer in an Transaktionen
gebundene Sperren eingebettet ({\em Lock-Commit\/}-Modell
\cite[\citepage{38}]{bib:ca88}); ferner wird beim Zugriff auf einzelne
Worte gepr\"{u}ft, ob sie innerhalb des Referenz- bzw.\ Wertefeldes
liegen.
%
\item[Wurzelobjekt]
Struktur und Zustand des 2.~Wurzelobjektes werden innerhalb der
2.~Schicht definiert und verwaltet; es kann nicht explizit gelesen
oder \"{u}berschrieben werden au\ss{}er der frei zugreifbaren Komponente,
die die \sobjid\/ des Wurzelobjektes der 3.~Schicht ent\-h\"{a}lt.
%
\item[Speicherr\"{u}ckgewinnung]
Wie in der 1.~Schicht mu\ss{} die Speicherr\"{u}ckgewinnung explizit
veranla\ss{}t werden.
%
\end{description}
%
\section{Schicht 3: Die \plob\ Schicht}
%
F\"{u}r die Realisierung der dritten Schicht wurde \lwcl\ benutzt; eine
Portierung nach \allegrocl\ ist geplant.
%
\par{}\figurename~\ref{fig:pldflow} %
%
\begin{figure}[htbp]%
\ifbuch%
\centerline{\psfig{figure=twelvept/pldflow.eps}}%
\else%
\centerline{\psfig{figure=tenpt/pldflow.eps}}%
\fi%
\caption{Zugriff auf persistente Objekte}%
\label{fig:pldflow}%
\end{figure}%
%
zeigt einen Gesamt\"{u}berblick, wie eine Sitzung Zugriff auf persistente
Objekte erh\"{a}lt. Die Aufgabe der 3.~Schicht besteht darin, die von der
Sitzung kommenden Anfragen (Funktionsaufrufe) zu bearbeiten und unter
Benutzung der 2.~Schicht den Zugriff auf die persistenten Objekte zu
erm\"{o}glichen. Ein persistentes Objekt erh\"{a}lt zu\-s\"{a}tz\-lich zu
seiner persistenten eine transiente \clos\ representation, auf die
die Sitzung transparent zugreifen kann. Zus\"{a}tzlich kann eine Sitzung
auch direkt die persistente \representation{} eines persistenten
Objektes manipulieren.
%
\subsection{Der \protect\cache}
%
Der \cache\/ ist als assoziativer transienter Zwei-Wege-Speicher in der
3.~Schicht realisiert; er bildet sowohl die innerhalb des persistenten
Speichers zur Identita"tsrepr\"{a}sentation verwendeten \objid[s]\ auf
die Identit\"{a}tsrepr\"{a}sentation der transienten \representation{} eines
persistenten Objektes ab als auch umgekehrt. Da der Zustand eines
\immval[s]\/ bereits in seiner Identit\"{a}tsrepr\"{a}sentation enthalten
ist, wird f\"{u}r \immval[s]\/ kein Speicher zu ihrer Repr\"{a}sentation
alloziert; damit entf\"{a}llt auch die Notwendigkeit, in den
\cache\/ Referenzen auf \immval[s]\/ einzutragen.
%
\subsubsection{Effizienzsteigerung}
%
Eine Referenz auf die transiente \representation{} eines persistenten
Objektes werden assoziiert zur \sobjid\/ in den \cache\/ eingetragen;
f\"{u}r nachfolgende Zugriffe auf die transiente Re\-pr\"{a}\-sen\-ta\-tion
desselben persistenten Objektes werden so erneute Aufrufe der unteren
Schichten unn\"{o}tig.
%
\subsubsection{Identit\"{a}tserhaltung}
%
Soweit m\"{o}glich, sollte die transiente \representation{} eines
persistenten Objektes die Iden\-ti\-t\"{a}ts\-re\-pr\"{a}\-sen\-ta\-tion
(d.h.\ die \sobjid\/) der persistenten \representation{} enthalten;
damit kann  die f\"{u}r ein Objekt vergebene \sobjid\/ aus der
transienten \representationform{} des Objektes selbst ermittelt
werden. Die \representationform{} der Instanzen von Basistypen kann
nicht zur zur Aufnahme der Identit\"{a}tsrepr\"{a}sentation erweitert
werden;  mit der Umkehrabbildung des \cache\/ wird eine Abbildung der
Identit\"{a}t der transienten \representation{} (d.h.\ der Adresse) auf die
Identit\"{a}t der persistenten \representation{} (d.h.\ die \sobjid\/)
eines persistenten Objektes einer Basisklasse realisiert.
%
\par{}F\"{u}r \immval[s]\/ ergibt sich ihre Identit\"{a}tsrepr\"{a}sentation
un\-ab\-h\"{a}n\-gig von ihrer Umgebung aus ihrem Zustand; damit wird
ihre Identit\"{a}t durch den Zustand selbst erhalten.
%
\subsubsection{Speicherr\"{u}ckgewinnung des transienten Systems}
%
Da die Referenzen auf die transienten \representations{} aller
dereferenzierten persistenten Objekte in den \cache\/ eingetragen
werden, sind sie f\"{u}r das LISP-System erreichbar; damit werden sie bei
einer Speicherr\"{u}ckgewinnung des LISP-Systems nicht freigegeben, auch
wenn sie ausschlie\ss{}lich durch den \cache\/ referenziert werden.
%
\par{}Die L\"{o}sung besteht darin, f\"{u}r die Referenzen auf die
transienten \representations{} im \cache\/ schwache Referenzen zu
verwenden. Leider stehen diese in dem zur Realisierung verwendetem
\lwcl\ nicht zur Verf\"{u}gung; statt dessen mu\ss{} der \cache\/ falls
n\"{o}tig von den BenutzerInnen durch Aufruf einer entsprechenden
Funktion gel\"{o}scht werden. Bei der Wahl des Zeitpunktes mu\ss{}
ber\"{u}cksichtigt werden, da\ss{} durch das L\"{o}schen der \cache\/ nicht
l\"{a}nger identit\"{a}tserhaltend wirkt.
%
\subsubsection{Konsistenz}%
\label{sec:plcon}
%
Im \cache\/ befinden sich die transienten \representations{} von
persistenten Objekten ohne eine feste Kopplung an die persistente
\representation. Sofern mehrere Sitzungen gleichzeitig modifizierend
auf die persistenten Objekte zugreifen, kann es passieren, da\ss{} die
Zust\"{a}nde der transienten \representation{} im \cache\/ und der
persistenten \representation{} nicht mehr \"{u}ber\-ein\-stim\-men.
%
\par{}Bei der Modifikation eines persistenten Objektes sollten daher
alle Sitzungen, die eine transiente \representation{} dieses Objektes im
\cache\/ stehen haben, benachrichtigt werden, da\ss{} sich der Zustand
ge\-\"{a}n\-dert hat \cite[\citepage{17--21}]{bib:ro87}. Realisiert werden
kann dies durch einen Sperrmodus, der lediglich anzeigt, da\ss{} das
gesperrte Objekt im \cache\/ einer Sitzung eingetragen ist; bei
Zustands\"{a}nderungen werden dann alle Sitzungen, die diese Sperre
gesetzt haben, benachrichtigt und k\"{o}n\-nen ihren
\cache\/ aktualisieren
(\gande{Benachrichtigungssperre}{notify lock}\ %
\cite[\citepage{342}]{bib:fe89}). Die Realisierung dieses 
Sperrmodus erfolgte bisher nicht; sie ist f\"{u}r eine sp\"{a}tere
Programmversion vorgesehen.
%
\subsection{Datentypen}
%
Dieser Abschnitt beschreibt die realisierten Strukturen verschiedener
Datentypen, die benutzt werden, um transiente LISP Objekte persistent
zu halten.
%
\subsubsection{Instanzen von Basistypen}
%
Der Zustand von Instanzen von LISP Basistypen wird in eine
entsprechende Instanz eines in der 2.~Schicht definierten Basistyps
abgebildet.
%
\subsubsection{Klassenbeschreibungen von
\protect\clos\ \protect\cls[n]}%
\label{sec:mopdescr}%
%
Bei der Verwendung von relationalen Datenbanken als persistenter
Speicher werden Klassen oft nicht explizit gespeichert, sondern ihre
Strukturdefinition geht lediglich implizit in die Festlegung der zur
Instanzrepr\"{a}sentation verwendeten Relationen ein
(Abschnitt~\ref{sec:rdbpr}, \citepage{\pageref{sec:rdbpr}}). Ein
Objektspeicher hingegen bietet die M\"{o}glichkeit, auch Klassen
\"{a}hn\-lich wie \clsmo[e]\ explizit zu repr\"{a}sentieren.
%
\par{}In \clos\ werden \clsmo[e]\ zur \representation{} von
\cls[n]\ verwendet; sie enthalten alle Informationen \"{u}ber die
Struktur und das Verhalten der Instanzen der repr\"{a}sentierten
\cls\ (\figurename~\ref{fig:clicmc}).
%
\begin{figure}[htbp]%[htbp]%
\ifbuch%
\centerline{\psfig{figure=twelvept/clicmc.eps}}%
\else%
\centerline{\psfig{figure=tenpt/clicmc.eps}}%
\fi%
\caption{\protect\clos-Instanz und \protect\clsmo}\label{fig:clicmc}%
\end{figure}%
%
\clos-Objekte sind damit selbstbeschreibend.
Die im persistenten System enthaltenen Objekte sind durch ihre
\representationform{} (Abbildungen~\ref{fig:pivec},
\citepage{\pageref{fig:pivec}}\ und \ref{fig:pcivec},
\citepage{\pageref{fig:pcivec}}) ebenfalls selbstbeschreibend.
In der 3.~Schicht wurden wie in Abschnitt~\ref{sec:plpsol}
(\citepage{\pageref{sec:plpsol}}) erkl\"{a}rt \clsdc[n]\ f\"{u}r
Klassenbeschreibungen von persistenten Instanzen definiert, die
Informationen \"{u}ber die Struktur
der beschriebenen Instanzen enthalten (Abschnitt~\ref{sec:ocls},
\citepage{\pageref{sec:ocls}}). Wegen der in Abschnitt~\ref{sec:fnmtd}
(\citepage{\pageref{sec:fnmtd}}) genannten Restriktionen wird das
Verhalten von persistenten Instanzen nicht in einem \clsdo\ mit
abgelegt. Der Begriff \rglq{}Beschreibung\rgrq\ wurde
somit gew\"{a}hlt, um zu verdeutlichen, da\ss{} ein \clsdo\ im Gegensatz zu
einem \clsmo\ nicht alle Informationen \"{u}ber die repr\"{a}sentierte
Klasse enth\"{a}lt.
%
\par{}Aufgrund der \"{A}hnlichkeit zwischen \clsmo[en]\ und der f\"{u}r
\clsdo[e]\ geforderten Eigenschaften in Bezug auf die Repr\"{a}sentation
von Informationen \"{u}ber die Struktur von Instanzen wurden die Klassen
der von einem \clsdo\ referenzierten \deo[e]\ aus den in
\cite{bib:amop91} gegebenen Spezifikationen f\"{u}r
\std-\mc[n]\ abgeleitet. Die Spezifikation in \cite{bib:amop91}
vermeidet allerdings eine Festlegung der
Struktur der \std-\mc[n], sondern legt ausschlie\ss{}lich ihre Stellung
innerhalb der \mc[n]hierarchie und ihr Verhalten fest; die
spezifizierten
\gande{generischen {\protect\std[]}-Lese-Funktionen}%
{standard reader generic functions}%
\addglossary{Generische Standard-Lese-Funktion}%
[standard reader generic function]%
[{Alle in\protect\\%
\protect\amopcite{212--224}\ definierten
generischen \protect\Std-Lese-\protect\fn[en].}]\ gaben
% Die Dinger heissen tatsaechlich `reader generic functions';
% siehe z.B. [AMOP 91], S. 212 oben. Der Praefix `standard' wurde von
% mir hinzugefuegt, da ich die in [AMOP 91] definierten generischen
% Lese-Funktionen meine.
aber indirekt Hinweise auf
die Festlegung ihrer Struktur, d.h.\ f\"{u}r jede generische
\std-Lese-Funktion wurde ein entsprechender \Slt\/ innerhalb der
\dec\ definiert.  Es ergibt sich ein \"{a}hn\-li\-cher Zusammenhang wie
f\"{u}r \std-\mc[n]\ (\figurename~\ref{fig:plicmc}).
%
\begin{figure}[hbtp]%
\ifbuch%
\centerline{\psfig{figure=twelvept/plicmc.eps}}%
\else%
\centerline{\psfig{figure=tenpt/plicmc.eps}}%
\fi%
\caption{Persistente Instanz und \protect\clsdo}%
\label{fig:plicmc}%
\end{figure}%
%
\par{} Um Portabilit\"{a}t der persistenten Objekte zu erm\"{o}glichen,
wurden nach der Bestimmung der \dec[n]\ ihre Strukturen in der
2.~Schicht festgelegt (Abschnitt~\ref{sec:plpsol},
\citepage{\pageref{sec:plpsol}}). Nachtr\"{a}gliche \"{A}nderungen an
\dec[n]\ in der 3.~Schicht m\"{u}ssen dementsprechend in ihre
Strukturdefinitionen in der 2.~Schicht propagiert werden.
%
\subsubsection{Klassenbeschreibungen von Strukturklassen}%
\label{sec:strdescr}%
%
Die \representation{} von Strukturklassen ist im
Gegensatz zu \clos\ \cls[n]\ nur teilweise definiert; in
\stcite{800}\ wird zwar definiert, da\ss{} eine Strukturklasse durch
ein \clsmo\ repr\"{a}sentiert wird, es wird aber nicht weiter festgelegt,
welche Informationen \"{u}ber die Strukturklasse von ihrem \clsmo\ erfragt
werden k\"{o}n\-nen. Wegen der \"{A}hnlichkeit zwischen Struktur- und
\clos-\cls[n]\ orientiert sich die letztendlich gew\"{a}hlte
\representation{} der Beschreibung von Strukturklassen ebenfalls an dem
in \cite{bib:amop91} definiertem \std.
%
\subsubsection{Objekte und Klassen}
%
Die Struktur eines persistenten Objektes wird im
persistenten Speicher durch sein \clsdo\ und die von ihm
referenzierten \deo[e], speziell \sltdo[e], festgelegt; die im
persistenten \clsdo\ und den \sltdo[en]\ enthaltenen Informationen
stammen aus dem \clsmo\ des transienten Objektes sowie aus den
davon referenzierten \sltmo[en].
Es wird nur die Struktur eines persistenten Objektes beschrieben, da
der Funktionscode von Methoden aus den in Abschnitt~\ref{sec:fnmtd}
(\citepage{\pageref{sec:fnmtd}}) genannten Gr\"{u}nden nicht gespeichert
werden kann. Die Hierarchie der definierten \dec[n]\ zeigt
\figurename~\ref{fig:pltlcls}. %
%
\begin{figure}[htbp]%
\ifbuch%
\centerline{\psfig{figure=twelvept/pltlcls.eps}}%
\else%
\centerline{\psfig{figure=tenpt/pltlcls.eps}}%
\fi%
\caption{\protect\dec[n]-Hierarchie f\"{u}r \protect\clos\ Klassen}%
\label{fig:pltlcls}%
\end{figure}%
%
Wie bereits in Abschnitt~\ref{sec:mopdescr}
(\citepage{\pageref{sec:mopdescr}}) erl\"{a}utert, wurden die Strukturen
der \dec[n]\ aus den Strukturen der \std-\mc[n]\ abgeleitet
(\tablename~\ref{tab:mopdescr}). %
%
\begin{figure}[hbtp]%
\centering%
%                  #1  #3 tauschen, falls noetig
\def\makline#1#2#3{#1#2#3}%
\def\oneline#1#2{\makline{\class{#1}}{&}{\class{#2}}}%
\def\acls{%                             Kennzeichen abstrakte Klasse
%{\usefont{OT1}{pzd}{m}{n}\char111}%
{\usefont{U}{pzd}{m}{n}\char111}%
}%
%
\begin{minipage}{\textwidth}\centering% \minipage wg. Fussnote
\let\footnoterule\relax% Kein Trennungsstrich fuer Fussnoten
%
\begin{\figurefontsize}%
\begin{tabular}{|cc|c|}%
\hline
      & \makline {\tabularheader{\std-\mc}}
                 {&}{\tabularheader{\dec}}\\
\hline\hline
\acls & \oneline {metaobject}
                 {plob-description}\\
      & \oneline {standard-class}
                 {class-description}\\
\acls & \oneline {slot-definition}
                 {slot-description}\\
      & \oneline {direct-slot-definition}
                 {direct-slot-description}\\
      & \oneline {effective-slot-definition}
                 {effective-slot-description}\\
      & \oneline {standard-method}
                 {method-description}%
\footnote{Die \dec\ \class{method-description}\ wird zur Zeit nicht
weiter benutzt.}\\
\hline
\multicolumn{3}{r}{\acls\ $\equiv$\ Abstrakte Superklasse}
\end{tabular}%
\end{\figurefontsize}%
\end{minipage}%
%
\let\figurename\tablename%
\caption{Gegen\"{u}berstellung von
 \makline{\protect\mc[n]}{\ und\ }{\protect\dec[n]}}%
\label{tab:mopdescr}%
\end{figure}%
%
F\"{u}r das Lesen der \Slt\/-Zust\"{a}nde der \deo[e]\ wurden
\mtd[n]\ der generischen \std-Lese-Funktionen auf die entsprechenden
\dec[n]\ spezialisiert, so da\ss{} sich \deo[e]\ \"{a}hn\-lich wie
\mo[e]\ verhalten.
%
\paragraph{Die Struktur der
\protect\clsdc\ \protect\class{class-description}}
Eine Instanz der \clsdc\ \class{class-description}\ beschreibt eine
Klasse; sie enth\"{a}lt \ua\ folgende aus dem transienten
\clsmo\ ermittelten Informationen:
\begin{itemize}
%
\item Den Namen der Klasse.
%
\item Die Superklassen und die Klassen-Pr\"{a}zedenzliste als Liste von
Instanzen der \clsdc\ \class{class-description}.
%
\item Die Liste der direkten bzw.\ effektiven \Slt[s]\/ als Liste von
Instanzen der \sltdc[n]\ \class{direct-slot-description}%
\ bzw.\ \class{effective-slot-de\-scrip\-ti\-on}.
%
\end{itemize}
%
Zus\"{a}tzlich werden \ua\ noch folgende Informationen abgelegt:
%
\begin{itemize}
%
\item Das \clsdo\ der Klasse, deren Namen in der Klassendefinition im
Makro \lisp{defclass}\ der beschriebenen Klasse mit der Klassenoption
\lisp{:metaclass}\ angegebenen wurde. Fehlt diese Klassenoption, wird
das \clsdo\ der Klasse \class{standard-class}\ abgelegt.
%
\item Die Art der \gande{Schemaentwicklung}{schema evolution}, die f\"{u}r
die Klasse benutzt werden soll.
%
\item Der Name einer Konstruktor-Funktion. Sofern der Name
\nonnil\ ist, wird die an den Namen gebundene Funktion zur Erzeugung
der transienten \representation{} eines persistenten Objektes der
beschriebenen Klasse aufgerufen, ansonsten wird die Funktion
\stfn{make-instance}\ aufgerufen.
%
\end{itemize}
%
\paragraph{Die Struktur der
\protect\sltdc\ \protect\class{slot-description}}
Eine Instanz der \sltdc\ \class{slot-description}\ beschreibt einen
\Slt\/ einer Klasse; sie enth\"{a}lt \ua\ folgende aus dem
transienten \sltmo\ ermittelten Informationen:
\begin{itemize}
%
\item Den Namen des \Slt[s].
%
\item Die Liste der \gande{Initialisierungs-Argumentnamen}%
{initargs}%
%
\addglossary{Initialisierungs-Argumentname}[initarg]%
[{Ein Initialisierungs-Argumentname wird zur Initialisierung eines
\protect\Slt[s]\/ bei der Erzeugung eines Struktur- oder
\protect\clos-Objektes als Schl\"{u}s\-sel eines
\protect\gande{Schl\"{u}sselwort-Parameters}%
{\protect\&key parameter}\ zusammen mit dem Initialisierungswert an
die ob\-jekt-er\-zeu\-gen\-de Funktion \"{u}bergeben.}]%
\addglossary{Schl\"{u}sselwort-Symbol}[keyword symbol][Jedes
Schl\"{u}sselwort-Symbol ist per Definition immer global sichtbar und hat
sich selbst als Wert.],
%
den \gande{Initialisierungsausdruck}{initform}%
%
[Initialisierungsausdruck][{LISP Ausdruck, dessen Evaluierung im
Verlauf der Initialisierung eines \protect\Slt[s]\/ eines
\protect\clos-Objektes als
Initialisierungswert verwendet wird. Nicht der
Initialisierungsausdruck selbst wird zur Initialisierung benutzt,
sondern eine aus ihm durch eine \"{U}bersetzung erzeugte
\protect\gande{Initialisierungsfunktion}%
{\protect\see{initfunction}}.}]\ % 
%
und die \gande{Initialisierungsfunktion}{initfunction}%
%
[Initialisierungsfunktion][{LISP
Funktion ohne Ar\-gu\-men\-te, deren R\"{u}ck\-ga\-be\-wert im Verlauf
der Initialisierung eines \protect\Slt[s]\/ eines
\protect\clos-Objektes als Initialisierungswert verwendet wird. Die
Initialisierungsfunktion wird w\"{a}hrend der Klasseninitialisierung
durch eine \"{U}bersetzung aus dem
\protect\gande{Initialisierungsausdruck}%
{\protect\see{initform}}\ gewonnen.}].
%
\item Die \gande{Allozierungsart}{slot allocation}[{Allozierungsart
eines \protect\Slt[s]}][{Die Allozierungsart eines
\protect\Slt[s]\/ bestimmt, wie der \protect\Slt\/-Zustand von
\protect\clos\ repr\"{a}sentiert werden soll. F\"{u}r die Art 
\protect\lisp{:in\-stan\-ce}\ wird der \protect\Slt\/-Zustand f\"{u}r
jede Instanz repr\"{a}sentiert, f\"{u}r die Art \protect\lisp{:class}\ wird
der \protect\Slt\/-Zustand genau einmal repr\"{a}sentiert, f\"{u}r alle
anderen Allozierungsarten wird der \protect\Slt\/-Zustand nicht von
\protect\clos\ repr\"{a}sentiert.}]\ des \Slt[s].
%
\end{itemize}
%
Zus\"{a}tzlich werden \ua\ noch folgende Informationen abgelegt:
%
\begin{itemize}
%
\item Die \gande{Lebensdauer}{extent}[]\ des \Slt\/-Zustands.
%
\item In einem direkten \sltdo\ befindet sich ein S-Ausdruck, der den
f\"{u}r den \Slt\/ zu haltenden Index definiert;
ein effektives \sltdo\ enth\"{a}lt eine daraus generierte Indextabelle,
die assoziativ jeden Zustand des beschriebenen \Slt[s]\/ auf eine
dazugeh\"{o}rige persistente \clos-Instanz abbildet. Damit k\"{o}n\-nen
persistente \clos-Instanzen assoziativ gesucht werden; der
Suchschl\"{u}ssel ist ein bestimmter \Slt\/-Zustand, das assoziierte Datum
ist die dazugeh\"{o}rige persistente \clos-Instanz. Als Indextabellen
werden persisistente B-B\"{a}ume benutzt, die eine Abfrage
\"{u}ber einen Schl\"{u}sselbereich erm\"{o}glichen.
%
\end{itemize}
%
\paragraph{Die Struktur der
\protect\sltdc\ \protect\class{di\-rect-slot-des\-crip\-tion}}
Eine Instanz der \clsdc\ \class{direct-slot-description}\ beschreibt
einen direkten \Slt\/ einer Klasse; sie enth\"{a}lt zu\-s\"{a}tz\-lich zu
den ererbten Informationen eine aus dem transienten direkten
\sltmo\ ermittelte Liste mit den Namen der generischen
Zugriffs-\fn[en].
%
\paragraph{Die Struktur der
\protect\sltdc\ \protect\class{ef\-fec\-ti\-ve-slot-des\-crip\-tion}}
%
Eine Instanz der \sltdc\ \class{effective-slot-description}\ beschreibt
einen effektiven \Slt\/ einer Klasse; sie enth\"{a}lt zu\-s\"{a}tz\-lich zu
den ererbten Informationen eine
\gande{Po\-si\-tions-In\-for\-ma\-tion}%
{slot location}. F\"{u}r \Slt[s]\/ mit der Allozierungsart
\lisp{:instance}\ ist dies die Position des Elementes mit
dem \Slt\/-Zustand innerhalb des Zustandsvektors
(\figurename~\ref{fig:pcivec}, \citepage{\pageref{fig:pcivec}}).
Bei \Slt[s]\/ mit der Allozierungsart
\lisp{:class}\ wird in der Po\-si\-tions-In\-for\-ma\-tion der Zustand
des \Slt[s]\/ abgelegt.
%
\subsubsection{Strukturobjekte und Strukturklassen}
%
\figurename~\ref{fig:pltlscls} zeigt die definierten \dec[n]\ f\"{u}r
Strukturklassen. Im Gegensatz zu \clos\ Klassen gibt es f\"{u}r
Strukturklassen keine
\std[isie\-rung]\ im Sinne des \mop; die in einem Struktur-\clsdo\
abgelegten Informationen m\"{u}ssen daher anderweitig besorgt werden.
Abschnitt~\ref{sec:stdiz} (\citepage{\pageref{sec:stdiz}}) erl\"{a}utert
die gew\"{a}hlte L\"{o}sung.
%
\begin{figure}[htbp]%
\ifbuch%
\centerline{\psfig{figure=twelvept/pltlscls.eps}}%
\else%
\centerline{\psfig{figure=tenpt/pltlscls.eps}}%
\fi%
\caption{\protect\dec[n]-Hierarchie f\"{u}r Strukturklassen}%
\label{fig:pltlscls}%
\end{figure}%
%
\paragraph{Die Struktur der
\protect\clsdc\ \protect\class{struc\-ture-des\-crip\-tion}}%
%
Eine Instanz der
\clsdc\ \class{struc\-ture-des\-crip\-tion}\ repr\"{a}sentiert eine
Beschreibung einer Strukturklasse; sie enth\"{a}lt \ua\ folgende
Informationen:
\begin{itemize}
%
\item Den Namen der Klasse.
%
\item Die Liste der effektiven \Slt[s]\/ als Liste von
Instanzen der \sltdc\ \class{structure-slot-description}.
%
\item Die Art der Schemaentwicklung.
%
\item Den Namen der Konstruktor-Funktion. 
%
\end{itemize}
%
\paragraph{Die Struktur der
\protect\sltdc\ \protect\class{struc\-ture-slot-des\-crip\-tion}}%
%
Eine Instanz der
\sltdc\ \class{struc\-ture-slot-des\-crip\-tion}\ beschreibt einen
effektiven \Slt\/ einer Strukturklasse; sie enth\"{a}lt \ua\ folgende
Informationen:
\begin{itemize}
%
\item Den Namen des \Slt[s].
%
\item Das Initialisierungs-Schl\"{u}sselwort-Symbol und den
Initialisierungsausdruck.
%
\item Die Lebensdauer des \Slt\/-Zustands.
%
\item Die Position des Elementes mit dem \Slt\/-Zustand innerhalb des
Referenzfeldes (\figurename~\ref{fig:pivec},
\citepage{\pageref{fig:pivec}}).
%
\end{itemize}
%
\par{}Da f\"{u}r Strukturklassen keine multiple Vererbung m\"{o}glich ist,
gibt es hier keine Unterscheidung zwischen direkten und effektiven
\Slt[s]; sie entsprechen in etwa den effektiven \Slt[s]\/ einer
\clos\ Klasse.
%
\subsubsection{Funktionen}
%
Der hier gew\"{a}hlte Ansatz sieht Funktionen als extern-persistente
Objekte an und erm\"{o}glicht damit eine teilweise Speicherung.  Von
einer Funktion wird nur der Name gespeichert; der Name einer
persistenten Funktion wird durch ein persistentes Symbol
repr\"{a}sentiert. Kann der Name einer transienten Funktion nicht
ermittelt werden, wird statt der \sobjid\/ einer gespeicherten Funktion
die \sobjid\/ des Markierungsobjektes \rglq{}nicht speicherbares
Objekt\rgrq\ als Ergebnis der Speicherung zur\"{u}ckgegeben.
%
\par{}Beim Laden wird eine (transiente) Funktion im LISP
{\em image\/} gesucht, die an das transiente Symbol mit dem durch den
Namen der persistenten Funktion vorgegebenen persistenten Symbolnamen
und persistenten Paketnamen gebunden ist.  Damit wird implizit
angenommen, da\ss{} eine unter dem gleichen Namen im transienten System
\rglq{}wiedergefundene\rgrq\ Funktion identisch zu der urspr\"{u}nglich
gespeicherten Funktion ist. Wenn keine Funktion
gefunden wird, wird stattdessen eine Funktion zur\"{u}ckgegeben, die bei
einem Aufruf einen Fehler signalisiert.
%
%\subsubsection{Extern persistente Objekte}
%
% Werden in \plob\ nicht behandelt.
%
\subsection{\protect\Spc\ \protect\mc[n]}
%
Anders als in \soh\ \cite[\citepage{9}]{bib:ro87} und
\pclos\ \cite[\citepage{11}]{bib:pa91a} ist in \plob\ die M\"{o}glichkeit
der Persistenz grunds\"{a}tzlich f\"{u}r alle Objekte un\-ab\-h\"{a}n\-gig von
ihrer Klasse gegeben; die realisierten \spc[n]\ \mc[n]\ werden
lediglich als Hilfsmittel benutzt, um transiente Instanzen der aus der
\spc[n]\ \clsmc\ \class{persistent-metaclass}\ hervorgegangenen
Klassen mit zu\-s\"{a}tz\-li\-chen Eigenschaften bez\"{u}glich Persistenz zu
versehen. Die durch ein
\clsmo\ der
\spc[n]\ \clsmc\ \class{per\-si\-stent-""me\-ta\-class}\ %
repr\"{a}sentierten Klassen sind aktiv-persistente Klassen; alle anderen
Klassen sind pas\-siv-""per\-si\-stent.
%
\par{}Aktiv-persistente Klassen haben gegen\"{u}ber passiv-persistenten
folgende zu\-s\"{a}tz\-li\-che Eigenschaften:
\begin{itemize}
%
\item In der Klassendefinition kann innerhalb einer \Slt\/-Definition
mit einer \Slt\/-Option ein Index f\"{u}r den \Slt\/ deklariert werden;
die Verwaltung des Indexes wird vollst\"{a}ndig in der 3.~Schicht
durchgef\"{u}hrt.
%
\item Die Lebensdauer jedes \Slt\/-Zustands kann einfacher deklariert
werden; die Deklaration kann direkt als \Slt\/-Option in die
\Slt\/-Definition innerhalb der Klassendefinition eingef\"{u}gt werden.
F\"{u}r einen \Slt\/ einer passiv-persistenten Instanz mu\ss{} die
Deklaration \"{u}ber die Lebensdauer seines Zustands \"{u}ber einen
Funktionsaufruf an das persistente System \"{u}ber\-ge\-ben werden.
%
\item F\"{u}r den \Slt\/-Zustand stehen mehr Repr\"{a}sentationsarten zur
Verf\"{u}gung als bei pas\-siv-""per\-si\-sten\-ten Klassen
\ifbericht%
\cite[\citepage{120}]{bib:ki94a}%
\else%
(\figurename~\ref{fig:extent}, \citepage{\pageref{fig:extent}})%
\fi.
%
\item Die Art der Schemaentwicklung kann einfacher deklariert
werden; sie kann direkt als \cls[n]\/-Option in die Klassendefinition
eingef\"{u}gt werden.  F\"{u}r eine passiv-persistente Klasse mu\ss{} die
Deklaration f\"{u}r die Schemaentwicklung \"{u}ber einen Funktionsaufruf an
das persistente System \"{u}bergeben werden.
%
\end{itemize}
%
Dieser Abschnitt beschreibt zun\"{a}chst die zur Realisierung dieser
zu\-s\"{a}tz\-li\-chen Eigenschaften festgelegten Strukturen der
\spc[n]\ \mc[n]; ihr Verhalten wird in 
\ifbericht%
\cite[\citepage{111}]{bib:ki94a}%
\else%
Abschnitt~\ref{sec:plpr} (\citepage{\pageref{sec:plpr}})
\fi\ erl\"{a}utert. \figurename~\ref{fig:pltlmcls} zeigt die Hierarchie
der \spc[n]\ \cls[n]- und \Slt\/-\mc[n].
%
\begin{figure}[htbp]%
\ifbuch%
\centerline{\psfig{figure=twelvept/pltlmcls.eps}}%
\else%
\centerline{\psfig{figure=tenpt/pltlmcls.eps}}%
\fi%
\caption{\protect\Spc\ \protect\cls[n]- und
\protect\Slt\/-\protect\mc[n]}\label{fig:pltlmcls}%
\end{figure}%
%
\subsubsection{Die Struktur der \protect\spc[n]\ \protect\clsmc\ %
\protect\class{persistent-metaclass}}
%
Eine Instanz der \clsmc\ \class{persistent-metaclass}\ repr\"{a}sentiert
eine ak\-tiv-per\-si\-sten\-te Klasse. Sie enth\"{a}lt \ua\ folgende aus
den zu\-s\"{a}tz\-li\-chen Klassenoptionen der \lisp{defclass}-Anweisung
ermittelten Informationen, die zum Aufbau des \clsdo[es]\ benutzt
werden:
%
\begin{description}
%
\item[Klassenoption \lisp{:extent}]
Die auf Klassenebene deklarierte Lebensdauer der
\Slt\/-Zust\"{a}nde; sie wird standardm\"{a}\ss{}ig f\"{u}r alle \Slt[s]\/ benutzt,
in deren \Slt\/-Definition keine andere Lebensdauer deklariert wird.
%
\item[Klassenoption \lisp{:schema-evolution}]
Die Art der Schemaentwicklung, die f\"{u}r die Klasse deklariert wurde.
%
\item[Klassenoption \lisp{:constructor}]
Der Name der Konstruktor-Funktion.
%
\end{description}
%
Zus\"{a}tzlich enth\"{a}lt sie noch folgende Informationen:
%
\begin{itemize}
%
\item Eine \gande{Statusvariable}{flag}[], die angibt, ob die
Klassendefinition seit der Speicherung der letzten Version des
\clsdo[es]\ ge\-\"{a}n\-dert wurde.
%
\item Eine Referenz auf die transiente \representation{} des zur Klasse
geh\"{o}rigen \clsdo[es].
%
\item Eine Statusvariable mit der Information, ob die
\gande{Initialisierung des \protect\clsdo[es]\ endg\"{u}ltig
abgeschlossen ist}{finalized}[{Endg\"{u}ltig abgeschlossene
Initialisierung eines \protect\clsmo[es]\ oder \protect\clsdo[es].}].
%
\end{itemize}
%
\subsubsection{Die Strukturen der
\protect\spc[n]\ \protect\sltmc[n]\ %
\protect\class{per\-si\-stent-di\-rect-slot-de\-fi\-ni\-tion}\ und
\protect\class{per\-si\-stent-ef\-fec\-tive-slot-de\-fi\-ni\-tion}}
%
Instanzen der
\sltdc[n]\ \class{per\-si\-stent-di\-rect-slot-de\-fi\-ni\-tion}\ bzw.\ %
\class{per\-si\-stent-ef\-fec\-tive-slot-de\-fi\-ni\-tion}\ %
repr\"{a}sentieren  einen direkten bzw.\ effektiven \Slt\/ einer
ak\-tiv-""per\-si\-sten\-ten Klasse.
%
\par{}Die in einer Instanz der
\sltdc\ \class{persistent-direct-slot-definition}\ enthaltenen
Informationen stammen aus den zu\-s\"{a}tz\-li\-chen \Slt\/-Optionen
einer \Slt\/-Definition; diese Informationen werden zum Aufbau eines
direkten \sltdo[es]\ benutzt. 
%
\begin{description}
%
\item[{\Slt\/-Option \lisp{:extent}}]
Die Lebensdauer des \Slt\/-Zustands.
%
\item[{\Slt\/-Option \lisp{:index}}]
Ein S-Ausdruck mit einer Beschreibung des Indexes, der f\"{u}r den
\Slt\/ angelegt und verwaltet werden soll.
%
\end{description}
%
Gleichnamige direkte \Slt[s]\/ werden im Verlauf der
Klasseninitialisierung zu einem einzigen effektiven
\Slt\/ zusammengefa\ss{}t\ifbericht\else\ (Protokoll~\ref{pro:fi}
\stfn{(finalize-inheritance)}, \citepage{\pageref{pro:fi}})%
\fi. Eine Instanz der
\sltdc\ \class{persistent-effective-slot-definition}\ enth\"{a}lt 
au\ss{}erdem noch die urspr\"{u}nglich f\"{u}r den \Slt\/ spezifizierte
Allozierungsart.
%
\subsubsection{Die \protect\spc[n]\ \protect\mtdmc[n]\ %
\protect\class{per\-si\-stent-rea\-der-me\-thod}\ und
\protect\class{per\-si\-stent-wri\-ter-me\-thod}}
%
F\"{u}r den \Slt\/-Zugriff auf persistente \clos-Instanzen wurden die
beiden \spc[n]\ \mtdmc[n]\ \class{persistent-reader-method}\ und
\class{persistent-writer-method}\ definiert
(\figurename~\ref{fig:pltlacls}); Instanzen dieser
\spc[n]\ \mtdmc[n]\ ersetzen die \std-Zugriffs-\mtd[n].
%
\begin{figure}[htbp]%[tbp]%
\ifbuch%
\centerline{\psfig{figure=twelvept/pltlacls.eps}}%
\else%
\centerline{\psfig{figure=tenpt/pltlacls.eps}}%
\fi%
\caption{\protect\Spc\ \protect\mtd[n]-\protect\mc[n]}%
\label{fig:pltlacls}%
\end{figure}%
%
\subsection{Schnitt zwischen Objekten}
%
Beim Transfer eines Objektes zwischen transientem und persistentem
Speicher werden normalerweise alle transitiv referenzierten Objekte
mit transferiert; wie sich bei der Evaluierung des Systems
\soh\ zeigte, kann dies unter Umst\"{a}nden dazu f\"{u}hren, da\ss{} das
persistente System versucht, den gesamten Zustand des LISP-Systems zu
\"{u}bertragen. Um die Anzahl der transferierten Objekte zu begrenzen,
k\"{o}n\-nen die BenutzerInnen daher Hinweise geben, ab wann die transitiv
referenzierten Objekte nicht mit \"{u}bertragen werden sollen.
%
\par{}Bei Instanzen von Basisklassen kann beim Speichern bzw.\ Laden
ein Parameter angegeben werden, bis zu dem
\"{u}bertragen werden soll; beispielsweise werden bei einer bestimmten
Angabe Objekte, die bereits zu einem fr\"{u}heren Zeitpunkt
gespeichert oder geladen wurden, nicht weiter betrachtet. Andere
Angaben schr\"{a}nken die Anzahl der Objektbestandteile, die
gespeichert werden sollen, ein; so kann z.B.\ f\"{u}r ein Symbol
angegeben werden, da\ss{} nur der Name des Symbols in die
persistente \representation{} \"{u}bertragen wird.
%
\par{}F\"{u}r Struktur- und \clos-Objekte kann deklarativ die
Lebensdauer eines \Slt\/-Zustands als transient festgelegt werden;
transiente \Slt[s]\/ werden beim Abspeichern nicht weiter traversiert.
Deklarationen beziehen sich auf Klassen und gelten dann f\"{u}r alle ihre
Instanzen.
%
\subsection{Protokolle}%
\label{sec:plpr}%
%
\ifbericht%
%
\let\nextstep\relax
In \cite[\citepage{111 ff.}]{bib:ki94a}\ werden die Protokolle, mit
denen \plob\ Persistenz f\"{u}r \clos\ und LISP Objekte realisiert,
spezifiziert und erl\"{a}utert.
%
\else%
\def\nextstep{%
In den folgenden Beschreibungen werden die verschiedenen
\ifbericht von \plob\ \fi im Verlauf
eines Protokolls durchgef\"{u}hrten Aktivit\"{a}ten in der
Spezifikationssprache {\sc Infol} ({\em Info\/}rmal {\em L\/}anguage)
beschrieben. Einger\"{u}ckter Text spezifiziert eine Verfeinerung der in
seiner \"{U}berschrift genannten Aktivit\"{a}t; da die Sprache informal ist,
existieren keine weiteren Festlegungen bez\"{u}glich ihrer Syntax oder
Semantik. Die hier benutzte Beschreibungsform richtet sich nach
\cite[\citepage{18, 31--33}]{bib:pa91b}.
%
\subsubsection{Definition von aktiv-persistenten \protect\cls[n]}
%
Im Verlauf der Definition von aktiv-persistenten
\cls[n]\ werden folgende Aufgaben erledigt:
\begin{enumerate}
%
\item Die Deklarationen der aktiv-persistenten Klassen werden
ausgewertet; sie befinden sich in der
\gande{Klassendefinitions}{defclass}-Anweisung:
\begin{itemize}
%
\item Deklarationen auf Klassenebene werden als zu\-s\"{a}tz\-li\-che
Klassenoptionen spezifiziert und ausgewertet; sie werden in das
die Klasse repr\"{a}sentierende \clsmo\ eingetragen.
%
\item Deklarationen auf \Slt\/-Ebene werden als zu\-s\"{a}tz\-li\-che
\Slt\/-Optionen spezifiziert und ausgewertet; sie werden in
die die \Slt[s]\/ der Klasse repr\"{a}sentierenden \sltmo[e]\ eingetragen.
%
\end{itemize}
%
\item F\"{u}r den Zugriff auf den \Slt\/-Zustand der Instanzen von
aktiv-persistenten Klassen \"{u}ber die in der
Klassendefinitions-Anweisung spezifizierten generischen
\gande{Lese}{:reader}- und \gande{Schreib}{:writer}-Funktionen
werden \spc\ \mtdmc[n]\ eingebunden, die die verschiedenen m\"{o}glichen
\Slt\/-\representations{} (\figurename~\ref{fig:extent},
\citepage{\pageref{fig:extent}}) korrekt verarbeiten.
%
\end{enumerate}
In \soh\ wurde bei jeder Evaluierung einer Klassendefinition die
Relation zur \representation{} der Instanzen in der Datenbank angelegt;
dies f\"{u}hrte dazu, da\ss{} diese Evaluierung relativ viel Zeit in
Anspruch nahm. In \plob\ wird daher das die Klasse beschreibende
\clsdo\ nicht im Verlauf der Definitionsevaluierung, sondern zum
sp\"{a}testm\"{o}glichen Zeitpunkt (d.h.\ beim n\"{a}chsten Dereferenzieren des
Zustands einer Instanz der Klasse) im Speicher abgelegt.
%
% Neuer Schritt im Vergleich zu \cite[\citepage{31}]{bib:pa91b}:
\def\addeditemchar{%
%{\usefont{OT1}{pzd}{m}{n}\char58}%
{\usefont{U}{pzd}{m}{n}\char58}%
}%
%
% Implementationsspezifischer (d.h. \lwcl) Schritt:
\def\lwclitemchar{%
%{\usefont{OT1}{pzd}{m}{n}\char54}%
{\usefont{U}{pzd}{m}{n}\char54}%
}%
%
\def\plobmtdchar{$\uparrow$}%
%
\def\addeditem{~\hfill{\small\addeditemchar}}
\def\lwclitem{~\hfill{\small\lwclitemchar}}
%
\begin{infol}
%
\captionfrom%
 {Protokoll f\"{u}r die Definition einer aktiv-persistenten Klasse}%
 {\cite[\citepage{31}]{bib:pa91b}}%
 [(defclass)]%
 \label{pro:defclass}%
\nopagebreak[4]%
%
\item\label{enu:dcsynt} Syntaxpr\"{u}fung.
%
\item Das \clsmo\ wird ermittelt:\\
\stfn{(ensure-class), (ensure-class-using-class)}
%
\begin{block}
%
\item Suche das \clsmo; wird es nicht gefunden, erzeuge ein
\clsmo\ als Instanz der \clsmc, dessen Name mit der Klassenoption
\lisp{:metaclass}\ angegeben wurde:%
\footnote{F\"{u}r aktiv-persistente Klassen ist der Wert der
Klassenoption \lisp{:metaclass}\ immer
\class{persistent-metaclass}\ oder der Name einer Subklasse der Klasse
\class{persistent-metaclass}.}\\
\stfn{(make-instance (find-class %
{\usefont{OT1}{\rmdefault}{m}{n}\it\lt{}Wert der Klassenoption
\lisp{:metaclass}\gt\/}))}
%
\item Die Klassenoptionen werden in eine kanonische Form gebracht:%
\footnote{Dieser Schritt wird nur in \lw\ an dieser Stelle
ausgef\"{u}hrt; nach \cite[\citepage{31}]{bib:pa91b} befindet er sich
normalerweise hinter Schritt~\ref{enu:dcsynt}.}%
\lwclitem\\
\stfn{(clos::canonicalize-class-options)}
%
\item\label{enu:dcinit} (Re)initialisiere das \clsmo:\\
\stfn{((re)initialize-instance)}
\item[\plobmtdchar] Bei der Initialisierung werden die Klassenoptionen
als Ini\-tia\-li\-sie\-rungs-Schl\"{u}s\-sel\-wort-Pa\-ra\-me\-ter an die
Methode \stfn{(re)initialize-instance}\ \"{u}bergeben und im
\clsmo\ gespeichert.
\par{}Die Liste der Superklassen wird um die Klasse
\class{persistent-clos-object}\ erweitert; damit wird an die
definierte Klasse u.a.\ ein \Slt\/ vererbt, der f\"{u}r jede Instanz der
definierten Klasse die \sobjid\/ der persistenten Repr\"{a}sentation
enth\"{a}lt.
%
\begin{block}
%
\item Nicht \"{u}bergebene Initialisierungs-Schl\"{u}sselwort-Parameter
werden mit \std[\-wer\-ten]\ versehen; die
Initialisierungs-Schl\"{u}sselwort-Parameter werden auf Konsistenz
gepr\"{u}ft.
%
\item Es wird gepr\"{u}ft, ob die Superklassen kompatibel sind:\\
\stfn{(validate-superclasses)}
\item[\plobmtdchar] Die Klassen \class{persistent-metaclass}\ und
\class{standard-class}\ sind kompatibel; die Methode gibt daher immer
\lispt\ zur\"{u}ck.
%
\item Die Superklassen werden zum \clsmo\ assoziiert.
% Die Liste der Superklassen wird in das \clsmo\ eingetragen.
%
\item Bestimme die \sltmc\ der direkten \Slt[s]:\\
\stfn{(direct-slot-definition-class)}
\item[\plobmtdchar] Die direkten \Slt[s]\/ einer aktiv-persistenten
Klasse werden durch die \spc\ \sltmc\ %
\class{persistent-direct-slot-definition}\ repr\"{a}sentiert; sie
enth\"{a}lt im Vergleich zur
\std-\sltmc\ \class{stan\-dard-""di\-rect-""slot-""de\-fi\-ni\-tion}\ %
zu\-s\"{a}tz\-li\-che \Slt[s]\/ zur Speicherung der \Slt\/-Optionen.
%
\item Die \Slt\/-Optionen werden in eine kanonische Form
gebracht:\lwclitem\\
\stfn{(clos::canonicalize-defclass-slot)}
%
\item Die direkten \sltmo[e]\ werden erzeugt und initialisiert:\\
\stfn{(make-instance), (initialize-instance)}
\item[\plobmtdchar] Bei der Initialisierung werden die \Slt\/-Optionen
als Ini\-tia\-li\-sie\-rungs-Schl\"{u}s\-sel\-wort-Pa\-ra\-me\-ter an die
Methode \stfn{initialize-instance}\ \"{u}bergeben und im direkten
\sltmo\ gespeichert.
%
\item Die direkten \sltmo[e]\ werden zum \clsmo\ assoziiert.
%
\item Die
\gande{\std-Ini\-tia\-li\-sie\-rungs-Ar\-gu\-men\-te}%
{default initargs}\ werden zum \clsmo\ assoziiert.
%
\item Die Klasse wird in die Subklassenliste aller Superklassen
eingetragen:\\
\stfn{(add-direct-subclass), (remove-direct-subclass)}
%
\item Falls m\"{o}glich, wird die Vererbungs-Initialisierung wie in
Protokoll~\ref{pro:fi}
\stfn{(finalize-inheritance)}\ (\citepage{\pageref{pro:fi}}) angegeben
endg\"{u}ltig abgeschlossen:\\
\stfn{(finalize-inheritance)}
%
\item\label{enu:dcmcl} Bestimme die \mtdmc\ der
Zugriffs-\mtd[n]:\addeditem\\
\stfn{(reader-method-class), (writer-method-class)}
\item[\plobmtdchar] Die auf
\class{persistent-metaclass}\ \spc\ \mtd\ der
\gfn[n]\ \stfn{rea\-der-""me\-thod-""class}\ gibt das
\mtdmo\ der \spc[n]\ \mtdmc\ %
\class{per\-si\-stent-""rea\-der-""me\-thod},
die auf \class{per\-si\-stent-""me\-ta\-class}\ \spc\ \mtd\ der
\gfn[n]\ \stfn{wri\-ter-""me\-thod-""class}\ gibt das \mtdmo\ der
\spc[n]\ \mtdmc\ \class{per\-si\-stent-""wri\-ter-""me\-thod}\ %
zur\"{u}ck;
damit werden diese \mtdmc[n]\ f\"{u}r die Generierung der
Zugriffs-\mtd[n]\ benutzt.
%
\item\label{enu:dcmcr} Die \mtdmo[e]\ der Zugriffs-\mtd[n]\ werden
erzeugt und initialisiert:\addeditem\\
\stfn{(make-instance), (initialize-instance) [, (make-method-lambda)]}
\item[\plobmtdchar] Da die $\lambda$-Liste der in \lw\ realisierten
\gfn[n]\ \stfn{ma\-ke-""me\-thod-""lamb\-da}\ nicht mit dem in
\amopcite{207}\ definierten \std\ \"{u}ber\-ein\-stimmt, wurde die
\gfn\ \stfn{make-accessor-lambda}\ definiert. Ihre
Methoden erzeugen die $\lambda$-Ausdr\"{u}cke der Zugriffs-\mtdmo[e]; sie
werden \"{u}ber\-setzt und im Verlauf der Aufrufe der generischen
Zugriffs-\fn[en]\ ausgef\"{u}hrt. Die erzeugten $\lambda$-Ausdr\"{u}cke sind
im Moment so aufgebaut, da\ss{} sie immer die
\gfn\ \stfn{slot-value-using-class}\ aufrufen;  
in einer sp\"{a}teren Programmversion k\"{o}n\-nen optimierte
$\lambda$-Ausdr\"{u}cke f\"{u}r den \Slt\/-Zugriff auf persistente Instanzen
erzeugt werden.
%
\item\label{enu:dcmad} Die \mtdmo[e]\ der Zugriffs-\mtd[n]\ werden zum
\clsmo\ assoziiert:\addeditem\\
\stfn{(add-direct-method)}
%
\end{block}\end{block}
%
\vspace{1ex}\hspace{-\leftmargin}\parbox{\textwidth}{%
 \footnotesize%
%
 \plobmtdchar:~Beschreibung der realisierten Methode%
 \hspace*{\fill} System-(d.h.\ \mbox{\lw-)} spezifischer
 Schritt:~\lwclitemchar\\
%
 \hspace*{\fill} Hinzugef\"{u}gter oder ge\"{a}nderter Schritt im Vergleich
 zu \cite[\citepage{31}]{bib:pa91b}:~\addeditemchar}
%
\end{infol}
%
Das in \cite[\citepage{31}]{bib:pa91b} angegebene Protokoll ist wegen
einiger fehlender Schritte nicht voll\-st\"{a}n\-dig \mop-konform.
Der hinzugef\"{u}gte Schritt~\ref{enu:dcmcl} l\"{a}\ss{}t sich
aus den Definitionen der
\gfn[n]en \stfn{rea\-der-me\-thod-class}\ \amopcite{224}\ und
\stfn{wri\-ter-me\-thod-class}\ \amopcite{242}\ herleiten.
F\"{u}r die neuen Schritte~\ref{enu:dcmcr} und \ref{enu:dcmad} findet
sich keine direkte Begr\"{u}ndung in \cite{bib:amop91}. Ich habe sie aus
der Tatsache hergeleitet, da\ss{} in \cite{bib:amop91} nicht zwischen
\rglq{}normalen\rgrq\ be\-nut\-zer\-In\-nen-de\-fi\-nier\-ten Methoden
und aus der Klassendefinition generierten
Zugriffs-\mtd[n]\ unterschieden wird; demzufolge sollten auch sie wie
normale Methoden behandelt werden. Leider wird diese Ansicht (noch)
nicht von den Autoren des \lwcl\ geteilt, da das dort realisierte
\mop\ die Schritte~\ref{enu:dcmcl}--\ref{enu:dcmad} nicht befolgt;
statt dessen werden die \mtdmo[e]\ der Zugriffs-\mtd[n]\ im
Verlauf der Initialisierung des \clsmo[es]\ erzeugt und
eingetragen, ohne da\ss{} weitere \gfn{}en dazu aufgerufen
werden. Um trotzdem die Zugriffs-\mtd[n]\ durch Instanzen
der \mtdmc[n]\ \class{persistent-reader-method}\ bzw.\ %
\class{persistent-writer-method}\ repr\"{a}sentieren zu k\"{o}n\-nen, werden
die vom System erzeugten \mtdmo[e]\ nach ihrer Generierung ersetzt;
sinngem\"{a}\ss{} wird wie in Protokoll~\ref{pro:defclass}
\stfn{(defclass)}, Schritt~\ref{enu:dcmcl}--\ref{enu:dcmad},
spezifiziert vorgegangen.
%
\begin{infol}
%
\captionfrom%
 {Protokoll f\"{u}r den endg\"{u}ltigen Abschlu\ss{} der
  Vererbungs-Initialisierung}%
 {\cite[\citepage{32}]{bib:pa91b}}%
 [(finalize-inheritance)]%
 \label{pro:fi}%
\nopagebreak[4]%
%
\item Die Klassen-Pr\"{a}zedenzliste wird berechnet:\\
\stfn{(compute-class-precedence-list)}
%
\item Konflikte zwischen gleichnamigen ererbten \Slt[s]\/ werden
aufgel\"{o}st:
%
\begin{block}
%
\item Bestimme die \sltmc\ der effektiven \Slt[s]:\\
\stfn{(effective-slot-definition-class)}
\item[\plobmtdchar] Die effektiven \Slt[s]\/ einer
aktiv-persistenten Klasse werden durch die \spc\ \sltmc\ %
\class{persistent-effective-slot-definition}\ repr\"{a}sentiert; sie
enth\"{a}lt im Vergleich zur
\std-\sltmc\ %
\class{standard-effective-slot-definition}\ zu\-s\"{a}tz\-li\-che
\Slt[s]\/ zur Speicherung der aus den direkten \Slt[s]\/ abgeleiteten
Informationen.
%
\item Die effektiven \sltmo[e]\ werden erzeugt:\\
\stfn{(make-instance)}
%
\item Die effektiven \sltmo[e]\ werden initialisiert:\\
\stfn{(initialize-instance), (compute-effective-slot-definition)}
\item[\plobmtdchar] Gleichnamige direkte
\Slt[s]\/ werden in der Methode
\stfn{compute-effective-slot-definition}\ zu einem effektiven
\Slt\/ zusammengefa\ss{}t.
%
\item Die effektiven \sltmo[e]\ werden zum \clsmo\ assoziiert.
%
\end{block}
%
\end{infol}
%
In der Methode \stfn{compute-effective-slot-definition}\ wird die
Lebensdauer des \Slt\/-Zustands bestimmt; dazu werden hierarchische
Deklarationen ausgewertet. Eine Lebensdauer kann auf einer der Ebenen
\rglq\Slt\/\rgrq\ -- \rglq\cls\rgrq\ -- \rglq{}Paket des
\cls[n]-Namens\rgrq\  deklariert werden; die Lebensdauer
des \Slt\/-Zustands ergibt sich aus der ersten Deklaration, die
in der angebenen Reihenfolge f\"{u}r eine der Ebenen gefunden wird. Wurde
keine Deklaration gefunden, wird f\"{u}r die Lebensdauer des
\Slt\/-Zustands der in der Konstanten
\stvar{+de\-fault-clos-slot-ex\-tent+}\ enthaltene
\std[wert]\ \lisp{:cached}\ benutzt. 
%
\subsubsection{Passiv-persistente \protect\clos- und
Struktur-\protect\cls[n]}%
\label{sec:nrmlp}
%
Die M\"{o}glichkeiten des \mop, Zustand und Verhalten eines
\clsmo[es]\ im  Verlauf seiner Definition wie in
Protokoll~\ref{pro:defclass}
\stfn{(defclass)}\ (\citepage{\pageref{pro:defclass}}) angegeben \"{u}ber
spezialisierte \mtd[n]\ der vom
\mop\ aufgerufenen \gfn[n]en zu modifizieren, ist zum einen an die
Form (Spezifikation der Klassenoption \lisp{:metaclass}) und zum
anderen an die Evaluierung der Klassendefinition gebunden.
F\"{u}r Klassen, deren Definition bereits evaluiert wurde, besteht daher
nicht die M\"{o}glichkeit, ihr \clsmo\ wie in Protokoll~\ref{pro:defclass}
\stfn{(defclass)}\ angegeben zu beeinflussen. Um unvorhersehbare
Seiteneffekte auszuschlie\ss{}en, verzichtete ich darauf, diese Klassen
bei Bedarf neu zu definieren, um eine Modifikation zu erreichen. Statt
dessen sind die M\"{o}glichkeiten dieser passiv-persistenten Klassen
eingeschr\"{a}nkt:
\begin{itemize}
%
\item Da die Zugriffs-\mtd[n]\ der Klasse nicht ver\"{a}ndert
werden, kann auf den Objektzustand nur \"{u}ber \swizzling\ zugegriffen
werden.
%
\item Aus dem gleichen Grund kann f\"{u}r \Slt[s]\/ von
passiv-persistenten Klassen kein Index definiert werden.
%
\item Da keine zu\-s\"{a}tz\-li\-chen \cls[n]- oder \Slt\/-Optionen
spezifiziert werden k\"{o}n\-nen, m\"{u}ssen Deklarationen bez\"{u}glich
Persistenz bei passiv-persistenten Klassen \"{u}ber Aufrufe
entsprechender Funktionen erfolgen, die die Deklarationen
in Tabellen speichern.
%
\end{itemize}
%
Die eingeschr\"{a}nkten M\"{o}glichkeiten gelten sowohl f\"{u}r
passiv-persistente \clos\ Klassen als auch f\"{u}r
Strukturklassen.
%
\subsubsection{Abgleich zwischen einer Klassendefinition und
seiner Klassenbeschreibung}
%
Der Abgleich sorgt daf\"{u}r, da\ss{} die aktuelle Klassendefinition im
transienten System durch eine \"{a}quivalente Klassenbeschreibung im
persistenten Speicher repr\"{a}sentiert wird. Der Abgleich erfolgt immer
vor der \"{U}bertragung des Zustands eines Struktur- oder \clos-Objektes
zwischen dem transienten und persistenten Speicher f\"{u}r die
Klasse der zu \"{u}bertragenden Instanz. F\"{u}r Instanzen von Basistypen
ist dieser Abgleich nicht notwendig, da ihre Strukturen als
un\-ver\-\"{a}n\-der\-lich festgelegt sind.
%
\begin{infol}%
%
\acaption%
 {Protokoll f\"{u}r den Abgleich einer transienten
  \protect\clos\ Klassendefinition mit seiner
  persistentenKlassenbeschreibung}%
 [\protect\enscldescr]%
 \label{pro:escl}%
%
\item\label{enu:ensdmo} Unter dem Namen der Klasse
wird das dazugeh\"{o}rige \clsmo\ gesucht:\\
\stfn{(find-class)}
%
\item\label{enu:ensddo} In der persistenten Klassentabelle
wird unter dem Namen der Klasse nach einem bereits existierendem
\clsdo\ gesucht:\\
\stfn{(p-find-class)}
%
\item Falls in Schritt~\ref{enu:ensddo} ein \clsdo\ gefunden wurde:
%
\begin{block}%
%
\item Falls in Schritt~\ref{enu:ensdmo} ein \clsmo\ gefunden wurde:
%
\begin{block}%
%
\item\label{enu:ensdmp} Es wird gepr\"{u}ft, ob sich das transiente
  \clsmo\ aus Schritt~\ref{enu:ensdmo} seit dem letzten Aufruf von
  \stfn{\enscldescr}\ ge\"{a}ndert haben k\"{o}nnte:\\
  \stfn{(mismatch-p)}\\
  Der Aufruf von \stfn{(mismatch-p)}\ gibt den Wert einer an das
  transiente \clsmo\ assoziierten Statusvariablen zur\"{u}ck, das in
  den auf die Klasse \class{class}\ \spc[n]\ \mtd[n]\ der \gfn[n]en
  \stfn{initialize-instance}\ und \stfn{reinitialize-instance}\ auf
  \lispt\ gesetzt wird; die \mbox{(Re-)}Initialisierung eines
  \clsmo[es]\ wird damit als Indikator einer eventuell stattgefundenen
  \mbox{(Re-)}Definition benutzt.
%
\begin{block}%
%
\item\label{enu:ensdndo} Aus dem transienten \clsmo\ aus
Schritt~\ref{enu:ensdmo} wird ein transientes \clsdo\ erzeugt:\\
\stfn{(fill-description)}\\
Das erzeugte transiente \clsdo\ enth\"{a}lt alle Informationen zur
Beschreibung der durch das transiente \clsmo\ re\-pr\"{a}\-sen\-tier\-ten
Klasse, insbesondere auch die \sltdo[e]\ der direkten und effektiven
\Slt[s].
%
\item Es wird gepr\"{u}ft, ob das \clsdo\ aus
Schritt~\ref{enu:ensddo} \"{a}quivalent zu dem \clsdo\ aus
Schritt~\ref{enu:ensdndo} ist:\\
\stfn{(class-description-equal-p)}
%
\item\label{enu:ensdupd} Sind die \clsdo[e]\ nicht \"{a}quivalent, wird
das neu erzeugte \clsdo\ aus Schritt~\ref{enu:ensdndo}
nicht-destruktiv im persistenten Speicher abgelegt, d.h.\ das
vorhandene alte \clsdo\ bleibt unver\"{a}ndert erhalten. In der
persistenten Klassentabelle wird die unter dem Namen der Klasse
eingetragene Referenz auf das neue \clsdo\ gesetzt:\\
\stfn{(update-class)}\\
Dabei wird auch die in Schritt~\ref{enu:ensdmp} abgefragte
Statusvariable einer eventuellen Klassen\"{a}nderung auf
\lispnil\ gesetzt.
Jetzt ist das \clsdo\ aus Schritt~\ref{enu:ensdndo} das aktuelle
\clsdo.
%
\end{block}%
%
\end{block}%
%
\item sonst exisitiert ein persistentes \clsdo, aber kein transientes
\clsmo. Dies kann beim Laden eines persistenten Objektes passieren,
dessen Klassendefinition nicht anderweitig evaluiert oder geladen
wurde:
%
\begin{block}%
%
\item Erzeuge die Klassendefinition aus dem \clsdo\ und evaluiere sie:\\
\stfn{(compile-description)}
%
\end{block}%
\end{block}%
%
\item sonst existiert noch kein \clsdo\ der Klasse:
%
\begin{block}%
%
\item Aus dem transienten \clsmo\ von
Schritt~\ref{enu:ensdmo} wird ein transientes \clsdo\ erzeugt:\\
\stfn{(fill-description)}\\
Das erzeugte transiente \clsdo\ enth\"{a}lt alle Informationen zur
Beschreibung der durch das transiente \clsmo\ repr\"{a}sentierten Klasse,
insbesondere auch die \sltdo[e]\ der direkten und effektiven \Slt[s].
%
\item\label{enu:ensdsdo} Speichere das erzeugte \clsdo:\\
\stfn{(update-class)}
%
\end{block}%
%
\item Jetzt ist sichergestellt, da\ss{} die Klassendefinition im
transienten System und das dazugeh\"{o}rige aktuelle \clsdo\ zueinander
passen. Als Ergebnis werden zwei Werte zur\"{u}ckgegeben:
\begin{itemize}
%
\item Das aktuelle transiente \clsdo.
%
\item Eine Statusinformation, die angibt, ob in
Schritt~\ref{enu:ensdupd} oder \ref{enu:ensdsdo} ein (neues)
\clsdo\ gespeichert wurde.
%
\end{itemize}
%
\end{infol}
%
F\"{u}r Strukturklassen gibt es ein \"{a}hn\-li\-ches Protokoll
\stfn{\ensstdescr}, das sich vom Protokoll
\stfn{\enscldescr}\ bez\"{u}glich der hier gew\"{a}hlten Abstraktionsebene
nur wenig unterscheidet.
%
\par{}Die Struktur- bzw.\ Klassen-Beschreibung einer neu erzeugten
persistenten \representation{} verweist immer auf das im Moment der
Erzeugung aktuelle \clsdo. In Schritt~\ref{enu:ensdupd} wird bei
\"{A}nderung der Klassendefinition eine neue \gande{Version einer
Klassenbeschreibung}{type version}\ \cite[\citepage{401}]{bib:sk87}
mit neuer Identit\"{a}t erzeugt; beim n\"{a}chsten Zugriff auf die
persistente \representation{} wird durch Vergleich des durch sie
referenzierten \clsdo[es]\ mit dem aktuellen
\clsdo\ gegebenenfalls eine Anpassung der 
\representation{} und eine \"{A}nderung der Referenz auf das
\clsdo\ veranla\ss{}t.
%
\par{}Die persistente Instanz \oiii\ in \figurename~\ref{fig:plsche} %
%
\begin{figure}[htbp]%
\ifbuch%
\centerline{\psfig{figure=twelvept/plsche.eps}}%
\else%
\centerline{\psfig{figure=tenpt/plsche.eps}}%
\fi%
\caption{Schemaentwicklung f\"{u}r eine persistente Instanz}%
\label{fig:plsche}%
\end{figure}%
%
wurde erzeugt, als das \clsdo\ mit der Version 1.00 aktuell war
(Referenz \oi); bei der n\"{a}chsten Dereferenzierung der persistenten
Instanz \oiii\ wird die Referenz \oi\ auf das
aktuelle \clsdo\ \oii\ ge\-\"{a}n\-dert.
%
\subsubsection{Speichern eines transienten Objektes}
%
Das Speichern eines transienten Objektes beinhaltet die \"{U}bertragung
des Objektzustands in eine persistente \representation. In das
Protokoll der \"{U}bertragung flossen folgende \"{U}berlegungen ein:
\begin{enumerate}
%
\item\label{enu:sobjcmod} Die persistente \representation{} soll den
transienten Zustand angemessen widerspiegeln. Daraus folgt, da\ss{}
sowohl die im transienten Objekt enthaltenen Referenzen als auch die
Zust\"{a}nde der referenzierten Objekte ebenfalls mit abgelegt werden
m\"{u}ssen.
%
\item\label{enu:sobjceff} Die \"{U}bertragung soll effizient sein; da
ein Objekt transitiv sehr viele Instanzen referenzieren kann, sollte
die M\"{o}glichkeit eines Schnitts bestehen. 
%
\item\label{enu:sobjcid} Die Identit\"{a}t der Objekte soll
ber\"{u}cksichtigt werden; sofern f\"{u}r den Zustand eines transienten
Objektes eine entsprechende persistente \representation{} existiert,
soll eine erneute \"{U}bertragung des Zustands immer in dieselbe
persistente \representation{} stattfinden.
%
\end{enumerate}
%
Die Eigenschaften von Punkt~\ref{enu:sobjcmod} und \ref{enu:sobjceff}
widersprechen sich; je nach Verwendungszweck des persistenten Systems
kann die mehr gew\"{u}nschte Eigenschaft durch Deklarationen oder
Parameter hervorgehoben werden.
%
\begin{infol}%
%
\acaption%
 {Protokoll f\"{u}r das Speichern eines transienten Objektes}%
 [(store-object)]%
 \label{pro:sobj}%
%
\item Es wird gepr\"{u}ft, ob die \sh-Datei\ ge\"{o}ffnet ist:\\
\stfn{(assert-sh-open-p)}
%
\item\label{enu:sotop} Mit dem Aufruf der \gfn[n]\\
\stfn{(t-object-to-p-objid)}\\
wird das transiente Objekt gespeichert:
%
\begin{block}%
%
\item Ist das zu speichernde Objekt ein \immval, werden als Ergebnis
von Schritt~\ref{enu:sotop} zwei Werte zur\"{u}ckgegeben:
\begin{itemize}
%
\item Der erste R\"{u}ckgabewert ist die \sobjid\/ des gespeicherten
Objektes; f\"{u}r \immval[s]\/ ist dies der Zustand des Objektes.
%
\item Der zweite R\"{u}ckgabewert ist die Typkennung des
\immval[s]\/ nach \tablename~\ref{tab:typetags}
(\citepage{\pageref{tab:typetags}}).
%
\end{itemize}
%
\item\label{enu:sobjed} F\"{u}r die Instanz einer Struktur-
bzw.\ \clos-Klasse wird das aktuelle transiente \clsdo\ wie im
Protokoll~\ref{pro:escl}
\stfn{\enscldescr}\ (\citepage{\pageref{pro:escl}}) angegeben
ermittelt:\\
\stfn{\ensstdescr}\ bzw.\ \stfn{\enscldescr}
%
\item\label{enu:sobjid} Die \sobjid\/ des Objektes wird ermittelt:
%
\begin{block}%
%
\item Das transiente Objekt wird im \cache\/ gesucht:\\
\stfn{(is-registered-object)}\\
Bei erfolgreicher Suche wird als Ergebnis von Schritt~\ref{enu:sobjid}
die zum Objekt geh\"{o}rige \sobjid\/ der persistenten Repr\"{a}sentation
zur\"{u}ckgegeben.
\par{}Der \cache\/ stellt somit sicher, da\ss{} der Zustand eines
transienten Objektes immer in dieselbe persistente
\representation{} \"{u}bertragen wird.
%
\item Wurde das Objekt nicht im \cache\/ gefunden, wird ein
persistentes Objekt mit einer der transienten Klasse entsprechenden
Klasse im persistenten Speicher alloziert:\\
\stfn{(p-allocate-%
{\usefont{OT1}{\rmdefault}{m}{n}\it\lt{}Klassenname\gt\/})}\\
Die erhaltene \sobjid\/ und das transiente Objekt werden in den
\cache\/ eingetragen:\\
\stfn{(register-to-cache)}\\
Die \sobjid\/ wird als Ergebnis von Schritt~\ref{enu:sobjid}
zur\"{u}ckgegeben.
%
\end{block}%
%
\item\label{enu:sotrbeg} Sofern noch keine Transaktion aktiv ist, wird
eine Transaktion gestartet:\\
\stfn{(with-transaction)}\\
Eine hier gestartete Transaktion wird von rekursiven Aufrufen des
Protokolls~\ref{pro:sobj} \stfn{(store-object)}\ mitbenutzt.
%
\item\label{enu:sowrl} F\"{u}r das durch die \sobjid\/ identifizierte
persistente Objekt wird eine Schreibsperre angefordert:\\
\stfn{(with-write-lock)}\\
Dabei wird festgestellt, ob auf das durch die \sobjid\/ referenzierte
persistente Objekt bereits eine Schreibsperre innerhalb der
aktiven Transaktion gesetzt wurde. Wenn ja, wurde der Zustand des
transienten Objektes in der aktiven Transaktion bereits in die
persistente \representation{} \"{u}ber\-tra\-gen; in diesem Fall verzweigt
das Protokoll zu Schritt~\ref{enu:sotrend}.
%
\item Der Zustand des transienten Objektes wird in die persistente
\representation{} \"{u}bertragen:
%
\begin{block}%
%
\item F\"{u}r Instanzen von Basisklassen siehe das
Subprotokoll~\ref{pro:sbi} f\"{u}r das Speichern einer transienten
Instanz einer Basisklasse (\citepage{\pageref{pro:sbi}}).
%
\item F\"{u}r Strukturobjekte siehe das Subprotokoll~\ref{pro:sstr}
\stfn{(setf p-structure)}\ (\citepage{\pageref{pro:sstr}}).
%
\item F\"{u}r \clos-Instanzen siehe das Subprotokoll~\ref{pro:scli}
\stfn{(setf p-instance)}\ (\citepage{\pageref{pro:scli}}).
%
\end{block}%
%
\item\label{enu:sotrend} Wenn in Schritt~\ref{enu:sotrbeg} eine
Transaktion gestartet wurde, wird sie in diesem Schritt beendet. Damit
werden alle innerhalb der Transaktion gesetzten Sperren
(insbesondere die Schreibsperre aus Schritt~\ref{enu:sowrl})
aufgehoben.
%
\item Als Ergebnis von Schritt~\ref{enu:sotop} werden zwei Werte
zur\"{u}ckgegeben:
\begin{itemize}
%
\item Der erste R\"{u}ckgabewert ist die in Schritt~\ref{enu:sobjid}
ermittelte \sobjid\/ des gespeicherten Objektes.
%
\item Der zweite R\"{u}ckgabewert ist die Typkennung, die den ersten
R\"{u}ckgabewert als \sobjid\/ eines persistentes Objektes typisiert
(\tablename~\ref{tab:typetags}, \citepage{\pageref{tab:typetags}},
Kennung~0).
%
\end{itemize}%
%
\end{block}%
%
\item Wenn aus dem zweiten R\"{u}ckgabewert von Schritt~\ref{enu:sotop}
hervorgeht, da\ss{} ein \immval\/ gespeichert wurde
(\tablename~\ref{tab:typetags}, \citepage{\pageref{tab:typetags}},
Kennung\,$\neq$\,0), wird als Ergebnis des Protokolls der 
\immval\/ selbst zur\"{u}ckgegeben; andernfalls wird die \sobjid\/ des
persistenten Objektes innerhalb einer transienten Instanz der
Strukturklasse \class{persistent-object}\ zur\"{u}ckgegeben, damit sich
die Klasse der zur\"{u}ckgegebenen \sobjid\/ vom intern zur
\representation{} von \sobjid[s]\ verwendeten Typ
\class{fixnum}\ unterscheidet.
%
\end{infol}%
%
\begin{infol}%
%
\acaption%
 {Subprotokoll f\"{u}r das Speichern einer transienten Instanz einer
  Basisklasse}%
 \label{pro:sbi}%
\nopagebreak[4]%
%
\item F\"{u}r Instanzen von Basisklassen, die nicht Werte sind, wird f\"{u}r
jeden \Slt\/ des Objektes die \gfn\\
\stfn{(t-object-to-p-objid)}\\
aufgerufen, d.h.\ das Protokoll~\ref{pro:sobj}
\stfn{(store-object)}\ (\citepage{\pageref{pro:sobj}}) wird rekursiv
mit Schritt~\ref{enu:sotop} fortgef\"{u}hrt. Die aus dem rekursiven
Aufruf erhaltene \sobjid\/ und die Typkennung des \Slt\/-Zustands wird
in die zum \Slt\/ geh\"{o}rige Komponente des Referenzfeldes
(\figurename~\ref{fig:povec}, \citepage{\pageref{fig:povec}}) des
persistenten Objektes eingetragen.
%
\item F\"{u}r transiente Werte wird ihr Zustand mit der Funktion\\
\stfn{(setf p-values)}\\
in ihrer bin\"{a}ren \representation{} in das Wertefeld des persistenten
Objektes (\figurename~\ref{fig:povec}, \citepage{\pageref{fig:povec}})
kopiert.
%
\end{infol}
%
\par{}F\"{u}r Symbole kann durch einen dem Protokoll~\ref{pro:sobj}
mitgegebenen Parameter die Anzahl der \Slt[s], deren Zustand vom
transienten in das persistente Symbol \"{u}bertragen werden soll,
eingeschr\"{a}nkt werden; so wird f\"{u}r die Angabe
\lisp{:name}\ lediglich der Name des Symbols gespeichert. Beim
Speichern des Wertes eines Symbols wird immer der dynamisch gebundene
Wert benutzt, da zum einen der lexikalisch gebundene Wert nicht
einfach ermittelt werden kann \stcite{119}\ und zum anderen in diesem
Fall die (unter Umst\"{a}nden sehr umfangreiche) lexikalische Umgebung
des Symbols mit gespeichert werden m\"{u}\ss{}te.
%
\begin{infol}%
%
\acaption%
 {Subprotokoll f\"{u}r das Speichern eines transienten
  Strukturobjektes}%
 [(setf p-structure)]%
 \label{pro:sstr}%
%
\item Aus dem in Schritt~\ref{enu:sobjed} von Protokoll~\ref{pro:sobj}
\stfn{(store-object)} (\citepage{\pageref{enu:sobjed}}) erhaltenem
aktuellem transientem \clsdo\ wird die Liste der \sltdo[e]\ gelesen:\\
\stfn{(class-effective-slots)}
%
\item F\"{u}r jeden \Slt, dessen Zustands-Lebensdauer nicht als
\lisp{:transient}\ deklariert wurde:
%
\begin{block}
%
\item Rufe f\"{u}r den \Slt\/-Zustand die \gfn\\
\stfn{(t-object-to-p-objid)}\\
auf, d.h.\ das Protokoll~\ref{pro:sobj}
\stfn{(store-object)}\ (\citepage{\pageref{pro:sobj}}) wird ab
Schritt~\ref{enu:sotop} aufgerufen. Die daraus erhaltene \sobjid\/ und
die Typkennung des \Slt\/-Zustands wird in die zum \Slt\/ geh\"{o}rige
Komponente des Referenzfeldes (\figurename~\ref{fig:pivec},
\citepage{\pageref{fig:pivec}}) des persistenten Strukturobjektes
eingetragen.
%
\end{block}
%
\end{infol}
%
Die Iteration \"{u}ber alle \Slt[s]\/ sorgt daf\"{u}r, da\ss{} alle Referenzen
und referenzierten Objekte mit im persistenten Speicher abgelegt
werden; die Deklaration eines transienten \Slt\/-Zustands
er\-m\"{o}g\-licht einen Schnitt, ab dem Objekte nicht weiter gespeichert
werden.
%
\par{}Ein transientes Paket wird als Instanz der Basis-Strukturklasse
\class{persistent-package}\ abgespeichert. Die im transienten Paket
enthaltene Liste der ihm zugeordneten transienten Symbole wird beim
Speichern eines Paketes nicht mit abgelegt; statt dessen werden
persistente Symbole im Moment des Speicherns ihrem persistenten Paket
zugeordnet.
%
\begin{infol}%
%
\acaption%
 {Subprotokoll f\"{u}r das Speichern einer transienten
  \protect\clos-Instanz}%
 [(setf p-instance)]%
 \label{pro:scli}%
%
\item Aus dem in Schritt~\ref{enu:sobjed} von Protokoll~\ref{pro:sobj}
\stfn{(store-object)} (\citepage{\pageref{enu:sobjed}}) erhaltenem
aktuellem transientem \clsdo\ wird die Liste der \sltdo[e]\ gelesen:\\
\stfn{(class-effective-slots)}
%
\item F\"{u}r jeden \Slt, dessen Zustands-Lebensdauer als
\lisp{:cached}\ deklariert wurde:
%
\begin{block}
%
\item Es wird gepr\"{u}ft, ob der \Slt\/ an einen Wert gebunden ist:\\
\stfn{(slot-boundp)}
%
\item Wenn der \Slt\/ ungebunden ist, wird in die zum \Slt\/ geh\"{o}rige
Komponente des durch das Referenzfeld referenzierten Zustandsvektors
(\figurename~\ref{fig:pcivec},
\citepage{\pageref{fig:pcivec}}) der persistenten \clos-Instanz
die \sobjid\/ der Markierung \rglq{}ungebundener \Slt\rgrq\ und die
Typkennung Markierung (\tablename~\ref{tab:typetags},
\citepage{\pageref{tab:typetags}}, Kennung~4) eingetragen:\\
\stfn{(setf p-index)}
%
\item Wenn der \Slt\/ an einen Wert gebunden ist, rufe f\"{u}r den
\Slt\/-Zustand die \gfn\\ 
\stfn{(t-object-to-p-objid)}\\
auf, d.h.\ das Protokoll~\ref{pro:sobj}
\stfn{(store-object)}\ (\citepage{\pageref{pro:sobj}}) wird ab
Schritt~\ref{enu:sotop} aufgerufen. Die daraus erhaltene \sobjid\/ und
Typkennung des \Slt\/-Zustands wird in die zum \Slt\/ geh\"{o}rige
Komponente des durch das Referenzfeld referenzierten Zustandsvektors
(\figurename~\ref{fig:pcivec}, \citepage{\pageref{fig:pcivec}}) der
persistenten \clos-Instanz eingetragen:\\
\stfn{(setf p-index)}
%
\end{block}%
%
\end{infol}%
%
Um \mo[e]\ mit einer fest definierten, sy\-stem\-un\-ab\-h\"{a}n\-gi\-gen
Struktur zu speichern, werden alle \mo[e]\ als \deo[e]\ gespeichert;
die Abbildung einer \mc\ auf seine \dec\ ergibt sich nach
\tablename~\ref{tab:mopdescr} (\citepage{\pageref{tab:mopdescr}}).
%
\begin{figure}[htbp]%
\ifbuch%
\centerline{\psfig{figure=twelvept/extent.eps}}%
\else%
\centerline{\psfig{figure=tenpt/extent.eps}}%
\fi%
\caption{Lebensdauer von \protect\Slt\/-Zust\"{a}nden}%
\label{fig:extent}%
\end{figure}%
%
\par{}\figurename~\ref{fig:extent} zeigt den Zusammenhang zwischen der
Deklaration einer Lebensdauer f\"{u}r einen \Slt\/-Zustand einer
\clos-Instanz (mittlere Spalte in \figurename~\ref{fig:extent}) und der
f\"{u}r die jeweilige Lebensdauer allozierten \representation{} des
\Slt\/-Zustands im transienten (2.~Spalte) und persistenten
(3.~Spalte) Speicher, die Zeitpunkte, zu denen die Zust\"{a}nde zwischen
der transienten und persistenten \representation{} ausgetauscht werden,
sowie die Verf\"{u}gbarkeit der Deklaration f\"{u}r \Slt[s]\/ von Struktur-
(2.~Spalte von rechts) und \clos- (rechte Spalte) Klassen.  Die
Zust\"{a}nde der als \lisp{:cached-\lb{}write-\lb{}through}\ und
\lisp{:persistent}\ deklarierten \Slt[s]\/ werden durch spezialisierte
Methoden der \gfn[n]\ \stfn{(setf
slot-\lb{}value-\lb{}using-\lb{}class)}\ (Protokoll~\ref{pro:stsl},
\citepage{\pageref{pro:stsl}}) gespeichert und stehen damit nur
aktiv-persistenten Instanzen zur Verf\"{u}gung.
%
%\subsubsection{Nicht speicherbare Objekte}
%
\subsubsection{Laden eines persistenten Objektes}
%
Die Eigenschaften der \"{U}bertragung eines transienten Objektzustands in
eine persistente Re\-pr\"{a}\-sen\-ta\-tion gelten sinngem\"{a}\ss{} auch f\"{u}r
das Laden eines persistenten Objektes.
%
\begin{infol}%
%
\acaption%
 {Protokoll f\"{u}r das Laden eines transienten Objektes}%
 [(load-object)]%
 \label{pro:lobj}%
%
\item Es wird gepr\"{u}ft, ob die \sh-Datei\ ge\"{o}ffnet ist:\\
\stfn{(assert-sh-open-p)}
%
\item\label{enu:lotyp} Die Typkennung des durch die
\sobjid\/ referenzierten persistenten Objektes wird gelesen:\\
\stfn{(p-type-tag-of)}
%
\item\label{enu:lotop} Mit dem Aufruf der \gfn[n]\\
\stfn{(p-objid-to-t-object)}\\
wird die transiente \representation{} eines persistenten Objektes
geladen; beim Aufruf werden die \sobjid\/ und die Typkennung des
persistenten Objektes, die in Schritt~\ref{enu:lotyp} ermittelt
wurde, mitgegeben:
%
\begin{block}%
%
\item Wenn aus der Typkennung hervorgeht, da\ss{} ein \immval\/ geladen
wird (\tablename~\ref{tab:typetags}, \citepage{\pageref{tab:typetags}},
Kennung\,$\neq$\,0), wird als Ergebnis von Schritt~\ref{enu:lotop} ein
transienter \immval\/ mit dem in der \sobjid\/ abgelegten Zustand
zur\"{u}ckgegeben.
%
\item\label{enu:lobjed} F\"{u}r die Instanz einer Struktur-
bzw.\ \clos-Klasse wird das aktuelle transiente \clsdo\ wie im
Protokoll~\ref{pro:escl}
\stfn{\enscldescr}\ (\citepage{\pageref{pro:escl}}) angegeben
ermittelt:\\
\stfn{\ensstdescr}\ bzw.\ \stfn{\enscldescr}
%
\item\label{enu:lotrr} Die transiente \representation{} des
persistenten Objektes wird ermittelt:
%
\begin{block}%
%
\item Das transiente Objekt wird im \cache\/ gesucht:\\
\stfn{(is-registered-objid)}\\
Bei erfolgreicher Suche wird als Ergebnis von Schritt~\ref{enu:lotrr}
die zum Objekt geh\"{o}rige transiente \representation{} zur\"{u}ckgegeben.
\par{}Der \cache\/ stellt somit sicher, da\ss{} der Zustand eines
persistenten Objektes immer in dieselbe transiente Repr\"{a}sentation
\"{u}bertragen wird.
%
\item Wurde das Objekt nicht im \cache\/ gefunden, wird
klassenabh\"{a}ngig ein transientes Objekt erzeugt:\\
\stfn{(make-%
{\usefont{OT1}{\rmdefault}{m}{n}\it\lt{}Klassenname\gt\/})}\\
F\"{u}r Instanzen von Struktur- und \clos\ Klassen wird gegebenenfalls die
f\"{u}r die Klasse deklarierte Konstruktor-Funktion aufgerufen.
Das erzeugte transiente Objekt und seine \sobjid\/ werden in den
\cache\/ eingetragen:\\
\stfn{(register-to-cache)}\\
Die transiente \representation{} wird als Ergebnis von
Schritt~\ref{enu:lotrr} zur\"{u}ckgegeben.
%
\end{block}%
%
\item\label{enu:locut} Sofern in Schritt~\ref{enu:lotrr} eine
transiente \representation{} im \cache\/ gefunden wurde, bleibt der
Zustand der transienten \representation{} unver\"{a}ndert; das Protokoll
verzweigt zu Schritt \ref{enu:loend}.
\par{}Dieses Kriterium sorgt damit f\"{u}r den Schnitt beim Laden eines
Objektes.
%
\item\label{enu:lotrbeg} Sofern noch keine Transaktion aktiv ist, wird
eine Transaktion gestartet:\\
\stfn{(with-transaction)}\\
Eine hier gestartete Transaktion wird von rekursiven Aufrufen des
Protokolls~\ref{pro:lobj} \stfn{(load-object)}\ mitbenutzt.
%
\item\label{enu:lowrl} F\"{u}r das durch die \sobjid\/ identifizierte
persistente Objekt wird eine Lesesperre angefordert:\\
\stfn{(with-read-lock)}\\
Dabei wird festgestellt, ob auf das durch die \sobjid\/ referenzierte
persistente Objekt bereits eine Lesesperre innerhalb der
aktiven Transaktion gesetzt wurde. Wenn ja, wurde der Zustand des
persistenten Objektes in der aktiven Transaktion bereits in seine
transiente \representation{} \"{u}ber\-tra\-gen; in diesem Fall verzweigt
das Protokoll zu Schritt~\ref{enu:lotrend}.
%
\item Der Zustand des persistenten Objektes wird in die transiente
\representation{} \"{u}bertragen:
%
\begin{block}%
%
\item F\"{u}r Instanzen von Basisklassen siehe das
Subprotokoll~\ref{pro:lbi} f\"{u}r das Laden einer transienten
Instanz einer Basisklasse (\citepage{\pageref{pro:lbi}}).
%
\item F\"{u}r Strukturobjekte siehe das Subprotokoll~\ref{pro:lstr}
\stfn{(p-structure)}\ (\citepage{\pageref{pro:lstr}}).
%
\item F\"{u}r \clos-Instanzen siehe das Subprotokoll~\ref{pro:lcli}
\stfn{(p-instance)}\ (\citepage{\pageref{pro:lcli}}).
%
\end{block}%
%
\item\label{enu:lotrend} Wenn in Schritt~\ref{enu:lotrbeg} eine
Transaktion gestartet wurde, wird sie in diesem Schritt beendet. Damit
werden alle innerhalb der Transaktion gesetzten Sperren
(insbesondere die Lesesperre aus Schritt~\ref{enu:lowrl})
aufgehoben.
%
\item\label{enu:loend} Als Ergebnis wird die in
Schritt~\ref{enu:lotrr} ermittelte transiente \representation{} des
persistenten Objektes zur\"{u}ckgegeben.
%
\end{block}%
\end{infol}%
%
Durch die Verzweigung in Schritt~\ref{enu:locut} existiert im Moment
keine M\"{o}glichkeit, den Zustand eines Objektes erneut zu laden; dies
kann in einer zuk\"{u}nftigen Programmversion realisiert werden. N\"{o}tig
wird das erneute Laden des Zustands von persistenten Objekten
nach \"{A}nderung ihrer persistenten \representation{} durch parallelen
Zugriff; eine \"{A}nderung kann \"{u}ber die in Abschnitt~\ref{sec:plcon}
(\citepage{\pageref{sec:plcon}}) erl\"{a}uterten
Benachrichtigungssperren erkannt werden.
%
\begin{infol}%
%
\acaption{Subprotokoll f\"{u}r das Laden einer transienten
Instanz einer Basisklasse}\label{pro:lbi}
%
\item F\"{u}r Instanzen von Basisklassen, die nicht Werte sind,
wird f\"{u}r jeden \Slt\/ des Objektes die \sobjid\/ und die Typkennung des
im entsprechenden \Slt\/ in der persistenten Re\-pr\"{a}\-sen\-ta\-tion
enthaltenen persistenten Objektes gelesen; mit diesen Parametern wird
die \gfn\\
\stfn{(p-objid-to-t-object)}\\
aufgerufen, d.h.\ das Protokoll~\ref{pro:lobj}
\stfn{(load-object)}\ (\citepage{\pageref{pro:lobj}}) wird rekursiv 
mit Schritt~\ref{enu:lotop} fortgef\"{u}hrt. Die aus dem rekursiven
Aufruf erhaltene transiente \representation{} des \Slt\/-""Zu\-stands
wird in den \Slt\/ der transienten \representation{} eingetragen.
%
\item F\"{u}r Werte wird ihr Zustand mit der Funktion\\
\stfn{(p-values)}\\
in ihrer bin\"{a}ren \representation{} aus dem Wertefeld des
persistenten Objektes (\figurename~\ref{fig:povec},
\citepage{\pageref{fig:povec}}) in die transiente Repr\"{a}sentation
kopiert.
%
\end{infol}
%
\par{}Beim Laden von Symbolen wird der Zustand nur dann vom
persistenten in das transiente Symbol kopiert, wenn das transiente
Symbol durch den Ladevorgang selbst erzeugt wurde; damit wird eine
eventuell von den BenutzerInnen unerwartete \"{A}nderung des
Zustands des transienten Symbols verhindert.
%
\par{}Ein transientes Paket wird genau dann erzeugt, wenn ein
persistentes Symbol geladen wird, das sich in einem persistenten Paket
befindet, dessen Name kein transientes Paket identifiziert.
Vorhandene transiente Pakete werden \rglq{}mitbenutzt\rgrq; insbesondere
wird f\"{u}r sie niemals die Funktion
\stfn{make-package}\ \stcite{262}\ aufgerufen, da die Semantik
f\"{u}r ein bereits exisitierendes transientes Paket nicht spezifiziert
ist.
%
\begin{infol}%
%
\acaption%
 {Subprotokoll f\"{u}r das Laden eines transienten Strukturobjektes}%
 [(p-structure)]%
 \label{pro:lstr}%
\nopagebreak[4]%
%
\item\label{enu:lstrcld} Das referenzierte \clsdo\ wird aus dem Feld
mit der Strukturbeschreibung (\figurename~\ref{fig:pivec},
\citepage{\pageref{fig:pivec}}) des Strukturobjektes wie im
Protokoll~\ref{pro:lobj}
\stfn{(load-object)}\ (\citepage{\pageref{pro:lobj}}) angegeben
geladen.
%
\item\label{enu:lstrse} Wenn das in Schritt~\ref{enu:lobjed} von
Protokoll~\ref{pro:lobj}
\stfn{(load-object)}\ (\citepage{\pageref{enu:lobjed}}) erhaltene
aktuelle transiente \clsdo\ und das in Schritt~\ref{enu:lstrcld}
erhaltene referenzierte transiente \clsdo\ nicht identisch sind, wird
die persistente \representation{} des Strukturobjektes in die durch das
aktuelle transiente \clsdo\ von Schritt~\ref{enu:lobjed} von
Protokoll~\ref{pro:lobj} \stfn{(load-object)}\
(\citepage{\pageref{enu:lobjed}}) vorgegebene Repr\"{a}sentation
transformiert, sofern sich die Anzahl der \Slt[s]\/ nicht erh\"{o}ht hat:
%
\begin{block}
%
\item Die \Slt\/-Zust\"{a}nde werden in der persistenten Repr\"{a}sentation
von ihrer bisherigen Position an die Position des gleichnamigen
\Slt[s]\/ der aktuellen transienten Strukturbeschreibung verschoben.
%
\item\label{enu:lstrupd} Die Strukturbeschreibung des persistenten
Strukturobjektes wird auf das in Schritt~\ref{enu:lobjed} von
Protokoll~\ref{pro:lobj}
\stfn{(load-object)}\ (\citepage{\pageref{enu:lobjed}}) ermittelte
aktuelle transiente \clsdo\ ge\-\"{a}n\-dert, d.h.\ in das Feld mit der
Strukturbeschreibung wird die \objid\ der zum aktuellen transienten
\clsdo\ geh\"{o}rigen persistenten \representation{} eingetragen.
%
\end{block}%
%
\item Wenn in Schritt~\ref{enu:lstrupd} die Strukturbeschreibung des
Strukturobjektes ge\-\"{a}n\-dert wurde, wird die Liste der \sltdo[e]\ aus
dem in Schritt~\ref{enu:lobjed} von Protokoll~\ref{pro:lobj}
\stfn{(load-object)}\ (\citepage{\pageref{enu:lobjed}}) erhaltenem
aktuellem transientem \clsdo\ gelesen; ansonsten wird die Liste der
\sltdo[e]\ aus dem in Schritt~\ref{enu:lstrcld} erhaltenem
referenziertem transientem \clsdo\ gelesen:\\ 
\stfn{(class-effective-slots)}
%
\item F\"{u}r jeden \Slt, dessen Zustands-Lebensdauer nicht als
\lisp{:transient}\ deklariert wurde:
%
\begin{block}
%
\item Rufe f\"{u}r die im \Slt\/ enthaltene \sobjid\/ die \gfn\\
\stfn{(p-objid-to-t-object)}\\
auf, d.h.\ das Protokoll~\ref{pro:lobj}
\stfn{(load-object)}\ (\citepage{\pageref{pro:lobj}}) wird mit
Schritt~\ref{enu:lotop} aufgerufen. Die daraus erhaltene transiente
\representation{} des \Slt\/-Zustands wird in den \Slt\/ des transienten
Strukturobjektes eingetragen.
%
\end{block}
%
\end{infol}
%
%
\begin{infol}%
%
\acaption%
 {Subprotokoll f\"{u}r das Laden einer transienten
  \protect\clos-Instanz}%
 [(p-instance)]%
 \label{pro:lcli}%
%
\item\label{enu:lclcld} Das referenzierte \clsdo\ wird aus dem Feld
mit der Klassenbeschreibung (\figurename~\ref{fig:pcivec},
\citepage{\pageref{fig:pcivec}}) der \clos-Instanz wie im
Protokoll~\ref{pro:lobj}
\stfn{(load-object)}\ (\citepage{\pageref{pro:lobj}}) angegeben
geladen.
%
\item\label{enu:lclse} Wenn das in Schritt~\ref{enu:lobjed} von
Protokoll~\ref{pro:lobj}
\stfn{(load-object)}\ (\citepage{\pageref{enu:lobjed}}) erhaltene
aktuelle transiente \clsdo\ und das in Schritt~\ref{enu:lclcld}
erhaltene referenzierte transiente \clsdo\ nicht identisch sind:
%
\begin{block}%
%
\item Der Zustandsvektor der persistenten \clos-Instanz wird in
die durch das aktuelle transiente \clsdo\ von Schritt~\ref{enu:lobjed}
von Protokoll~\ref{pro:lobj}
\stfn{(load-object)}\ (\citepage{\pageref{enu:lobjed}}) vorgegebene
\representation{} transformiert. F\"{u}r die Transformation werden die
Zust\"{a}nde gleichnamiger \Slt[s]\/ aus dem alten in einen neu
allozierten persistenten Zustandsvektor kopiert.
%
\item Der neu aufgebaute persistente Zustandsvektor wird in die
persistente \clos-Instanz eingetragen.
%
\item Die Klassenbeschreibung der persistenten \clos-Instanz wird auf
das in Schritt~\ref{enu:lobjed} von Protokoll~\ref{pro:lobj}
\stfn{(load-object)}\ (\citepage{\pageref{enu:lobjed}}) ermittelte
aktuelle transiente \clsdo\ ge\-\"{a}n\-dert, d.h.\ in das Feld mit der
Klassenbeschreibung wird die \objid\ der zum aktuellen transienten
\clsdo\ geh\"{o}rigen persistenten \representation{} eingetragen. 
%
\end{block}%
%
Jetzt ist sichergestellt, da\ss{} das \clsdo\ der \clos-Instanz
und die Klassendefinition im transienten System zueinander passen.
%
\item Aus dem in Schritt~\ref{enu:lobjed} von Protokoll~\ref{pro:lobj}
\stfn{(load-object)} (\citepage{\pageref{enu:lobjed}}) erhaltenem
aktuellem transientem \clsdo\ wird die Liste der \sltdo[e]\ gelesen:\\ 
\stfn{(class-effective-slots)}
%
\item F\"{u}r jeden \Slt, dessen Zustands-Lebensdauer als
\lisp{:cached}\ deklariert wurde:
%
\begin{block}
%
\item\label{enu:lclidx} Aus der zum \Slt\/ geh\"{o}rigen Komponente des
durch das Referenzfeld referenzierten Zustandsvektors
(\figurename~\ref{fig:pcivec}, \citepage{\pageref{fig:pcivec}}) der
persistenten \clos-Instanz werden die \sobjid\/ und die Typkennung des
vom \Slt\/ referenzierten persistenten Objektes gelesen:\\
\stfn{(p-index)}
%
\item Referenziert die in Schritt~\ref{enu:lclidx} erhaltene
\sobjid\/ in Verbindung mit der Typkennung das Markierungsobjekt
\rglq{}ungebundener \Slt\rgrq, ist der persistente \Slt\/ ungebunden;
dem\-ent\-spre\-chend wird die Bindung des \Slt[s]\/ in der
transienten \representation{} aufgehoben:\\
\stfn{(slot-makunbound)}
%
\item Ist der persistente \Slt\/ gebunden, rufe mit der
der in Schritt~\ref{enu:lclidx} enthaltenen \sobjid\/ und
Typkennung die \gfn\\
\stfn{(p-objid-to-t-object)}\\
auf, d.h.\ das Protokoll~\ref{pro:lobj}
\stfn{(load-object)}\ (\citepage{\pageref{pro:lobj}}) wird mit
Schritt~\ref{enu:lotop} aufgerufen. Die daraus erhaltene transiente
\representation{} des \Slt\/-Zustands wird in den \Slt\/ der transienten
\clos-Instanz eingetragen.
%
\end{block}
%
\end{infol}
%
\par{}Der Zustand eines nur im transienten Speicher repr\"{a}sentierten
\Slt[s]\/ einer Instanz einer \clos\ Klasse
(\figurename~\ref{fig:extent}, \citepage{\pageref{fig:extent}},
\Slt\/-Lebensdauer \lisp{:transient}) wird, sofern die transiente
\clos\ \representation{} im Verlauf des Ladens erzeugt wurde, auf den
Zustand gesetzt, der sich aus seinem Initialisierungsausdruck ergibt.
F\"{u}r \Slt[s]\/ von Strukturobjekten wird analog dazu der f\"{u}r den
\Slt\/ in der Definition angegebene Initialisierungswert benutzt.
%
\subsubsection{Schemaentwicklung}
%
%\iffalse%
\iftrue%
Schritt~\ref{enu:lstrse} von Subprotokoll~\ref{pro:lstr}
bzw.\ \ref{pro:lcli} beschreibt
\else%
Schritt~\ref{enu:lstrse} von Subprotokoll~\ref{pro:lstr}
bzw.\ Schritt~\ref{enu:lclse} von Subprotokoll \ref{pro:lcli}
beschreiben
\fi%
die momentan realisierte Schemaentwicklung f\"{u}r Struktur-
bzw.\ \clos-Objekte. Sie besteht darin, die persistente
\representation{} eines
persistenten Objektes beim n\"{a}chsten Zugriff nach einer Neuablage
einer Klassenbeschreibung destruktiv auf die durch
die aktuelle Klassenbeschreibung gegebene \representationform{}
anzupassen (\gande{inkrementelle Konvertierung}{incremental
conversion}\ \cite[\citepage{454}]{bib:bj89}) und die Referenz auf das
\clsdo\ entsprechend zu \"{a}ndern (\figurename~\ref{fig:plsche},
\citepage{\pageref{fig:plsche}}). F\"{u}r persistente \clos-Instanzen
bleibt ihre Identit\"{a}t auch nach der \"{A}nderung immer erhalten; f\"{u}r
persistente Strukturobjekte ist eine Schemaentwicklung zur Zeit nur
dann m\"{o}glich, wenn sich die Anzahl der \Slt[s]\/ nicht erh\"{o}ht hat.
%
\par{}Es ist geplant, in einer sp\"{a}teren Programmversion mehrere Arten
f\"{u}r die Schemaentwicklung zur Verf\"{u}gung zu stellen; welche Art benutzt
werden soll, geht dann aus der f\"{u}r die Klasse deklarierten Art der
Schemaentwicklung hervor. Um eine Schemaentwicklung auch f\"{u}r verschieden
realisierte 3.~Schichten zu erhalten, soll ihre Verarbeitung
aus der 3.\ in die 2.~Schicht verlegt werden.
%
\subsubsection{Zugriff auf den Objektzustand}
%
\paragraph{Passiv-persistente Objekte}
%
Auf pas\-siv-per\-si\-sten\-te Instanzen kann entweder mit
\swizzling\ oder \"{u}ber die direkte Manipulation der persistenten
\representation{} zugegriffen werden.
%
\subparagraph{\protect\swizzling}%
%
Die transiente \representation{} eines passiv-persistenten Objektes wird
entweder explizit durch einen Aufruf einer entsprechenden Funktion
(Protokoll~\ref{pro:lobj} \stfn{(load-object)},
\citepage{\pageref{pro:lobj}}) geladen oder implizit erzeugt, wenn das
passiv-persistente Objekt von einer zu ladenden Instanz referenziert
wird. Auf die transiente \representation{} kann mit den \"{u}blichen
\cl\ Funktionen zugegriffen werden. Der Abgleich zwischen der
transienten und persistenten Re\-pr\"{a}\-sen\-ta\-tion eines
passiv-persistenten Objektes erfolgt entweder explizit durch Aufruf
einer entsprechenden Funktion (Protokoll~\ref{pro:sobj}
\stfn{(store-object)}, \citepage{\pageref{pro:sobj}}) oder implizit,
sofern es von einer zu sichernden Instanz referenziert wird.
%
\subparagraph{Direkte Manipulation der persistenten Repr\"{a}sentation}
%
Zur direkten Manipulation der persistenten \representations{} gibt es 
entsprechende Funktionen; analog zu \wood\ \cite{bib:wo93} wurde
als Name einer direkt auf einer persistenten Repr\"{a}sentation
arbeitenden Funktion eine Konkatenation des Pr\"{a}fixes
\rglq{}p-\rgrq\ mit dem in \cite{bib:st90} definierten Namen der auf
einer transienten \representation{} gleichen Typs arbeitenden Funktion
gebildet; beispielsweise gibt die Funktion \stfn{p-car}\ das erste
Element einer persistenten {\em cons\/}-Zelle zur\"{u}ck.
%
\par{}Auf die Zust\"{a}nde der \Slt[s]\/ von persistenten Struktur- und
\clos-Objekten kann direkt mit den Funktionen
\stfn{slot-\ldots}\footnote{Im einzelnen sind dies die Funktionen
\stfn{slot-value}, \stfn{(setf slot-value)},
\stfn{slot-boundp}\ und \stfn{slot-makunbound}.}\ zugegriffen werden.
Dazu wird an diese Funktionen anstatt einer transienten
\representation{} eines persistenten Struktur- oder \clos-Objektes
eine \sobjid\/ \"{u}ber\-ge\-ben; die \spc[n]\ \mtd[n]\ greifen in
diesem Fall immer direkt auf die persistente Re\-pr\"{a}\-sen\-ta\-tion
zu.
%
\paragraph{Aktiv-persistente Instanzen}
%
F\"{u}r Instanzen von aktiv-persistenten Klassen erfolgt der Zugriff auf
den Objektzustand \"{u}ber Methoden der
\gfn[n]en\ \stfn{slot-\ldots-using-class}, die auf die
\clsmc\ \class{persistent-metaclass}\ spezialisiert sind.
%
\par{}Die \spc[n]\ \mtd[n]\ der generischen Funktionen \stfn{(setf
slot-value-using-class)}\ und
\stfn{slot-\lb{}makunbound-\lb{}using-\lb{}class}\ pflegen den f\"{u}r
einen \Slt\/ deklarierten Index; in der Methode zu \stfn{(setf
slot-value-using-class)}\ werden persistente \clos-Instanzen zu ihrem
\Slt\/-Zustand assoziiert, in der Methode zu
\stfn{slot-makunbound-using-class}\ wird eine bestehende Assoziation
aufgehoben. F\"{u}r eine sp\"{a}tere Programmversion ist geplant, diese
Indexverwaltung in die 2.~Schicht zu verschieben.
%
\begin{infol}
%
\acaption%
 {{Protokoll f\"{u}r das Schreiben eines \protect\Slt\/-Zustands einer
   aktiv-persistenten \clos-Instanz}}%
 [(setf slot-value-using-class)]\label{pro:stsl}%
%
\item\label{enu:stsltr} Sofern noch keine Transaktion aktiv ist, wird
eine Transaktion gestartet:\\
\stfn{(with-transaction)}
%
\item Ermittle die \sobjid\/ der persistenten \clos-Instanz:\\
\stfn{(persistent-object-objid)}\\
Damit wird der von der Superklasse
\class{persistent-clos-object}\ geerbte
\Slt\/ mit der \sobjid\/ (Protokoll~\ref{pro:defclass}
\stfn{(defclass)}, Schritt~\ref{enu:dcinit},
\citepage{\pageref{enu:dcinit}}) der persistenten \representation{} der
\clos-Instanz gelesen. Ist die \sobjid\/ \lispnil, wird die
persistente \representation{} des Objektes alloziert:\\
\stfn{(p-allocate-instance)}
%
\item Das den \Slt\/ repr\"{a}sentierende \sltdo\ wird gesucht:\\
\stfn{(find-effective-slot-description)}\\
Im Verlauf der Auswertung von
\stfn{(find-effective-slot-description)}\ wird das
Protokoll~\ref{pro:escl}
\stfn{\enscldescr}\ (\citepage{\pageref{pro:escl}}) aufgerufen.
%
\item Wenn die Lebensdauer des \Slt\/-Zustands als
\lisp{:persistent}\ oder
\lisp{:cached-\lb{}write-\lb{}through}\ (\figurename~\ref{fig:extent},
\citepage{\pageref{fig:extent}}) deklariert wurde, speichere den
neuen \Slt\/-Zustand in der persistenten Repr\"{a}sentation:
%
\begin{block}
%
\item\label{enu:stslov} Wenn f\"{u}r den \Slt\/ ein Index deklariert
wurde, wird zun\"{a}chst der alte \Slt\/-Zustand wie in
Protokoll~\ref{pro:losl} \stfn{(slot-value-using-class)}\ %
(\citepage{\pageref{pro:losl}}) angegeben
gelesen:\\
\stfn{(slot-value-using-class)}
%
\item Aus der persistenten \representation{} des Objektes wird die
\sobjid\/ des Zustandsvektors (\figurename~\ref{fig:pcivec},
\citepage{\pageref{fig:pcivec}}) gelesen:\\
\stfn{(get-instance-vector-and-location)}
%
\item\label{enu:stslnv} Der neue \Slt\/-Zustand wird mit einen Aufruf
der \gfn\\
\stfn{(t-object-to-p-objid)}\\
im persistenten Speicher abgelegt (Protokoll~\ref{pro:sobj}
\stfn{(store-object)}, Schritt~\ref{enu:sotop},
\citepage{\pageref{enu:sotop}}). Die erhaltene
\sobjid\/ und Typkennung wird in den Zustandsvektor an der durch
die Po\-si\-tions-In\-for\-ma\-tion des \sltdo[es]\ gegebenen Position
eingetragen.
%
\item Wenn f\"{u}r den \Slt\/ ein Index deklariert wurde:
%
\begin{block}
%
\item\label{enu:stsldk} Pr\"{u}fe, ob der unter Schritt~\ref{enu:stslnv}
gespeicherte neue \Slt\/-Zustand in der Indextabelle bereits zu einer
persistenten \clos-Instanz assoziiert ist:\\
\stfn{(getindex-by-tag)}\\
Wenn eine persistente Instanz gefunden wird und sie
nicht identisch zu der an dieses Protokoll \"{u}bergebenen Instanz ist,
ist der neue \Slt\/-Zustand bereits als Schl\"{u}s\-sel auf eine andere
persistente \clos-Instanz vergeben; in diesem Fall wird ein Fehler
signalisiert.
%
\item Entferne aus der Indextabelle die Assoziation vom alten in
Schritt~\ref{enu:stslov} ermittelten \Slt\/-Zustand auf die
persistente \clos-Instanz:\\
\stfn{(remindex-by-tag)}
%
\item Assoziiere in der Indextabelle den neuen in
Schritt~\ref{enu:stslnv} gespeicherten \Slt\/-Zustand zur persistenten
\clos-Instanz:\\
\stfn{(setf getindex-by-tag)}
%
\end{block}
%
\item Wenn die Lebensdauer des \Slt\/-Zustands als
\lisp{:cached-write-through}\ deklariert wurde, speichere den
neuen \Slt\/-Zustand auch in der transienten Repr\"{a}sentation:\\
\stfn{(call-next-method)}
%
\end{block}
%
\item sonst wird der neue \Slt\/-Zustand mithilfe der \std-Methode
lediglich in die transiente \representation{} eingetragen:\\
\stfn{(call-next-method)}
%
\item Wenn in Schritt~\ref{enu:stsltr} eine Transaktion gestartet
wurde, wird sie in diesem Schritt beendet. Damit werden alle innerhalb
der Transaktion gesetzten Sperren aufgehoben.
%
\end{infol}
%
In der momentanen Programmversion sind keine \gande{doppelten
Schl\"{u}ssel}{duplicate keys}\ f\"{u}r einen Index zugelassen; die Pr\"{u}fung
auf einen doppelten Schl\"{u}ssel erfolgt in Schritt~\ref{enu:stsldk} und
signalisiert gegebenenfalls einen Fehler. F\"{u}r eine sp\"{a}tere
Programmversion ist geplant, beim Auftreten von doppelten Schl\"{u}sseln
die zum \Slt\/-Zustand assoziierten persistenten \clos-Instanzen in
einer Liste abzulegen.
%
\begin{infol}
%
\acaption%
 {{Protokoll f\"{u}r das Aufheben der Bindung eines
   \protect\Slt[s]\/ einer ak\-tiv-per\-si\-sten\-ten
   \clos-Instanz an einen Wert}}%
 [(slot-makunbound-using-class)]%
 \label{pro:musl}%
%
\item\label{enu:musltr} Sofern noch keine Transaktion aktiv ist, wird
eine Transaktion gestartet:\\
\stfn{(with-transaction)}
%
\item Ermittle die \sobjid\/ der persistenten \clos-Instanz:\\
\stfn{(persistent-object-objid)}
%
\item Das den \Slt\/ repr\"{a}sentierende \sltdo\ wird gesucht:\\
\stfn{(find-effective-slot-description)}\\
Im Verlauf der Auswertung von
\stfn{(find-effective-slot-description)}\ wird das
Protokoll~\ref{pro:escl}
\stfn{\enscldescr}\ (\citepage{\pageref{pro:escl}}) aufgerufen.
%
\item Wenn die Lebensdauer des \Slt\/-Zustands als
\lisp{:persistent}\ oder
\lisp{:cached-\lb{}write-\lb{}through}\ (\figurename~\ref{fig:extent},
\citepage{\pageref{fig:extent}}) deklariert wurde, hebe die
Bindung des \Slt[s]\/ in der persistenten \representation{} auf:
%
\begin{block}
%
\item\label{enu:muslov} Wenn f\"{u}r den \Slt\/ ein Index deklariert
wurde, wird zun\"{a}chst der alte \Slt\/-Zustand wie in
Protokoll~\ref{pro:losl} \stfn{(slot-value-using-class)}\ %
(\citepage{\pageref{pro:losl}}) angegeben gelesen:\\
\stfn{(slot-value-using-class)}
%
\item Aus der persistenten \representation{} des Objektes wird die
\sobjid\/ des Zustandsvektors (\figurename~\ref{fig:pcivec},
\citepage{\pageref{fig:pcivec}}) gelesen:\\
\stfn{(get-instance-vector-and-location)}
%
\item In den Zustandsvektor wird an der durch
die Po\-si\-tions-In\-for\-ma\-tion des \sltdo[es]\ gegebenen Position
das Markierungsobjekt \rglq{}ungebundener \Slt\rgrq\ eingetragen.
%
\item Wenn f\"{u}r den \Slt\/ ein Index deklariert wurde, wird
aus der Indextabelle die Assoziation vom in
Schritt~\ref{enu:muslov} ermittelten \Slt\/-Zustand auf die
persistente \clos-Instanz entfernt:\\
\stfn{(remindex-by-tag)}
%
\item Wenn die Lebensdauer des \Slt\/-Zustands als
\lisp{:cached-write-through}\ deklariert wurde, wird auch die Bindung
des \Slt[s]\/ in der transienten \representation{} aufgehoben:\\
\stfn{(call-next-method)}
%
\end{block}
%
\item sonst wird lediglich die Bindung des \Slt[s]\/ mithilfe der
\std-Methode in der transienten \representation{} aufgehoben:\\
\stfn{(call-next-method)}
%
\item Wenn in Schritt~\ref{enu:musltr} eine Transaktion gestartet
wurde, wird sie in diesem Schritt beendet. Damit werden alle innerhalb
der Transaktion gesetzten Sperren aufgehoben.
%
\end{infol}
%
%
\begin{infol}
%
\acaption%
 {{Protokoll f\"{u}r das Lesen eines \protect\Slt\/-Zustands einer
   aktiv-persistenten \protect\clos-Instanz}}%
 [(slot-value-using-class)]%
 \label{pro:losl}%
%
\item\label{enu:ldsltr} Sofern noch keine Transaktion aktiv ist, wird
eine Transaktion gestartet:\\
\stfn{(with-transaction)}
%
\item Ermittle die \sobjid\/ der persistenten \clos-Instanz:\\
\stfn{(persistent-object-objid)}
%
\item Das den \Slt\/ repr\"{a}sentierende \sltdo\ wird gesucht:\\
\stfn{(find-effective-slot-description)}\\
Im Verlauf der Auswertung von
\stfn{(find-effective-slot-description)}\ wird das
Protokoll~\ref{pro:escl}
\stfn{\enscldescr}\ (\citepage{\pageref{pro:escl}}) aufgerufen.
%
\item Wenn die Lebensdauer des \Slt\/-Zustands als
\lisp{:persistent}\ deklariert wurde, lade den
\Slt\/-Zustand aus der persistenten Repr\"{a}sentation:
%
\begin{block}
%
\item Aus der persistenten \representation{} des Objektes wird die
\sobjid\/ des Zustandsvektors gelesen:\\
\stfn{(get-instance-vector-and-location)}
%
\item\label{enu:ldslld} Der \Slt\/-Zustand befindet sich im
Zustandsvektor an der durch die Po\-si\-tions-In\-for\-ma\-tion des
\sltdo[es]\ gegebenen Position; er wird mit einem Aufruf der \gfn\\
\stfn{(p-objid-to-t-object)}\\
mit der \sobjid\/ und der Typkennung des vom \Slt\/ referenzierten
persistenten Objektes geladen (Protokoll~\ref{pro:lobj}
\stfn{(load-object)}, Schritt~\ref{enu:lotop},
\citepage{\pageref{enu:lotop}}).
%
\end{block}
%
\item\label{enu:ldslcn} sonst wird der \Slt\/-Zustand mithilfe der
\std-Methode lediglich aus der transienten \representation{} geladen:\\
\stfn{(call-next-method)}
%
\item Wenn in Schritt~\ref{enu:ldsltr} eine Transaktion gestartet
wurde, wird sie in diesem Schritt beendet. Damit werden alle innerhalb
der Transaktion gesetzten Sperren aufgehoben.
%
\item Als Ergebnis des Protokolls~\ref{pro:losl}
\stfn{(slot-value-using-class)}\ wird das Ergebnis von
Schritt~\ref{enu:ldslld} bzw.\ \ref{enu:ldslcn}
zu\-r\"{u}ck\-ge\-ge\-ben.
%
\end{infol}
%
%
\begin{infol}
%
\acaption%
 {{Protokoll f\"{u}r die Pr\"{u}fung, ob ein \protect\Slt\/ einer
   aktiv-persistenten \clos-Instanz an einen Wert gebunden ist}}%
 [(slot-boundp-using-class)]\label{pro:bpsl}%
%
\item\label{enu:bpsltr} Sofern noch keine Transaktion aktiv ist, wird
eine Transaktion gestartet:\\
\stfn{(with-transaction)}
%
\item Ermittle die \sobjid\/ der persistenten \clos-Instanz:\\
\stfn{(persistent-object-objid)}
%
\item Das den \Slt\/ repr\"{a}sentierende \sltdo\ wird gesucht:\\
\stfn{(find-effective-slot-description)}\\
Im Verlauf der Auswertung von
\stfn{(find-effective-slot-description)}\ wird das
Protokoll~\ref{pro:escl}
\stfn{\enscldescr}\ (\citepage{\pageref{pro:escl}}) aufgerufen.
%
\item Wenn die Lebensdauer des \Slt\/-Zustands als
\lisp{:persistent}\ deklariert wurde, pr\"{u}fe die Bindung
des \Slt[s]\/ in der persistenten Repr\"{a}sentation:
%
\begin{block}
%
\item Aus der persistenten \representation{} des Objektes wird die
\sobjid\/ des Zustandsvektors gelesen:\\
\stfn{(get-instance-vector-and-location)}
%
\item\label{enu:bpslld} Im Zustandsvektor wird an der durch
die Po\-si\-tions-In\-for\-ma\-tion des \sltdo[es]\ gegebenen Position
gepr\"{u}ft, ob sich dort nicht das Markierungsobjekt \rglq{}ungebundener
\Slt\rgrq\ befindet.
%
\end{block}
%
\item\label{enu:bpslcn} sonst wird lediglich die Bindung des \Slt[s]\
mithilfe der \std-Methode in der transienten Repr\"{a}sentation
gepr\"{u}ft:\\
\stfn{(call-next-method)}
%
\item Wenn in Schritt~\ref{enu:ldsltr} eine Transaktion gestartet
wurde, wird sie in diesem Schritt beendet. Damit werden alle innerhalb
der Transaktion gesetzten Sperren aufgehoben.
%
\item Als Ergebnis des Protokolls~\ref{pro:bpsl}
\stfn{(slot-boundp-using-class)}\ wird das Ergebnis von
Schritt \ref{enu:bpslld} bzw.\ \ref{enu:bpslcn} zur\"{u}ckgegeben.
%
\end{infol}
%
%\subsubsection{Funktionen}
%
%%
%\subsubsection{Versionen}
%%
%Versionen werden f\"{u}r \clsdo[e]\ unterst\"{u}tzt.
%%
%Beim Speichern oder Laden eines persistenten Objektes wird gepr\"{u}ft,
%ob sein \clsdo\ gleich ist zu dem aus der aktuellen transienten
%Klassendefinition gewonnenem \clsdo; sofern Unterschiede festgestellt
%werden, wird die neue Version des
%\clsdo[es]\ im persistenten Speicher abgelegt.
%%
%
%%% Local Variables: 
%%% mode: latex
%%% TeX-master: "main"
%%% buffer-file-coding-system: raw-text-unix
%%% End: 
}%
\fi%
\nextstep%
%
\subsection{Lokalisierung}
%
\subsubsection{Binden von Objekten an persistente Symbole}%
\label{sec:psym}
%
Ein Objekt ist genau dann persistent, wenn es erreichbar ist, d.h.\
wenn es transitiv vom Wurzelobjekt referenziert wird. In
\wood\ wird beispielsweise deswegen die M\"{o}glichkeit angeboten, das
Wurzelobjekt explizit zu setzen \cite{bib:wo93}; seine Struktur mu\ss{}
von den BenutzerInnen so angelegt und verwaltet werden, da\ss{} alle
persistenten Objekte \"{u}ber dieses Wurzelobjekt erreichbar sind.
%
\par{}In \plob\ wurde ein anderer, LISP-""\"{a}hn\-li\-cher Weg gew\"{a}hlt.
In LISP k\"{o}n\-nen Symbole \"{u}ber ihren Namen referenziert werden
\stcite{13, 27}; an ein Symbol kann ein Objekt gebunden werden.
Analog dazu k\"{o}n\-nen hier persistente Symbole benutzt werden, an die
ein persistentes Objekt gebunden werden kann. Persistente Symbole
k\"{o}n\-nen ebenfalls \"{u}ber ihren Namen lokalisiert werden; f\"{u}r die
Suche kann als Name entweder eine transiente oder persistente
Zeichenkette oder ein transientes Symbol angegeben werden. Bei Angabe
eines transienten Symbols wird ein persistentes Symbol mit gleichem
Namen und Paket gesucht.  F\"{u}r jedes persistente Symbol ist
garantiert, da\ss{} es erreichbar ist; damit ist auch ein an ein
persistentes Symbol gebundenes Objekt erreichbar.
%
\subsubsection{Indizes \"{u}ber die Zust\"{a}nde eines \protect\Slt[s]}
%
F\"{u}r den \Slt\/ einer aktiv-persistenten Klasse mit deklariertem
Index kann nach den persistenten \clos-Instanzen gesucht
werden, die in diesem \Slt\/ einen bestimmten Zustand enthalten;
Abfragen \"{u}ber einen Zustandsbereich sind ebenfalls m\"{o}glich. F\"{u}r die
Suche werden folgende Parameter angegeben:
\begin{itemize}
\item Der Name der aktiv-persistenten Klasse.
\item Der Name des \Slt[s], \"{u}ber den gesucht werden soll. F\"{u}r diesen
\Slt\/ mu\ss{} ein Index deklariert worden sein.
\item Der Zustand oder Zustandsbereich, nach dem gesucht werden soll.
\end{itemize}
Als Ergebnis wird eine Liste der passenden Instanzen der Klasse oder
Superklassen zu\-r\"{u}ck\-ge\-ge\-ben.
%
\par{}Durch Erstellen entsprechender Methoden k\"{o}n\-nen relativ einfach
zu\-s\"{a}tz\-li\-che Indexarten eingebunden werden; genaueres dazu
findet sich im Referenzhandbuch \cite{bib:ki94b}\ im Abschnitt
\rglq{}index \ldots\rgrq.
%
\subsection{\protect\std[isierung]}\label{sec:stdiz}
%
In \plob\ wird eine orthogonale Sichtweise f\"{u}r Persistenz realisiert;
dies bedeutet, da\ss{} jedes transiente Objekt persistent werden kann.
Ein transientes Objekt wird persistent, indem sein Zustand bestehend
aus Referenzen und Werten vollst\"{a}ndig in ein persistentes Objekt
\"{u}bertragen wird. Daraus ergibt sich die Notwendigkeit, \"{u}ber das
Verhalten jedes transienten Objektes auf seinen Gesamtzustand
zugreifen k\"{o}n\-nen zu m\"{u}ssen. Das Verhalten der Instanzen von
\cl\ Typen wird in \cite{bib:st90} weitestgehend normiert.
%
\par{}F\"{u}r fehlende Normierungen gibt es prinzipiell die zwei
L\"{o}sungen, sie entweder hinzunehmen oder eigene Normierungen
festzulegen. Die erste L\"{o}sung schr\"{a}nkt die Transparenz oder
Orthogonalit\"{a}t von Persistenz ein; deswegen wurde hier die zweite
L\"{o}sung vorgezogen.
Im folgenden Text werden die Arten der fehlenden Normierungen
sowie die sich bietenden M\"{o}glichkeiten, doch noch zu einer Normierung
zu kommen, geschildert.
%
\subsubsection{{\protect\rglq}Vergessen{\protect\rgrq}\ einer Norm}
%
In diese Kategorie fallen Normierungen, die offensichtlich w\"{a}ren,
aber trotzdem nicht erfolgten, obwohl der Normierungsproze\ss{}
abgeschlossen ist.
%
\par{}Ein Beispiel daf\"{u}r sind die fehlenden
\gande{Feld-Informationsfunktionen}
{array information functions}%
\addglossary{Feld-Informationsfunktion}[array information function]%
[{Eine
Feld-\protect\discretionary{}{}{}In\-for\-ma\-ti\-ons\-funk\-tion
liefert Informationen \"{u}ber ein LISP Datenobjekt der Klasse
\protect\gande{Feld}{array}, wie z.B.\ die Anzahl Feldelemente, eine
Liste der Felddimensionen, den Elementtyp, f\"{u}r den das Feld
spezialisiert wurde usw.}]\ \stcite{448--452}\ im Zusammenhang mit
\gande{Feldern, die Elemente anderer Felder referenzieren}{displaced
arrays}%
\addglossary{displaced array}%
[Ein {\protect\em displaced array\/} ist ein Feld, das Elemente eines
anderen Feldes referenziert. Jeder Zugriff auf ein Feldelement liest
bzw.\ schreibt ein Element des referenzierten Feldes.]\ \stcite{444};
f\"{u}r diese Felder gibt es keine Informationsfunktionen, die dar\"{u}ber
Auskunft geben, zu welchem Feld die referenzierten Elemente geh\"{o}ren
und ab welchem
\gande{Index}{displaced index offset}[{Der erste Index f\"{u}r ein
\protect\gande{Feld, da\ss{} ein anderes Feld
referenziert}{\protect\see{displaced array}}\ des gemeinsam
referenzierten Teils.}]\ \stcite{445}\ der gemeinsam referenzierte
Teil beginnt.
%
\subsubsection{Fehlende Norm}
%
Fehlende Normen resultieren aus einem noch nicht begonnenem
Normierungsproze\ss{}.
%
\par{}Ein Beispiel ist die fehlende Normierung des Verhaltens
von Strukturklassen. In \stcite{800}\ wird nur festgelegt, da\ss{}
Strukturklassen repr\"{a}sentiert werden; eine Festlegung ihres
Verhaltens erfolgte nicht. In \cite{bib:amop91} werden
Strukturklassen \"{u}berhaupt nicht erw\"{a}hnt.
%
\par{}Eine Normierung f\"{u}r LISP-interne Prozesse erfolgte ebenfalls
nicht; viele LISP-Systeme, darunter auch \lwcl\ und \allegrocl, halten
sich aber an den durch die LISP Implementation auf den Maschinen der
Firma \symbolics\ gegebenen \std.
%
\paragraph{L\"{o}sung}
Das Problem der \rglq{}vergessenen\rgrq\ und fehlenden Normen stellte
sich bereits den AutorInnen der verschiedenen LISP-Systeme, die
daf\"{u}r meist untereinander \"{a}quivalente vern\"{u}nftige
Festlegungen trafen, die als Norm im Rahmen \ifbericht dieses
Berichtes \else\ifbuch dieses Buches \else dieser Arbeit \fi\fi
verwendet werden konnten. Es wurde Wert darauf gelegt, die
entsprechenden, meist systemabh\"{a}ngigen Konstrukte nicht direkt zu
nutzen, sondern eine dokumentierte Schnittstelle zu schaffen.
%
\subsubsection{Unvollst\"{a}ndige oder unklare Norm}
%
Unvollst\"{a}ndige oder unklare Normen resultieren aus einem noch nicht
abgeschlossenem Normierungsproze\ss{}.
%
\par{}In \cite{bib:st90}\ wurde beispielsweise keine Normierung des
\mop\ vogenommen; statt dessen wurde der Hinweis gegeben, da\ss{} die
dem ANSI Kommitee im Juni 1988 vorgeschlagene \std[isierung]\ des
\mop\ nicht angenommen wurde \stcite{770}. Die AutorInnen des zur
Realisierung von \plob\ verwendeten \lwcl\ hatten also zu diesem
Zeitpunkt keine verbindliche Norm zur Implementation des \mop;
letztendlich erfolgte sie doch nach der nicht angenommenen
\std[isierung]\ vom Juni 1988 \cite{bib:sn94}. Im Jahr 1991 erschien
dann \cite{bib:amop91}, das inzwischen als de-fakto Normierung des
\mop\ angesehen wird; es unterscheidet sich aber von dem 1988 nicht
angenommenen \std\ in einigen Punkten, die f\"{u}r die Realisierung von
transparenter Persistenz f\"{u}r aktiv-persistente \clos-Objekte nicht
unerhebliche Schwierigkeiten bereiteten.
%
\par{}Eine Unklarheit findet sich in
\cite{bib:amop91} bez\"{u}glich des Protokolls zum Erzeugen des
effektiven Funktionscodes der \mtdmo[e]\ der generischen
Zugriffs-\fn[en], deren Namen in einer
\cls[n]definition in den Argumenten der \Slt\/-Optionen
\lisp{:reader}, \lisp{:writer}\ oder \lisp{:accessor}\ angegeben
wurden. Der effektive Funktionscode aller \mtdmo[e]\ wird nach
\amopcite{207--209}\ \"{u}ber einen Aufruf der
\gfn[n]\ \stfn{make-method-lambda}\ erzeugt, die u.a.\ den (von den
BenutzerInnen erstellten) direkten Code der Methode als Argument
erh\"{a}lt. Das Problem im Zusammenhang mit generischen
Zugriffs-\fn[en]\ besteht darin, da\ss{} kein expliziter direkter Code
f\"{u}r ihre Methoden spezifiziert wird, sondern vom LISP-System selbst
generiert wird. Diese Tatsache wird von \lw\ so
interpretiert, da\ss{} die \gfn\ \stfn{make-method-lambda}\ nur f\"{u}r
\mtdmo[e]\ aufgerufen wird, die aus explizit
% vom Benutzer
erstellten Methoden erzeugt wurden und nicht f\"{u}r die implizit
erzeugten \mtdmo[e]\ der Methoden der generischen
Zugriffs-\fn[en]\ifbericht\else\ (Protokoll~\ref{pro:defclass}
\stfn{(defclass)}, Schritte~\ref{enu:dcmcl}--\ref{enu:dcmad},
\citepage{\pageref{enu:dcmcl}})\fi.
%
\paragraph{L\"{o}sung}
Die unklaren oder unvollst\"{a}ndigen Normen resultieren in einem
unerwartetem Verhalten des LISP-Systems; eine gr\"{u}ndliche
L\"{o}sung dieses Problems besteht in der Beseitigung der Unklarheiten
bzw.\ in der Erg\"{a}nzung der Normierung und einer Korrektur des LISP
Systems. Dies war im Rahmen \ifbericht dieses Berichtes \else\ifbuch
dieses Buches \else dieser Arbeit \fi\fi nicht m\"{o}glich.
%
\par{}Ich vertrete die Meinung, da\ss{}, sofern der Produzent eines
Systems angibt, einem gewissem \std\ zu gen\"{u}gen, der Produzent auch
daf\"{u}r zust\"{a}ndig ist, da\ss{} das System den \std\ auch tats\"{a}chlich
einh\"{a}lt. Deswegen wurde hier darauf verzichtet, f\"{u}r den Ausgleich
der Inkompatibilit\"{a}ten der LISP-Systeme zum gegebenen
\std\ zu\-s\"{a}tz\-lich einen \rglq{}Meta\rgrq-\std\ zu definieren; statt
dessen wurden die Inkompatibilit\"{a}ten durch dokumentierte
ad-hoc-L\"{o}sungen ausgeglichen.
%
\subsubsection{Keine Festlegung von Strukturen}
%
Die \std[isierung]\ in \cite{bib:st90} legt keine Strukturen f\"{u}r
Instanzen der dort definierten Typen und Klassen fest; damit kann
diese Festlegung von den AutorInnen eines LISP-Systems vorgenommen
werden und erm\"{o}glicht so beispielsweise Optimierungen. Statt dessen
wurde ihr Verhalten so normiert, da\ss{} % (bis auf die \og[e]\ Ausnahme)
dar\"{u}ber auf den Gesamtzustand einer Instanz als Summe ihrer
Teilzust\"{a}nde zugegriffen werden kann. Bei der Realisation zeigte
sich, da\ss{} diese \rglq{}Aufsummierung\rgrq\ mit erheblichen
Effizienzverlusten verbunden sein kann und also unter Umst\"{a}nden
vermieden werden sollte; es ist dann g\"{u}nstiger, auf den Gesamtzustand
zugreifen zu k\"{o}n\-nen.  Dies bedingt aber die Verf\"{u}gbarkeit von
Informationen \"{u}ber die Struktur dieser Instanzen.
%
\par{}Beispiele daf\"{u}r sind Zeichenketten und Bit-Vektoren, auf die
lediglich elementweise zugegriffen werden kann; die \"{U}bertragung
eines Zustands w\"{a}re in diesem Fall sehr ineffizient.
%
\paragraph{L\"{o}sung}
Die in den verschiedenen LISP-Systemen verwendeten Strukturen sind oft
\"{a}quivalent; daher wurden ihre internen Zugriffsfunktionen als Norm
im Rahmen \ifbericht dieses Berichtes \else\ifbuch dieses Buches \else
dieser Arbeit \fi\fi verwendet. Es wurde Wert darauf gelegt, die
entsprechenden systemabh\"{a}ngigen Zugriffsfunktionen nicht direkt
aufzurufen, sondern eine dokumentierte Schnittstelle zu schaffen.
%
\subsubsection{System-interne Konstrukte}
%
Der \cl\ \std\ verbietet es einem LISP-System selbstverst\"{a}ndlich
nicht, zu\-s\"{a}tz\-li\-che Konstrukte zu benutzen oder anzubieten,
solange sie nicht inkompatibel zum \std\ sind. Dies k\"{o}n\-nen
beispielsweise systemintern definierte Typen sein, deren Instanzen zur
Re\-pr\"{a}\-sen\-ta\-tion von Instanzen von \std-Typen benutzt werden.
%
\paragraph{L\"{o}sung}
Die systeminternen Konstrukte wurden entweder als Basiskonstrukte zu
\plob\ hinzugef\"{u}gt oder durch andere Basiskonstrukte nachgebildet.
%
\subsection{Zusammenfassung der 3.\ Schicht}
%
F\"{u}r die Einhaltung der \acid\ Kriterien (\citepage{\pageref{ref:acid}})
sorgt die 2.~Schicht; die 3.~Schicht benutzt lediglich die
dort realisierten Funktionalit\"{a}ten und bietet h\"{o}heren Schichten
einfache Ein\-bin\-dungs\-m\"{o}g\-lich\-kei\-ten f\"{u}r sie an.
%
Der folgende Abschnitt fa\ss{}t noch einmal kurz die Funktionalit\"{a}ten
der 3.~Schicht aus der Sicht der n\"{a}chsth\"{o}heren 4.~Schicht zusammen.
\begin{description}
%
\item[Abbildung zwischen transienten und persistenten Objekten]
LISP Objekte werden auf persistente Objekte mit einer der
transienten Klasse angemessenen Beschreibung abgebildet.
%
\item[Orthoghonalit\"{a}t; aktiv-persistente Klassen]
Jedes transiente LISP Objekt kann persistent werden;
Instanzen von aktiv-persistenten \clos\ Klassen haben bez\"{u}glich
Persistenz mehr M\"{o}g\-lich\-kei\-ten als passiv-persistente Instanzen.
%
\item[Zugriff auf den Zustand von persistenten Objekten]
F\"{u}r den Zugriff auf persistente Objekte gibt es die
M\"{o}glichkeit, die persistente \representation{} eines persistenten
Objektes auf eine transiente \representation{} abzubilden und auf diese
zuzugreifen (\swizzling\/). F\"{u}r aktiv-persistente Objekte kann
auch transparent direkt auf die persistente Repr\"{a}sentation
zugegriffen werden, f\"{u}r alle anderen Objekte gibt es die M\"{o}glichkeit
des intransparenten direkten Zugriffs.
%
\item[Wurzelobjekt; persistente Symbole]
Das Wurzelobjekt wird lediglich intern in der 3.~Schicht verwendet;
damit ein persistentes Objekt vom Wurzelobjekt aus erreichbar wird,
gen\"{u}gt es, wenn es als Wert an ein persistentes Symbol gebunden wird.
Persistente Symbole und damit auch der jeweils an sie gebundende Wert
k\"{o}n\-nen \"{u}ber Namen lokalisiert werden.
%
\item[Assoziativer Zugriff]
F\"{u}r \Slt[s]\/ von aktiv-persistenten Instanzen kann ein Index
deklariert werden, der persistente \clos-Instanzen zu
\Slt\/-Zust\"{a}nden assoziiert. Die Verwaltung der Indextabellen erfolgt
transparent in \spc[n]\ \mtd[n]\ der \gfn[n]en
\stfn{slot-\ldots-using-class}.
%
\end{description}
%
Die 3.~Schicht wird ausf\"{u}hrlich im Re\-fe\-renz-Hand\-buch
\cite{bib:ki94b}\ dokumentiert.
%
\section{Schicht 4: Die Sitzungs-Schicht}
%
Dieser Abschnitt erl\"{a}utert die zur Verf\"{u}gung gestellten Konzepte und
zeigt Beispiele f\"{u}r ihre Benutzung.
%
\subsection{Konzepte}
%
\subsubsection{Persistente Pakete}
%
Persistente Pakete dienen zur Gruppierung von persistenten Symbolen.
Alle persistenten Pakete sind erreichbar und damit immer persistent.
Zur Manipulation von persistenten Paketen gibt es Funktionen mit
gleicher Funktionalit\"{a}t wie f\"{u}r transiente Pakete; die
entsprechenden Funktionsnamen stammen aus \cite{bib:st90} und wurden
zur Unterscheidung zu den auf transienten Paketen arbeitenden
Funktionen mit dem Pr\"{a}fix \rglq{}p-\rgrq\ versehen; im einzelnen sind
dies die Funktionen \stfn{p-find-package},
\stfn{p-delete-package}\footnote{Da persistente Pakete implizit
erreichbar sind, gibt es f\"{u}r sie einen expliziten Destruktor.},
\stfn{p-package-name}\ und \stfn{p-package-internals}.
%
\subsubsection{Persistente Symbole}
%
Persistente Symbole sind mit Namen versehene persistente Objekte. Sie
geh\"{o}ren zu genau einem persistenten Paket und sind von diesem
erreichbar; damit sind sie persistent. Ein an ein persistentes Symbol
gebundenes Objekt ist, da
es von einem persistenten Symbol referenziert wird, ebenfalls
persistent und erreichbar. Aus der Sicht der BenutzerInnen gibt es
kein explizites Wurzelobjekt; diese Aufgabe haben in
\plob\ persistente Pakete und speziell die von ihnen erreichbaren
persistenten Symbole \"{u}bernommen. Dies entspricht vom Konzept her der
Erreichbarkeit von transienten Symbolen in \cl\ \"{u}ber ihr transientes
Paket.
%
\par{}Zur Manipulation von persistenten Symbolen gibt es Funktionen
mit gleicher Funktionalit\"{a}t wie f\"{u}r transiente Symbole. Die
entsprechenden Funktionsnamen stammen aus \cite{bib:st90} und wurden
zur Unterscheidung zu den auf transienten Symbolen arbeitenden
Funktionen mit dem Pr\"{a}fix \rglq{}p-\rgrq\ versehen; im einzelnen sind
dies die Funktionen \stfn{p-boundp}, \stfn{p-fboundp},
\stfn{p-symbol-function}, \stfn{(setf p-symbol-function)},
\stfn{p-symbol-package}, \stfn{p-symbol-plist}, 
\stfn{(setf p-symbol-plist)}, \stfn{p-symbol-name}, 
\stfn{p-symbol-value}, \stfn{(setf p-symbol-value)},
\stfn{p-fmakunbound}\ und \stfn{p-mak\-un\-bound}.
Bei den Manipulationsfunktionen f\"{u}r persistente Symbole kann
zur Referenzierung des persistenten Symbols das transiente
Symbol angegeben werden; die Manipulationsfunktionen suchen dann
das persistente Symbol mit gleichem Namen und Paket. Der Wert des
persistenten Symbols \lisp{'foo} kann beispielsweise mit dem Ausdruck
\lisp{(setf (p-symbol-value 'foo) 5)}\ auf den Wert 5 gesetzt werden;
das transiente Symbol \lisp{'foo}\ dient zur Lokalisierung des
persistenten Symbols \lisp{'foo}. Das Setzen des Wertes eines
persistenten Symbols auf ein transientes Objektes impliziert die
Speicherung des transienten Objektes.
%
\par{}Eine der h\"{a}ufigsten Operationen mit Symbolen ist das Setzen
bzw.\ Abfragen des an das Symbol gebundenen Wertes; dazu gibt es das
{\em macro-reader\/}-Zeichen%
%
\begin{listing}[\tt]%
\#l{\rm\lt{}symbol-name\gt{}\/}
\end{listing}%
%
\noindent das zu
%
\begin{listing}[\tt]%
(p-symbol-value {\rm\lt{}symbol\gt\/})
\end{listing}%
%
\noindent evaluiert; der Buchstabe \lisp{l}\ ist eine Abk\"{u}rzung f\"{u}r
\rglq{}load\rgrq.
%
\par{}Zur Verwaltung von persistenten Symbolen gibt es die
Funktionen \stfn{p-find-symbol}, \stfn{p-intern}\ und
\stfn{p-unintern}\ mit gleicher Funktionalit\"{a}t wie in
\cite{bib:st90}.
%
%\subsubsection{Aktiv-persistente Klassen}
%
%Aktiv-persistente Klassen 
%
\subsubsection{Deklarationen}
%
\"{U}ber Deklarationen kann die Lebensdauer und \representationform{} von
\Slt\/-Zust\"{a}nden, die Art der Schemaentwicklung f\"{u}r Klassen und die
f\"{u}r die Erzeugung der transienten \representations{} von persistenten
Objekten einer Klasse zu benutzende Konstruktor-Funktion festgelegt
werden. Die M\"{o}glichkeiten der Beeinflussung von Persistenz \"{u}ber
Deklarationen besteht f\"{u}r alle Klassen, un\-ab\-h\"{a}n\-gig davon, ob
sie aktiv- oder passiv-persistent sind; lediglich die Art und Weise,
in der die Deklarationen angegeben werden k\"{o}nnen, ist
unterschiedlich.
%
\par{}F\"{u}r aktiv-persistente Klassen werden die Deklarationen als
zu\-s\"{a}tz\-li\-che Klassen- und \Slt\/-Optionen in die
Klassendefinition eingef\"{u}gt. Ihre Auswertung erfolgt im Laufe der
Initialisierung des die Klasse repr\"{a}sentierenden \clsmo[es].  Aus dem
\clsmo\ und den davon referenzierten \mo[en]\ werden die Informationen
in das persistente \clsdo\ und die davon referenzierten \deo[e]\ der
Klasse \"{u}bernommen.
%
\par{}Bei passiv-persistenten Klassen werden die Deklarationen \"{u}ber
Aufrufe von Funktionen an \plob\ weitergegeben; Deklarationen auf
Klassenebene werden an das \clsmo\ der Klasse und Deklarationen auf
\Slt\/-Ebene werden an das \clsmo\ und den Namen des
\Slt[s]\/ assoziiert.
%
\paragraph{Deklaration der Lebensdauer von
\protect\Slt\/-Zust\"{a}nden}
%
Die Deklaration der Lebensdauer von \Slt\/-Zust\"{a}nden beeinflu\ss{}t
sowohl die Lebensdauer als auch die \representationform{} von
\Slt\/-Zust\"{a}nden 
\ifbericht%
\cite[\citepage{120}]{bib:ki94a}%
\else%
(\figurename~\ref{fig:extent},
\citepage{\pageref{fig:extent}})\fi. Die einzelnen Deklarationswerte
bedeuten:
\begin{description}
%
\item[Lebensdauer \lisp{:cached-demand-load}]
Diese Lebensdauer gibt es nur f\"{u}r
\Slt[s]\/ von Strukturobjekten. Der transiente \Slt\/-Zustand wird bei
der ersten lesenden Dereferenzierung des \Slt[s]\/ geladen, es sei
denn, der persistente \Slt\/-Zustand referenziert ein \immval\/ oder
ein bereits geladenes  Objekt; in diesem Fall wird der transiente
\Slt\/-Zustand bereits beim Laden des Strukturobjektes gesetzt.
%
\par{}Da Strukturobjekte passiv-persistente Instanzen sind, erfolgt
das Speichern des transienten \Slt\/-Zustands beim Speichern des
transienten Strukturobjektes.
%
\item[Lebensdauer \lisp{:transient}]
Mit der Deklaration einer transienten Lebensdauer
wird der \Slt\/ nicht im persistenten Speicher repr\"{a}sentiert;
demzufolge wird der Zustand eines als transient deklarierten
\Slt[s]\/ beim Speichern des transienten Objektes auch nicht
weiter traversiert.
%
\item[Lebensdauer \lisp{:cached}]
Der transiente \Slt\/-Zustand wird nur beim Laden bzw.\ Speichern des
transienten Objektes \"{u}bertragen.
%
\item[Lebensdauer \lisp{:cached-write-through}]
Diese Lebensdauer gibt es nur f\"{u}r \Slt[s]\/ von aktiv-persistenten
Instanzen.  Der Zustand des transienten \Slt[s]\/ wird beim Laden auf
den Zustand des persistenten \Slt[s]\/ gesetzt; Schreibzugriffe werden
in den \Slt\/ sowohl der transienten als auch der persistenten
\representation{} propagiert.
%
\item[Lebensdauer \lisp{:persistent}]
Diese Lebensdauer gibt es nur f\"{u}r \Slt[s]\/ von aktiv-persistenten
Instanzen.  Der \Slt\/ wird nur im persistenten Speicher
repr\"{a}sentiert; alle Zugriffe finden direkt auf die persistente
\representation{} statt. 
%
\end{description}
Eine Lebensdauer kann auf einer der Ebenen \rglq\Slt\/\rgrq\ --
\rglq\cls\rgrq\ -- \rglq{}Paket des
\cls[n]-Namens\rgrq\  deklariert werden; die Lebensdauer
des \Slt\/-Zustands ergibt sich aus der ersten Deklaration, die
in der angebenen Reihenfolge f\"{u}r eine der Ebenen gefunden wird.
%
\par{}F\"{u}r aktiv-persistente Klassen erfolgt die Einbindung der
\og[en]\ Deklarationswerte auf \Slt\/- und Klassen-Ebene mittels der
zu\-s\"{a}tz\-li\-chen \Slt\/- bzw.\ Klassen-Option
%
\begin{listing}[\tt]%
:extent {\rm\it\lt{}Lebensdauer\gt\/}
\end{listing}%
%
\par{}Bei passiv-persistenten Klassen erfolgt die
Einbindung der Deklarationswerte auf \Slt\/-Ebene \"{u}ber die
Evaluierung des Ausdrucks
%
\begin{listing}[\tt]%
(setf \=(slot-extent {\rm\it\lt{}Slot-Name\gt\/}
                     (find-class {\rm\it\lt{}Klassenname\gt\/}))\\
      \>{\rm\it\lt{}Lebensdauer\gt\/})
\end{listing}%
%
\noindent und auf Klassenebene \"{u}ber die Evaluierung des Ausdrucks
%
\begin{listing}[\tt]%
(setf (class-extent (find-class {\rm\it\lt{}Klassenname\gt\/}))
      {\rm\it\lt{}Lebensdauer\gt\/})
\end{listing}%
%
\par{}Auf Paketebene erfolgt die Einbindung der Deklaration einer
Lebensdauer \"{u}ber die Evaluierung des Ausdrucks
%
\begin{listing}[\tt]%
(setf (package-extent (find-package {\rm\it\lt{}Paketname\gt\/}))
      {\rm\it\lt{}Lebensdauer\gt\/})
\end{listing}%
%
\noindent Damit k\"{o}nnen alle Klassen, die innerhalb
eines Paketes definiert werden, mit einer Lebensdauer versehen werden;
\ia\ wird die Lebensdauer auf Paketebene f\"{u}r Pakete mit
den Namen von \rglq{}uninteressanten Klassen\rgrq\ auf transient
gesetzt.
%
%\paragraph{Deklaration der Schemaentwicklung}
%
%
%\subsubsection{\"{O}ffnen und Schlie\ss{}en des \protec\postore}
%
%
\subsubsection{Sitzungen und Transaktionen}
%
Eine Sitzung wird implizit von \plob\ bei Bedarf f\"{u}r den laufenden
\lw\ {\em lightweight\/}-Proze\ss{} ge\"{o}ffnet; durch Aufruf der Funktion
%
\begin{listing}[\tt]%
(open-session)
\end{listing}%
%
\noindent kann sie auch explizit ge\"{o}ffnet werden. Die Funktion
%
\begin{listing}[\tt]%
(show-sessions)
\end{listing}%
%
\noindent zeigt alle momentan offenen Sitzungen an.
%
\par{}Das Makro \stfn{with-transaction}\ bettet die
{\it\lt{}Anweisungen\gt\/} in eine an die Sitzung gebundene
Transaktion ein:
%
\begin{listing}[\tt]%
(with-transaction {\rm\it\{ \lt{}Anweisungen\gt\ \}*\/})
\end{listing}%
%
\noindent Werden die {\it\lt{}An\-wei\-sun\-gen\gt\/} vollst\"{a}ndig
abgearbeitet, wird die Transaktion \gande{beendet}{committed};
andernfalls wird sie \gande{abgebrochen}{aborted}. Eine
nicht-vollst\"{a}ndige Abarbeitung ist durch jeden nicht-""lo\-ka\-len
Sprung aus den {\it\lt{}Anweisungen\gt\/} heraus in den umfassenden
Block des Makros gegeben; dies tritt insbesondere bei folgenden
Ereignissen auf:
\begin{description}
%
\item[Laufzeitfehler]
Bei Fehlern, die nicht innerhalb der {\it\lt{}Anweisungen\gt\/}
behandelt werden, ist eine nicht-vollst\"{a}ndige Abarbeitung gegeben.
%
\item[\lisp{block}/\lisp{return}-Konstrukte]
Bei \lisp{block}/\lisp{return}-Konstrukten \stcite{161}\ ist zu
beachten, da\ss{} bei Einbettung des Makros \stfn{with-transaction}\ in
einen Block die Ausf\"{u}hrung der Anweisung
\lisp{return}\ bzw.\ \lisp{return-from}\ eine nicht
vollst\"{a}ndige Abarbeitung der {\it\lt{}An\-wei\-sun\-gen\gt\/} zur
Folge hat.
%
\item[\lisp{catch}/\lisp{throw}-Konstrukte]
Die Konstrukte \lisp{catch}\ und \lisp{throw}\ \stcite{187}\ mit der
Anweisung \lisp{catch}\  im umfassenden Block des Makros
\stfn{with-transaction}\ und der Anweisung \lisp{throw}\ innerhalb der
{\it\lt{}Anweisungen\gt\/} f\"{u}hren bei der Ausf\"{u}hrung der Anweisung
\lisp{throw}\ zu einer nicht vollst\"{a}ndigen Abarbeitung der
{\it\lt{}Anweisungen\gt}.
%
\end{description}
Beim Auftreten eines der \og[en]\ Ereignisse wird eine Transaktion
demzufolge abgebrochen.
%
\subsubsection{Assoziativer Zugriff}
%
F\"{u}r den \Slt\/ einer aktiv-persistenten Instanz kann ein Index
definiert werden; die Art des Indexes wird \"{u}ber die
zu\-s\"{a}tz\-li\-che \Slt\/-Option \lisp{:index}\ deklariert:
%
\begin{listing}[\tt]%
:index {\rm\it\lt{}index-definierender S-Ausdruck\gt\/}
\end{listing}%
%
\noindent Der {\rm\it\lt{}index-definierende S-Ausdruck\gt\/}
definiert den anzulegenden Index; momentan stehen persistente B-B\"{a}ume
(Abschnitt~\ref{sec:plbtr}, \citepage{\pageref{sec:plbtr}}) als
\representation{} f\"{u}r Indextabellen zur Verf\"{u}gung. Die \Slt\/-Option
f\"{u}r Verwendung eines B-Baumes als Indextabelle lautet:
%
\begin{listing}[\tt]%
:index (btree :test {\rm\it\lt{}Ordnungskriterium\gt\/}
                    :cached {\rm\it\lt{}cached\gt\/})
\end{listing}%
%
\noindent Die m\"{o}glichen {\rm\it\lt{}Ordnungskriterien\gt\/} werden
in Abschnitt~\ref{sec:plbtr} (\citepage{\pageref{sec:plbtr}})
erl\"{a}utert. Wenn f\"{u}r die Angabe {\rm\it\lt{}cached\gt\/}
\nonnil\ angegeben wird, finden alle Einzelanfragen (d.h.\ alle
Anfragen, die nicht \"{u}ber einen Schl\"{u}sselbereich gehen) auf den
B-Baum \"{u}ber einen an ihn gebundenen transienten \cache\/ statt; damit
wird der lesende Zugriff f\"{u}r einzelne Schl\"{u}sselwerte beschleunigt,
aber gleichzeitig der Speicherbedarf erh\"{o}ht.
%
\par{}Die Deklaration eines Indexes hat zwei Seiteneffekte. Der erste
Seiteneffekt besteht darin, da\ss{} durch die Deklaration eines Indexes
mit dem Ordnungskriterium \lisp{equal}\ der \Slt\/ auf den Typ des
ersten \Slt\/-Zustands festgelegt wird, der in den B-Baum eingef\"{u}gt
wird. Der Grund liegt darin, da\ss{} die Schl\"{u}sselobjekte in einem
B-Baum mit dem Ordnungskriterium \lisp{equal}\ alle den gleichen Typ
haben m\"{u}ssen, damit sie sortiert abgelegt werden k\"{o}nnen.
%
\par{}Der zweite Seiteneffekt ist, da\ss{} alle Objekte der Klassen,
bei denen f\"{u}r einen \Slt\/ ein Index definiert wurde, erreichbar sind
und daher bei einer Speicherr\"{u}ckgewinnung nicht gel\"{o}scht werden
(\figurename~\ref{fig:adridx}, \citepage{\pageref{fig:adridx}}).
Damit ein Objekt bei einer Speicherr\"{u}ckgewinnung gel\"{o}scht
werden kann, darf keine Referenz aus einer Indextabelle auf das Objekt
existieren. Erreicht wird dies durch Entfernen des Objektes aus den
Indextabellen, indem die Bindung jedes \Slt[s]\/ des Objektes, f\"{u}r den
ein Index deklariert wurde, aufgehoben wird.
%
\par{}Objekte k\"{o}nnen mit der Funktion
%
\begin{listing}[\tt]%
(p-select {\rm\it\lt{}Klassenname\gt\/}
          :where {\rm\it\lt{}Slot-Name\gt\/}
          {\rm\it\lt{}Auswahlkriterien\gt\/})
\end{listing}%
%
\noindent nach einem \Slt\/-Zustand ausgew\"{a}hlt werden. F\"{u}r den durch
{\rm\it\lt{}Slot-Namen\gt\/} identifizierten \Slt\/ in der durch
{\rm\it\lt{}Klassenname\gt\/} benannten Klasse mu\ss{} ein Index
deklariert worden sein. F\"{u}r Einzelheiten der
{\rm\it\lt{}Auswahlkriterien\gt\/} verweise ich auf das
Referenzhandbuch \cite{bib:ki94b}, Abschnitt \rglq{}p-select\rgrq.
%
\subsubsection{L\"{o}schen von Objekten und Speicherr\"{u}ckgewinnung}
%
Der Speicherplatz eines Objektes kann dann und nur dann freigegeben
werden, wenn keine Referenzen mehr auf das Objekt existieren; ein
Objekt wird durch nicht-referenzieren als zu l\"{o}\-schen\-des Objekt
f\"{u}r die n\"{a}chste Speicherr\"{u}ckgewinnung \rglq{}markiert\rgrq{}.
Eine Speicherr\"{u}ckgewinnung wird durch den Aufruf der Funktion
%
\begin{listing}[\tt]%
(plob-gc)
\end{listing}%
%
\noindent veranla\ss{}t; vor einer Speicherr\"{u}ckgewinnung sollte der
\postore\ durch einen Aufruf der Funktion
%
\begin{listing}[\tt]%
(plob-close)
\end{listing}%
%
\noindent geschlossen werden, da sich durch die in der letzten Phase
der Speicherr\"{u}ckgewinnung durchgef\"{u}hrten Kompaktierung des
\postore\ die \objid[s]\ der persistenten Objekte \"{a}ndern k\"{o}nnen und
damit bei ge\"{o}ffnetem \postore\ Inkonsistenzen mit den transienten
\representations{} der persistenten Objekte auftreten k\"{o}nnen.
%
\par{}Der Zustand eines Objektes kann explizit mit der Funktion
%
\begin{listing}[\tt]%
(p-destroy {\rm\it\lt{}short objid\gt\/})
\end{listing}%
%
\noindent gel\"{o}scht werden; nach dem Aufruf besteht der Zustand nur
noch aus \immval[s], die keine weiteren persistenten Objekte
referenzieren. Bei diesem Aufruf wird auch die Klasse des Objektes auf
die Klasse \class{Zombie}\ ge\"{a}ndert; eine n\"{a}chste Dereferenzierung
des persistenten Objektes f\"{u}hrt \ia\ zu einer Fehlermeldung.
%
\par{}Beim Abbruch einer Transaktion erhalten alle im Verlauf der
Transaktion erzeugten Zombies ihre vorherige Klasse und ihren
vorherigen Zustand zur\"{u}ck.
%
\subsection{Beispiele}
%
Dieser Abschnitt zeigt anhand von Beispielen die realisierten Konzepte
von \plob.
%
\subsubsection{Anzeigen von \protect\clsdo[en]}
%
In \figurename~\ref{fig:odclsdo} werden durch Aufruf der Funktion
\stfn{p-apropos-classes}\ alle \clsdo[e]\ angezeigt, deren Namen die
an diese Funktion \"{u}bergebene Zeichenkette
\lisp{\dq{}des\-crip\-tion\dq}\ enthalten. 
%
\begin{listing}%
\listener\label{enu:odclsdopa}
{\em(p-apropos-classes \dq{}description\dq)}\\
\oi\ Er\=ror: The stable heap is not open.\\
  \>1 (continue) Open it.\\
  \>2 (abort) Return to level 0.\\
Type :c followed by a number to proceed\\
\errlistener\label{enu:odclsdoer} 1 > {\em:c}\\
; Bootstrap, phase 1: \=Loaded structure description\\
; \>plob::structure-description, 10/12 slots.\\
; Bootstrap, phase 1: Loaded structure description\\
; \>plob::structure-slot-description, 8/10 slots.\\ 
{\rm usw.\ bis}\\
; LISP root formatted at 18.07.1994 14:15\\
plob::\=method-description, data:\ \#<instance class-description\\
\oii  \>method-description 1.00 slots=5/6 short-objid=8388579>\\
plob::\=effective-slot-description, data:\ \#<instance class-description\\
      \>effective-slot-description 1.00 slots=9/10 short-objid=8388581>\\
{\rm usw.\ bis}\\
plob::\=structure-description, data:\ \#<structure structure-description\\
      \>structure-description 1.00 slots=10/12 short-objid=8388602>
\caption{\"{O}ffnen des \protect\sh\ und Anzeigen der \protect\clsdo[e]}%
\label{fig:odclsdo}%
\end{listing}%
%
Da der \sh\ vor dem Aufruf geschlossen war, wird er nach R\"{u}ckfrage
bei den BenutzerInnen ge\"{o}ffnet \oi. Bei der Form der Ausgabe der
Funktion \stfn{p-apropos-classes}\ (ab \oii) habe ich mich nach der in
\lw\ realisierten \fn\ \stfn{apropos}\ \stcite{701}\ gerichtet. Vor
der Angabe \lisp{data:}\ befindet sich der Name der Klasse, dahinter
die in der 2.~Schicht generierte Druckrepr\"{a}sentation der \clsdo[e];
sie enth\"{a}lt eine Klasseninformation, den Klassennamen, die
Versionsnummer, die Anzahl der im persistenten Speicher
repr\"{a}sentierten sowie der gesamten
\Slt[s]\/ und die \sobjid\/ des persistenten \clsdo[es].
%
\subsubsection{Definition einer aktiv-persistenten Klasse}
%
Die Definition der aktiv-persistenten Klasse \class{person}\ zeigt
\figurename~\ref{fig:depcls}; ak\-tiv-per\-si\-stent wird die Klasse
durch die Klassenoption \lisp{(:metaclass
persistent-\lb{}metaclass)}\ \oii.
%
\begin{listing}%
\listener\label{enu:depclsdc}
{\em(d}\={\em efclass person ()}\\
       \>{\em(}\={\em(name }\=\em{:initarg :name
                                  :accessor person-name}\\ 
\oi    \>      \>           \>{\em :index (btree :test equal))}\\
       \>      \>{\em(age :initarg :age :accessor person-age))}\\
\oii   \>{\em(:metaclass persistent-metaclass))}\\
\#<persistent-metaclass person 100A952C>\\
\listener\label{enu:depclsmi}
{\em(setf \#l*p-1* (make-instance 'person
                                           :name \dq\nameii\dq))}\\ 
; Stored \#<instance \=class-description person 1.00 slots=2/3\\
;                    \>short-objid=8384713>.\\
\#<instance person short-objid=8384679>\\
\listener\label{enu:depclsmii}
{\em(setf \#l*p-2* (make-instance 'person :name \dq\namei\dq))}\\
\#<instance person short-objid=8384672>\\
\listener\label{enu:depclsps}
{\em (mapcar }\={\em\#'person-name}\\
 \>{\em (p-select 'person :where 'name :below \dq{}Sch\dq))}\\
(\dq\nameii\dq)
\caption{Definition einer aktiv-persistenten Klasse}%
\label{fig:depcls}%
\end{listing}%
%
F\"{u}r den \Slt\/ \stslt{name}\ wird \"{u}ber die \Slt\/-Option
\lisp{:index}\ ein Index definiert \oi. Der Index wird durch einen
persistenten B-Baum repr\"{a}sentiert (Argument \lisp{btree}\ der
\Slt\/-Option), der f\"{u}r den  Schl\"{u}sselvergleich die Zust\"{a}nde des
\Slt[s]\/ \stslt{name}\ benutzt (Argumente \lisp{:test equal}\ der
\Slt\/-Option).
%
\par{}Beim ersten Erzeugen einer Instanz einer aktiv-persistenten
Klasse wird das \clsdo\ der Klasse erzeugt und gespeichert
(\lprompt{\ref{enu:depclsmi}}). Der \Slt\/ \stslt{name}\ wird durch das
Initialisierungs-Schl\"{u}sselwort-Argument \lisp{:name}\ auf den
\"{u}bergebenen Namen gesetzt; da f\"{u}r den \Slt\/ ein Index definiert ist,
wird das erzeugte persistente Objekt assoziiert zum Namen in die
Indextabelle des \Slt[s]\/ eingetragen. In diesem Zusammenhang ist zu
beachten, da\ss{} die Schl\"{u}ssel einer Indextabelle (d.h.\ die Zust\"{a}nde
des \Slt[s], f\"{u}r den ein Index deklariert wurde, \"{u}ber alle Instanzen
einer Klasse) eindeutig sein m\"{u}ssen und damit der \Slt\/ keinen
Initialisierungsausdruck haben darf, der f\"{u}r jeden Aufruf den
gleichen Wert zur\"{u}ckgibt. Damit w\"{u}rde im Verlauf der Erzeugung einer
persistenten Instanz versucht werden, diesen immer gleichen Wert
assoziiert zur jeweils erzeugten persistenten Instanz in die
Indextabelle einzutragen; da keine doppelten Schl\"{u}ssel zul\"{a}ssig
sind, w\"{u}rde ab der zweiten erzeugten Instanz immer ein Fehler
signalisiert werden. Deswegen sollte der Initialisierungsausdruck
\ia\ weggelassen werden; damit wird der \Slt\/ nicht implizit
w\"{a}hrend der Objekterzeugung an einen Wert gebunden und sein Zustand
wird nicht implizit in die Indextabelle eingetragen.
%
\par{}Der Ausdruck \lisp{\#l*p-1*}\ (\lprompt{\ref{enu:depclsmi}})
referenziert den Wert des persistenten Symbols mit dem Namen
\lisp{*p-1*}. Zwischen dem transienten und dem persistenten Symbol
\lisp{*p-1*}\ gibt es bei dieser Art der Zuweisung keine Kopplung der
Symbolwerte; insbesondere bleibt der Wert des transienten Symbols
\lisp{*p-1*}\ unver\"{a}ndert.
%
\par{}Mit der Funktion \stfn{p-select}\ kann assoziativ nach Instanzen
gesucht werden (\lprompt{\ref{enu:depclsps}}); in diesem Fall wird
nach allen Instanzen der Klasse \class{person}\ (oder einer ihrer
Superklassen) gesucht, deren im \Slt\/ \stslt{name}\ enthaltener
Zustand literal kleiner als die Zeichenkette \lisp{\dq{}Sch\dq}\ ist.
%
\par{}\figurename~\ref{fig:plinesna} %
%
\begin{figure}[htbp]%
\centerline{\psfig{figure=plinesna.eps,width=0.83333\textwidth}}%
\caption{Beispiel f\"{u}r ein \protect\sltdo}%
\label{fig:plinesna}%
\end{figure}%
%
zeigt das effektive \sltdo\ des \Slt[s]\/ \stslt{name}\ der Klasse
\class{person}\ so, wie es vom \lw\ Inspektor visualisiert wird. Der
\Slt\/ \stslt{plob::p-location}\ gibt die Position des Elementes
innerhalb des Zustandsvektors von persistenten Instanzen der Klasse
\class{person}\ an, in dem der \Slt\/-Zustand gespeichert wird
(\figurename~\ref{fig:pcivec}, \citepage{\pageref{fig:pcivec}}). Die
im \Slt\/ \stslt{plob::p-extent}\ enthaltene Angabe
\lisp{:persistent}\ bedeutet, da\ss{} der \Slt\/ \stslt{name}\ lediglich
in der persistenten \representation{} alloziert wird; es existiert keine
\representation{} f\"{u}r den \Slt\/-Zustand in der transienten
\representation{} einer Instanz der Klasse
\class{person}\ \ifbericht%
\cite[\citepage{120}]{bib:ki94a}%
\else%
(\figurename~\ref{fig:extent},
\citepage{\pageref{fig:extent}})\fi. Der \Slt\/ \stslt{plob::p-index}\
enth\"{a}lt die Indextabelle des \Slt[s],
in diesem Fall ein B-Baum mit den zwei bisher erzeugten Elementen
(aus \figurename~\ref{fig:depcls}, \citepage{\pageref{fig:depcls}},
\lprompt{\ref{enu:depclsmi}--\ref{enu:depclsmii}}). Der B-Baum
enth\"{a}lt im Moment eine Seite mit maximal 680 Elementen vor der
n\"{a}chsten Vergr\"{o}\ss{}erung.
%
\subsubsection{Deklarationen f\"{u}r passiv-persistente Klassen}
%
Instanzen von systeminternen Klassen k\"{o}n\-nen lediglich
passiv-persistent gehalten werden \ifbericht%
\cite[\citepage{114}]{bib:ki94a}%
\else%
(\citepage{\pageref{sec:nrmlp}})\fi.
Das Problem f\"{u}r diese Klassen besteht darin, da\ss{} ihre Struktur oft
redundant ist und sehr viele Objekte referenziert. \"{U}ber die
angebotenen Deklarationen kann 
%zum einen
veranla\ss{}t werden, bestimmte
\Slt[s]\/ der Klasse nicht persistent zu halten.
%und zum anderen kann erreicht werden, da\ss{} beim Erzeugen der
%transienten \representation{} eines persistenten Objektes eine von den
%BenutzerInnen deklarierte Konstruktor-Funktion aufgerufen wird.
\figurename~\ref{fig:dclpn}
zeigt die in \lw\ f\"{u}r die Strukturklasse
\class{logical-path\-name}\ notwendigen Deklarationen:
%
\begin{listing}%
\oi\   (setf \=(class-constructor (find-class 'logical-pathname))\\
             \>'make-logical-pathname-by-plob)\\
\oii\  (d\=efu\=n make-logical-pathname-by-plob\\
         \>   \>(\&key host directory name type version)\\
         \>(\=system::make-logical-pathname-from-components\\
         \> \>host directory name type version))\\
\oiii\ (setf (class-extent (find-class 'logical-pathname))
             :cached)\\
\oiv\  (setf \=(slot-extent 'system::device
                            (find-class 'logical-pathname))\\
             \>:transient)
\caption%
 [Deklarationen f\"{u}r die systeminterne Strukturklasse
  \protect\rglq{}logical-pathname\protect\rgrq]%
 {Deklarationen f\"{u}r die systeminterne Strukturklasse
  \protect\class{logical-pathname}}%
\label{fig:dclpn}%
\end{listing}%
%
\begin{description}
%
\item[Deklaration \oi, Funktion \oii]
Die Deklaration \oi\ veranla\ss{}t, da\ss{} die in \oii\ definierte Funktion
f\"{u}r die Erzeugung der transienten \representation{} einer persistenten
Instanz der Klasse \class{lo\-gi\-cal-""path\-name}\ aufgerufen wird.
%
\item[Deklaration \oiii]
Damit wird die Lebensdauer der Zust\"{a}nde aller \Slt[s]\/ der Klasse
\class{logi\-cal-path\-name}\ als
\lisp{:cached}\ \ifbericht%
\cite[\citepage{120}]{bib:ki94a}%
\else%
(\figurename~\ref{fig:extent},
\citepage{\pageref{fig:extent}})\fi\ deklariert.
%
\item[Deklaration \oiv]
Durch die Deklaration der Lebensdauer \lisp{:transient}\ f\"{u}r die
Lebensdauer des \Slt[s]\/ \stslt{system::device}\ wird erreicht, da\ss{}
der \Slt\/ nicht in der persistenten \representation{} des Objektes
gehalten wird.
%
\end{description}
Die angegebenen Deklarationen f\"{u}r eine Klasse m\"{u}ssen vor dem
Speichern ihres \clsdo[es]\ evaluiert werden.
\iffalse
Weiter Beispiele von
Deklarationen f\"{u}r systeminterne \lw\ Klassen befinden sich im Modul
\lisp{lisp-works}; das Modul \lisp{plob-obvius}\ enth\"{a}lt
Deklarationen, die es erm\"{o}glichen, Objekte des bildverarbeitenden
Systems {\sc Obvius} persistent zu halten.
\fi
%
\ifbericht%
\def\nextstep{\relax}%
\else%
\def\nextstep{%
\let\pfsection\section%
\let\pfsubsection\subsection%
\let\pfsubsubsection\subsubsection%
% This is -*-LaTeX-*-
%
\pfsection{Performanz}\label{chap:perf}
%
\begin{fortune}[6cm]%
Don't let the sun go down on me.
\from{Elton John}
\end{fortune}%
%
\ifx\pfsection\chapter%
Dieses Kapitel \else Dieser Abschnitt\fi\ enth\"{a}lt
verschiedene Messungen der Performanz von
\plob. Gemessen wurden die elementaren Funktionen, wie Allozieren und
Zugriff auf den Zustand von persistenten Objekten, sowie Einf\"{u}gen und
Lesen von aktiv-persistenten Instanzen, bei denen ein Index f\"{u}r einen
\Slt\/ deklariert wurde. Die Tabellen geben in der Spalte 
\rglq{}LISP\rgrq\ die Zeit an, die \lw\ f\"{u}r die Ausf\"{u}hrung der Aktion
ben\"{o}tigte; in der Spalte \rglq{}BS\rgrq\ findet sich die Zeit, die
das Betriebssystem mit der Ausf\"{u}hrung der Aktion besch\"{a}ftigt
war. Die insgesamt von \plob\ ben\"{o}tigte Zeit ergibt sich aus der
Summe der in der Spalten
\rglq{}LISP\rgrq\ und \rglq{}BS\rgrq\ aufgef\"{u}hrten Zeiten.  Die Einheiten
sind in eckigen Klammern in der
Spalten\"{u}berschrift angegeben; [s] steht f\"{u}r Sekunden und [ms] f\"{u}r
Millisekunden.  Bei Ladevorg\"{a}ngen wurde f\"{u}r die Messungen darauf
geachtet, da\ss{} die Objekte tats\"{a}chlich aus dem persistenten Speicher
und nicht aus dem \cache\/ geladen wurden; bei der Benutzung von
\plob\ wird man
\ia\ daher eine wesentlich bessere Performanz als hier angegeben
erhalten. Die f\"{u}r die Messung benutzte Rechnerkonfiguration ist in
\cite{bib:ki94b}\ im Abschnitt \rglq{}Installation\rgrq\ beschrieben.
%
\newlength{\perfw}\setlength{\perfw}{0.6\textwidth}%
\newlength{\smallperfw}\setlength{\smallperfw}{0.45\textwidth}%
\newlength{\perfh}\setlength{\perfh}{\medskipamount}%
%
\pfsubsection{Transaktionen}
%
Die ermittelte Zeit ist der Verwaltungsaufwand f\"{u}r das Starten und
Beenden einer Transaktion:\\[\perfh]%
%
\begin{tabular}{|p{\perfw}|r@{,}l|r@{,}l|}
\hline
\tabularheader{Erkl\"{a}rung}
        &\multicolumn{2}{|c}{\tabularheader{LISP [ms]}}
                &\multicolumn{2}{|c|}{\tabularheader{BS [ms]}}\\
\hline\hline
\parbox[t]{\perfw}{%
Transaktion mit leerer Anweisung:\\
{\small\tt(with-transaction nil)}}\smallskip
        & 0&250  & 0&0\\
\hline
\end{tabular}
%
\pfsubsection{Sperren}
%
Bei den nachfolgenden Messungen wurde ein Vektor mit 1000
\immval[s]\/ benutzt:
%
\begin{listing}%
(setf *v* (make-array 1000 :initial-element
                           {\rm\it\lt\rm\immval\it\gt\/}))
\end{listing}%
%
\noindent Auf die transiente \representation{} wurde mit
\swizzling\ (Zugriffsart~1) sowie der direkten Manipulation der
persistenten \representation{} (Zugriffsarten~2--4)
schreibend (1.~Tabelle) und lesend (2.~Tabelle) zugegriffen. Die
Zugriffsarten unterscheiden sich in der Anzahl der Transaktionen und
der Anzahl der gesetzten Sperren, die f\"{u}r den Zugriff ausgef\"{u}hrt
wurden; die Spalten \rglq{}Tr.\rgrq\ bzw.\ \rglq{}Sp.\rgrq\ geben die
Anzahl der durchgef\"{u}hrten Transaktionen bzw.\ Sperren
an:\\[\perfh]
%
\begin{tabular}{|c|p{\smallperfw}|r|r|r@{,}l|r@{,}l|}
\hline
\tabularheader{Nr}
 &\tabularheader{Erkl\"{a}rung}
  &\tabularheader{Tr.}
   &\tabularheader{Sp.}
    &\multicolumn{2}{|c}{\tabularheader{LISP [s]}}
     &\multicolumn{2}{|c|}{\tabularheader{BS [s]}}\\
\hline\hline
1&
\parbox[t]{\smallperfw}{%
Speichern des transienten Vektors \lisp{*v*}:\\
{\small\tt(setf *objid* (store-object *v*))}}\smallskip
        & 1 & 1
                & 0&290  & 0&0\\
\hline
2&
Schreiben eines \immval[s]\/ in jede Komponente 
von \lisp{*objid*}\ mit einer Transaktion f\"{u}r alle
Schreibzugriffe und einer Sperre auf Vektorebene f\"{u}r
alle Schreibzugriffe
        & 1 & 1
                & 0&320  & 0&010\\
\hline
3&
Schreiben eines \immval[s]\/ in jede Komponente
von \lisp{*objid*}\ mit einer Transaktion f\"{u}r alle
Schreibzugriffe und einer Sperre auf Elementebene pro
Schreibzugriff
        & 1 & 1000
                & 4&020  & 0&020\\
\hline
4&
Schreiben eines \immval[s]\/ in jede Komponente
von \lisp{*objid*}\ mit einer Transaktion pro
Schreibzugriff und einer Sperre auf Elementebene pro
Schreibzugriff
        & 1000 & 1000
                & 15&470 & 0&050\\
\hline
\end{tabular}\\[\perfh]%
%
Die Me\ss{}werte zeigen, da\ss{} die Kosten des Zugriffs auf den Zustand
eines persistenten Objektes \"{u}berwiegend bei der
Transaktionsverarbeitung und beim Sperrprotokoll liegen.  Die
Zugriffsart~2 entspricht vom Zeitaufwand her der Zugriffsart~1;
Zugriffsart~3 lohnt sich f\"{u}r den schreibenden Zugriff bei bis zu
$\frac{\mbox{0,290}}{\mbox{4,020}/\mbox{1000}}$\,$\approx$\,72
Elementen, Zugriffsart~4 bei bis zu
$\frac{\mbox{0,290}}{\mbox{15,470}/\mbox{1000}}$\,$\approx$\,19
Elementen, d.h.\ ab der jeweils genannten Zahl wird der Aufwand f\"{u}r
das Sperrprotokoll so gro\ss{}, da\ss{} es sich lohnt, auf den Vektor nur
mit
\swizzling\ zuzugreifen.
\\[\perfh]
%
\begin{tabular}{|c|p{\smallperfw}|r|r|r@{,}l|r@{,}l|}
\hline
\tabularheader{Nr}
 &\tabularheader{Erkl\"{a}rung}
  &\tabularheader{Tr.}
   &\tabularheader{Sp.}
    &\multicolumn{2}{|c}{\tabularheader{LISP [s]}}
     &\multicolumn{2}{|c|}{\tabularheader{BS [s]}}\\
\hline\hline
1&
\parbox[t]{\smallperfw}{%
Laden des persistenten Vektors \lisp{*objid*}:\\
{\small\tt(setf *v* (load-object *objid*))}}\smallskip
        & 1 & 1
                & 0&280  & 0&0\\
\hline
2&
Lesen jeder Komponente von \lisp{*objid*}\ mit einer Transaktion f\"{u}r
alle Lesezugriffe und einer Sperre auf Vektorebene f\"{u}r
alle Lesezugriffe
        & 1 & 1
                & 0&280  & 0&030\\
\hline
3&
Lesen jeder Komponente von \lisp{*objid*}\ mit einer Transaktion f\"{u}r
alle Lesezugriffe und einer Sperre auf Elementebene pro
Lesezugriff
        & 1 & 1000
                & 4&230  & 0&010\\
\hline
4&
Lesen jeder Komponente von \lisp{*objid*}\ mit einer Transaktion pro
Lesezugriff und einer Sperre auf Elementebene pro Lesezugriff
        & 1000 & 1000
                & 15&580 & 0&250\\
\hline
\end{tabular}\\[\perfh]%
%
Auch hier wird der Gesamtzeitbedarf in erster Linie durch das
Sperrprotokoll bestimmt; die eigentliche \"{U}bertragung des Zustands
f\"{a}llt kaum ins Gewicht. Die Zugriffsarten~2 und 1 sind vom
Zeitaufwand her gleich; Zugriffsart~3 lohnt sich f\"{u}r den
lesenden Zugriff bei bis zu
$\frac{\mbox{0,280}}{\mbox{4,230}/\mbox{1000}}$\,$\approx$\,66
Elementen, Zugriffsart~4 bei bis zu
$\frac{\mbox{0,280}}{\mbox{15,580}/\mbox{1000}}$\,$\approx$\,18
Elementen.
%
\pfsubsection{Listen}
%
\begin{tabular}{|p{\perfw}|r@{,}l|r@{,}l|}
\hline
\tabularheader{Erkl\"{a}rung}
        &\multicolumn{2}{|c}{\tabularheader{LISP [ms]}}
                &\multicolumn{2}{|c|}{\tabularheader{BS [ms]}}\\
\hline\hline
Speichern einer einelementigen Liste
\lisp{'({\rm\it\lt\rm\immval\it\gt\/})}
        & 14&890  & 0&020\\
\hline
Laden einer einelementigen Liste
\lisp{'({\rm\it\lt\rm\immval\it\gt\/})}
        & 15&200  & 0&020\\
\hline
\end{tabular}\\[\perfh]%
%
Auch beim Speichern und Laden von Listen wird die meiste Zeit
f\"{u}r das Sperren der persistenten {\em cons\/}-Zellen ben\"{o}tigt.
%
\pfsubsection{Felder}
%
Bei den nachfolgenden Messungen wurde ein Feld benutzt, dessen
Dimensionen in etwa denen eines RGB-Farbbildes entsprechen:
%
\begin{listing}%
(setf *array* \=(make-array \='(256 256 3)\\
              \>            \>:element-type 'single-float\\
              \>            \>:initial-element 1.0e0))
\end{listing}%
%
\noindent\begin{tabular}{|p{\perfw}|r@{,}l|r@{,}l|}
\hline
\tabularheader{Erkl\"{a}rung}
        &\multicolumn{2}{|c}{\tabularheader{LISP [ms]}}
                &\multicolumn{2}{|c|}{\tabularheader{BS [ms]}}\\
\hline\hline
\parbox[t]{\perfw}{%
Erzeugen des transienten Feldes \lisp{*array*}:\\
{\small\tt(setf *array*
                (make-array {\rm\it\lt{}siehe
                oben\gt\/}))}}\smallskip
        &  35&0  &   6&0\\
\hline
\parbox[t]{\perfw}{%
Speichern des transienten Feldes \lisp{*array*}:\\
{\small\tt(setf *objid* (store-object *array*))}}\smallskip
        & 222&0  & 366&0\\
\hline
\parbox[t]{\perfw}{%
Erzeugen der transienten \representation{} und Laden des persistenten
Feldes \lisp{*objid*}:\\
{\small\tt(setf *array* (load-object *objid*))}}\smallskip
        &  97&0  &  59&0\\
\hline
\end{tabular}\\[\perfh]%
%
Das Feld besteht aus \immval[s]\/ des Typs \class{single-float}; daher
wird der Zustand der Feldelemente sowohl in der transienten als auch
in der persistenten \representation{} in einem {\em ivector\/}
abgelegt (\figurename~\ref{fig:plexarr},
\citepage{\pageref{fig:plexarr}}). Bei der \"{U}bertragung des Feldes
zwischen transientem und persistentem Speicher wird der Zustand des
{\em ivectors\/} unkonvertiert zwischen den Umgebungen kopiert; damit
erkl\"{a}rt sich die relativ hohe Betriebssystemzeit bei der \"{U}bertragung.
F\"{u}r das Kopieren des Feldzustands wird die
\clogo\ \fn\ \stfn{memcpy}\ benutzt, die wahrscheinlich byteweise
kopiert; in diesem Fall w\"{u}rde ein Ersetzen der
\Std-\clogo-\fn\ \stfn{memcpy}\ durch eine selbst realisierte
Kopierfunktion, die wortweise arbeitet,
einen erheblichen Effizienzgewinn bringen.
%
\pfsubsection{Aktiv-persistente Instanzen}
%
Die \Slt\/-Namen und -Lebensdauern in der Definition der zur Messung
verwendeten ak\-tiv-""per\-si\-sten\-ten Klasse
\class{p-example-class}\ wurden so gew\"{a}hlt, da\ss{} sie zu
\ifbericht der in \cite[\citepage{120}]{bib:ki94a} gezeigten
\figurename \else \figurename~\ref{fig:extent}
(\citepage{\pageref{fig:extent}})\fi\ passen.  Transiente Instanzen
der Klasse \class{t-example-class}\ werden zu Vergleichsmessungen
benutzt.
%
\begin{listing}%
(d\=efclass p-example-class ()\\
  \>(\=(slot-4\ \ :initform nil :extent :transient)\\
  \> \>(slot-6\ \  :initform nil :extent :cached)\\
  \> \>(slot-8\ \  :initform nil :extent :cached-write-through)\\
  \> \>(slot-10\  :initform nil :extent :persistent))\\
  \>(:metaclass persistent-metaclass))\\
\pagebreak[3]\\
(d\=efclass t-example-class ()\\
  \>(\=(slot-3 :initform nil) (slot-5 :initform nil)\\
  \> \>(slot-7 :initform nil) (slot-9 :initform nil)))
\end{listing}
%
\pfsubsubsection{Erzeugung von aktiv-persistenten Instanzen}
%
\noindent\begin{tabular}{|p{\perfw}|r@{,}l|r@{,}l|}
\hline
\tabularheader{Erkl\"{a}rung}
        &\multicolumn{2}{|c}{\tabularheader{LISP [ms]}}
                &\multicolumn{2}{|c|}{\tabularheader{BS [ms]}}\\
\hline\hline
Erzeugen einer (rein transienten) Instanz der Klasse
\class{t-example-class}
        & 0&090 & 0&0\\
\hline
Erzeugen einer persistenten Instanz der Klasse
\class{p-example-class}
        & 45&810 & 0&770\\
\hline
Laden einer persistenten Instanz der Klasse
\class{p-example-class}
        & 38&610 & 0&240\\
\hline
\end{tabular}\\[\perfh]%
%
Die Werte in der ersten Zeile zeigen zum Vergleich die Werte der
Erzeugung einer Instanz f\"{u}r eine rein transiente Klasse.
%
\pfsubsubsection{Zugriff auf den Zustand von aktiv-persistenten
Instanzen}
%
Bei der folgenden Messung wurde jeder \Slt\/-Zustand einmal auf ein
\immval\/ gesetzt.
%
\begin{listing}
(setf *objid* (make-instance 'p-example-class))\\
(setf (slot-value *objid* {\rm\it\lt{}Slot-Name\gt\/})
      {\rm\it\lt{}Immediate\gt\/})
\end{listing}
%
\noindent\begin{tabular}{|l|l|r@{,}l|r@{,}l|}
\hline
\tabularheader{\Slt\/-Name}
 &\tabularheader{\Slt\/-Lebensdauer}
  &\multicolumn{2}{|c}{\tabularheader{LISP [ms]}}
   &\multicolumn{2}{|c|}{\tabularheader{BS [ms]}}\\
\hline\hline
\lisp{slot-3} & \Slt\/ aus \class{t-example-class}
        & 0&007 & 0&0\\
\lisp{slot-4} & \lisp{:transient}
        & 0&430 & 0&0\\
\lisp{slot-6} & \lisp{:cached}
        & 0&440 & 0&0\\
\lisp{slot-8} & \lisp{:cached-write-through}
        & 1&340 & 0&0\\
\lisp{slot-10} & \lisp{:persistent}
        & 1&280 & 0&0\\
\hline
\end{tabular}\\[\perfh]%
%
Wie sich zeigt, ist die Performanz schlechter als bei rein
transienten Klassen, auch bei der \Slt\/-Lebensdauer
\lisp{:transient}\ des \Slt[s]\/ \stslt{slot-4}\ der aktiv-persistenten
Klasse \class{p-example-class}. Dies
ist darin begr\"{u}ndet, da\ss{} die entsprechenden \Slt\/-Lebensdauern aus
dem im \clsdo\ enthaltenen \sltdo\ gelesen werden; offensichtlich ist
die Ermittlung des \sltdo[es]\ relativ zeitintensiv.
%
\par{}F\"{u}r die \Slt\/-Lebensdauer \lisp{:cached-write-through}\ wird
der \Slt\/-Zustand auch in die transiente \representation{} propagiert;
damit ist der schreibende Zugriff langsamer als bei der
\Slt\/-Lebensdauer \lisp{:persistent}.
%
\par{}Bei der folgenden Messung wurde jeder \Slt\/-Zustand 
einmal gelesen.
%
\begin{listing}
(slot-value *objid* {\rm\it\lt{}Slot-Name\gt\/})
\end{listing}
%
\noindent\begin{tabular}{|l|l|r@{,}l|r@{,}l|}
\hline
\tabularheader{\Slt\/-Name}
 &\tabularheader{\Slt\/-Lebensdauer}
  &\multicolumn{2}{|c}{\tabularheader{LISP [ms]}}
   &\multicolumn{2}{|c|}{\tabularheader{BS [ms]}}\\
\hline\hline
\lisp{slot-3} & \Slt\/ aus \class{t-example-class}
        & 0&006 & 0&0\\
\lisp{slot-4} & \lisp{:transient}
        & 0&390 & 0&0\\
\lisp{slot-6} & \lisp{:cached}
        & 0&400 & 0&0\\
\lisp{slot-8} & \lisp{:cached-write-through}
        & 0&400 & 0&0\\
\lisp{slot-10} & \lisp{:persistent}
        & 1&170 & 0&0\\
\hline
\end{tabular}\\[\perfh]%
%
Erwartungsgem\"{a}\ss{} unterscheiden sich die Zeiten f\"{u}r das \Slt\/-Lesen
und -Schreiben signifikant bei der \Slt\/-Lebensdauer
\lisp{:cached-write-through}\ des \Slt[s]\/ \stslt{slot-8}\ der
aktiv-persistenten Klasse \class{p-example-class}; Schreibzugriffe auf
den \Slt\/ werden sowohl in die transiente als auch in die persistente
\representation{} des persistenten Objektes propagiert, w\"{a}hrend
Lesezugriffe immer die transiente \representation{} referenzieren.
%
\pfsubsubsection{Indexdeklarationen f\"{u}r aktiv-persistente Klassen}
%
F\"{u}r die letzte Messung wurde eine aktiv-persistente Klasse mit einem
Index benutzt. Der \Slt\/ \stslt{slot-i}, f\"{u}r den der Index
\lisp{(btree :test equal)}\ deklariert wurde, wurde bei der
Objekterzeugung initialisiert; damit wurde die Assoziation vom
\Slt\/-Zustand auf das Objekt im Verlauf der Erzeugung der
aktiv-persistenten Instanz aufgebaut:%
%
\begin{listing}%
(d\=efclass p-example-index-class ()\\
  \>(\=(slot-i \=:initarg :slot-i\\
  \> \>        \>:index (btree :test equal)\\
  \> \>        \>:extent :persistent))\\
  \>(:metaclass persistent-metaclass))
\end{listing}%
%
\newlength{\onespace}%
\noindent\begin{tabular}{|p{\perfw}|r@{,}l|r@{,}l|}
\hline
\tabularheader{Erkl\"{a}rung}
        &\multicolumn{2}{|c}{\tabularheader{LISP [ms]}}
                &\multicolumn{2}{|c|}{\tabularheader{BS [ms]}}\\
\hline\hline
\parbox[t]{\perfw}{%
Erzeugung einer Instanz der Klasse \class{p-example-index-class}:\\
{\tt\small\settowidth{\onespace}{\ }%
(make-instance\\
\hspace*{\onespace}'p-example-index-class\\
\hspace*{\onespace}:slot-i \dq{}Slot-I with string
                               {\rm\it\lt{}Nummer\gt\/}\dq)}}\smallskip
        & 114&800 & 0&970\\
\hline
\parbox[t]{\perfw}{%
Laden einer Instanz  der Klasse \class{p-example-index-class}:\\
{\tt\small\settowidth{\onespace}{\ }%
(p-select 'p-example-index-class\\
\hspace*{10\onespace}:where 'slot-i)}}\smallskip
        & 64&490 & 0&250\\
\hline
\end{tabular}\\[\perfh]%
%
Im Verlauf der Objektinitialisierung wird die Zeichenkette
abgespeichert und die Assoziation vom \Slt\/-Zustand auf das Objekt
in die Indextabelle eingetragen; damit ergibt sich ein h\"{o}herer
Zeitbedarf.
%
\pfsubsection{Bewertung der Performanz}
%
Wie insbesondere die erste Messung zeigt, ist der Zeitbedarf f\"{u}r das
Sperrprotokoll in Relation zur eigentlichen Zustands\"{u}bertragung recht
hoch. In der 2.~Schicht  k\"{o}nnten noch Optimierungen durchgef\"{u}hrt
werden, da dort im Moment das Sichern des Zustands der
\postore-Vektoren in die UNDO-{\em Log\/}-Datei mit den relativ
ineffizienten {\em Stream\/}-Funktionen der \std-\clogo-Bibliothek
erfolgt.
%
\par{}In der 3.~Schicht lie\ss{}e sich eine Verbesserung der Performanz
beim Zugriff auf \Slt[s]\/ von aktiv-persistenten Instanzen durch
Optimierung der \spc[n]\ \mtd[n]\ der \gfn[n]\ 
\stfn{make-accessor-lambda}\ \ifbericht
\cite[\citepage{111}]{bib:ki94a} \else (Protkoll~\ref{pro:defclass}
\stfn{(defclass)}, Schritt~\ref{enu:dcmcr},
\citepage{\pageref{enu:dcmcr}})\fi\ erreichen. Sofern Wert auf
Performanz gelegt wird, sollten in der 4.~Schicht Transaktionen
m\"{o}glichst gro\ss{}e Bl\"{o}cke umfassen.
%

%%% Local Variables: 
%%% mode: latex
%%% TeX-master: "main"
%%% End: 
}%
\fi%
\nextstep%
%
\section{Zusammenfassung und Bewertung}%
\label{sec:plsac}%
%
Realisiert wurde ein persistentes System unter Verwendung eines
Objektspeichers, dessen Datenmodellierung sich an LISP orientiert.
Die M\"{o}glichkeit der Persistenz ist f\"{u}r den Zustand aller LISP
Objekte un\-ab\-h\"{a}n\-gig vom Typ gegeben; insbesondere ist f\"{u}r
Persistenz von \clos-Objekten nicht die Einbindung einer
\spc[n]\ \clsmc\ in ihre Klassendefinition notwendig. Die
Schnittstelle f\"{u}r die Benutzung des persistenten Systems orientiert
sich soweit m\"{o}glich an den in \cite{bib:st90} dargestellten
Konzepten; damit wird die Benutzung des Systems sehr einfach und die
Einbindung in vorhandene Programmodule erleichtert.
%
\par{}Durch die realisierte M\"{o}glichkeit, Persistenz deklarativ auch
f\"{u}r beliebige, im System vorgefundene Klassen beeinflussen zu
k\"{o}nnen, wird eine hohe Effizienz des Systems erreicht.
%
\par{}Die Realisierung von Sitzungen, \twophasetrs,
des hierarchischen Sperrprotokolls und einer Indexverwaltung f\"{u}r
eindimensionale Schl\"{u}ssel bietet die grundlegenden
Funktionalit\"{a}ten einer Datenbank.
%
\par{}Insgesamt bietet das hier realisierte System Persistenz mit
LISP-gem\"{a}\ss{}er Datenmodellierung mit guter Performanz
\ifbericht(Anhang~\ref{chap:perf},
\citepage{\pageref{chap:perf}})\fi\ an.
%
\par{}Noch nicht realisiert ist assoziativer Zugriff mit
mehrdimensionalen Schl\"{u}sseln. Ebenso sind die Anfragem\"{o}glichkeiten
noch sehr einfach gehalten. Echte Mehr-BenutzerInnen-Zugriffe sind
im Moment nicht m\"{o}glich, da der in der 1.~Schicht verwendete
Objektspeicher \postore\ nicht {\em reentrant\/}-f\"{a}hig ist.
%%% Local Variables: 
%%% mode: latex
%%% TeX-master: "main"
%%% buffer-file-coding-system: raw-text-unix
%%% End: 
