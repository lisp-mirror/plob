%
\ifbericht
\title{%
Persistenz in der objekt-orientierten
Programmiersprache \clos\ am Beispiel
des \plob\ Systems}
\else\ifbuch
\title{%
{\large Heiko Kirschke}\\[1.5em]%
Persistenz in objekt-orientierten
Programmier-\\
sprachen am Beispiel von \clos}
\else
\title{%
Persistenz in objekt-orientierten
Programmier-\\
sprachen am Beispiel von \clos%
\\[\bigskipamount]
{\rm\large Diplomarbeit}}
\fi\fi
%
\ifbuch\else\author{%
Heiko Kirschke\ifbericht\else%
%\\Beim Alten Sch\\"{u}tzenhof 4\\22083 Hamburg%
\\[3em]
%
Betreuung:\\[\smallskipamount]
Prof.\ Dr.\ Leonie Dreschler-Fischer\\
Universit\"{a}t Hamburg\\
Fachbereich Informatik\\
Arbeitsbereich Kognitive Systeme\\[\smallskipamount]
%
und\\[\smallskipamount]
%
Dr.\ Volker Haarslev\\
Universit\"{a}t Hamburg\\
Fachbereich Informatik\\
Arbeitsbereich Kognitive Systeme\\[\bigskipamount]
%
\vfill%
{\small Universit\"{a}t Hamburg $\bullet$
Fachbereich Informatik $\bullet$
Vogt-K\"{o}lln-Stra\ss{}e 30 $\bullet$
22527 Hamburg}\fi}
\fi%\ifbuch
%
\maketitle
%
\thispagestyle{empty}\clearpage%
%
\ifbuch\else\par\noindent\centerline{\textbf{\Large Zusammenfassung}}
%
\vspace*{\smallskipamount}\par
Die bisher gebr\"{a}uchliche Methode zum Speichern von langlebigen,
strukturierten Daten ist die Benutzung einer Datenbank, die den
BenutzerInnen Datentypen und Operationen \"{u}ber die abgelegten Daten
zur Verf\"{u}gung stellt. Neben der Anfragesprache bietet eine Datenbank
oft eine Schnittstelle zu einer Programmiersprache an. Diese
Schnittstelle stellt lediglich die Funktionalit\"{a}t der Datenbank zur
Verf\"{u}gung, richtet sich \ia\ jedoch nicht nach den Konzepten der
Programmiersprache, die die Datenbank verwendet.  Eine L\"{o}sung dieses
Problems besteht darin, die Trennung zwischen Datenbank und
Programmiersprache aufzuheben, d.h.\ die Datenbankkonzepte m\"{o}glichst
vollst\"{a}ndig in eine Programmiersprache zu integrieren. In diesem
Bericht wurde der Ansatz untersucht und realisiert, das
\cl\ Object System um Persistenz zu erweitern.
%
\vspace*{2\bigskipamount}%
\par\noindent\centerline{\textbf{\Large Abstract}}
%
\vspace*{\smallskipamount}\par
The common approach used for storing longlived structured data is to
employ a relational database which provides its users with types for
structuring data and operations to manipulate them. Along with its
data definition and manipulation language, a database often offers an
interface to a general programming language. This interface is a
possibility to access the database's functionality, but it neglects
the concepts of the programming language using the database.  A
solution of this problem is to remove the dichotomy between a database
and a programming language, i.e.\ to integrate the concepts of a
database into the general programming language. This report presents
research done on persistency and a realization of extending the
\cl\ Object System by persistency.
%
\vspace*{2cm}%
%
\par\noindent\textit{Schl\"{u}sselw\"{o}rter}\ --
Persistenz, objekt-orientierte Datenbank, persistente Objekte,
persistente \clos-Objekte.
%
\vspace*{\smallskipamount}%
%
\par\noindent\textit{Index Terms}\ --
Persistency,
object-oriented database,
persistent objects,
persistent \clos\ objects.
%
\vspace*{3cm}
%
\begin{small}%
\ifbericht\else
\par\noindent%
Eine leicht gek\"{u}rzte Fassung dieser Diplomarbeit ist als Bericht
FBI-HH-B-\thisreportnumber/95 beim Fachbereich Informatik der
Universit\"{a}t Hamburg erh\"{a}ltlich.\\[\medskipamount]
\fi%
\par\noindent%
EMail-Adresse des Autors:
\texttt{kirschke@informatik.uni-hamburg.de}
\par\noindent%
World Wide Web:
\texttt{http://lki-www.informatik.uni-hamburg.de/%
\td{}kirschke/home.html}
\end{small}
\thispagestyle{empty}\clearpage%
\fi%\ifbuch
%
% 13.09.93: Der Lichtenberg-Spruch wird nur ausgegeben, wenn \nothink
% eingeschaltet ist, d.h. wenn ein Text ohne Kommentare erzeugt werden
% soll.
\ifdothink\else%
%
\thispagestyle{empty}\cleardoublepage%
%
\vspace*{\fill}
\begin{fortune}
Es ist fast mit der Mathematik wie mit der Theologie. So wie die der
letzteren Beflissenen, zumal wenn sie in \"{A}mtern stehen, Anspruch auf
einen besonderen Kredit von Heiligkeit und eine n\"{a}here Verwandschaft
mit Gott machen (obgleich sehr viele darunter wahre Taugenichtse sind),
so verlangt sehr oft der sogenannte Mathematiker f\"{u}r einen tiefen
Denker gehalten zu werden, ob es gleich darunter die gr\"{o}\ss{}ten
Plunderk\"{o}pfe gibt, die man nur finden kann, untauglich zu irgendeinem
Gesch\"{a}ft, das Nachdenken erfordert, wenn es nicht unmittelbar durch
jene leichte Verbindung von Zeichen geschehen kann, die mehr das Werk
der Routine als des Denkens sind.
%\from{Georg Christoph Lichtenberg \cite[S.\ 46]{li50}}
\from{Georg Christoph Lichtenberg}
\end{fortune}
%
\fi
\ifbericht\else\ifbuch\else%
\thispagestyle{empty}\clearpage%
%
\vspace*{100pt}\par\noindent Die Erstellung dieser Arbeit w\"{a}re ohne die
Liebe und das Vertrauen meiner Frau Nicola nicht m\"{o}glich gewesen; ihr
ist diese Arbeit gewidmet.
%
\bigskip\par\noindent Danken m\"{o}chte ich meiner Betreuerin Frau
Prof.\ Dr.\ Leonie Dreschler-Fischer f\"{u}r die Anregung zum Thema
dieser Arbeit; mein Dank gilt Herrn Dr.\ Volker Haarslev f\"{u}r die
\"{U}bernahme der Zweitbetreuung.
%
\par Herr Carsten Schr\"{o}der hat mir durch seine kompetente
Un\-ter\-st\"{u}t\-zung sehr geholfen; dies begann mit den ersten
Hinweisen auf interessante Literatur und setzte sich fort mit der
Un\-ter\-st\"{u}t\-zung bei der Einarbeitung in das Thema und die
verwendeten Systeme. Seine konstruktiven Rat\-schl\"{a}\-ge flossen an
vielen Stellen sowohl in die schriftliche Darstellung meiner Konzepte
in dieser Arbeit als auch in deren Realisierung ein. Daf\"{u}r bin ich
ihm zu gro\ss{}em Dank verpflichtet.
%
\par Viele andere Mitarbeiter des Fachbereichs Informatik haben mich
ebenfalls unterst\"{u}tzt; danken m\"{o}chte ich Herrn Harald Lange f\"{u}r die
Durchsicht meines ersten Entwurfs und Herrn Ralf M\"{o}ller f\"{u}r die
Hinweise, die er mir zukommen lie\ss{}. Last but not least m\"{o}chte ich
mich bei Frau Ingeborg Heer-M\"{u}ck und Herrn J\"{o}rn Tellkamp f\"{u}r den
reibungslosen Betrieb der Rechner des Arbeitsbereiches Kognitive
Systeme bedanken.
%
\fi\fi%
\thispagestyle{empty}\clearpage%
\thispagestyle{empty}\cleardoublepage
%%% Local Variables: 
%%% mode: latex
%%% TeX-master: "main"
%%% buffer-file-coding-system: raw-text-unix
%%% End: 
