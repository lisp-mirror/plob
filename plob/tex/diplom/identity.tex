% This is -*-LaTeX-*-
%
\chapter{Identit\"{a}t}%
\label{chap:idty}%
%
\begin{fortune}
Heute betrachtet sich das Vieh darin [im Spiegelbild eines Flusses,
Anm.\ d.\ Verf.]\ und sieht sich schwimmen, wenn es daran
entlangl\"{a}uft, und H\"{a}nfling und Buchfink verdoppeln sich darin, ohne
ihre Einzigkeit zu verlieren. Sie entstehen darin wieder, ohne zu
sterben, und wundern sich, da\ss{} ihnen in solch kaltem Nest im Nu Junge
ausschl\"{u}pfen, die genauso gro\ss{} sind wie sie selber.
\from{Cyrano De Bergerac: Herzstiche}
\end{fortune}
%
Die Objekte innerhalb eines Objektsystems stehen nicht f\"{u}r sich
alleine, sondern sind durch Referenzen untereinander verbunden. Eine 
Referenz stellt einen eineindeutigen Bezug zur referenzierten Instanz
her. Ein Objekt wird durch seine Identit\"{a}t unterscheidbar von allen
anderen Objekten; Eineindeutigkeit f\"{u}r eine Referenz ist daher dann
gew\"{a}hrleistet, wenn sie aus der Identit\"{a}t des Objektes bestimmt
wird. Dieses Kapitel besch\"{a}ftigt sich deswegen mit dem
Identit\"{a}tsbegriff allgemein und spezieller mit den m\"{o}glichen
\representationforms\ f\"{u}r Identit\"{a}t von transienten und
persistenten Objekten zum Aufbau von Referenzen zwischen Instanzen.
%
\par{}Mit Identit\"{a}t wird die Eigenschaft bezeichnet, da\ss{} ein Ding
oder ein Sachverhalt nur mit sich selbst in allen
\gande{Merkmalen}{properties}\ \"{u}bereinstimmt
\cite[\citepage{543}]{bib:kl76} \cite[\citepage{9}]{bib:zd90}.%
\footnote{Dagegen ist Gleichheit lediglich die \"{U}bereinstimmung in
wesentlichen Merkmalen zweier nicht-identischer Dinge
\cite[\citepage{303}]{bib:schm82}.} Laut {\em Identit\"{a}tsprinzip\/}
sind zwei Dinge {\it x\/} und {\it y\/} genau dann identisch, wenn
alle Merkmale {\it P\/} zwischen diesen beiden Dingen \"{a}quivalent
sind:
%
\begin{equation}
(x \equiv y) %
\stackrel{\mbox{\rm\footnotesize\it def}}{=}
\bigwedge_{P} P(x) \leftrightarrow P(y)
\end{equation}
%
\par{}In der hier vorliegenden Form kann das Identit\"{a}tsprinzip
(lediglich) verwendet werden, um ein Individuum gegen andere
abzugrenzen, indem z.B.\ alle Individuen zu einem bestimmten Zeitpunkt
betrachtet werden; das Identit\"{a}tsprinzip erm\"{o}glicht in diesem Fall
die Bestimmung einer Identit\"{a}t f\"{u}r jedes einzelne Individuum, da es
sich von allen anderen in mindestens einem Merkmal
unterscheidet. Diese sogenannte {\em abstrakte Identit\"{a}t\/} ist die
triviale Feststellung, da\ss{} jedes Individuum genau mit sich selbst
identisch ist und wird auch als {\it x}\,=\,{\it x} formuliert.
%
\par{}Der nicht-triviale Fall besteht darin, da\ss{} sich Merkmale
eines Individuums \"{a}ndern k\"{o}nnen. Bei einer \"{A}nderung bleibt die
abstrakte Identit\"{a}t gegen\"{u}ber allen anderen Individuen erhalten,
aber das Identit\"{a}tsprinzip gilt in der \og[en]\ Form nicht mehr f\"{u}r
das Individuum selbst vor und nach einer \"{A}nderung.  Trotzdem bleibt
das Individuum mit sich selbst gleich, es ist {\em
kon\-kret-""iden\-tisch}. Die Merkmale eines Individuums lassen sich
in konstante und variable Merkmale einteilen; da die sich ergebenden
\"{A}nderungen der variablen Merkmale \ia\ nicht vorhersehbar sind, kann
ein eingeschr\"{a}nktes Identit\"{a}tsprinzip zur Identit\"{a}tsfeststellung
eines Individuums verwendet werden, wobei die Einschr\"{a}nkung darin
besteht, nicht alle, sondern ausschlie\ss{}lich die konstanten Merkmale
zu betrachten. Die Einteilung in konstante und variable Merkmale mu\ss{}
je nach Verwendungszweck angemessen vorgenommen werden, so da\ss{} die
konkrete Identit\"{a}t eines Individuums hinreichend feststellbar wird.
%
\par{}Beispielsweise sind variable Merkmale bei einer Person die
Kleidung oder die Umgebung, in der sich die Person befindet, wie die
aktuelle Postanschrift; ein oft konstantes Merkmal ist der Name; immer
konstante Merkmale sind Geburtstag und -ort. Bei Immobilien ist
hingegen die Ortsangabe ein konstantes Merkmal, w\"{a}hrend die aktuellen
BesitzerInnen oder BewohnerInnen eines solchen Objektes variable
Merkmale sind.
%
\par{}Die in diesem Zusammenhang interessierenden Individuen sind
Daten im Sinne der (objekt-orientierten) Programmierung.
Programmierung verfolgt die Absicht, einen Teil der realen Welt als
Modell nachzubilden. Dazu wird \"{u}ber die realen Gegebenheiten
abstrahiert. Aus den einzelnen konkreten Sachverhalten werden
allgemeing\"{u}ltige Gesetzm\"{a}\ss{}igkeiten abgeleitet und in eine
Modellwelt \"{u}bertragen; unwesentliche Dinge werden im Modell
weggelassen \cite[\citepage{538}]{bib:wa89}. Ein Datenobjekt
ist die Abbildung eines Dings oder Sachverhaltes der realen Welt in
die Modellwelt. Ein reales Ding besitzt eine Identit\"{a}t; deswegen ist
es sinnvoll, den Identit\"{a}tsbegriff der realen Welt auch auf die durch
die Programmierung geschaffene Modellwelt zu \"{u}ber\-tra\-gen.
%
\section{Identit\"{a}tsrepr\"{a}sentation}\label{sec:idrepr}
%
Identit\"{a}t ist eine zu jedem Individuum geh\"{o}rige Eigenschaft; wenn
ein System die Identit\"{a}t der in ihm enthaltenen Individuen
widerspiegelt, mu\ss{} die Identit\"{a}t der einzelnen Individuen durch eine
geeignete Form repr\"{a}sentiert werden. Diese Form kann zur
Identifikation (lat.\ {\em identitas\/} \rglq{}Wesenseinheit\rgrq\ und
{\em facere\/} \rglq{}machen\rgrq{} \cite{bib:wa68}) eines Individuums
benutzt werden. F\"{u}r Individuen, Identit\"{a}t und
\representationforms\ f\"{u}r Identit\"{a}t gelten folgende
Grunds\"{a}tze:
%
\begin{enumerate}
%
\item Es gibt Individuen; jedes Individuum hat eine eigene, genau ihm
zu\-ge\-h\"{o}\-ri\-ge Iden\-ti\-t\"{a}t\ %
\linebreak[4]\cite[\citepage{284}]{bib:pat88}.
%
\item Es gibt keine Identit\"{a}t ohne ein dazugeh\"{o}riges Individuum.
%
\item Die Identit\"{a}t eines Individuums kann durch eine geeignete Form
repr\"{a}sentiert werden.
%
\end{enumerate}
%
Der Rest dieses Kapitels besch\"{a}ftigt sich mit
Identit\"{a}tsrepr\"{a}sentationen und geht insbesondere der Frage nach, was
eine \rglq{}geeignete Form\rgrq\ ist.
%
\subsection{Referenzen zwischen Individuen}
%
Eine Identit\"{a}tsrepr\"{a}sentation kann als Stellvertreter f\"{u}r das
Individuum selbst benutzt werden. Damit wird es erst m\"{o}glich,
Referenzen zwischen Individuen aufzubauen. Umgekehrt macht es das
Fehlen einer solchen Repr\"{a}sentation unm\"{o}glich, Referenzen zwischen
Objekten zu etablieren; in diesem Fall kann ein Objekt selbst
lediglich in einem \rglq{}umfassenden\rgrq\ Objekt als Wert enthalten
sein, da es nicht referenzierbar ist.
%
\par{}\figurename~\ref{fig:kartei} %
%
\begin{figure}[htbp]%
\ifloriot%
\ifbuch%
\centerline{\psfig{figure=twelvept/kartei.eps}}%
\else%
\centerline{\psfig{figure=tenpt/kartei.eps}}%
\fi%
\else%
\ifbuch%
\centerline{\psfig{figure=twelvept/karteic.eps}}%
\else%
\centerline{\psfig{figure=tenpt/karteic.eps}}%
\fi%
\fi%
\caption{Identit\"{a}ten, Identit\"{a}tsrepr\"{a}sentationen
und Referenzen in einer Personenkartei}%
\label{fig:kartei}%
\end{figure}%
%
zeigt als Beispiel den Ausschnitt einer Personenkartei. Es sind
insgesamt vier Individuen beteiligt: zwei Personen (Dr.\ Kl\"{o}b\-ner
\oiv\ und Herr M\"{u}l\-ler-""L\"{u}\-den\-scheid \ovii) sowie zwei
Karteikarten (Karten \oi\ und \ov). Jedes dieser Individuen hat eine
eigene Identit\"{a}t. Die Identit\"{a}ten der Individuen werden \"{u}ber ihre Namen
repr\"{a}sentiert (\tablename~\ref{tab:kartei}).
%
\begin{figure}[hbtp]\centering
%
\begin{minipage}{\textwidth}\centering% \minipage wg. Fussnote
\let\footnoterule\relax% Kein Trennungsstrich fuer Fussnoten
\newlength{\dqwidth}%
%
\begin{\figurefontsize}%
\begin{tabular}{|p{16em}|p{8em}|}%
\hline%
$\downarrow$%
\footnote{Zu lesen: Person mit dem Namen Kl\"{o}bner wird
repr\"{a}sentiert durch den Namen \lisp{\dq{}Kl\"{o}bner\dq}}%
\ \tabularheader{wird repr\"{a}sentiert durch den}
        &\tabularheader{Namen}\ $\downarrow$\\
\hline\hline
Person mit dem Namen \rglq{}Kl\"{o}bner\rgrq%
        &\lisp{\dq{}Kl\"{o}bner\dq}\\
\hline
Person mit dem Namen \rglq{}M\"{u}ller-\lb{}L\"{u}denscheid\rgrq%
        &\lisp{\settowidth{\dqwidth}{\dq}%
               \dq{}M\"{u}ller-\newline\hspace*{\dqwidth}L\"{u}denscheid\dq}\\
\hline
Karte, die die Person mit dem Namen
\rglq{}Kl\"{o}bner\rgrq\ repr\"{a}sentiert%
        &\lisp{<KARTE\us{}12>}\\
\hline
Karte, die die Person mit dem Namen
\rglq{}M\"{u}ller-\lb{}L\"{u}denscheid\rgrq\ repr\"{a}sentiert%
        &\lisp{<KARTE\us{}25>}\\
\hline
\end{tabular}%
\end{\figurefontsize}%
\end{minipage}%
%
\let\figurename\tablename%
\caption{Identit\"{a}tsrepr\"{a}sentation in einer Kartei}%
\label{tab:kartei}%
\end{figure}%
%
\"{U}ber diese Re\-pr\"{a}\-sen\-ta\-ti\-ons\-for\-men k\"{o}nnen die in
\figurename~\ref{fig:kartei} eingezeichneten Referenzen aufgebaut
werden. Da die Referenz zwischen Person und dazugeh\"{o}riger Karteikarte
\"{u}ber den Namen der Person aufgebaut wird, darf er sich nicht \"{a}ndern,
da sonst die Aufl\"{o}sung der Referenz nicht mehr m\"{o}glich ist. Ebenso
mu\ss{} der Name eindeutig sein.
%
\subsection{Bedingungen f\"{u}r eine
Identit\"{a}tsrepr\"{a}sentation}\label{sec:cobjid}%
%
Folgende Bedingungen gelten f\"{u}r eine Identit\"{a}tsrepr\"{a}sentation
von Datenobjekten:
%
\begin{description}
%
\item[Eindeutigkeit] Sie mu\ss{} eineindeutig sein. Jedes
Datum mu\ss{} eine eigene Iden\-ti\-t\"{a}ts\-re\-pr\"{a}\-sen\-ta\-tion
erhalten, die sich von allen anderen gleichzeitig erreichbaren
Datenobjekten unterscheidet, da sonst keine eindeutige Identifikation
\"{u}ber diese \representationform{} m\"{o}glich ist.
%
\item[Unabh\"{a}ngigkeit vom aktuellen Zustand] Sie
darf nicht von Bestandteilen des Datenobjektes abh\"{a}ngig sein, deren
Zustand sich w\"{a}hrend seiner Lebensdauer \"{a}ndern k\"{o}nnten, da sonst
Referenzen auf das Datenobjekt ung\"{u}ltig werden.
%
\item[Unabh\"{a}ngigkeit von der Umgebung] Reale Objekte behalten ihre
Identit\"{a}t bei einem Umgebungswechsel; analog dazu beh\"{a}lt auch ein
Datenobjekt unabh\"{a}ngig von der aktuellen Umgebung seine Identit\"{a}t.
F\"{u}r die gew\"{a}hlte \representationform{} von Identit\"{a}t folgt daraus,
da\ss{} sie davon unabh\"{a}ngig sein mu\ss{}, falls sich diese w\"{a}hrend der
Lebensdauer des Objektes \"{a}ndern k\"{o}nnte.
%
\item[Zeitliche Unabh\"{a}ngigkeit] Identit\"{a}t von realen Objekten ist
invariant gegen\"{u}ber Ver\"{a}nderung der Zeit w\"{a}h\-rend ihrer
Lebensdauer; gleiches gilt f\"{u}r Datenobjekte. Daher mu\ss{} die
\representationform{} f\"{u}r Identit\"{a}t konstant bleiben \"{u}ber die
Zeitdauer, w\"{a}hrend der das Datenobjekt existiert.
%
\end{description}
%
% Fuer den folgenden Satz:
% (C) by Carsten Schroeder, 15.09.1993, ca. 14:30h :
Das Umfeld, in dem sich die zusammengeh\"{o}rigen Objekte befinden,
bestimmt, welche Re\-pr\"{a}\-sen\-ta\-tion f\"{u}r Identit\"{a}t gew\"{a}hlt
werden kann, damit diese Bedingungen erf\"{u}llt sind.
%
Wenn beispielsweise alle Objekte ausschlie\ss{}lich in einem
nicht-verteilten abgeschlossenem System enthalten sind, gen\"{u}gt es,
wenn die Identit\"{a}tsrepr\"{a}sentation innerhalb dieses Systems eindeutig
ist.  Bei verteilten Systemen ist es notwendig, da\ss{} die
Repr\"{a}sentation eindeutig f\"{u}r alle beteiligten Prozesse ist.
%
\subsection{Forderungen an eine
Identit\"{a}tsrepr\"{a}sentation}
%
Au\ss{}er diesen notwendigen Bedingungen sollte eine
Identit\"{a}tsrepr\"{a}sentation zus\"{a}tzliche Forderungen erf\"{u}llen.
%
\begin{description}
%
\item[Effiziente Realisierung] Bei Systemen mit sehr vielen
Objekten sollte die Iden\-ti\-t\"{a}ts\-re\-pr\"{a}\-sen\-ta\-tion
m\"{o}glichst kurz sein; sie sollte insbesondere in einem vern\"{u}nftigen
Verh\"{a}ltnis zur mittleren Gr\"{o}\ss{}e der Datenobjekte selbst stehen.
%
\item[M\"{o}glichst hoher lokaler Informationsgehalt] Eine
Identit\"{a}tsrepr\"{a}sentation sollte soviel Information wie m\"{o}glich
\"{u}ber das sie repr\"{a}sentierende Objekt enthalten
\cite[\citepage{244}]{bib:he89}. Die Art der Information
ist abh\"{a}ngig vom beabsichtigten Verwendungszweck der Objekte. Wenn
eine Identit\"{a}tsrepr\"{a}sentation z.B.\ Komponenten enth\"{a}lt, die eine
Lokalisierung des dazugeh\"{o}rigen Datenobjektes erm\"{o}glichen, kann der
Zugriff auf den Objektzustand effizienter realisiert werden. Bei
Systemen mit Typbindung der Datenobjekte zur Laufzeit ist es f\"{u}r
Typpr\"{u}fungen von Vorteil, wenn bereits die
Iden\-ti\-t\"{a}ts\-re\-pr\"{a}\-sen\-ta\-tion R\"{u}ck\-schl\"{u}s\-se auf den
Typ des Datenobjektes zul\"{a}\ss{}t, da dann kein Zugriff auf die
Typinformation des Datenobjektes selbst notwendig ist. Je mehr
Information in einer Iden\-ti\-t\"{a}ts\-re\-pr\"{a}\-sen\-ta\-tion
enthalten ist, um so gr\"{o}\ss{}er ist sie; damit bedingt ein hoher
Informationsgehalt eine geringere Platzeffizienz.
%
\par{}Objekte werden durch Identit\"{a}tsrepr\"{a}sentationen
referenziert; f\"{u}r sie ergibt sich bei Identit\"{a}tsrepr\"{a}sentationen
mit Informationsgehalt die Einschr\"{a}nkung, da\ss{} sich diese, das Objekt
betreffenden Informationen nicht \"{a}ndern d\"{u}rfen, weil sonst die in
allen das Objekt referenzierenden Identit\"{a}tsrepr\"{a}sentationen
enthaltenen Informationen ung\"{u}ltig werden. Eine Pr\"{u}fung, ob dieser
Fall eingetreten ist mit anschlie\ss{}ender Korrektur der Informationen
in der Identit\"{a}tsrepr\"{a}sentation ist nicht sinnvoll zu realisieren,
da die Pr\"{u}fung selbst weitere Dereferenzierungen (z.B.\ des
Objektzustands) nach sich zieht und damit der Effizienzvorteil
verloren geht.
%
\item[Kanonische Identit\"{a}tsrepr\"{a}sentation]
Eine kanonische Identit\"{a}tsrepr\"{a}sentation erm\"{o}glicht eine
einheitliche Verarbeitung aller im System auftretenden
Identit\"{a}tsrepr\"{a}sentationen unabh\"{a}ngig vom Typ des Objektes, das sie
repr\"{a}sentieren. Damit wird die Erstellung eines solchen Systems
vereinfacht und die \"{U}bersichtlichkeit erh\"{o}ht. Eine kanonische Form
kann bei Datenobjekten von einfachen, skalaren Typen dazu f\"{u}hren,
da\ss{} die Re\-pr\"{a}\-sen\-ta\-tions\-form mehr Platz verbraucht als das
Objekt, das sie repr\"{a}sentiert; zur Vermeidung dessen d\"{u}rfen in
\cl\ Daten von einfachen Typen (z.B.\ Daten der Typen
\class{fixnum}\ \stcite{288}\ oder \class{character}\ \stcite{371})
nicht-kanonisch repr\"{a}sentiert werden.
%
\par{}Ein Beispiel f\"{u}r eine kanonische Form ist die in diesem Kapitel
verwendete Notation f\"{u}r die Identit\"{a}tsrepr\"{a}sentation von
Objekten. Sie besteht aus der Konkatenation einer \"{o}ff\-nen\-den
spitzen Klammer mit dem Namen des Objekt-Typs, einem Unterstrich,
einer beliebig ge\-w\"{a}hl\-ten und innerhalb des Typs eindeutigen
Nummer und einer schlie\ss{}enden spitzen Klammer
(z.B.\ \lisp{<ANGESTELLTER\us{}23>}).  Aus dieser kanonischen Form
l\"{a}\ss{}t sich damit zus\"{a}tzlich bereits der Typ des durch sie
repr\"{a}sentierten Objektes ableiten.
%
\item[Sp\"{a}tes \rglq{}Materialisieren\rgrq\ des Datenobjektes] Eine
Identit\"{a}tsrepr\"{a}sentation dient \ua\ als Referenz auf das
dazugeh\"{o}rige Datenobjekt. Eine Dereferenzierung besteht daraus, eine
Abbildung zu evaluieren, die mit der Identit\"{a}tsrepr\"{a}sentation als
Argument den Zugriff auf den Zustand des sie repr\"{a}sentierenden
Objektes erm\"{o}glicht. Sp\"{a}testens zu diesem Moment mu\ss{} das
Datenobjekt \rglq{}materialisiert\rgrq\ sein, d.h.\ in einer direkt
dereferenzierbaren Repr\"{a}sentation (z.B.\ im transienten Speicher)
vorliegen \cite[\citepage{240}]{bib:he89}. Vorher besteht diese
Notwendigkeit nicht; es ist also ausreichend, bis zur ersten
Dereferenzierung lediglich die Identit\"{a}tsrepr\"{a}sentation eines
Datenobjektes als Stellvertreter f\"{u}r die persistente Repr\"{a}sentation
zu benutzen, die dann beim (ersten) Zugriff den Aufbau einer direkt
referenzierbaren (transienten) Repr\"{a}sentation nach sich zieht.
%
\par{}Sp\"{a}tes Materialisieren l\"{a}\ss{}t sich demzufolge nur dann
erreichen, wenn zwischen Iden\-ti\-t\"{a}ts\-re\-pr\"{a}\-sen\-ta\-tion und
Datenobjekt eine \rglq{}schwache\rgrq\ Kopplung existiert.
%
\end{description}
%
\begin{figure}[htbp]%
\ifbuch%
\centerline{\psfig{figure=twelvept/idtaxon.eps}}%
\else%
\centerline{\psfig{figure=tenpt/idtaxon.eps}}%
\fi%
\captionfrom%
 {\"{U}bersicht \"{u}ber Identit\"{a}tsrepr\"{a}sentationen}%
 {\cite[\citepage{43}]{bib:ko90}}%
\label{fig:idtaxon}%
\end{figure}%
%
\section{\"{U}bersicht \"{u}ber Identit\"{a}tsrepr\"{a}sentationen}%
\label{sec:ovidrepr}
%
\figurename~\ref{fig:idtaxon} %
%
zeigt eine \"{U}bersicht f\"{u}r verschiedene Realisierungen von
Iden\-ti\-t\"{a}ts\-re\-pr\"{a}\-sen\-ta\-tio\-nen, die zum einen in
vorhandenen transienten Systemen verwendet werden (Adressen) und die
sich zum anderen f\"{u}r persistente Umgebungen eignen (Bezeichner und
Surrogate). In den folgenden Abschnitten werden die einzelnen
Repr\"{a}sentationen erl\"{a}utert und bewertet.
%
\subsection{Adressen}
%
Eine der einfachsten M\"{o}glichkeiten zur Identit\"{a}tsrepr\"{a}sentation
eines Datums ist seine Adresse. Bei direkten Adressen reicht eine
einmalige Dereferenzierung f\"{u}r den Zugriff auf den Zustand des
dazugeh\"{o}rigen Datenobjektes aus (\figurename~\ref{fig:dadr}). %
Indirekte Adressen referenzieren einen sogenannten
{\em Forwarder}. Ein {\em Forwarder\/} ist eine Datenstruktur mit
meist fester Gr\"{o}\ss{}e, die eine direkte Referenz auf das Datenobjekt
sowie gegebenenfalls weitere Informationen \"{u}ber das Datenobjekt
enth\"{a}lt; indirekte Adressen m\"{u}ssen damit f\"{u}r den Zugriff auf den
Objektzustand doppelt dereferenziert werden.  Oft werden alle
{\em Forwarder\/} in einer Tabelle zusammengefa\ss{}t; eine indirekte
Adresse kann dann als Index in diese Tabelle realisiert werden.
%
\par{}Die Interpretation einer Adresse erfolgt auf der {\em
Hardware\/}-Ebene und damit auf einer sehr niedrigen
Abstraktionsebene.
%
\subsubsection{Eindeutigkeit, Unabh\"{a}ngigkeit vom aktuellen
Objektzustand}
%
Sofern die Allozierung des die Daten enthaltenen Speicherbereichs
korrekt erfolgt, sind Adressen innerhalb ihres G\"{u}ltigkeitsbereichs
immer eindeutig. Sie bleiben bei Zustands\"{a}nderungen von Datenobjekten
konstant.
%
\begin{figure}[htbp]%[btp]%
\ifbuch%
\centerline{\psfig{figure=twelvept/dadr.eps}}%
\else%
\centerline{\psfig{figure=tenpt/dadr.eps}}%
\fi%
\caption{Direkte Adressen}\label{fig:dadr}%
\end{figure}%
%
\subsubsection{Umgebungsunabh\"{a}ngigkeit}
%
Indirekte Adressen (\figurename~\ref{fig:indadr})
%
erm\"{o}glichen eine Relozierung der referenzierten Objekte innerhalb des
Adre\ss{}raumes. Beim Wechsel der Umgebung von Objekten aus ihrem
Adre\ss{}raum heraus wird eine Identit\"{a}tsrepr\"{a}sentation \"{u}ber die
Adresse entweder nicht mehr m\"{o}glich (z.B.\ weil die neue Umgebung
keine vergleichbaren [Hauptspeicher-]Adressen hat, wie bei der
\"{U}ber\-tra\-gung von Objekten in Dateien), oder es kann nicht
sichergestellt werden, da\ss{} die Objekte in der neuen Umgebung die
gleiche Adresse wie in der vorherigen Umgebung haben (z.B.\ Transfer
von Daten auf eine andere Maschine).  Bei der Verwendung von Adressen
als Iden\-ti\-t\"{a}ts\-re\-pr\"{a}\-sen\-ta\-tion ist
Umgebungsunabh\"{a}ngigkeit damit nicht gegeben.
%
\begin{figure}[htbp]%[tbp]%
\ifbuch%
\centerline{\psfig{figure=twelvept/indadr.eps}}%
\else%
\centerline{\psfig{figure=tenpt/indadr.eps}}%
\fi%
\caption{Indirekte Adressen}\label{fig:indadr}%
\end{figure}%
%
\subsubsection{Zeitliche Unabh\"{a}ngigkeit}
%
Damit Daten, die sich innerhalb eines transienten Speichers befinden,
persistent werden, m\"{u}ssen sie in eine nicht-fl\"{u}chtige Umgebung
transferiert werden. Damit treten f\"{u}r persistente Daten die im
letzten Absatz geschilderten Probleme auf; zeitliche Unabh\"{a}ngigkeit
ist also bei Adressen als Identit\"{a}tsrepr\"{a}sentation f\"{u}r persistente
Objekte nicht gegeben.
%
\par{}Sofern indirekte Adressen wie \og\ als Index in eine
Tabelle mit {\em Forwardern\/} realisiert werden, l\"{a}\ss{}t sich das
Problem der Umgebungs- und Zeitunabh\"{a}ngigkeit dadurch l\"{o}sen, da\ss{} in
der persistenten Umgebung au\ss{}er den Instanzen die Tabelle aller
{\em Forwarder\/} ebenfalls persistent abgelegt wird; die f\"{u}r die
Referenz zwischen {\em Forwarder\/} und seiner dazugeh\"{o}rigen Instanz
verwendete Identit\"{a}tsrepr\"{a}sentation mu\ss{} dann jeweils der Umgebung
angepa\ss{}t werden, in der sich das Objekt befindet. Ein Beispiel daf\"{u}r
ist die in GemStone verwendete Adressierung \"{u}ber eine
\gande{Objekttabelle}{object table}, die eine indirekte Adresse auf
den physikalischen Ort eines Objektes abbildet
\cite[\citepage{370}]{bib:ma87}. In \gande{LOOM}{Large Object-Oriented
Memory}\ werden f\"{u}r Referenzen zwischen Objekten indirekte Adressen
benutzt, deren \representationform{} davon abh\"{a}ngt, ob sich das Objekt
im transienten oder persistenten Speicher befindet; bei
\"{U}ber\-tra\-gung eines Objektes zwischen den Speicherarten werden die
indirekten Adressen
konvertiert\ \linebreak[4]\cite[\citepage{299}]{bib:ka90}.
%
\subsubsection{Effizienz}
%
Adressierung von Daten wird auf allen Rechnern bis auf niedrigste
Ebene ({\em Hardware\/}) unterst\"{u}tzt und ist damit als
Identit\"{a}tsrepr\"{a}sentation f\"{u}r Datenobjekte, die im transienten
Speicher liegen, sehr effizient. Der Zugriff auf den Objektzustand
\"{u}ber eine indirekte Adresse ist wegen der doppelten Dereferenzierung
langsamer als bei einer direkten Adresse.
%
\subsubsection{Informationsgehalt}
%
Wieviel Informationen bereits eine Adresse \"{u}ber das von ihr
dereferenzierte Objekt enthalten kann, ist von der verwendeten
Rechnerarchitektur abh\"{a}ngig. Rechner mit einfachen Prozessoren
k\"{o}nnen in einer Adresse meist keine Informationen \"{u}ber das
dereferenzierte Objekt ablegen; in diesem Fall m\"{u}ssen die ben\"{o}tigten
Informationen immer \"{u}ber eine Dereferenzierung der Adresse besorgt
werden.
%
\par{}Bei anderen Rechnern bestehen Adressen aus
der Speicheradresse und einer Menge von
\gande{Markierungsbits}{tag bits}, in denen begrenzt Informationen
\"{u}ber das dereferenzierte Objekt abgelegt werden k\"{o}nnen; damit
k\"{o}nnen diese Informationen ohne eine Dereferenzierung des Objektes
selbst ermittelt werden. Wenn das mit der Adresse repr\"{a}sentierte
Objekt ein referenzierbares Objekt ist, d\"{u}rfen sich die
in den Markierungsbits abgelegten Informationen \"{u}ber das
repr\"{a}sentierte Objekt nicht
\"{a}ndern, da sonst die in allen das Objekt referenzierenden Adressen
enthaltenen Informationen ung\"{u}ltig werden.
%
\par{}Meist enthalten die Markierungsbits Informationen \"{u}ber den Typ
des referenzierten Objektes; aus der \og[en]\ Einschr\"{a}nkung
folgt, da\ss{} sich in diesem Fall der Typ des Objektes nicht \"{a}ndern
darf.\footnote{In diesem Zusammenhang sei auf den Unterschied zwischen
dem Typ eines Objektes (der in diesem Fall nicht ge\"{a}ndert werden
darf) und dem Typ selbst (der durchaus ge\"{a}ndert werden k\"{o}nnte,
sofern er w\"{a}hrend der \rglq{}Lebenszeit\rgrq\ der Objekte zur
Verf\"{u}gung steht) hingewiesen.} Eine m\"{o}gliche Markierung w\"{a}re z.B.,
da\ss{} eine Adresse selbst als Wert zu interpretieren ist.
%
\par{}Bei indirekten Adressen lassen sich im {\em Forwarder\/}
zus\"{a}tzliche Informationen \"{u}ber das referenzierte Datenobjekt
ablegen.  Um diese Informationen zu erhalten, reicht damit eine
einmalige Dereferenzierung der indirekten Adresse aus; das Objekt
selbst braucht nicht dereferenziert zu werden. Die Ablage von
Informationen im {\em Forwarder\/} f\"{u}hrt dazu, da\ss{} ein Teil der
Informationen vom Objekt weg in den {\em Forwarder\/} verlagert wird;
damit m\"{u}ssen die Informationen im {\em Forwarder\/} durch das System
besonders ber\"{u}cksichtigt werden, z.B.\ m\"{u}ssen sie beim Aufbau des
{\em Forwarders\/} aus dem Objekt bestimmt werden und in den {\em
Forwarder\/} geschrieben werden.
%
\subsubsection{Kanonische Form}
%
Direkte und indirekte Adressen als Identit\"{a}tsrepr\"{a}sentation sind
dann kanonisch, wenn jede Adresse unabh\"{a}ngig von Merkmalen (wie
z.B.\ dem Typ) des referenzierten Datenobjektes die gleiche Form hat.
Aus Effizienzgr\"{u}nden wird oft eine nicht-kanonische Form gew\"{a}hlt,
bei der z.B.\ die Anzahl der verwendeten Markierungsbits vom Typ des
Datenobjektes abh\"{a}ngt. In diesem Fall ist die Repr\"{a}sentation
speicherplatz-sparend, verlangt aber einen h\"{o}heren Aufwand bei der
Verarbeitung, weil jeder Typ spezialisierte Methoden ben\"{o}tigt.
%
\subsubsection{Sp\"{a}tes Materialisieren}
%
Eine direkte Adresse verweist immer auf ein existierendes Datenobjekt;
es gibt keine direkten Adressen von (noch) nicht existierenden
Datenobjekten. Damit die Identit\"{a}t eines Datums \"{u}ber seine direkte
Adresse repr\"{a}sentiert werden kann, mu\ss{} also das Datum selbst im
Speicher alloziert werden.  Dies gilt ebenso f\"{u}r virtuellen Speicher,
der allerdings nach erfolgter Allozierung die M\"{o}glichkeit bietet, den
physikalischen Hauptspeicher durch
\gande{Aus- und Einlagerung von Daten}{paging}\ zu
entlasten; damit findet eine sp\"{a}te Materialisierung des Objektes
f\"{u}r die nach der Allozierung folgenden Zugriffe auf den Objektzustand
statt.
%
\par{}Eine indirekte Adresse setzt f\"{u}r ihre Existenz nicht die
Allozierung des Speicherplatzes f\"{u}r das durch sie repr\"{a}sentierte
Datenobjekt, sondern lediglich die Allozierung eines {\em
Forwarders\/} voraus. Damit lie\ss{}e sich ein
sp\"{a}tes Materialisieren dadurch erreichen, da\ss{} innerhalb des
{\em Forwarders\/} Informationen abgelegt werden, wie das Objekt
materialisiert werden kann; das Objekt selbst wird dann beim ersten
Zugriff auf den Objektzustand nach der Dereferenzierung des
{\em Forwarders\/} materialisert.
%
\subsubsection{Bewertung}
%
Adressen erm\"{o}glichen eindeutige und effiziente Referenzen innerhalb
eines transienten physikalischen oder virtuellen Adre\ss{}raumes; f\"{u}r
persistente Objekte sind sie nicht geeignet, da ihre G\"{u}ltigkeit an
den Adre\ss{}raum selbst und seine \ia\ endliche Lebensdauer gebunden
ist. Da die Effizienz von Adressen extrem hoch ist, kann es sich
lohnen, ein persistentes Objekt in den transienten Adre\ss{}raum zu
\"{u}bertragen; im Verlauf der \"{U}bertragung m\"{u}\ss{}te dann \ua\ \gande{eine
Konvertierung der persistenten Identit\"{a}tsrepr\"{a}sentation in eine
Adresse vorgenommen werden}{Pointer-Swizzling}%
\addglossary{Pointer-Swizzling}[Konvertierung
einer persistenten Identit\"{a}tsrepr\"{a}sentation in eine Adresse.].
%
\subsection{Strukturierte Bezeichner}
%
\gande{Strukturierte Bezeichner}{structured identifier}\ sind
Identit\"{a}tsrepr\"{a}sentationen, die Informationen \"{u}ber die jeweils
repr\"{a}sentierten Objekte enthalten.  Ein Teil der die Objekte
betreffenden Informationen wird von den Objekten weg in die
strukturierten Bezeichner verlagert. Die Strukturierung wird so
gew\"{a}hlt, da\ss{} diese Informationen relativ einfach aus den Bezeichnern
ermittelt werden k\"{o}nnen. Die Interpretation eines strukturierten
Bezeichners erfolgt auf einer h\"{o}heren Abstraktionsebene als bei einer
Adresse.
%
\subsubsection{Arten von strukturierten Bezeichnern}
%
In persistenten Systemen realisierte strukturierte Bezeichner
lassen sich in wertstrukturierte, typ\-struk\-tu\-rier\-te und
ortsstrukturierte Bezeichner einteilen.
%
\paragraph{Wertstrukturierte Bezeichner}
%
\begin{figure}[htbp]%
\ifbuch%
\centerline{\psfig{figure=twelvept/stridval.eps}}%
\else%
\centerline{\psfig{figure=tenpt/stridval.eps}}%
\fi%
\caption{Wertstrukturierter Bezeichner}\label{fig:stridval}%
\end{figure}%
%
Der Zustand eines Wertes l\"{a}\ss{}t sich vollst\"{a}ndig in einem
wertstrukturierten Bezeichner ablegen (\figurename~\ref{fig:stridval}).
Die repr\"{a}sentierten Instanzen werden nicht als tats\"{a}chlich
allozierte Objekte im Speicher abgelegt, da jeweils ein gesamter
Objektzustand im Bezeichner enthalten ist; damit besteht keine
Notwendigkeit, zus\"{a}tzlich explizite Speicherrepr\"{a}sentationen f\"{u}r
die Objekte vorzusehen.
%
\par{}Die in den Bezeichnern enthaltenen Informationen
bestehen aus dem Typ und dem Zustand der Objekte (in diesem Beispiel
dem Typ \lisp{INTEGER}\ und dem Zustand~\lisp{5}). Bei
strukturierten Bezeichnern f\"{u}r Instanzen von Basistypen kann oft auf
die Typinformation verzichtet werden, da der Typ implizit aus dem
repr\"{a}sentierten Zustand selbst hervorgeht; so w\"{a}re in diesem
Beispiel die Identit\"{a}tsrepr\"{a}sentation \lisp{<5>}\ f\"{u}r die
ganze Zahl~5 ausreichend.
%
\begin{figure}[htbp]%
\ifbuch%
\centerline{\psfig{figure=twelvept/stridtyp.eps}}%
\else%
\centerline{\psfig{figure=tenpt/stridtyp.eps}}%
\fi%
\caption{Typstrukturierter Bezeichner}\label{fig:stridtyp}%
\end{figure}%
%
\paragraph{Typstrukturierte Bezeichner}%
\label{enu:stridtyp}%
%
Typstrukturierte Bezeichner enthalten Informationen \"{u}ber den Typ der
Objekte, die sie repr\"{a}sentieren (\figurename~\ref{fig:stridtyp}).
Genauso wie bei Adressen mit Markierungsbits k\"{o}nnen diese
Informationen aus den Bezeichnern ausgelesen werden, ohne das
dazugeh\"{o}rige Datenobjekt zu dereferenzieren. Auch gilt hier die
Einschr\"{a}nkung, da\ss{} sich die in den Bezeichnern enthaltenen
Informationen \"{u}ber die Objekte nicht \"{a}ndern d\"{u}rfen, da sonst die in
allen ein Objekt referenzierenden Bezeichnern enthaltenen
Informationen ung\"{u}ltig werden; in diesem Beispiel darf sich also der
Typ eines Objektes nicht \"{a}ndern.
%
\par{}Au\ss{}er dem Typ enth\"{a}lt der Bezeichner Informationen
zur Lokalisierung des Objektes (in diesem Beispiel die Information
\lisp{23}), die durch die Zugriffsmethoden in eine direkte
Referenz auf das Objekt umgewandelt wird
(\figurename~\ref{fig:exidrepr}).
%
\begin{figure}[hbtp]%
\ifbuch%
\centerline{\psfig{figure=twelvept/exidrepr.eps}}%
\else%
\centerline{\psfig{figure=tenpt/exidrepr.eps}}%
\fi%
\caption{Abbildung eines typstrukturierten Bezeichners auf ein
Datenobjekt}%
\label{fig:exidrepr}%
\end{figure}%
%
\paragraph{Ortsstrukturierte Bezeichner}
%
Die Identit\"{a}tsrepr\"{a}sentation mit ortsstrukturierten Bezeichnern
(\figurename~\ref{fig:stridcs}) %
%
\begin{figure}[htbp]%
\ifbuch%
\centerline{\psfig{figure=twelvept/stridcs.eps}}%
\else%
\centerline{\psfig{figure=tenpt/stridcs.eps}}%
\fi%
\caption{Ortsstrukturierter Bezeichner}\label{fig:stridcs}%
\end{figure}%
%
kann in verteilten Systemen verwendet werden; sie entspricht vom
Konzept her einer Erweiterung der indirekten Adressierung.  Ein {\em
Client\/} kann \"{u}ber einen strukturierten Bezeichner eine Referenz auf
ein Objekt herstellen, das sich auf einem {\em Server\/}
befindet. Hier entspricht ein strukturierter Bezeichner dem bei der
indirekten Adressierung verwendeten {\em Forwarder}. Er besteht aus
einer zur Umgebung der Instanz (d.h.\ dem {\em Server\/}) lokalen
Iden\-ti\-t\"{a}ts\-re\-pr\"{a}\-sen\-ta\-tion und einer Information, mit
der die Umgebung des Objektes bestimmt werden kann. Damit k\"{o}n\-nen
vom {\em Client\/} \rglq{}externe\rgrq\ Objekte referenziert werden.
%
\subsubsection{Eindeutigkeit, Unabh\"{a}ngigkeit vom aktuellen
Objektzustand}
%
Beides mu\ss{} f\"{u}r strukturierte Bezeichner durch die ge\-w\"{a}hl\-te
Realisierung sichergestellt werden.
%
\subsubsection{Umgebungsunabh\"{a}ngigkeit, Zeitliche Unabh\"{a}ngigkeit}
%
Wertstrukturierte Bezeichner enthalten implizit das Objekt, das sie
repr\"{a}sentieren; sie bilden eine abgeschlossene atomare Einheit. Zur
\rglq{}Materialisierung\rgrq\ des repr\"{a}sentierten Objektes ist nur die
Anwendung einer Methode \"{u}ber den Bezeichner notwendig, die ohne
weitere Seiteneffekte auskommt; damit sind wertstrukturierte
Bezeichner unabh\"{a}ngig von Umgebung und Zeit.
%
\par{}Bei typ- und ortsstrukturierten Bezeichnern mu\ss{} Umgebungs- und
Zeitunabh\"{a}ngigkeit durch die gew\"{a}hlte Realisierung erfolgen.
%
\subsubsection{Effizienz, kanonische Form}
%
Strukturierte Bezeichner k\"{o}nnen f\"{u}r einfache, kleine Objekte sehr
gro\ss{} werden, insbesondere wenn eine kanonische Repr\"{a}sentation
verwendet wird.  F\"{u}r den Zugriff auf den Objektzustand m\"{u}ssen aus
dem Bezeichner die f\"{u}r die Lokalisierung notwendigen Informationen
durch eine Methode ermittelt werden; damit sind strukturierte
Bezeichner ineffektiver als Adressen.
%
\subsubsection{Informationsgehalt}
%
Die Absicht bei der Verwendung von strukturierten Bezeichnern ist die
Ablage von m\"{o}glichst viel Informationen \"{u}ber ein Objekt in seiner
Identit\"{a}tsrepr\"{a}sentation.
%
\subsubsection{Sp\"{a}tes Materialisieren}
%
Anders als bei Adressen gibt es bei Bezeichnern keine
\rglq{}feste\rgrq\ Kopplung zwischen
Iden\-ti\-t\"{a}ts\-re\-pr\"{a}\-sen\-ta\-tion und dem Speicherplatz, den
das repr\"{a}sentierte Datum belegt, sondern Bezeichner bilden eine
\rglq{}schwache\rgrq\ Kopplung an das Datenobjekt \"{u}ber Methoden; daher
lie\ss{}e sich in diese Methoden ein sp\"{a}tes Materialisieren integrieren.
Die f\"{u}r die Materialisierung notwendigen Informationen m\"{u}ssen aus
dem Bezeichner ableitbar sein.
%
\subsubsection{Bewertung}
%
Die Aufl\"{o}sung einer Referenz \"{u}ber einen strukturierten Bezeichner
bedingt seine Interpretation auf einer relativ hohen
Abstraktionsebene. Damit sind strukturierte Bezeichner ineffizienter
als Adressen, bieten aber durch eine geeignet realisierte
Interpretation die M\"{o}glichkeit der Unabh\"{a}ngigkeit von Umgebung und
Zeit und sind damit abh\"{a}ngig von ihrer Realisierung f\"{u}r persistente
Objekte geeignet. Da sich die in ihnen enthaltenen Informationen nicht
\"{a}ndern d\"{u}rfen, sind bei ihrer Verwendung in \cl\ gewisse
Restriktionen zu beachten; so sollten z.B.\ keine typstrukturierten
Bezeichner f\"{u}r die Repr\"{a}sentation der Identit\"{a}t von \clos-Objekten
verwendet werden, da die \cls\ einer \clos-Instanz ge\"{a}ndert werden
kann.
%
\subsection{Benutzer-definierte Bezeichner}
%
\begin{fortune}[6cm]
%I really don't know her -- I only know her name
%\from{Phil Collins: Invisible Touch}
Ich habe keinen Namen\\
Daf\"{u}r! {[}\ldots{]}\\
Name ist Schall und Rauch,\\
Umnebelnd Himmelsglut.%
\from{Goethe: Faust}
\end{fortune}
%
Wie bereits bei den wertstrukturierten Bezeichnern gezeigt, kann der
gesamte Zustand eines konstanten Datenobjektes zur
Identit\"{a}tsrepr\"{a}sentation genutzt werden. Bei nicht-konstanten
Datenobjekten gibt es eine vergleichbare M\"{o}glichkeit, indem zur
Identit\"{a}tsrepr\"{a}sentation ein Teilzustand des Datenobjektes
ausgew\"{a}hlt wird und daf\"{u}r implizit angenommen wird, da\ss{} er konstant
bleibt. Der Teilzustand wird \ia\ so gew\"{a}hlt, da\ss{} er
genau einen Bezeichner enth\"{a}lt, der zur Identit\"{a}tsrepr\"{a}sentation
des gesamten Datenobjektes benutzt wird. Anders als bei den
wertstrukturierten Bezeichnern gibt es zu jeder
Identit\"{a}tsrepr\"{a}sentation ein explizit vorhandenes Datenobjekt.
%
\par{}In rein relationalen Datenbanken wird aus externer Sicht diese
Art der Iden\-ti\-t\"{a}ts\-re\-pr\"{a}\-sen\-ta\-tion verwendet; aus einem
Tupel wird von den BenutzerInnen als Teilzustand ein Attribut
ausgew\"{a}hlt, dessen Wert die Identit\"{a}t des Tupels repr\"{a}sentiert
(\figurename~\ref{fig:usridnty}).
%
\begin{figure}[htbp]%
\ifbuch%
\centerline{\psfig{figure=twelvept/usridnty.eps}}%
\else%
\centerline{\psfig{figure=tenpt/usridnty.eps}}%
\fi%
\caption{Benutzer-definierte Bezeichner}\label{fig:usridnty}%
\end{figure}%
%
Rein relationale Datenbanken haben keine
\rglq{}eingebaute\rgrq\ Repr\"{a}sentation f\"{u}r Identit\"{a}t der in ihnen
enthaltenen Daten.
%
\par{}Eine genauere Betrachtung von Identit\"{a}t und
Identit\"{a}tsrepr\"{a}sentation in relationalen Datenbanken folgt in
Abschnitt~\ref{sec:idmsys}.
%
\subsubsection{Eindeutigkeit}
%
Der Wert dieses Attributes wird von den BenutzerInnen vergeben; damit
m\"{u}ssen sie auch die Eindeutigkeit sicherstellen.
%
\subsubsection{Unabh\"{a}ngigkeit vom aktuellen Objektzustand}
%
Der Wert des ausgew\"{a}hlten Attributes ist ein Teilzustand des Tupels
und korrespondiert mit einem Merkmalswert des durch den Tupel
modellierten realen Objektes; bei \"{A}nderungen dieses Teilzustands
\"{a}ndert sich damit auch die Repr\"{a}sentation f\"{u}r die Identit\"{a}t des
Tupels.
%
\par{}Zudem ist die G\"{u}ltigkeit des Wertes als
Identit\"{a}tsrepr\"{a}sentation von Tupeln auf eine Relation
eingeschr\"{a}nkt; zur Benutzung des Wertes als
Identit\"{a}tsrepr\"{a}sentation m\"{u}ssen die BenutzerInnen diesen
G\"{u}ltigkeitsbereich kennen.
%
\subsubsection{Umgebungsunabh\"{a}ngigkeit, Zeitliche Unabh\"{a}ngigkeit}
%
Identit\"{a}tsrepr\"{a}sentierende Bezeichner sind unabh\"{a}ngig vom
Ort, an dem sich die repr\"{a}sentierten Objekte befinden. Sie bleiben
auch \"{u}ber einen beliebig langen Zeitraum g\"{u}ltig.
%
\subsubsection{Effizienz}
%
Referenzen zwischen Tupeln werden von den BenutzerInnen \"{u}ber
{\em Joins\/} aufgel\"{o}st; die Effizienz dieser {\em Joins\/} wird in
erster Linie durch den in der relationalen Datenbank realisierten
\gande{Anfragen-Optimierer}{query optimizer}\ bestimmt. Die
BenutzerInnen k\"{o}nnen die Datenbank veranlassen, die Werte der
identit\"{a}tsrepr\"{a}sentierenden Attribute geordnet abzulegen, um einen
schnelleren assoziativen Zugriff zu erm\"{o}glichen.
%
\subsubsection{Sp\"{a}tes Materialisieren}
%
Zwischen Bezeichner und Tupel besteht eine schwache Kopplung; damit
kann prinzipiell eine sp\"{a}te Materialisierung erfolgen. Der Zeitpunkt
wird durch die Realisation innerhalb der Datenbank bestimmt.
%
\subsubsection{Kanonische Form, Informationsgehalt}
%
Inwieweit diese Forderungen erf\"{u}llt werden, h\"{a}ngt von der durch die
BenutzerInnen gew\"{a}hlten Form der Bezeichner ab; die sich daraus
ergebenen Informationen \"{u}ber die repr\"{a}sentierten Objekte stehen
ausschlie\ss{}lich den BenutzerInnen (und nicht der Datenbank) zur
Verf\"{u}gung.
%
\subsubsection{Bewertung}
%
Die durch benutzer-definierte Bezeichner realisierten Eigenschaften
und M\"{o}glichkeiten stehen ausschlie\ss{}lich den BenutzerInnen zur
Verf\"{u}gung; das persistente System repr\"{a}sentiert zwar die
Bezeichner, hat aber kein Wissen um ihre Bedeutung. Dieses fehlende
Wissen wirkt sich \ua\ darin aus, da\ss{} Referenzen zwischen Objekten
nicht innerhalb der Datenbank aufgel\"{o}st werden k\"{o}nnen; damit
lassen sich bestimmte, f\"{u}r persistente Objektsysteme wichtige
Konzepte, wie Traversieren der von einem Objekt referenzierten
Instanzen oder Speicherr\"{u}ckgewinnung \"{u}ber Erreichbarkeit,
nicht innerhalb des persistenten Systems realisieren. Im Rahmen
\ifbericht dieses Berichtes \else\ifbuch dieses Buches \else dieser
Arbeit \fi\fi sind benutzer-definierte Bezeichner daher zur
Identit\"{a}tsrepr\"{a}sentation ungeeignet.
%
\subsection{Tupel-Bezeichner und Tupel-Surrogate}
%
Innerhalb einer relationalen Datenbanken wird f\"{u}r viele Operationen
eine Referenzierung von Tupeln ben\"{o}tigt:
%
\begin{description}
%
\item[Transaktionen] W\"{a}hrend einer laufenden Transaktion werden die
Zustands\"{a}nderungen an Tupeln festgehalten, um bei Abbruch einer
Transaktion den Zustand der ge\"{a}nderten Tupel wieder \gande{auf den
Stand bei Beginn der Transaktion zur\"{u}ckzusetzen}{rollback}. Aus
dieser Protokollierung m\"{u}ssen sich daher Referenzen auf die
modifizierten Tupel aufbauen lassen.
%
\item[{\em Update}, Selektion,
Aggregate]\hspace*{-\labelsep}\footnote{Ein Aggregat ist eine Funktion
\"{u}ber alle Werte eines Attributs einer Relation, z.B.\ Summenbildung,
Mittelwert.}\hspace*{\labelsep} W\"{a}hrend der internen Verarbeitung
durch eine Datenbank werden in einer unteren Schicht aus der Datenbank
Tupel ausgew\"{a}hlt und in einer h\"{o}heren Schicht verarbeitet
(z.B.\ Iteration \"{u}ber diese Tupel mit \"{A}ndern, Ausgeben oder
Aufsummieren von Attributwerten).  Schreibende Zugriffe auf Tupel sind nur
m\"{o}glich, wenn die Datenbank eine Referenz auf die in ihr enthaltenen
Tupel aufbauen kann.
%
\item[Projektion] F\"{u}r schreibenden Zugriff m\"{u}ssen die Referenzen auf
die zu einer Projektion ge\-h\"{o}\-ri\-gen Tupel in der Datenbank
abgelegt werden.
%
\end{description}
%
Damit mu\ss{} die Identit\"{a}t von Tupeln innerhalb der Datenbank
repr\"{a}sentiert werden. Realisiert wird die Identit\"{a}tsrepr\"{a}sentation
in einigen Datenbanken durch generierte Tupel-Bezeichner, die
innerhalb einer Relation eindeutig sind; durch die Einschr\"{a}nkung der
Eindeutigkeit auf eine Relation sind sie nicht unabh\"{a}ngig von ihrer
Umgebung.
%
\par{}Tupel-Surrogate unterscheiden sich von Tupel-Bezeichnern darin,
da\ss{} sie \"{u}ber alle Relationen eindeutig sind. In \postgres\ werden
beispielsweise Tupel durch \gande{eineindeutige
Tupel-Surrogate}{IID}\ identifiziert \cite[\citepage{18}]{bib:sto86a}.
%
\par{}Tupel-Bezeichner und -Surrogate werden ausschlie\ss{}lich
innerhalb der Datenbank verwendet; den Datenbank-BenutzerInnen ist
diese Identit\"{a}tsrepr\"{a}sentation nicht zug\"{a}nglich. Zudem wird nur die
Identit\"{a}t von Tupeln repr\"{a}sentiert; f\"{u}r die in den Attributen
befindlichen Instanzen gibt es keine Identit\"{a}tsrepr\"{a}sentation.
%
\subsubsection{Bewertung}
%
Tupel-Bezeichner und Tupel-Surrogate sind sehr spezielle, f\"{u}r
relationale Datenbanken geeignete Identit\"{a}tsrepr\"{a}sentationen.
Im Rahmen \ifbericht dieses Berichtes \else\ifbuch dieses Buches \else
dieser Arbeit \fi\fi wird eine Identit\"{a}tsrepr\"{a}sentation
be\-n\"{o}\-tigt, die f\"{u}r alle in einem Objektsystem auftretenden
Instanzen benutzt werden kann; da Tu\-pel-""Be\-zeich\-ner und
-Surrogate lediglich die Identit\"{a}t von Tupeln repr\"{a}sentieren,
sind sie damit f\"{u}r den hier beabsichtigten Zweck nicht geeignet.
%
\subsection{Surrogate}%
\label{sec:idsur}%
%
Surrogate sind globale Identit\"{a}tsrepr\"{a}sentationen, die
selbst keine direkten Informationen \"{u}ber die jeweils repr\"{a}sentierten
Instanzen enthalten (\figurename~\ref{fig:suridnty}).
%
\begin{figure}[htbp]%
\ifbuch%
\centerline{\psfig{figure=twelvept/suridnty.eps}}%
\else%
\centerline{\psfig{figure=tenpt/suridnty.eps}}%
\fi%
\caption{Surrogate}\label{fig:suridnty}%
\end{figure}%
%
\subsubsection{Eindeutigkeit, Unabh\"{a}ngigkeit vom aktuellen Zustand,
Umgebungsunabh\"{a}ngigkeit, Zeitliche Unabh\"{a}ngigkeit}
%
Diese Bedingungen k\"{o}nnen durch eine Realisierung relativ leicht
eingehalten werden, indem die Surrogate z.B.\ von einem globalen
Proze\ss{} generiert werden.
%
\subsubsection{Kanonische Form}
%
Eine nicht-kanonische Form ist bei allen Identit\"{a}tsrepr\"{a}sentationen
ein Hinweis darauf, da\ss{} die Form selbst bereits direkte Informationen
\"{u}ber das repr\"{a}sentierte Objekt enth\"{a}lt; daraus folgt f\"{u}r
Surrogate, da\ss{} sie eine kanonische Form haben m\"{u}ssen.
%
\subsubsection{Informationsgehalt} 
%
In Surrogaten selbst sind keine Informationen \"{u}ber die
repr\"{a}sentierten Objekte abgelegt; eine Realisierung mu\ss{} daher diese
Informationen z.B.\ in internen Tabellen halten, die \"{u}ber Surrogate
indiziert werden und als Ergebnis der Indizierung die ben\"{o}tigten
Objektinformationen ergeben.
%
\par{}Das Fehlen jeglicher Information \"{u}ber Objekte in Surrogaten hat
den Vorteil, da\ss{} ein Surrogat immer seine G\"{u}ltigkeit als
Identit\"{a}tsrepr\"{a}sentation beh\"{a}lt; \"{A}nderungen von
Objektinformationen wirken sich nicht auf Surrogate aus.
%
\subsubsection{Effizienz}
%
Der in Surrogaten fehlende direkte Informationsgehalt hat folgende
Auswirkungen:
%
\begin{description}
%
\item[Zugriffe auf den Objektzustand] Surrogate erfordern im
Vergleich zu den bisherigen
Iden\-ti\-t\"{a}ts\-re\-pr\"{a}\-sen\-ta\-tio\-nen einen h\"{o}heren Aufwand
f\"{u}r Zugriffe auf den Objektzustand.
%
\item[{Ablage von \immval[s]}] Der Zustand eines \immval[s]\/ l\"{a}st
sich nicht als Information in einem Surrogat ablegen. Eine Realisierung
mu\ss{} daher intern auch f\"{u}r \immval[s]\/ Speicher allozieren.
Viele Realisierungen benutzen aus diesem Grund Surrogate nur
f\"{u}r explizit repr\"{a}sentierte Objekte und nicht f\"{u}r \immval[s].
%
\item[{Identit\"{a}t von \immval[s]}] Identische \immval[s]\/ m\"{u}ssen auch
identische Surrogate erhalten; die L\"{o}sung dieses Problems ist
nicht-trivial.
%
\end{description}
%
\subsubsection{Sp\"{a}tes Materialisieren}
%
Wegen der schwachen Kopplung \"{u}ber Methoden zwischen Surrogaten und
repr\"{a}sentierten Objekten ist ein sp\"{a}tes Materialisieren m\"{o}glich.
%
\subsubsection{Bewertung}
%
Surrogate sind v\"{o}llig unabh\"{a}ngig von Umgebung, Zeit und Zustand des
Objektes, dessen Identit\"{a}t sie repr\"{a}sentieren; damit sind sie sehr
gut als Identit\"{a}tsrepr\"{a}sentation f\"{u}r persistente Objekte
geeignet. Problematisch ist die Behandlung von \immval[s]; eine
konkrete Realisierung kann dieses Problem aber dadurch l\"{o}sen, da\ss{}
\rglq{}abgeschw\"{a}chte\rgrq\ Surrogate mit einer nicht vollst\"{a}ndig
kanonischen Form realisiert werden.
%
\section{Objekt-Identifikation (\protect\objid)}%
\label{sec:handle}
%
\begin{fortune}
{[\ldots]} Zeichen oder Zeichen von Zeichen benutzen wir nur,
solange wir keinen Zugang zu den Dingen selbst haben.
\from{Umberto Eco: Der Name der Rose}
\end{fortune}
%
Die Identit\"{a}tsrepr\"{a}sentation eines Datenobjektes wird als seine
\objid\/ bezeichnet. In einer Programmiersprache werden Datenobjekte
durch unevaluierte Ausdr\"{u}cke, sogenannte \handle[s],
referenziert. Die Evaluierung eines \handle\ hat als Ergebnis das
referenzierte Datenobjekt selbst; damit entspricht sie einer
Dereferenzierung des sie repr\"{a}sentierenden Objektes
\cite[\citepage{238}]{bib:he89} (Beispiele siehe
\tablename~\ref{tab:exhandle}).
%
\begin{figure}[htbp]%
%
%\begin{minipage}{\textwidth}% \minipage wg. Fussnote
%\let\footnoterule\relax% Kein Trennungsstrich fuer Fussnoten
\centering%
%
\def\coli{5em}\def\colii{11em}\def\coliii{16em}% Breite Spalten 1-3
\def\pbrr#1#2{\parbox[t]{#1}{\raggedright{}#2}}% \parbox\raggedright
%                                                ~~  ~   ~     ~
\begin{\figurefontsize}%
\begin{tabular}{|p{\coli}|p{\colii}|p{\coliii}|}
%
\hline
\tabularheader{In}\ $\downarrow$%
%\footnote{Zu lesen: In \clogo\ ergibt
%\handle\ \lisp{malloc(2048)}\ bei Evaluierung Referenz auf ein
%Speicherblockobjekt der Gr\"{o}\ss{}e 2048 Byte}%
 &\tabularheader{ergibt \handle}\ $\downarrow$
  &\tabularheader{bei Evaluierung Referenz auf}\ $\downarrow$ \ldots\\
\hline
\hline
%
%\clogo\smallskip
% &\lisp{malloc(2048)}\smallskip
%  &\pbrr{\coliii}{\ldots\ ein Speicherblockobjekt der Gr\"{o}\-\ss{}e 2048
%   Byte}\smallskip\\
%\hline
%
\pbrr{\coli}{SQL Datenbank}\smallskip
 &\pbrr{\colii}{\lisp{SELECT A.ALL WHERE A.GEHALT > 2000 \& A IN
  ANGESTELLTER}}\smallskip
  &\ldots\ ein (tempor\"{a}res) Mengenobjekt, das alle
   Tupelobjekte enth\"{a}lt, die Angestellte mit einem Gehalt gr\"{o}\ss{}er
   2000 repr\"{a}sentieren\smallskip\\
\hline
%
%LISP\smallskip
% &\lisp{(cons 1 nil)}\smallskip
%  &\ldots\ ein neues Listenobjekt \lisp{'(1)}\smallskip\\
%\hline
%
\clos\smallskip
 &%
%  \pbrr{\colii}{\lisp{(make-instance\\{}\hspace*{0.5em}'foo)}}\smallskip
%  &\ldots\ ein neues Objekt der \cls\ \class{foo}\smallskip\\
% \cline{2-3}
% &%
  \pbrr{\colii}{\lisp{(find-class\\{}\hspace*{0.5em}'standard-class)}}%
  \smallskip
  &%
   \ldots\ das \std\ \clsmo\ der Klasse
   \class{standard-class}\smallskip\\
\hline
%
\pbrr{\coli}{Per\-si\-sten\-tes \clos}\smallskip
 &\pbrr{\colii}{\lisp{(get-object%
%\footnote{siehe Definition~(\ref{equ:getobj})}%
%\\{}\hspace*{0.5em}%
 \ \dq{}FOO\us{}15\dq)}}\smallskip
  &\ldots\ das Objekt, dessen Identit\"{a}t durch die Zeichenkette
  \lisp{\dq{}FOO\us{}15\dq}\ re\-pr\"{a}\-sen\-tiert wird\smallskip\\
\hline
%
\end{tabular}%
\end{\figurefontsize}%
%\end{minipage}%
%
\let\figurename\tablename%
\caption{Beispiele f\"{u}r \protect\handle\ und durch Evaluierung
referenzierte Objekte}%
\label{tab:exhandle}%
%
\end{figure}%
%
Die Identit\"{a}t der als Ergebnis der Evaluierung entstehenden
Objekte wird durch einen Objektidentifikator (\objid\/)
repr\"{a}sentiert.
Die Bestimmung der \objid\/ eines Objektes ist eine bijektive
polymorphe Abbildung von der Menge der Objekte $O$ in die Menge
der Objekt-Identifikatoren $I$:
%
\begin{equation}
\mbox{\em get-objid} : O \rightarrow I
\end{equation}
%
Je nach Vorgehensweise konstruiert oder generiert diese Abbildung eine 
\representationform{} f\"{u}r die Identit\"{a}t eines Objektes.
Die Umkehrabbildung $\mbox{\em get-objid}^{-1}$, im folgenden
{\em get-object\/} genannt, bildet einen Objektidentifikator
auf das dazugeh\"{o}rige Datenobjekt ab:
%
\begin{equation}\label{equ:getobj}
\mbox{\em get-objid\/}^{-1} \equiv \mbox{\em get-object} :
I \rightarrow O
\end{equation}
%
Da die \objid\/ die Identit\"{a}t eines Datenobjektes eineindeutig
repr\"{a}sentiert, ergibt sich daraus:
%
\begin{enumerate}
%
\item Zwei Objekte $o_1$ und $o_2$ sind genau dann identisch, wenn
ihre jeweiligen \objid[s]\ gleich sind:
\begin{equation}
\mbox{\em identical\/}(o_1,o_2) \leftrightarrow
\mbox{\em get-objid\/}(o_1) = \mbox{\em get-objid\/}(o_2) 
\end{equation}
%
\item Die \objid\/ dient als \rglq{}Stellvertreter\rgrq\ f\"{u}r das Objekt
selbst und kann benutzt werden, um Referenzen auf ein Objekt
aufzubauen. Die Abbildung {\em get-object\/} erm\"{o}glicht den Zugriff
auf den Zustand des Objektes.
%
\end{enumerate}
%
\section{Identit\"{a}t in verschiedenen Systemen}%
\label{sec:idmsys}
%
Jedes in einem System existierende Datum hat eine Identit\"{a}t;
Programmiersprachen und Systeme unterscheiden sich aber stark
in ihrer Auffassung davon, ob diese Identit\"{a}t \"{u}berhaupt
repr\"{a}sentiert werden soll und wie diese Repr\"{a}sentation
aussieht.
%
\subsection{Identit\"{a}t in statisch typisierten Programmiersprachen}
%
\begin{fortune}[8cm]
GO IN GREEN\\[\smallskipamount]
Go in green. Go in, case. Go in green.\hspace*{\fill}\\
\hspace*{\fill}Go in green. go in.\ in, go in green.\\
\hspace*{\fill}Go in green.\\
\hspace*{\fill}Go in green.
\from{Gertrude Stein}
\end{fortune}
%
In den klassischen Programmiersprachen wie \clogo\ und
\pascal\ werden typisierte Variablen zur Repr\"{a}sentation von Daten
benutzt.
Eine Variable wird \"{u}ber einen benutzer-definierten Bezeichner
(Variablennamen) adressiert. Das Datum selbst
enth\"{a}lt keinerlei Informationen \"{u}ber seinen Typ; er wird lediglich
an den Bezeichner des Datums gebunden und steht nur w\"{a}hrend des
\"{U}ber\-set\-zer\-laufs zur Verf\"{u}gung.
%
\par{}Ein Konzept f\"{u}r Datenidentit\"{a}t gibt es in diesen Sprachen
nicht; daraus folgt, da\ss{}
eine \representationform{} f\"{u}r die Identit\"{a}t eines Datums ebenso
fehlt. Der Variablenname erm\"{o}glicht nur die Adressierung eines
Datums; dessen Identit\"{a}t wird durch ihn nicht repr\"{a}sentiert.
Bei diesem Ansatz werden die Konzepte von Addressierung
und Identit\"{a}t vermischt. Die Adresse eines Datums ist eine
externe Eigenschaft, die es erm\"{o}glicht, auf dieses Datum zuzugreifen,
w\"{a}hrend die Identit\"{a}t die innere Eigenschaft der Einzigartigkeit
eines Datums widerspiegelt. Die Adresse ist zudem
abh\"{a}ngig von der Umgebung, in der sich das Datum befindet
\cite[\citepage{38}]{bib:ko90}. Kennzeichnend f\"{u}r diese Sprachen ist
das Fehlen eines Operators, mit dem auf Identit\"{a}t
gepr\"{u}ft werden kann; lediglich ein Test auf Gleichheit ist
m\"{o}glich (siehe z.B.\ \cite[\citepage{23}]{bib:wi83}). Dieser Test gibt
keinerlei Aufschlu\ss{} dar\"{u}ber, ob zwei Variablen sich auf ein und
dasselbe Datenobjekt beziehen, da jeweils nur die Zust\"{a}nde der
Datenobjekte verglichen werden (\figurename~\ref{fig:noiden}).
%
\begin{figure}[hbtp]%
\ifbuch%
\centerline{\psfig{figure=twelvept/noiden.eps}}%
\else%
\centerline{\psfig{figure=tenpt/noiden.eps}}%
\fi%
\caption{Identit\"{a}t in {\protect\rglq}klassischen{\protect\rgrq}\ 
  Programmiersprachen}\label{fig:noiden}%
\end{figure}%
%
\par{}Oft wird in diesen Sprachen die Adresse eines
Datums als seine Identit\"{a}tsrepr\"{a}sentation benutzt; damit ist
die Identit\"{a}tsrepr\"{a}sentation aber nicht in der Sprache
selbst realisiert, sondern wird explizit von den BenutzerInnen
gew\"{a}hlt. In einigen
Sprachen kann auch diese M\"{o}glichkeit nicht oder nur eingeschr\"{a}nkt
genutzt werden, da sie keine Operatoren zur Ermittlung der Adresse
eines Datums anbieten, wie z.B.\ \pascal. Die Adresse eines Datums in
\pascal\ ist nur dann bekannt, wenn f\"{u}r das Datum eine darauf
verweisende Zeigervariable deklariert wird und das Datum selbst im
\heap\ abgelegt wird; die Adresse des Datums wird bei der
Speicheralloziierung in der Zeigervariablen abgelegt. Die
Typinformation \"{u}ber das von der Zeigervariablen dereferenzierte Datum
ist an den Namen der Zeigervariablen gebunden. Ein Test auf
Identit\"{a}t von zwei Daten durch Vergleich ihrer Adressen kann in
\pascal\ nur erfolgen, wenn beide Daten vom gleichen Typ sind. Ein
Identit\"{a}tstest zwischen Daten verschiedenen Typs ist nicht
m\"{o}glich, da der \"{U}bersetzer keinen Vergleich von Zeigervariablen
zul\"{a}\ss{}t, die auf unterschiedliche Typen verweisen.
%
\par{}Durch die Verwendung von Bezeichnern f\"{u}r Objekte
in Systemen ohne Re\-pr\"{a}\-sen\-ta\-tions\-for\-men f\"{u}r Identit\"{a}t
tritt noch ein weiteres Problem auf. Auf dasselbe Objekt kann auf
verschiedene Art und Weise zugegriffen werden; wird das jeweils durch
den Zugriff adressierte Objekt an verschiedene Bezeichner gebunden,
ist es bei fehlender \representationform{} f\"{u}r Identit\"{a}t nicht
m\"{o}glich festzustellen, ob sich diese Bezeichner auf ein und dasselbe
Objekt beziehen. Dieser Effekt wird als {\em aliasing\/} bezeichnet.
%
\begin{figure}[htbp]%
\ifbuch%
\centerline{\psfig{figure=twelvept/ooiden.eps}}%
\else%
\centerline{\psfig{figure=tenpt/ooiden.eps}}%
\fi%
\caption{Identit\"{a}t in objekt-orientierten Programmiersprachen}%
\label{fig:ooiden}%
\end{figure}%
%
\subsection{Identit\"{a}t in dynamisch typisierten Programmiersprachen}
%
In dynamisch typisierten Programmiersprachen werden nicht Variablen,
sondern die Datenobjekte selbst typisiert; Variablen dienen lediglich
zur Referenzierung der Objekte.  \cl\ und viele der
objekt-orientierten Programmiersprachen wie \smalltalk\ und
\clos\ sind dynamisch typisierte Sprachen.
%
\par{}Die dynamisch typisierten Programmiersprachen benutzen Konstrukte,
die Datenidentit\"{a}t be\-r\"{u}ck\-sich\-ti\-gen; so repr\"{a}sentieren diese
Systeme die Identit\"{a}ten der in ihnen enthaltenen Daten. Aus der
internen Sicht dieser Sprachen enth\"{a}lt eine Variable keinen direkten
Wert, sondern eine Identit\"{a}tsrepr\"{a}sentation f\"{u}r ein Datenobjekt,
mittels der dann auf den Objektzustand zugegriffen werden kann.  Es
findet also eine Trennung zwischen dem Objekt als Individuum und
seiner Adressierung statt (\figurename~\ref{fig:ooiden}). Variablen
enthalten aus interner Sicht ausschlie\ss{}lich Referenzen auf
Datenobjekte und nicht mehr die Objekte selbst wie in den
\rglq{}klassischen\rgrq\ Pro\-gram\-mier\-spra\-chen.\footnote{Nat\"{u}rlich
gibt es auch hier Ausnahmen aus Effizienzgr\"{u}nden, wie z.B.\ Objekte
der Klasse \class{Smal\mbox{}lInt} in \smalltalk\ oder Instanzen der
Klasse \class{fixnum} in \clos.} Aus externer Sicht, also aus der
Sicht der BenutzerInnen, wird eine Variable an ein
ei\-gen\-st\"{a}n\-di\-ges Objekt gebunden; es ergeben sich zwei
\"{A}nderungen gegen\"{u}ber den klassischen Programmiersprachen:
%
\begin{enumerate}
%
\item Da Objekte \representationforms\ f\"{u}r ihre Identit\"{a}t haben und
Variablen diese Formen und nicht mehr die Objekte selbst enthalten,
kann festgestellt werden, ob sich zwei Variablen auf ein und dasselbe
Objekt beziehen; zus\"{a}tzlich zum Test auf Gleichheit gibt es jetzt noch
einen Test auf Identit\"{a}t.
%
\item Bei einer Zuweisung an eine Variable \"{a}ndert sich nicht der
Zustand des Objektes, sondern die Variable wird an das zugewiesene
Objekt gebunden; die vorherige Bindung wird aufgehoben. Der Zugriff
auf den Objektzustand erfolgt durch Aufrufe entsprechender
Zugriffsfunktionen durch die BenutzerInnen.
%
\end{enumerate}
%
\subsection{Identit\"{a}t in \protect\cl\ und \protect\clos}
%
\cl\ und \clos\ repr\"{a}sentieren die Identit\"{a}t von
Datenobjekten. In \cite{bib:st90} wird zwar keine bestimmte
Identit\"{a}tsrepr\"{a}sentation vorgeschrieben, es wird aber
angenommen, da\ss{} die Iden\-ti\-t\"{a}t eines Objektes \"{u}ber
seine Adresse repr\"{a}sentiert wird \stcite{103, 104}. Aus
Effizienzgr\"{u}nden wird f\"{u}r die Realisierung eines \cl-Systems
empfohlen, Werte der Klassen \class{short-float}\ \stcite{22}\ und
\class{character}\ \stcite{371}\ als \immval[s]\/ anzusehen; daraus
folgt eine nicht-kanonische Identit\"{a}tsrepr\"{a}sentation. F\"{u}r
Werte der Klasse \class{fixnum}\ wird ebenfalls angenommen, da\ss{}
sie effizienter als Instanzen der allgemeineren Klasse
\class{integer}\ repr\"{a}sentiert werden \stcite{16}. Numerische
Werte k\"{o}nnen ebenfalls nicht-kanonisch repr\"{a}sentiert werden
\stcite{288}. Die meisten \cl-Systeme halten sich an diese
Empfehlungen; in dem zur Realisierung von persistenten Objektsystemen
in \ifbericht diesem Bericht \else\ifbuch diesem Buch \else dieser
Arbeit \fi\fi verwendetem \lwcl\ werden Instanzen der Klassen
\class{short-float}, \class{character}\ und \class{fixnum}\ als
\immval[s]\/ angesehen, deren Identit\"{a}t nicht-kanonisch
repr\"{a}sentiert wird; bei allen anderen Objekten wird deren
jeweilige Identit\"{a}t \"{u}ber ihre Adresse repr\"{a}sentiert.
%
\subsection{Identit\"{a}t in relationalen Datenbanken}
%
Die in relationalen Datenbanken enthaltenen persistenten Datenobjekte
sind wie nicht-per\-si\-sten\-te Daten Dinge oder Sachverhalte der
realen Welt in abstrakter Form. Bei der Modellbildung werden reale
Dinge abstrahiert und anhand gleichartiger Merkmale klassifiziert;
jede gefundene Klasse wird in der Datenbank als Relation abgelegt.
Die Merkmale der realen Dinge einer Klasse werden als Attribute der
Relation definiert. Ein reales Objekt wird in der Datenbank als Tupel
der zu seiner Klasse geh\"{o}rigen Relation abgelegt.
%
\par{}Relationale Datenbanken lassen sich in rein relationale,
erweiterte und objekt-orientierte Datenbanken einteilen. In
rein relationalen und erweiterten relationalen Datenbanken wird eine
Relation als Tabelle verstanden, bei der f\"{u}r jedes Attribut eine
Spalte existiert; ein Tupel ist eine Zeile einer
solchen Tabelle. In die objekt-orientierten Datenbanken wurden
Konzepte der objekt-orientierten Programmierung integriert; so
werden Relationen als vererbbare Klassen, Tupel als Instanzen mit
Identit\"{a}t und Attribute als \Slt[s]\/ aufgefa\ss{}t. Das bedeutet jedoch
nicht, da\ss{} sie in der Lage sind, Datenobjekte im Sinne der
objekt-orientierten Programmierung (wie z.B.\ \clos-Objekte) direkt zu
speichern.
%
\subsubsection{Rein relationale Datenbanken}
%
Sie erf\"{u}llen die von Codd \cite[\citepage{399}]{bib:co79}
aufgestellte Bedingung, da\ss{} ein Tupel sich in mindestens einem
Attributwert von allen anderen Tupeln der gleichen Relation
unterscheidet \cite[\citepage{37}]{bib:on94}; damit gilt f\"{u}r jedes
Tupel das Identit\"{a}tsprinzip (es wird als individuelles,
unterscheidbares Datenobjekt aufgefa\ss{}t). Trotz dieser Auffassung von
Datenidentit\"{a}t fehlt in den rein relationalen Datenbanken nach Codd
eine datenbank-interne Identit\"{a}tsrepr\"{a}sentation; statt dessen wurde
daf\"{u}r eine Konvention definiert, die die Herstellung von Identit\"{a}t
den Datenbank-BenutzerInnen \"{u}berl\"{a}\ss{}t
(\figurename~\ref{fig:reldbref}).
%
\begin{figure}[htbp]%
\ifbuch%
\centerline{\psfig{figure=twelvept/reldbref.eps}}%
\else%
\centerline{\psfig{figure=tenpt/reldbref.eps}}%
\fi%
\caption{Identit\"{a}ten und Referenzen in relationalen
Datenbanken}\label{fig:reldbref}%
\end{figure}%
%
\begin{description}
%
\item[Identit\"{a}tsrepr\"{a}sentation in relationalen Datenbanken]
Die Identit\"{a}tsrepr\"{a}sentation in relationalen Datenbanken besteht
daraus, da\ss{} innerhalb einer Relation von den BenutzerInnen ein oder
mehrere Attribute ausgew\"{a}hlt werden, deren Werte die Identit\"{a}t eines
Tupels widerspiegeln (Prim\"{a}rschl\"{u}sselattribute); sie bilden somit
die \objid\/ des Tupels.  Aus den Bedingungen f\"{u}r
Iden\-ti\-t\"{a}ts\-re\-pr\"{a}\-sen\-ta\-tion folgt, da\ss{} die beteiligten
Attribute ihre Werte innerhalb eines Tupels nicht \"{a}ndern d\"{u}rfen und
mindestens innerhalb der Relation eindeutig sein m\"{u}ssen. Die korrekte
Vergabe der Attributwerte liegt vollst\"{a}ndig in der Hand der
BenutzerInnen.
%
\par{}Die BenutzerInnen der Relation, die diese
\objid\/ f\"{u}r den Aufbau von Referenzen ausnutzen m\"{o}chten, m\"{u}ssen
wissen, welche Attribute der Relation f\"{u}r diesen Zweck festgelegt
wurden (\figurename~\ref{fig:reldbref}, Attribute \oi, \oiii\ und \ov)
\cite[\citepage{84}]{bib:schl83}.
%
\item[Referenzen zwischen Datenobjekten]
Referenzen zwischen Datenobjekten werden aufgebaut,
indem der Wert des iden\-ti\-t\"{a}ts-re\-pr\"{a}\-sen\-tie\-ren\-den
Attributes des dereferenzierten Tupels in ein Attribut im
referenzierenden Tupel eingetragen wird. Im referenzierenden Tupel
wird dieses Attribut per Konvention von den BenutzerInnen nicht als
\rglq{}wertetragend\rgrq, sondern als
\rglq{}dereferenzierend\rgrq\ festgelegt (Sekund\"{a}rschl\"{u}ssel
oder Fremdschl\"{u}sselattribute, \figurename~\ref{fig:reldbref},
Referenzen zwischen
\oii\,$\leftrightarrow$\,\oiii, \oiii\,$\leftrightarrow$\,\ovi\ und
\oiv\,$\leftrightarrow$\,\ov).
%
\item[Aufl\"{o}sung der Referenzen]
Die Aufl\"{o}sung der Referenzen erfolgt \"{u}ber Selektion des
dereferenzierten Objektes \"{u}ber das identit\"{a}tsrepr\"{a}sentierende
Attribut. Dazu ist der Aufbau eines {\em Joins\/} zwischen
referenzierendem und dereferenziertem Tupel n\"{o}tig. Die Referenz ist
symmetrisch; da der {\em Join\/} kommutativ ist \cite[\citepage{85,
174}]{bib:schl83}, kann durch Umstrukturierung ebenso eine Referenz in
Gegenrichtung aufgel\"{o}st werden.
%
\par{}Die in \figurename~\ref{fig:reldbref} eingezeichneten Referenzen
zwischen den Objekten sind nicht explizit in der Datenbank enthalten.
%
\end{description}
%
\par{}Die korrekte Verwaltung sowohl der identit\"{a}ts-repr\"{a}sentierenden
Attribute als auch der Referenzen, d.h.\ der Schl\"{u}sselintegrit\"{a}t,
liegt in der Hand der BenutzerInnen, da eine rein relationale
Datenbank kein \rglq{}Wissen\rgrq\ \"{u}ber diese getroffenen Konventionen
hat.
%
\par{}Das Problem der Identit\"{a}tsrepr\"{a}sentation \"{u}ber einen
Attributwert ist seine doppelte Bedeutung. Zum einen dient er als
\objid\/ und zum anderen ist er ein Wert innerhalb des Tupels. Die
L\"{o}sung innerhalb einer rein relationalen Datenbank besteht darin,
diese beiden Bedeutungen aufzutrennen und f\"{u}r die
Identit\"{a}tsrepr\"{a}sentation ein zus\"{a}tzliches Attribut
\attr{}{OBJID}\ festzulegen. Beim Anlegen eines neuen Tupels mu\ss{} es
dann von den BenutzerInnen mit einem beliebigen aber innerhalb der
Relation eindeutigen Wert belegt werden. Der gew\"{a}hlte Wert kann mit
einem Merkmal des im Tupel modellierten realen Objektes
korrespondieren, sofern dieses Merkmal die Bedingungen f\"{u}r
Identit\"{a}tsrepr\"{a}sentation einh\"{a}lt (z.B.\ Personalausweisnummer als
\objid\/ f\"{u}r Tupel der Relation \relation{ANGESTELLTER}).
%
\begin{figure}[htbp]%
\ifbuch%
\centerline{\psfig{figure=twelvept/oordbref.eps}}%
\else%
\centerline{\psfig{figure=tenpt/oordbref.eps}}%
\fi%
\caption{Identit\"{a}ten und Referenzen in objekt-orientierten
relationalen Datenbanken}\label{fig:oordbref}%
\end{figure}%
%
\subsubsection{Erweiterte relationale Datenbanken}
%
Als m\"{o}gliches Hilfskonstrukt bieten sie den BenutzerInnen zur
Einhaltung der Bedingungen f\"{u}r eine Identit\"{a}tsrepr\"{a}sentation
sogenannte {\em Trigger\/} \cite[\citepage{407}]{bib:on94}
\cite[\citepage{49}]{bib:vaga89} (Synonym {\em Alerter\/}
\cite[\citepage{53}]{bib:ni88}) an. Ein {\em Trigger\/} ist eine
Prozedur, die bei Eintritt eines Ereignisses oder Erf\"{u}llung einer
Vorbedingung aktiviert wird (vergleichbar dem
{\em D\"{a}mon\/}-Konzept). Er kann beispielsweise benutzt werden, um
beim Ereignis \rglq{}Einf\"{u}gen eines Tupels\rgrq{} die dazugeh\"{o}rige
\objid\/ zu generieren und in das identit\"{a}ts-repr\"{a}sentierende
Attribut \attr{}{OBJID}\ einzutragen; ebenso kann er bei Zugriff auf
Attribute aktiviert werden und dabei die G\"{u}ltigkeit von Referenzen
\"{u}berwachen.
%
\par{}Bei neueren Systemen gibt es die M\"{o}glichkeit,
Integrit\"{a}tsbedingungen zu spezifizieren, die von der relationalen
Datenbank \"{u}berwacht werden; damit kann beispielsweise die
Schl\"{u}sselintegrit\"{a}t automatisch \"{u}berpr\"{u}ft werden. Problematisch
ist aber, da\ss{} die entsprechenden Bedingungen auch f\"{u}r den
\rglq{}einfachen\rgrq\ Fall von Referenzen zwischen Objekten
spezifiziert werden m\"{u}ssen, obwohl die Bedingung der
Schl\"{u}sselintegrit\"{a}t bereits inh\"{a}rent in der Referenz enthalten ist.
%
\subsubsection{Objekt-orientierte relationale Datenbanken}
%
Sie erweitern das Typkonzept und lassen Da\-ten\-bank-Re\-la\-tio\-nen
als Typen f\"{u}r Attribute zu (\figurename~\ref{fig:oordbref}).  Aus der
Sicht der BenutzerInnen ist der Wert eines derartigen Attributes ein
Tupel der typbildenden Relation; intern wird \"{a}hnlich wie bei
Variablen in objekt-orientierten Programmiersprachen das Tupel durch
seine \objid\/ repr\"{a}sentiert.  Dadurch kann eine direkte Referenz von
einem Attribut auf ein Tupel aufgebaut werden. Die in
\figurename~\ref{fig:oordbref} eingezeichneten Referenzen werden durch
die Datenbank und nicht durch die BenutzerInnen realisiert; f\"{u}r den
Zugriff ist daher auch kein {\em Join\/} mehr n\"{o}tig. Die Referenz kann
h\"{a}ufig nur in einer Richtung aufgel\"{o}st werden.
%
\par{}Als referenzierbare Objekte werden in einigen Systemen
ausschlie\ss{}lich Tupel
angesehen (z.B.\ \postgres\ \cite{bib:sto86a}). Die Identit\"{a}t aller
anderen in der Datenbank
enthaltenen Objekte wird in diesen Systemen nicht repr\"{a}sentiert;
damit sind keine Referenzen auf diese Daten m\"{o}glich.
%
\subsection{Identit\"{a}t in persistenten Objektsystemen}
%
\begin{fortune}
Entit\"{a}ten sollten nicht unn\"{o}tig vervielfacht werden.
%\from{\rglq{}Ockhams Rasiermesser\rgrq\ nach \cite[\citepage{220}]{ru92}}
\from{\rglq{}Ockhams Rasiermesser\rgrq}
\end{fortune}
%
In persistenten Systemen, die auf Maschinen mit fl\"{u}chtigem Speicher
realisiert werden, wird zur Ablage von persistenten Objekten ein
Sekund\"{a}rspeicher benutzt. Nach Erzeugung eines Objektes im
fl\"{u}chtigen Speicher legt das persistente System das Objekt entweder
explizit (z.B.\ durch Veranlassung von au\ss{}en, wie Anwendung einer
Methode durch die BenutzerInnen, die daf\"{u}r sorgt, da\ss{} das Objekt
persistent wird) oder implizit (z.B.\ wird f\"{u}r jedes neu
erzeugte Objekt angenommen, da\ss{} es persistent sein soll) im
nicht-fl\"{u}chtigen Speicher ab. Bei Zugriffen auf die Objekte 
von einem sp\"{a}teren oder anderen Proze\ss{} k\"{o}nnen die
referenzierten persistenten Objekte aus Effizienzgr\"{u}nden vom
persistenten System aus der persistenten in die transiente Umgebung
kopiert werden und dort direkt von den BenutzerInnen manipuliert
werden; in diesem Fall gibt es von einem Datum zwei
\rglq{}Ausgaben\rgrq{}, eine im fl\"{u}chtigen Speicher und eine im
nicht-fl\"{u}chtigen Speicher (\figurename~\ref{fig:periden}).
%
\begin{figure}[hbtp]%
\ifbuch%
\centerline{\psfig{figure=twelvept/periden.eps}}%
\else%
\centerline{\psfig{figure=tenpt/periden.eps}}%
\fi%
\caption{Identit\"{a}t eines Objektes in einem persistenten System}%
\label{fig:periden}%
\end{figure}%
%
F\"{u}r die BenutzerInnen der obersten Schicht eines persistenten Systems
sollte diese Dualit\"{a}t nicht sichtbar sein, sondern sie arbeiten
lediglich mit dem einen Objekt, das sie direkt im Zugriff haben; das
persistente System sorgt f\"{u}r die Verwaltung des Objektes, wie
Transfer zwischen fl\"{u}chtigem und nicht-fl\"{u}chtigem Speicher, Regelung
von Zugriffskonflikten von mehreren BenutzerInnen auf ein Objekt usw.
%
\par{}Die Identit\"{a}t eines Objektes ist damit abh\"{a}ngig von der
Schicht eines persistenten Systems. In der obersten Schicht gibt es
f\"{u}r jedes Objekt eine Identit\"{a}t; in den darunter liegenden Schichten
existiert das Objekt in einer oder mehreren zusammengeh\"{o}rigen
\rglq{}Ausgaben\rgrq{}, denen man in diesen Schichten eine eigene
Identit\"{a}t geben kann. Bei Transfer eines Objektes zwischen den
Schichten m\"{u}ssen dann die jeweiligen Identit\"{a}tsrepr\"{a}sentationen
umgeformt werden.
%
\section{Zusammenfassung}
%
Identit\"{a}t ist eine jedem Datenobjekt zugeh\"{o}rige Eigenschaft. Sie
zeichnet ein Datenobjekt als eigenst\"{a}ndig aus, d.h.\ es unterscheidet
sich von allen anderen Datenobjekten. Dies mu\ss{} in einem System von
Datenobjekten durch eine geeignete \representationform{} wiedergegeben
werden. Sie dient dazu, Referenzen auf das dazugeh\"{o}rige Objekt
aufzubauen. Eine \representationform{} mu\ss{} dabei bestimmten
Bedingungen gen\"{u}gen; sie mu\ss{} eindeutig und unabh\"{a}ngig von Zeit,
aktuellem Zustand und der Umgebung des Objektes sein, dessen
Identit\"{a}t repr\"{a}sentiert werden soll. F\"{u}r transiente und persistente
Umgebungen sind die in diesem Kapitel diskutierten
Iden\-ti\-t\"{a}ts\-re\-pr\"{a}\-sen\-ta\-tio\-nen unterschiedlich
geeignet. \clos-Systeme repr\"{a}sentieren die Identit\"{a}t der in ihnen
enthaltenen Objekte
\ia\ mit nicht-kanonischen Adressen, deren Verwendung als
Identit\"{a}tsrepr\"{a}sentation in einer persistenten Umgebung damit
ausscheidet.  F\"{u}r persistente Objekte mu\ss{} daher eine geeignete
Identit\"{a}tsrepr\"{a}sentation innerhalb der persistenten Umgebung
ausgew\"{a}hlt und realisiert werden; zu diesem Zweck sind Surrogate sehr
gut geeignet, da sie die Bedingungen f\"{u}r eine
Identit\"{a}tsrepr\"{a}sentationform am Besten erf\"{u}llen.  Sofern ein
persistentes Objekt seine Umgebung wechseln kann, mu\ss{}
bei jedem Wechsel \ua\ seine Identit\"{a}tsrepr\"{a}sentation konvertiert
werden; bei der Verwendung von Surrogaten bedeutet dies konkret, da\ss{}
beim Wechsel von persistenten Objekten zwischen der persistenten und
der transienten Umgebung \ua\ eine Konvertierung zwischen Surrogaten
und Adressen erfolgen mu\ss{}.

%%% Local Variables: 
%%% mode: latex
%%% TeX-master: "main"
%%% End: 
