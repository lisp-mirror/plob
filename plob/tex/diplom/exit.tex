%
\chapter{Zusammenfassung und Ausblick}%
\label{chap:exit}%
%
\section{Zusammenfassung}
%
Ziel \ifbericht dieser Forschungsarbeit \else\ifbuch der in diesem
Buch vorgestellten Forschungsarbeit \else dieser Arbeit \fi\fi war die
Realisierung eines Systems, das \cl\ Datenobjekte un\-ab\-h\"{a}n\-gig
vom daten-erzeugenden transienten Proze\ss{} persistent h\"{a}lt.
Wegen des beabsichtigten Einsatzgebietes kann die Anzahl der
persistent zu haltenden Objekte sehr gro\ss{} werden; ebenso
k\"{o}n\-nen die Objekte selbst sehr gro\ss{} sein, wie z.B.\ 
Satellitenbilder hoher Aufl\"{o}sung.
%
\par{}Der gew\"{a}hlte Weg bestand zun\"{a}chst darin, die f\"{u}r persistente
Systeme wichtigen Konzepte zu evaluieren, um sowohl vorhandene Systeme
bewerten als auch ein eigenes System entwerfen zu k\"{o}nnen:
\begin{itemize}
%
\item \cl\ Datenobjekte haben eine Identit\"{a}t, die explizit
repr\"{a}sentiert wird; dazu war es zun\"{a}chst n\"{o}tig, den Begriff der
Identit\"{a}t zu kl\"{a}ren und die verschiedenen
Re\-pr\"{a}\-sen\-ta\-tions\-for\-men f\"{u}r Identit\"{a}t im Hinblick auf
Persistenz zu analysieren und zu bewerten, um eine geeignete Wahl f\"{u}r
die Realisierung treffen zu k\"{o}nnen.
%
\item Aus dem beabsichtigten Einsatzzweck des zu realisierenden
  Systems ergab sich die Konzeption f\"{u}r die Gesamtarchitektur des
  Systems und die Auswahl der Subkomponenten. Als Speicher f\"{u}r die
  Ablage von Objektzust\"{a}nden ergaben sich im Rahmen \ifbericht
  dieses Berichtes \else\ifbuch der Forschungsarbeit \else dieser
  Arbeit \fi\fi die beiden M\"{o}glichkeiten der Verwendung einer
  relationalen Datenbank und eines Objektspeichers.  Eine relationale
  Datenbank hat den Vorteil m\"{a}chtiger
  Ma\-ni\-pu\-la\-tions\-m\"{o}g\-lich\-kei\-ten; nachteilig ist die
  f\"{u}r Objekte mit Referenzen ungeeignete Datenmodellierung. Ein
  Objektspeicher hingegen basiert nicht auf dem relationalen Modell
  und hat lediglich einfache Zugriffsm\"{o}glichkeiten, bietet aber
  daf\"{u}r die M\"{o}glichkeit, explizite Referenzen zwischen
  Objekten zu etablieren.
%
\end{itemize}
%
Nach der Konzeption folgte die Realisierungsphase:
\begin{itemize}
%
\item Zun\"{a}chst wurde gepr\"{u}ft, ob das mit einer relationalen
Datenbank als Speicher f\"{u}r die Zu\-st\"{a}n\-de von persistenten
Objekten realisierte System Shared Object Hierarchy (\soh) f\"{u}r den
beabsichtigten Zweck geeignet sei. Dazu wurde der Code an das zur
Verf\"{u}gung stehende LISP-System und an eine neuere Version der
Datenbank angepa\ss{}t. Im Verlauf der Portierung stellte sich heraus,
da\ss{} \soh\ die Probleme, die sich bei Verwendung einer Datenbank zur
Speicherung von Objekten ergeben, nur ungen\"{u}gend l\"{o}st.
%
\item Die entwickelten Konzepte und die Erfahrungen mit dem System
\soh\ m\"{u}ndeten in den Entwurf und die Realisierung des Systems \plob.
Wegen der prinzipiellen Schwierigkeiten der Repr\"{a}sentation von
Objektzust\"{a}nden und Referenzen zwischen Instanzen in relationalen
Datenbanken entschied ich mich daf\"{u}r, einen Objektspeicher f\"{u}r die
Ablage der Objektzust\"{a}nde zu verwenden.
%
\par{}Auf diesen Objektspeicher wurde eine Schicht mit
Datenbankfunktionalit\"{a}ten aufgesetzt, die Sitzungen, \twophasetrs,
ein hierarchisches Sperrprotokoll und die f\"{u}r eine Indexverwaltung
notwendigen Hilfsmittel \"{u}ber persistente Objekte realisiert. 
Diese Ebene realisiert transaktionsgesicherten und durch das
Sperrprotokoll unteilbaren Zugriff auf dynamisch typisierte,
selbstbeschreibende Objekte.
%
\par{}Die oberste Ebene realisiert unter Benutzung der Funktionalit\"{a}t
der n\"{a}chst-niedrigen Schicht transparente und orthogonale
Persistenz f\"{u}r \cl\ Daten sowie die M\"{o}glichkeit des assoziativen
Zugriffs auf persistente Objekte.
%
\end{itemize}
%
Bei der Realisierung zeigten sich die Vorteile von objekt-orientierten
prozedural-reflektiven Sprachen wie \clos\ bei der Erweiterung der
Sprache um Persistenz; f\"{u}r \pascalr\
\cite[\citepage{259}]{bib:schm77} und {\em E\/} \cite{bib:ri89} mu\ss{}ten
jeweils \"{U}bersetzer modifiziert bzw.\ erstellt werden, w\"{a}hrend f\"{u}r
die Einbindung von Persistenz in \clos\ eine Modifikation des Systems
mit Mitteln des Systems selbst m\"{o}glich ist.
%
\section{Anwendungen f\"{u}r persistente Systeme}
%
\subsection{Softwareentwicklung in einem transienten
und persistenten \protect\cl-System}
%
Das nachfolgende Beispiel zeigt, da\ss{} persistente Objektsysteme f\"{u}r
Soft\-ware-Ent\-wick\-lungs\-um\-ge\-bun\-gen verwendet werden
k\"{o}nnen; die persistenten Objekte ersetzen dabei das \"{u}bliche
Hilfsmittel f\"{u}r Persistenz in transienten Systemen, in diesem
Beispiel also Dateien mit Quell- und Objekt-Codes. Das hier angegebene
Beispiel stellt einige prinzipielle M\"{o}glichkeiten eines
persistenten Systems dar; nicht alle der folgenden Schritte lassen
sich mit dem aktuellen Entwicklungsstand von
\plob\ realisieren.\\[\parsep]
%
\refstepcounter{figure}%
\addcontentsline%
 {lof}%
 {figure}%
 {\protect\numberline{\thefigure}{\ignorespaces%
  Softwareentwicklung in einem transienten
  und persistenten System}}%
\begin{longtable}[c]{|rp{5.5cm}|p{5.5cm}|}%
\hline%
 &\tabularheader{Transientes \cl}
  &\tabularheader{Persistentes \cl}\\
\hline\hline\endhead
1
 &Starten des LISP-Systems
  &Starten des LISP-Systems\\
\hline
2
 &Laden der Pro\-gramm-De\-fi\-ni\-tions\-da\-tei (sie definiert die
  zum Programm geh\"{o}rigen Module)
  &Pro\-gramm-De\-fi\-ni\-tions\-da\-tei
   be\-fin\-det sich im per\-si\-sten\-ten Spei\-cher\\
\hline
3
 &Laden der Programm-Module
  &Programm-Module be\-fin\-den sich im per\-si\-sten\-ten Spei\-cher\\
\hline
4
 &Sichern des Quellcodes vor Beginn der Modifikationen
  &Start einer langandauernden Transaktion\\
\hline
5
 &Laden des zu \"{a}n\-dern\-den Quell\-co\-des in den Editor
  &Quellcode befindet sich im persistenten Speicher\\
\hline
6
 &Lokalisierung und \"{A}nderung einer Klassendefinition im Quellcode
  &Klassendefinition wird \"{u}ber einen {\em Browser\/} erst selektiert
   und dann editiert\\
\hline
7
 &\"{U}bersetzen und Testen der ge\"{a}nderten Klasse
  &\"{U}bersetzen und Testen der ge\"{a}nderten Klasse\\
\hline
8
 &Die Benutzerin stellt fest, da\ss{} ihre \"{A}nderungen falsch waren; sie
  mu\ss{} den in Schritt~4 gesicherten Quellcode erneut laden
  &Die Benutzerin stellt fest, da\ss{} ihre \"{A}nderungen falsch waren; sie
   bricht die in Schritt~4 begonnene langandauernde Transaktion ab.
   Damit werden alle betroffenen Module \gande{auf den alten Stand
   zur\"{u}ck gebracht}{rollback}\\
\hline
9
 &Erneutes Lokalisieren, \"{A}ndern (Schritt~6) und Testen (Schritt~7)
  &Erneutes Selektieren, \"{A}ndern (Schritt~6) und Testen (Schritt~7)\\
\hline
10
 &Die \"{A}nderungen wurden erfolgreich abgeschlossen; die Quellcodes
  werden gesichert
  &Die langandauernde Transaktion, die in Schritt 4 gestartet wurde,
   wird \gande{erfolgreich beendet}{committed}\\
\hline
\end{longtable}
%
\par{}F\"{u}r das persistente System wird in diesem Beispiel angenommen,
da\ss{} Persistenz ohne Einschr\"{a}nkung f\"{u}r alle im System auftretenden
Datentypen zur Verf\"{u}gung steht.  Ebenso wird angenommen, da\ss{} das
persistente System bereits einige Datenbankfunktionalit\"{a}t beinhaltet,
wie z.B.\ Transaktionen (Schritt~4 und 10) sowie Introspektions- und
Selektionsm\"{o}glichkeiten (Schritt~6).
%
\par{}Wie das Beispiel zeigt, sind keine Zugriffe auf Dateien n\"{o}tig;
damit wird die Abgeschlossenheit des Systems erh\"{o}ht. Ebenso lassen
sich \"{A}nderungen leicht r\"{u}ckg\"{a}ngig machen (Schritt~4 und~8). Sofern
das System zu\-s\"{a}tz\-lich noch eine Versionskontrolle bietet, lie\ss{}e
sich ebenfalls ein \"{a}lterer Zustand des Gesamtsystems wiederherstellen.
%
\section{Ausblick}
%
Dieser Abschnitt enth\"{a}lt Vorschl\"{a}ge, wie das in
Kapitel~\ref{chap:plob} beschriebene System erweitert werden k\"{o}nnte.
%
\subsection{Verlagerung von Funktionen aus der 3.\ in die 2.\ Schicht}
%
Einige Funktionalit\"{a}ten der 3.~Schicht sollten in die 2.~Schicht
verlagert werden; dies erleichtert eine Nutzung des persistenten
Systems durch nicht-LISP-Systeme. Ebenso w\"{a}re damit eine erh\"{o}hte
Effizienz des persistenten Systems verbunden.
%
\subsubsection{Sitzungsverwaltung}
%
Die Sitzungsverwaltung (\"{O}ffnen und Schlie\ss{}en einer Sitzung) sollte
in die 2.~Schicht verlagert werden.
%
\subsubsection{Indexverwaltung}
%
Die Verwaltung der \Slt-Indextabellen wird im Moment in der
3.~Schicht in den \spc[n]\ \mtd[n]\ der
\gfn[n]en \stfn{slot-\ldots-using-class}\ durchgef\"{u}hrt. Alle
Informationen zur Indexverwaltung sind in der 2.~Schicht verf\"{u}gbar;
die Zugriffs-\fn[en]\ der 2.~Schicht k\"{o}nnten dementsprechend die
Pflege des f\"{u}r einen \Slt\/ deklarierten Index bei Zugriff auf den
\Slt\/-Zustand \"{u}bernehmen.
%
\subsection{Erweiterungen der 2.\ Schicht}%
\label{sec:expiil}%
%
Dieser Abschnitt schl\"{a}gt diverse Erweiterungen f\"{u}r die 2.~Schicht
vor.
%
\subsubsection{Doppelte Schl\"{u}ssel f\"{u}r persistente B-B\"{a}ume}%
%
Die momentan realisierten persistenten B-B\"{a}ume k\"{o}nnen keine
doppelten Schl\"{u}ssel verarbeiten; sie sollten dementsprechend
erweitert werden. In diesem Zusammenhang w\"{u}rde sich f\"{u}r den
einfachen Zugriff auf B-B\"{a}ume mit doppelten Schl\"{u}sseln die
Realisierung von {\em Portals\/} \cite[\citepage{114}]{bib:we93}
anbieten. In etwa gleichbedeutend ist das {\em Cursor\/}-Konzept von
SQL \cite[\citepage{220}]{bib:on94}; {\em Portals\/} sind allgemeiner
als {\em Cursor}, da in ihnen Tupel verschiedener Relationen enthalten
sein k\"{o}nnen.
%
\subsubsection{Transaktionsarten}%
%
Momentan wurde nur das einfachste Transaktionsmodell der
\rglq{}flachen\rgrq\ Transaktionen realisiert; in
\cite[\citepage{187--210}]{bib:gr93}\ werden verschiedene andere
Transaktionsmodelle vorgestellt. Meiner Meinung nach w\"{a}re
insbesondere die Realisierung von \gande{verschachtelten
Transaktionen}{nested transactions}\ \cite[\citepage{195}]{bib:gr93}
lohnend, da sie die M\"{o}glichkeit bieten, die Transaktionsverarbeitung
an die Modulstruktur des persistenten Systems anzupassen.
%
%\subsubsection{Erweiterung des Sperrprotokolls}
%
%\paragraph{%
\subsubsection{Zus\"{a}tzliche Sperre}
%
Um im Betrieb mit mehreren BenutzerInnen den \cache\/ konsistent zu
halten, sollte eine zu\-s\"{a}tz\-li\-che, zu allen bisher definierten
Sperren kompatible Sperre \rglq{\em cached\/}\rgrq\ realisiert
werden, die im Verlauf der \"{U}bertragung eines Objektzustands zwischen
persistentem und transientem Speicher auf das \"{u}bertragene Objekt f\"{u}r
den \"{u}bertragenden Proze\ss{} gesetzt wird. Die 2.\ Schicht kann dann bei
Modifikation des im persistenten Speicher abgelegten Objektzustands
durch einen Proze\ss{} an alle Prozesse, die eine \rglq{\em
cached\/}\rgrq-Sperre auf das Objekt gesetzt haben,
benachrichtigen, da\ss{} der Zustand der vom \cache\/ referenzierten
transienten Repr\"{a}sentation des persistenten Objektes ung\"{u}ltig ist
und nachgeladen werden sollte.
%
%\paragraph{Zus\"{a}tzliche Sperrebenen}
%
%Res. auf Klassenebene
%
\subsubsection{Mehrdimensionale Indizes}
%
Persistente B-B\"{a}ume bieten lediglich eindimensionale Schl\"{u}ssel; f\"{u}r
den angestrebten assoziativen Zugriff auf geografische Daten w\"{a}re die
Realisierung von mehrdimensionalen Indizes n\"{o}tig.
%
\subsubsection{Trennung der Prozesse in einen {\em
Frontend\/}- und {\em Backend\/}-Proze\ss{}}
%
\"{A}hnlich wie in \postgres\ \cite[\citepage{30}]{bib:we93} w\"{a}re eine
Trennung in einen {\em Backend\/}-Proze\ss{} zur Verwaltung des
persistenten Speichers und einen benutzenden {\em Frontend\/}-Proze\ss{}
vorteilhaft. Einfacher zu realisieren w\"{a}re zun\"{a}chst die Bedienung
aller {\em Frontend\/}-Prozesse durch einen einzigen globalen {\em
Backend\/}-Proze\ss{}. N\"{o}tig w\"{a}re dazu die Realisierung einer
Kommunikationsschicht, die zwischen den Schichten~1 und 2 anzusiedeln
w\"{a}re.
%
\par{}Die Bedienung der {\em Frontend\/}-Prozesse durch jeweils einen 
lokalen, dazugeh\"{o}rigen {\em Backend\/}-Proze\ss{} wie in der
objekt-orientierten relationalen Datenbank \postgres\ ist im Moment
nicht realisierbar, da der in der 1.~Schicht verwendete Objektspeicher
\postore\ nicht mit mehreren Prozessen gleichzeitig auf einen
persistenten Speicher zugreifen kann \cite[\citepage{1}]{bib:br92}.
%
\subsubsection{Explizite Repr\"{a}sentation von Prozessen}
%
Benutzende Prozesse werden im Moment durch Instanzen der Klasse
\class{persistent-heap}\ implizit repr\"{a}sentiert; f\"{u}r Prozesse sollte
daher in der 2.~Schicht eine Basisklasse definiert werden. Die Klasse
\class{persistent-heap}\ sollte in diesem Zusammenhang um einen
\Slt\/ erweitert werden, der auf die Repr\"{a}sentation des benutzenden
Prozesses verweist; beim \"{O}ffnen einer Sitzung sollte die Referenz
auf die Proze\ss{}repr\"{a}sentation in diesen \Slt\/ eingetragen werden.
%%% Local Variables: 
%%% mode: latex
%%% TeX-master: "main"
%%% buffer-file-coding-system: raw-text-unix
%%% End: 
