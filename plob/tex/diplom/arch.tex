%
\chapter{Architekturkonzepte von persistenten Objektsystemen}%
\label{chap:arch}
%
\begin{fortune}
Beware of the Turing tar-pit in which everything is possible but
nothing of interest is easy.
\from{A.\ J.\ Perlis: Epigrams on Programming}
% SIGPLAN Notices, Sept. 1982, S. 7-13
\end{fortune}
%
In diesem Kapitel werden verschiedene Architekturkonzepte von
persistenten Objektsystemen er\-l\"{a}u\-tert. Die verschiedenen
M\"{o}glichkeiten zum Gesamtaufbau eines persistenten Systems werden
erkl\"{a}rt und es werden die letztendlich im weiteren Verlauf
\ifbericht dieses Berichtes \else\ifbuch dieses Buches \else der
Arbeit \fi\fi betrachteten Architekturen von persistenten
Objektsystemen in \clos\ festgelegt; f\"{u}r diese Architekturen
werden verschiedene, im Zusammenhang mit \cl\ wichtige Konzepte
vorgestellt.
%
\section{Systemarchitektur}
%
F\"{u}r die Systemarchitektur von persistenten Systemen gibt es je nach
Anforderung verschiedene Realisierungsans\"{a}tze.
%
\subsection{Verwendung von getrennten Subsystemen}
%
Der oft verwendete Weg zur Herstellung von Persistenz besteht darin,
innerhalb einer Applikation zus\"{a}tzlich zu einem
transienten Objektsystem einen persistenten Speicher einzubinden
(\figurename~\ref{fig:hperlay}).
%
\begin{figure}[htbp]%
\ifbuch%
\centerline{\psfig{figure=twelvept/hperlay.eps}}%
\else%
\centerline{\psfig{figure=tenpt/hperlay.eps}}%
\fi%
\caption{Aufbau eines persistenten Systems mit getrennten
Subsystemen}\label{fig:hperlay}%
\end{figure}%
%
Dabei werden f\"{u}r die Komponenten der untersten Schicht bereits
vorhandene Systeme verwendet. Die komplette Verwaltung der
persistenten Datenobjekte liegt in der Applikationsschicht. Dort
m\"{u}ssen die beiden Subsysteme integriert werden und erfordern
\ia\ f\"{u}r jede einzelne Applikation eine aufwendige spezielle
L\"{o}sung, insbesondere da sich oft die Konzepte von Objektsystem und
Applikation einerseits und persistentem Speicher andererseits nicht
decken.
%
\par{}Der Vorteil dieser L\"{o}sung ist die schnelle Verf\"{u}gbarkeit der
verwendeten unteren Komponenten, da sie bereits vorhanden sind. Die
Qualit\"{a}t der in der Applikationsschicht vorzunehmenden Integration
der beiden Subsysteme ist in erster Linie von der Erfahrung des
Applikationserstellers abh\"{a}ngig. Ebenso ist sie sowohl auf die
jeweilige Applikation als auch auf das f\"{u}r den persistenten Speicher
verwendete Subsystem spezialisiert.
%
\par{}Beispiele f\"{u}r diese Art von persistenten Systemen sind die
Verwendung von \cpp\ oder \clos\ als transientes Objektsystem und
einer Datei oder Datenbank als persistenten Speicher. In
Programmiersprachen
\gande{eingebettete Datenbanksprachen}{embedded database
languages}\ fallen ebenfalls unter diese Kategorie; sie bieten aber
bereits eine teilweise Integration in die verwendete
Programmiersprache an.
%
\par{}In den meisten unter \unix\ realisierten \cl-Systemen gibt es
als Hilfsmittel f\"{u}r Persistenz die M\"{o}glichkeit, den \gande{aktuellen
Zustand des laufenden \cl-Prozesses in einer (meist sogar direkt
ausf\"{u}hrbaren) Datei zu sichern}{workspace dump}; in dieser Datei
befinden sich s\"{a}mtliche Objekte des gesicherten Prozesses im Zustand
zum Zeitpunkt der Sicherung. Diese Datei kann dann zu einem sp\"{a}teren
Zeitpunkt wieder geladen werden; damit werden auch die gesicherten
Objekte wiederhergestellt. Diese Art der Sicherung umfa\ss{}t immer alle
im laufenden Proze\ss{} vorhandene Objekte; einzelne Speicherungen sind
nicht m\"{o}glich. Selektionsm\"{o}glichkeiten \"{u}ber Objekte in dieser Datei
sind nicht gegeben.
%
\par{}Die durch Verwendung von getrennten Subsystemen erreichte
Persistenz ist weder orthogonal noch transparent; Funktionalit\"{a}ten
\"{u}ber persistente Objekte m\"{u}ssen f\"{u}r jede Applikation neu
erstellt werden. Insgesamt gesehen scheidet damit diese Art der
Persistenz f\"{u}r den im Rahmen \ifbericht dieses Berichtes
\else\ifbuch dieses Buches \else dieser Arbeit \fi\fi beabsichtigten
Zweck aus.
%
\subsection{Vollst\"{a}ndig integrierte Systeme}%
\label{sec:arcis}%
%
Vollst\"{a}ndig integrierte Systeme (\figurename~\ref{fig:operlay}) bieten
Persistenz auf einem hohen Niveau. Eine auf einem solchen System
erstellte Applikation ist von oberer Sicht aus nicht auf spezielle
Hilfsmittel f\"{u}r persistente Objekte angewiessen.
%
\begin{figure}[hbtp]%
\ifbuch%
\centerline{\psfig{figure=twelvept/operlay.eps}}%
\else%
\centerline{\psfig{figure=tenpt/operlay.eps}}%
\fi%
\caption{Aufbau eines vollst\"{a}ndig integrierten persistenten
Systems}\label{fig:operlay}%
\end{figure}%
%
Die Verwaltung der persistenten Objekte befindet sich nicht mehr wie
beim ersten Ansatz in der Applikationsschicht, sondern wurde in die
unteren beiden Schichten verlagert. Damit entf\"{a}llt auch die
Notwendigkeit, f\"{u}r jede Applikation eine spezielle L\"{o}sung zu
erstellen.
%
\par{}Ein Nachteil ist der hohe Aufwand f\"{u}r die Erstellung eines
vollst\"{a}ndig persistenten Objektsystems. Eine effiziente Realisierung
wird \ia\ auch nur dann erreicht, wenn auf Persistenz
spezialisierte Rechnerarchitekturen verwendet werden.
%
\par{}Um den Aufwand f\"{u}r die Erstellung eines solchen Systems zu
verringern, k\"{o}nnte beispielsweise ein transientes Objektsystem
f\"{u}r eine Rechnerarchitektur mit nicht-fl\"{u}chtigem Speicher (wie
CMOS-RAM mit autarker Stromversorgung) angepa\ss{}t werden. Ebenso lie\ss{}e
sich in ein transientes System ein persistenter Objektspeicher
integrieren (siehe z.B.\ \cite[\citepage{12}]{bib:mu91}). Da
viele der bisher realisierten Objektsysteme nicht f\"{u}r Persistenz
ausgelegt sind, ist ein optimales System nur durch einen kompletten
Neuentwurf m\"{o}glich.
%
\par{}Damit scheidet dieser Ansatz ebenfalls aus, da ein bereits
vorhandenes \clos-System um Persistenz erweitert werden soll; die
vollst\"{a}ndige Integration von Persistenz wie in
\cite[\citepage{12}]{bib:mu91} geschildert ist hier nicht
durchf\"{u}hrbar, da ein derartiger Ansatz die Modifikation der unteren
Schichten des Systems voraussetzt, in das Persistenz integriert werden
soll, und dies f\"{u}r das vorgegebene \clos-System nicht m\"{o}glich ist.
%
\subsection{Teilweise integrierte Systeme}
%
Teilweise integrierte Systeme verbinden die beiden ersten Ans\"{a}tze.
Sie bestehen aus einem persistenten
Objektsystem, das seinerseits wieder auf einem transienten
Objektsystem beruht. Als Hilfsmittel zur Herstellung von Persistenz
wird ein nicht-fl\"{u}chtiger Speicher benutzt
(\figurename\ \ref{fig:rperlay}).
%
\begin{figure}[htbp]%
\ifbuch%
\centerline{\psfig{figure=twelvept/rperlay.eps}}%
\else%
\centerline{\psfig{figure=tenpt/rperlay.eps}}%
\fi%
\caption{Aufbau eines teilweise integrierten persistenten
Systems}\label{fig:rperlay}%
\end{figure}%
%
Das persistente Objektsystem ist kein eigenst\"{a}ndiges System, sondern
erweitert das vorhandene transiente System. Deswegen kann f\"{u}r
fl\"{u}chtige Objekte von einer Applikation weiterhin direkt das
transiente Objektsystem verwendet werden. Ebenso stehen die im
transienten Objektsystem realisierten Konzepte auch f\"{u}r persistente
Objekte zur Verf\"{u}gung. Die Verwaltung der
persistenten Objekte erfolgt wie beim zweiten Ansatz nicht in der
Applikationsschicht, sondern im persistenten Objektsystem.
%
\par{}Es ergeben sich f\"{u}r teilweise integrierte Systeme die gleichen
Vorteile wie bei den vollst\"{a}ndig integrierten Systemen. Zus\"{a}tzlich
l\"{a}\ss{}t sich der Aufwand f\"{u}r die Realisierung eines solchen Systems
relativ klein halten, da man f\"{u}r das transiente Objektsystem und den
persistenten Speicher bereits vorhandene Komponenten verwenden kann.
Um eine hohe Orthogonalit\"{a}t zwischen dem transienten und persistenten
Objektsystem zu erhalten, wird das persistente System meist in der
gleichen Programmiersprache wie das transiente System realisiert.
%
\par{}Alle in \ifbericht diesem Bericht \else\ifbuch diesem Buch
\else dieser Arbeit \fi\fi betrachteten Systeme beruhen auf diesem
Ansatz. Als transientes Objektsystem wird \clos\ verwendet. Das
persistente Objektsystem ist ebenfalls in \clos\ realisiert; einige
Systeme benutzen hier zus\"{a}tzlich in anderen Programmiersprachen
verfa\ss{}te Subsysteme. Der persistente Speicher ist je nach System
unterschiedlich; einige benutzen komplexe Subsysteme, wie relationale
und objekt-orientierte relationale Datenbanken (wie beispielsweise
\pclos\ \cite{bib:pa91a} und \soh\ \cite{bib:ro87}), andere verwenden
Objektspeicher und einfache Dateien (wie beispielsweise \wood\ 
\cite{bib:wo93} und \plob\ [\citepage{\pageref{chap:plob}}]).
%
\section{Persistenter Speicher}
%
Die im Rahmen \ifbericht dieses Berichtes \else\ifbuch dieses Buches
\else dieser Arbeit \fi\fi betrachteten persistenten Systeme verwenden
entweder eine relationale Datenbank oder einen Objektspeicher als
persistenten Speicher.
%
\subsection{Relationale Datenbank als persistenter Speicher}
%
Relationale Datenbanken sind im Bereich der Speicherung und
Manipulation von Daten sehr m\"{a}chtig; viele Realisierungen von
persistenten Objektsystemen vertreten daher den Ansatz, als
persistenten Speicher eine relationale Datenbank zu verwenden
(\figurename~\ref{fig:hlsecst}).
%
\begin{figure}[hbtp]%
\ifbuch%
\centerline{\psfig{figure=twelvept/hlsecst.eps}}%
\else%
\centerline{\psfig{figure=tenpt/hlsecst.eps}}%
\fi%
\caption{Persistentes Objektsystem mit relationaler
Datenbank}\label{fig:hlsecst}%
\end{figure}%
%
\par{}Im Rest dieses Abschnittes werden die Argumente f\"{u}r und gegen
diese Vorgehensweise diskutiert. Die hier getroffenen Aussagen sind
generell gehalten; auf spezielle Aspekte wird sp\"{a}ter in diesem
Kapitel eingegangen.
%
\subsubsection{Einfache Abbildung zwischen Objektsystem- und
Da\-ten\-bank-Kon\-zep\-ten}%
%
Die Hoffnung bei dieser Vorgehensweise ist zum einen, die im
transienten Objektsystem realisierten Konzepte m\"{o}glichst einfach auf
entsprechende Datenbank-Konzepte abbilden zu k\"{o}nnen und zum anderen
die von der Datenbank angebotenen M\"{o}glichkeiten auch f\"{u}r das
persistente Objektsystem verf\"{u}gbar zu machen.
%
\par{}Die Konzepte von \clos\ einerseits und relationalen Datenbanken
andererseits unterscheiden sich stark. Aus den in
Abschnitt~\ref{sec:rdb} (\citepage{\pageref{sec:rdb}}) geschilderten
Eigenheiten einer relationalen Datenbank folgt, da\ss{} die \"{U}bertragung
des Objektmodells von \clos\ auf das relationale Datenmodell
Schwierigkeiten bieten wird, sofern orthogonale Persistenz angestrebt
wird.
%
\par{}Reale Entit\"{a}ten werden in relationalen Datenbanken zu
Relationen und in \clos\ zu Klassen abstrahiert; damit ergeben sich
die in \tablename~\ref{tab:objvsdb} %
%
\begin{figure}[htbp]\centering%
\begin{\figurefontsize}%
\begin{tabular}{|c|c|}%
\hline
\tabularheader{Objektsystem}
                &\tabularheader{Relationale Datenbank}\\
\hline\hline
\cls            & Relation      \\
Instanz         & Tupel         \\
Effektiver \Slt & Attribut      \\
\hline
\end{tabular}%
\end{\figurefontsize}%
\let\figurename\tablename%
\caption{Datenstrukturen in Objektsystem und Datenbank}%
\label{tab:objvsdb}%
\end{figure}%
%
dargestellten Zusammenh\"{a}nge. Der \Slt\/-Zustand eines \clos-Objektes
kann einen Wert oder ein dynamisch typisiertes Objekt referenzieren;
in relationalen Datenbanken ist nur genau ein einfacher, auf Werte
festgelegter Bereich f\"{u}r ein Attribut erlaubt.  Deswegen sollte der
\Slt\/-Zustand so in einem Attribut repr\"{a}sentiert werden, da\ss{} aus
der Repr\"{a}sentation dynamisch der Typ des repr\"{a}sentierten Objektes
hervorgeht und so \ua\ eine Unterscheidung zwischen referenziertem
Wert und referenzierter Instanz m\"{o}glich wird.  Eine derartige
\representationform{} kann dazu f\"{u}hren, da\ss{} der Zustand des
repr\"{a}sentierten Objektes so konvertiert werden mu\ss{}, da\ss{} er von der
relationalen Datenbank in seiner urspr\"{u}nglichen Bedeutung nicht mehr
erfa\ss{}t werden kann und deswegen ihre Manipulationsm\"{o}glichkeiten f\"{u}r
persistente Objekte nicht vern\"{u}nftig genutzt werden k\"{o}nnen. Dies
kann nur vermieden werden, wenn der Objektzustand selbst und keine aus
einer Konvertierung hervorgangene Repr\"{a}sentation daf\"{u}r in der
Datenbank abgelegt wird. Da Attribute statisch auf einfache Bereiche
festgelegt werden m\"{u}ssen, wirkt sich dies auf persistente Objekte
r\"{u}ckwirkend dahingehend aus, da\ss{} ihre \Slt[s]\/ ebenfalls statisch
auf einen einfachen Basistyp typisiert werden m\"{u}ssen, der zu einem
einfachen Bereich des Datenbanksystems korrespondiert; damit geht
allerdings die Orthogonalit\"{a}t des persistenten Systems verloren.
%
\par{}Eine Abbildung der Konzepte ist entweder nur mit sehr
hohem Aufwand m\"{o}glich oder zieht Einschr\"{a}nkungen nach sich, die f\"{u}r
einzelne Applikationen dazu f\"{u}hren k\"{o}nnen, da\ss{} die Verwendung eines
derartig realisierten persistenten Objektsystems ausscheidet
\cite[\citepage{305}]{bib:broe89} \cite[\citepage{52}]{bib:ni88}.
%
\subsubsection{Verlagerung des {\em impedance mismatch}}%
%
In jeder Applikation, die eine relationale Datenbank direkt benutzt,
mu\ss{} ein Teil des Codes nur erstellt werden, um den {\em impedance
mismatch\/} zwischen Applikation einerseits und relationaler Datenbank
andererseits auszugleichen. Bei der Verwendung eines persistenten
Objektsystems unter Benutzung einer relationalen Datenbank als
persistenten Speicher wird der {\em impedance mismatch\/} von allen
Applikationen in das persistente Objektsystem verlagert, so da\ss{} er
f\"{u}r die BenutzerInnen des persistenten Systems weitestgehend
verschwindet; lediglich die AutorInnen des persistenten Objektsystems
m\"{u}ssen sich um eine L\"{o}sung des Problems bem\"{u}hen.
%
\par{}Eine hinreichend generelle L\"{o}sung dieses Problems innerhalb des
persistenten Objektsystems mu\ss{} sowohl definiert als auch (m\"{o}glichst
effizient) realisiert werden; der Aufwand daf\"{u}r kann sehr hoch
werden.
%
\subsubsection{Effizienz}%
%
Die Entwicklung von relationalen Datenbanken ist weit fortgeschritten;
inzwischen sind relationale Datenbanken erh\"{a}ltlich, die zum einen der
Definition nach Codd gen\"{u}gen und au\ss{}erdem \"{u}ber effiziente
M\"{o}glichkeiten der Daten-Haltung und -Manipulation verf\"{u}gen. Damit
wird zum einen die Realisierung eines persistenten Objektsystems
beschleunigt (da eine vorhandene Datenbank benutzt wird) und zum
anderen k\"{o}nnen die persistenten Instanzen effektiv manipuliert
werden.
%
\par{}Dies funktioniert nur, wenn die \og[en]\ Einschr\"{a}nkungen
akzeptabel sind, die aus der Abbildung zwischen
\clos- und Datenbank-Konzepten resultieren.
%
\subsection{Objektspeicher und Datei als persistenter Speicher}%
\label{sec:objstore}
%
Ein Objektspeicher bietet Persistenz auf \rglq{}niedrigem\rgrq\ Niveau
(\figurename~\ref{fig:llsecst}).
%
\begin{figure}[htbp]%
\ifbuch%
\centerline{\psfig{figure=twelvept/llsecst.eps}}%
\else%
\centerline{\psfig{figure=tenpt/llsecst.eps}}%
\fi%
\caption{Persistentes Objektsystem mit
Objektspeicher}\label{fig:llsecst}%
\end{figure}%
%
Er ist lediglich in der Lage, \cl-Daten in einer Datei abzulegen und
auf diese mit im Vergleich zu einer relationalen Datenbank sehr
bescheidenen Konstrukten zuzugreifen. Die M\"{o}glichkeit zur Typbildung
wie in relationalen Datenbanken ist nicht vorhanden; die im
Objektspeicher enthaltenen Daten sind auf Objektspeicherebene nicht
typisiert. Ein Objektspeicher wird oft auch als persistenter
\heap\ bezeichnet, da er \"{a}hnlich wie ein transienter
\heap\ realisiert wird.
%
\subsubsection{Referenzen zwischen Objekten}%
%
Innerhalb des Objektspeichers werden Objekte durch Datens\"{a}tze
repr\"{a}sentiert. Dem Objektspeicher wird mitgeteilt, welche
Bestandteile eines Datensatzes als Referenz auf (andere) Datens\"{a}tze
und welche als Werte zu interpretieren sind; die Aufteilung in
Referenzen und Werte wird aus der Strukturbeschreibung des Objektes
abgeleitet.  Damit sind wie in transienten Objektsystemen Referenzen
zwischen denen durch die Datens\"{a}tze repr\"{a}sentierten Objekte
innerhalb des Objektspeichers m\"{o}glich. Durch die explizite
Repr\"{a}sentation von Referenzen gen\"{u}gt ein Objektspeicher nicht dem
relationalen Modell.
%
\subsubsection{Effizienz}%
%
Der Aufwand f\"{u}r die Konvertierung eines Objektzustands in eine Form,
die vom Objektspeicher verarbeitet werden kann, ist \ia\ gering.
Objektzust\"{a}nde k\"{o}nnen teilweise unver\"{a}ndert zwischen
Objektspeicher und persistentem Objektsystem ausgetauscht werden;
lediglich f\"{u}r Referenzen m\"{u}ssen Konvertierungen vorgenommen werden.
%
\par{}Die im Objektspeicher abgelegten Daten werden bis auf die
Referenzen nicht weiter interpretiert oder besonders verarbeitet.
%
\subsubsection{Wahl der Strukturen im Objektspeicher}%
%
Da die Zust\"{a}nde der Datens\"{a}tze vom Objektspeicher nicht weiter
interpretiert werden, lassen sie sich mit beliebigen Strukturen
anlegen; insbesondere k\"{o}nnen die Strukturen so gew\"{a}hlt werden, da\ss{}
sie sich an dem transientem System orientieren, das den Objektspeicher
benutzt. Damit kann der {\em impedance mismatch\/} bez\"{u}glich der
Datenrepr\"{a}sentation zwischen \clos\ und Objektspeicher vermieden
werden, da f\"{u}r Zust\"{a}nde innerhalb des Objektspeichers die jeweilige
Repr\"{a}sentation des \cl-Systems verwendet werden kann. Eine
direkte \"{U}bernahme der Zust\"{a}nde bedeutet aber, da\ss{} ebenfalls die
Interpretation der Zust\"{a}nde, so wie sie im  verwendeten \cl-System
festgelegt ist, implizit mit \"{u}bernommen wird; wenn an den
persistenten Speicher verschiedene persistente Objektsysteme
angebunden werden sollen, mu\ss{} die jeweilige Interpretation des
Zustands den verschiedenen Systemen bekannt sein.
%
\subsubsection{Speicherr\"{u}ckgewinnung}%
\addglossary{Speicherr\"{u}ckgewinnung}[garbage collection]%
[Freigabe des Speicherplatzes von nicht-erreichbaren Objekten.]
Da Objektspeicher Referenzen zwischen den in ihnen enthaltenen
Datens\"{a}tzen erm\"{o}glichen, l\"{a}\ss{}t sich innerhalb eines Objektspeichers
eine auf Erreichbarkeit basierende Speicherr\"{u}ckgewinnung realisieren
(siehe Abschnitt~\ref{sec:gc}).
%
\subsubsection{Typfreiheit der Daten innerhalb des Objektspeichers} Ein
Objektspeicher schreibt dem benutzenden
System keine bestimmten Typisierungskonzepte vor. Die fehlenden Konzepte
k\"{o}nnen auf einer h\"{o}heren Ebene (hier im persistenten Objektsystem)
realisiert werden, die dann dem benutzenden System (hier \clos)
angepa\ss{}t werden k\"{o}n\-nen.
%
%
\section{Lebensdauer von Objekten}
%
Die Definition der Lebensdauer von Objekten erfolgt analog zu der
Festlegung in \cl\ \stcite{43}, da\ss{} jedes Objekt eine unendlich lange
Lebensdauer hat. Wenn das (persistente) System beweisen kann, da\ss{} ein
Objekt nicht transitiv von einem Wurzelobjekt aus erreichbar ist, darf
es gel\"{o}scht werden (\figurename~\ref{fig:reachabl}). %
%
\begin{figure}[htbp]%
\ifbuch%
\centerline{\psfig{figure=twelvept/reachabl.eps}}%
\else%
\centerline{\psfig{figure=tenpt/reachabl.eps}}%
\fi%
\captionfrom%
 {Erreichbarkeit von Objekten}%
 {\cite[\citepage{245}]{bib:th90}}%
 \label{fig:reachabl}%
\end{figure}%
%
Ein erreichbares Objekt \rglq{}wird noch gebraucht\rgrq\ und darf vom
System nicht gel\"{o}scht werden.
%
\subsection{Lebensdauer in relationalen Datenbanken}
%
Innerhalb einer rein relationalen Datenbank kann die Lebensdauer von
Daten nicht \"{u}ber ihre Erreichbarkeit definiert werden, da die
Datenbank die in ihr enthaltenen Referenzen zwischen den Tupeln nicht
explizit repr\"{a}sentiert. Das L\"{o}schen von Tupeln in relationalen
Datenbanken ist nur \"{u}ber explizite Aufrufe entsprechender Funktionen
durch die BenutzerInnen m\"{o}glich; zus\"{a}tzlich mu\ss{} dabei von den
BenutzerInnen sichergestellt werden, da\ss{} keine
\gande{ung\"{u}ltigen Referenzen}{dangling references}%
\addglossary{Ung\"{u}ltige Referenz}[dangling reference][Eine
un\-g\"{u}l\-ti\-ge Referenz ist eine
Iden\-ti\-t\"{a}ts\-re\-pr\"{a}\-sen\-ta\-tion, die kein g\"{u}ltiges Objekt
referenziert.]\ auf bereits gel\"{o}schte Objekte
entstehen. Beispielsweise k\"{o}nnen in dem auf einer relationalen
Datenbank basierendem persistenten Objektsystem \pclos\ Objekte
explizit ge\-l\"{o}scht werden, das System \"{u}berpr\"{u}ft aber nicht, ob auf
das gel\"{o}schte Objekt noch Referenzen existieren
\cite[\citepage{18}]{bib:pa91a}.
%
\subsection{Speicherr\"{u}ckgewinnung}%
\label{sec:gc}%
%
\begin{fortune}[11cm]
\hfill\begin{tabular}{lp{9cm}}
Don:    & I didn't know you had a cousin Penelope, Bill!  Was she
          pretty?\\
W.\ C.: & Well, her face was so wrinkled it looked like seven miles of
          bad road.  She had so many gold teeth, Don, she use
% Aus /usr/games/lib/fortunes.dat; da steht wirklich ,,use'', nicht
% das korrekte ,,used''
          {\em (sic!)\/} to have to sleep with her head in a safe.
          She died in Bolivia.\\
Don:    & Oh Bill, it must be hard to lose a relative.\\
W.\ C.: & It's almost impossible.
\end{tabular}
\from{W.\ C.\ Fields: %
 \rglqq{}The Further Adventures of Larson
 E.\ Whipsnade and other Tarradiddles\rgrqq}
\end{fortune}
%
Die Speicherr\"{u}ckgewinnung hat die Aufgabe, nicht erreichbare Objekte
zu finden und den von ihnen allozierten Speicherplatz freizugeben.
Aktiviert wird sie durch Eintritt einer Vorbedingung, wie z.B.\ bei
einem bestimmten F\"{u}llungsgrad des Speichers oder w\"{a}hrend
\gande{Freizeiten}{idle times}%
\addglossary{Freizeiten}[idle time]%
[Zeitabschnitte, in denen kein Proze\ss{} aktiv ist.]\ des Systems.
Damit ein System feststellen kann, ob ein Objekt erreichbar ist oder
nicht und damit gel\"{o}scht werden kann, mu\ss{} es Referenzen zwischen den
in ihm enthaltenen Objekten explizit repr\"{a}sentieren k\"{o}nnen.  Bei der
Begrenzung der Lebensdauer von Objekten \"{u}ber ihre Erreichbarkeit
sollte zur Vermeidung von inkonsistenten Referenzen auf eine nach
au\ss{}en zur Verf\"{u}gung gestellte explizite L\"{o}schoperation f\"{u}r Objekte
verzichtet werden; ein Objekt wird durch nicht-referenzieren f\"{u}r das
Objektsystem als zu l\"{o}schendes Objekt gekennzeichnet.
%
\par{}Eine Speicherr\"{u}ckgewinnung f\"{u}r persistente Objekte mu\ss{}
innerhalb des persistenten Objektsystems oder des persistenten
Speichers realisiert werden, da \ia\ die
Spei\-cher\-r\"{u}ck\-ge\-win\-nung des transienten Objektsystems nicht
f\"{u}r persistente Objekte genutzt werden kann.
%
\par{}Die in \ifbericht diesem Bericht \else\ifbuch diesem Buch \else
dieser Arbeit \fi\fi betrachteten persistenten Objektsysteme haben
entweder keine Speicherr\"{u}ckgewinnung oder realisieren sie
ebenfalls mittels Erreichbarkeit. Realisierte Vorgehensweisen f\"{u}r
R\"{u}ckgewinnung von persistentem Speicher lassen sich nach dem
Zeitpunkt ihres Geschehens in zwei Gruppen einteilen. Bei der ersten
Gruppe findet eine R\"{u}ckgewinnung von persistentem Speicher zur
\online\ Zeit statt, w\"{a}hrend der der persistente Speicher durch
ein persistentes Objektsystem benutzt wird. Die zweite Gruppe
f\"{u}hrt eine R\"{u}ckgewinnung zur \offline\ Zeit durch, w\"{a}hrend
der der persistente Speicher nicht genutzt wird.
%
\subsection{Keine R\"{u}ckgewinnung von persistentem Speicher}
%
Das Fehlen einer Speicherr\"{u}ckgewinnung ist oft durch die Konzepte des
zur Realisierung verwendeten persistenten Speichers begr\"{u}ndet. Bei
Verwendung einer relationalen Datenbank als persistenten Speicher
stellt sich das Problem, da\ss{} die Datenbank keine Referenzen der in
ihr enthaltenen Daten explizit re\-pr\"{a}\-sen\-tie\-ren kann; damit
m\"{u}ssen Referenzen zwischen persistenten Objekten auf einer h\"{o}heren
Ebene innerhalb des persistenten Objektsystems realisiert
werden. Daf\"{u}r werden Konstrukte gew\"{a}hlt, die zwar sehr effizient ein
persistentes Objekt mit allen von ihm referenzierten Objekten
materialisieren k\"{o}nnen, bei denen aber die Ermittlung der von einem
persistenten Objekt lediglich referenzierten Objekte eben genau diese
Materialisierung bedingt.
%
Eine R\"{u}ckgewinnung von persistentem Speicher wurde aus diesen
Gr\"{u}nden bei den in \ifbericht diesem Bericht \else\ifbuch diesem
Buch \else dieser Arbeit \fi\fi betrachteten Systemen, die eine
Datenbank als persistenten Speicher nutzen, nicht realisiert.
%
% Alter Text; von A. Gawecki mit dem Attribut ,,naiv'' belegt:
%Bei einer Speicherr\"{u}ckgewinnung w\"{u}rden also alle im persistenten
%Speicher abgelegten Objekte in den transienten Speicher geladen
%werden; da der persistente Speicher oft wesentlich gr\"{o}\ss{}er als der
%zur Verf\"{u}gung stehende transiente Speicher ist, wird eine
%Speicherr\"{u}ckgewinnung unm\"{o}glich.
%
\subsection{R\"{u}ckgewinnung von persistentem Speicher zur
\online\ Zeit}
%
Eine R\"{u}ckgewinnung zur Laufzeit einer Applikation hat den Vorteil
einer hohen Verf\"{u}gbarkeit des persistenten Speichers, da es keine
Ausfallzeiten gibt. Nicht erreichbare Objekte werden noch zur Laufzeit
als freier Speicher erkannt; damit wird eine gute Ausnutzung des
persistenten Speichers erreicht.
%
\par{}Persistente Objektsysteme lassen aus Effizienzgr\"{u}nden oft
direkte Referenzen von Variablen oder Objekten des transienten Systems
auf persistente Objekte zu; damit kann es bei einer
Speicherr\"{u}ckgewinnung zu Problemen kommen, wenn ein persistentes
Objekt zwar von einer transienten Variablen bzw.\ einem transienten
Objekt referenziert wird, diese Rerefenz aber nicht im Wurzelobjekt
\rglq{}eingetragen\rgrq\ ist (\figurename~\ref{fig:gctrpe}).
%
\begin{figure}[htbp]%
\ifbuch%
\centerline{\psfig{figure=twelvept/gctrpe.eps}}%
\else%
\centerline{\psfig{figure=tenpt/gctrpe.eps}}%
\fi%
\caption{Erreichbarkeit eines persistenten Objektes \"{u}ber eine Variable
des transienten Systems}\label{fig:gctrpe}%
\end{figure}%
%
Eine vom persistenten Objektsystem durch\-ge\-f\"{u}hr\-te R\"{u}ckgewinnung
von persistentem Speicher w\"{u}rde in diesem Beispiel die Objekte
\ov\ und \ovi\ als nicht erreichbare Objekte markieren und als freien
Speicher erneut zur Verf\"{u}gung stellen, obwohl sie von der
(transienten) Variablen \lisp{P}\ referenziert werden.
%
\par{}Zur Vermeidung einer solchen Situation mu\ss{} das persistente
Objektsystem eine Konvention oder ein Protokoll f\"{u}r seine
Speicherr\"{u}ckgewinnung festlegen. Eine Konvention k\"{o}nnte so aussehen,
da\ss{} solche Referenzen unber\"{u}cksichtigt bleiben. Ein Protokoll w\"{u}rde
z.B.\ festlegen, da\ss{} die Speicherr\"{u}ckgewinnung explizit und synchron
durch die BenutzerInnen aktiviert wird; in diesem Beispiel f\"{u}hrt eine
R\"{u}ckgewinnung von persistentem Speicher zur \online\ Zeit durch das
von den BenutzerInen einzuhaltende Protokoll zu einer verringert
transparenten Persistenz.
%
\subsection{R\"{u}ckgewinnung von persistentem Speicher zur
\offline\ Zeit}
%
Diese Art der Speicherr\"{u}ckgewinnung kann nur dann durchgef\"{u}hrt
werden, wenn keine Applikation den persistenten Speicher benutzt.
Dies setzt voraus, da\ss{} es solche Zeiten \"{u}berhaupt gibt. Au\ss{}erdem
ist der persistente Speicher w\"{a}hrend der Speicherr\"{u}ckgewinnung nicht
verf\"{u}gbar.
%
\par{}Nicht erreichbare Objekte werden zur Laufzeit einer
Applikation nicht als freier persistenter Speicher zur
Verf\"{u}gung gestellt; damit kann ein \"{U}berlauf des persistenten
Speichers auftreten, obwohl gen\"{u}gend nicht erreichbare
Objekte vorhanden sein k\"{o}nnten.
%
\par{}Vorteilhaft ist, da\ss{} bei dieser Art der Speicherr\"{u}ckgewinnung
keine Probleme mit Referenzen des transienten Systems auf persistente
Objekte auftreten k\"{o}nnen; ein Protokoll zwischen transientem und
persistentem Objektsystem ist nicht notwendig. Der Zeitpunkt der
Spei\-cher\-r\"{u}ck\-ge\-win\-nung kann so gelegt werden, da\ss{} die zur
Verf\"{u}gung stehenden Ressourcen besser oder intensiver genutzt werden
k\"{o}nnen.  Insbesondere verbraucht sie dann keine Ressourcen, die
ansonsten einer Applikation zugute kommen w\"{u}rden. Damit kann sie
neben ihrer eigentlichen Aufgabe auch Optimierungen innerhalb des
persistenten Speichers vornehmen, wie z.B.\ \gande{nahe Gruppierung
von Objekten bei ihren referenzierten Objekten}{clustering}[Nahe
Gruppierung von Objekten bei ihren referenzierten Objekten.].
%
\section{Typisierung der \protect\objid}
%
F\"{u}r \immval[s]\/ kann eine \objid\/ den gesamten Objektzustand selbst
enthalten.  F\"{u}r alle anderen Objekte existiert eine \objid\/ nur in
Verbindung mit einem Datenobjekt, da sie ausschlie\ss{}lich zur
Identit\"{a}tsrepr\"{a}sentation des dazugeh\"{o}rigen Objektes dient. In
beiden F\"{a}llen hat eine \objid\/ selbst den Charakter eines
Wertes,\footnote{Als Identit\"{a}tsrepr\"{a}sentation nehme man
beispielsweise die Speicheradresse eines Objektes an. Der erste Fall
entspricht der Situation, da\ss{} ein \immval\/ keine Adresse hat. Im
zweiten Fall wird ausgesagt, da\ss{} nur existierende Objekte eine
Adresse haben und da\ss{} diese Adresse selbst keine Adresse hat.}
allerdings keines \rglq{}gew\"{o}hnlichen\rgrq\ Wertes im Sinne einer
Instanz der in einem Objektsystem angebotenen Typen oder Klassen,
sondern eine \objid\/ wird als Repr\"{a}sentation f\"{u}r die Identit\"{a}t
eines Datums interpretiert. Diese Eigenschaft sollte sich soweit
m\"{o}glich in der tats\"{a}chlich gew\"{a}hlten Repr\"{a}sentation f\"{u}r
\objid[s]\ selbst wiederfinden. Im Sinne eines Objektsystems mit
Identit\"{a}tsrepr\"{a}sentation sollte damit
der Typ f\"{u}r \objid[s]\ \rglq{}neben\rgrq\ allen anderen Typen
bzw.\ Klassen stehen. In \cite[\citepage{325}]{bib:be87} wird
der Benutzung von \objid[s]\ die Bedeutung einer
\gande{\rglq{}Meta-Ebene\rgrq}{metalevel}\ zugewiesen.
%
\par{}Bei Realisierung eines persistenten Systems mittels eines
vorhandenen transienten Systems ist man f\"{u}r die Wahl einer
geeigneten Repr\"{a}sentation f\"{u}r \objid[s]\ selbst auf die
M\"{o}glichkeiten des transienten Systems angewiesen. Ein solches System
bietet zum einen \rglq{}eingebaute\rgrq\ Basistypen und -klassen, deren
Instanzen z.T.\ Werte oder \immval[s]\/ sind, und
zum anderen die M\"{o}g\-lich\-keit, \rglq{}h\"{o}here\rgrq\ Typen oder
Klassen zu definieren.
%
\par{}Kennzeichnend f\"{u}r Basisklassen, deren Instanzen als Werte oder
\immval[s]\/ angesehen werden, ist, da\ss{} jede von ihnen f\"{u}r ihre
Instanzen eine eigene, sehr spezielle Datenrepr\"{a}sentation hat. Ebenso
fehlt den Basisklassen im Gegensatz zu den \rglq{}h\"{o}heren\rgrq\ Klassen
die M\"{o}glichkeit zur Subklassenbildung. Vorteilhaft ist die hohe
Effizienz ihrer Instanzen bez\"{u}glich Speicherplatz und Verarbeitung.
Bei allen \rglq{}h\"{o}heren\rgrq\ Klassen hingegen werden einander
\"{a}hn\-li\-che In\-stanz\-re\-pr\"{a}\-sen\-ta\-tio\-nen verwendet, bei
denen ein Teil allen Instanzen aller Klassen gemeinsam ist und der
restliche Teil aus der Stellung der jeweiligen Klasse in der
Klassenhierarchie abgeleitet wird.
%
\par\Objid[s]\ sollten m\"{o}glichst effizient realisiert werden. Damit
bietet es sich an, den Typ f\"{u}r \objid[s]\ innerhalb des persistenten
Objektsystems aus den bereits vorhandenen Basistypen f\"{u}r Werte oder
\immval[s]\/ des transienten Systems abzuleiten. Da eine
Subklassenbildung nicht in Frage kommt,
wird aus den Basistypen eine m\"{o}glichst einfache Struktur als Typ f\"{u}r
\objid[s]\ festgelegt.  Die aus diesem Typ gebildeten
\objid[s]\ m\"{u}ssen f\"{u}r das Objektsystem als solche erkennbar sein,
d.h.\ sie m\"{u}ssen sich entweder von \rglq{}normalen\rgrq\ Werten von
Basistypen unterscheiden oder aus dem Kontext, in dem sich der Wert
befindet, mu\ss{} hervorgehen, da\ss{} er als \objid\/ oder als
\rglq{}normaler\rgrq\ Wert verstanden werden soll.  Wenn als Typ f\"{u}r
\objid[s]\ z.B.\ Zeichenketten verwendet werden (siehe auch letztes
Beispiel in \tablename~\ref{tab:exhandle},
\citepage{\pageref{tab:exhandle}}), mu\ss{} entweder aus einer
Zeichenkette hervorgehen, ob sie als \rglq{}normale\rgrq{} Zeichenkette
oder als \objid\/ verstanden werden soll oder die Zeichenkette mu\ss{}
sich in einem Kontext befinden, in dem ihre Bedeutung festgelegt ist.
%
\section{Strukturbeschreibungen}
%
Um den Zustand eines Objektes im persistenten Speicher abzulegen, mu\ss{}
die Struktur des Objektes bekannt sein. F\"{u}r m\"{o}glichst transparente
Persistenz sollte diese Information nicht beim Ablegen des Objektes im
persistenten Speicher f\"{u}r jede Instanz explizit angegeben werden
m\"{u}ssen, sondern sie sollte aus der die Struktur des Objektes
definierenden Klasse bestimmt werden. Bei Realisierung eines
persistenten Systems, das auf einem vorhandenem transientem System
beruht, sollten daher Informationen \"{u}ber die Klasse der persistent zu
haltenden Objekte verf\"{u}gbar sein. Programmiersysteme lassen sich
nach Verf\"{u}gbarkeit und Art von Strukturbeschreibungen einteilen.
%
\subsection{Keine Verf\"{u}gbarkeit der Strukturbeschreibungen}%
%
In den klassischen Programmiersprachen wie \pascal\ und
\clogo\ sowie den darauf zum Teil basierenden objekt-orientierten
Sprachen wie \cpp\ und \eiffel\ werden Strukturbeschreibungen
ausschlie\ss{}lich vom \gande{\"{U}bersetzer}{compiler}[]\ generiert und
benutzt. Zur Laufzeit kommen diese transienten Systeme mit sehr
marginalen Typmerkmalen aus, wie z.B.\ der Gr\"{o}\ss{}e der Instanzen f\"{u}r
Allozierung von Objekten auf dem \heap\/; weitere Informationen sind
nicht notwendig und werden demzufolge auch nicht repr\"{a}sentiert.
%
\subsection{Statische Strukturbeschreibungen}%
%
Statische Strukturbeschreibungen werden w\"{a}hrend der \"{U}bersetzung
erzeugt. Sie werden haupt\-s\"{a}ch\-lich in Objektsystemen verwendet,
die eine der \og[en]\ klassischen Sprachen mit typisierten Variablen
und nur eingeschr\"{a}nkt typisierten Datenobjekten um Persistenz
erweitern.
%
\par{}Einige \"{U}bersetzer bieten die M\"{o}glichkeit, die intern erzeugten
Strukturbeschreibungen \rglq{}nach au\ss{}en hin\rgrq\ verf\"{u}gbar zu machen
und zus\"{a}tzlich die Datenobjekte zu typisieren.  Die generierten
Strukturbeschreibungen werden dann zur Laufzeit vom persistenten
Objektsystem eingelesen und erm\"{o}glichen in Verbindung mit der
Typisierung der Datenobjekte eine Ermittlung ihrer
Strukturbeschreibung und damit eine Zerlegung der Instanzen in ihre
Komponenten. Ein Beispiel f\"{u}r diese Vorgehensweise ist das
Texas-System \cite[\citepage{147}]{bib:skw93}, das von
\gnu\ \cpp\ erzeugte Typinformationen und typisierte Datenobjekte
benutzt.
%
\par{}Diese Erweiterung erm\"{o}glicht zwar typisierte Datenobjekte und
Strukturbeschreibungen, gleich\-zei\-tig treten damit aber
Inkonsistenzen mit der urspr\"{u}nglich definierten Sprachsemantik
auf. Der Typ eines Objektes geht in diesem Fall nicht mehr wie bisher
aus der Variablendeklaration zur \"{U}bersetzungszeit, sondern aus der
Typinformation des Datenobjektes zur Laufzeit hervor; diese
Bedeutungs\"{a}nderung wird von solchen Erweiterungen \ia\ nicht
ber\"{u}cksichtigt.
%
\par{}Andere auf den \og[en]\ Sprachen basierende persistente
Objektsysteme verwenden
\gande{\protect\mbox{(Vor-)}\"{U}bersetzer}{(pre-)compiler}, die die
ben\"{o}tigten Strukturbeschreibungen aus dem Quellcode ermitteln.
Zus\"{a}tzlich modifizieren oder erweitern sie den Quellcode f\"{u}r
den nachfolgenden \"{U}bersetzerlauf so, da\ss{} die in diesen
Sprachen urspr\"{u}nglich fehlende Laufzeitunterst\"{u}tzung f\"{u}r
Persistenz durch das persistente Objektsystem m\"{o}glich wird.
Beispiele f\"{u}r diese Art von Persistenz sind {\em E\/}
\cite[\citepage{175, 181}]{bib:ri89} und OBST \cite{bib:uhl93}.
%
\par{}Kennzeichnend bei diesen Systemen ist, da\ss{} die statischen
Strukturbeschreibungen nicht in die verwendete Programmiersprache
integriert sind; in vielen persistenten Systemen werden die
Strukturbeschreibungen auch nicht zur Laufzeit explizit benutzt oder
repr\"{a}sentiert, sondern finden sich lediglich implizit wieder,
z.B.\ durch von einem Vor\"{u}bersetzer vorgenommene \"{A}n\-de\-run\-gen
oder Erweiterungen des Quellcodes. 
%
\par{}Die so erreichte Persistenz verlangt eine strenge Disziplin beim
Erstellen von Quellcodes, die das persistente Objektsystem benutzen,
da sich aus der Verwendung von statisch typisierten
Variablenbezeichern zur Repr\"{a}sentation von dynamisch typisierten
Datenobjekten Widerspr\"{u}che oder Versionskonflikte ergeben k\"{o}nnen. Je
nach verwendeter Programmiersprache d\"{u}rfen einige Konstrukte f\"{u}r
persistente Daten nicht benutzt werden, da sie die f\"{u}r das
persistente Objektsystem ermittelten statischen Typ-Informationen
invalidieren w\"{u}rden (z.B.\ \gande{Typwandlungen}{type casts}\ und
\gande{variante Verbunde}{unions}\ in derartig um Persistenz
erweitertem \cpp; diese Konstrukte erm\"{o}glichen es einem Objekt, bei
statisch ermittelter Typinformation den Typ zur Laufzeit zu wechseln
und sind damit widerspr\"{u}chlich).
%
\subsection{Dynamische Strukturbeschreibungen}%
%
In Systemen mit typisierten Datenobjekten werden oft auch ihre Klassen
explizit repr\"{a}sentiert, da sie zur Laufzeit die Realisierung von
Sprachkonzepten (wie Polymorphie, \"{A}nderungen innerhalb der
Klassenhierarchie zur Laufzeit) unterst\"{u}tzen oder sonstige Vorteile
bieten, wie z.B.\ einfachere Introspektionsm\"{o}glichkeiten f\"{u}r
Datenobjekte. Ein persistentes Objektsystem kann damit die vom
transienten System verwalteten Typinformationen benutzen, um den
Zustand von Instanzen in seine Bestandteile zu zerlegen.
%
\subsection{Strukturbeschreibungen von \protect\cl\ Daten}%
%
In reinem \cl\ sind zwar die Datenobjekte typisiert, im Gegensatz zu
\clos\ liegen aber keine expliziten Strukturbeschreibungen von
\cl\ Daten vor, sondern sie sind in das jeweilige System
\rglq{}eingebaut\rgrq. Direkte Informationen \"{u}ber diese
Strukturbeschreibungen sind damit im \cl-System nicht vorhanden; dies
bedeutet, da\ss{} sie anderweitig beschafft werden m\"{u}ssen, z.B.\ durch
entsprechende \cl\ Spezifikationen wie in
\cite{bib:st90}, Auswertung der Funktionen f\"{u}r den Zugriff auf den
Objektzustand, Offenlegung der Strukturbeschreibungen durch die
AutorInnen des \cl-Systems oder, schlimmstenfalls, durch Untersuchung
einer der verwendeten \representationforms\ von Instanzen eines Typs
mit Ableitung der ben\"{o}tigten Informationen daraus. Beispiele f\"{u}r
verschiedene \cl\ Daten und ihre Strukturbeschreibungen sind:
%
\begin{description}
%
\item[{\em cons\/}-Zelle] Eine {\em cons\/}-Zelle besteht aus
zwei Referenzen {\em car\/} und {\em cdr}.
%
\item[Symbol] Ein Symbol besteht aus einer Referenz auf seine externe
Repr\"{a}sentation (Namen), einer Referenz auf seinen Wert, einer
Referenz auf sein Funktionsobjekt und einer Referenz auf seine
{\em Property\/}-Liste.
%
\item[Felder] Fast alle relevanten Informationen \"{u}ber
\gande{Felder}{arrays}\ k\"{o}nnen \"{u}ber
\gande{Feld-In\-for\-ma\-tions\-funk\-tio\-nen}%
{array information functions}\ erfragt werden; aus den von diesen
Funktionen zur\"{u}ckgegebenen Werten kann daher implizit eine
Strukturbeschreibung f\"{u}r Felder generiert werden.
%
\end{description}
%
Die derartig erlangten Strukturbeschreibungen von reinen \cl\ Daten
sind statisch; da nicht zu erwarten ist, da\ss{} sich die in
\cite{bib:st90} gegebenen Definitionen von Basistypen signifikant
\"{a}ndern werden,  sind keine Inkompatibilit\"{a}ten der
Sturkturbeschreibungen \"{u}ber die Zeit zu bef\"{u}rchten.
%
\subsection{Strukturbeschreibungen von \clos-Instanzen}
%
Die Klassen von \clos-Instanzen werden zur Laufzeit durch
\clsmo[e]\ repr\"{a}sentiert, von denen alle notwendigen
Informationen f\"{u}r die Ablage ihrer Instanzen im Objektspeicher
erhalten werden k\"{o}nnen.  \clsmo[e]\ entsprechen dynamischen
Strukturbeschreibungen.
%
\section{Zusammenfassung}
%
F\"{u}r die Integration von Persistenz in \clos\ gibt es die beiden
Ans\"{a}tze, als persistenten Speicher eine relationale Datenbank oder
einen Objektspeicher zu verwenden.
%
\par{}Bei Verwendung einer relationalen Datenbank erh\"{a}lt das
persistente System bei entsprechender Realisierung m\"{a}chtige
M\"{o}glichkeiten zur Datenmanipulation; bei der Realisierung eines
persistenten Systems ist aber zu beachten, da\ss{} die von der Datenbank
vorgegebene Datenmodellierung insbesondere durch die statische
Festlegung der Attribute auf einfache Bereiche und die fehlende
explizite Repr\"{a}sentation von Referenzen gegen\"{u}ber LISP eine
Einschr\"{a}nkung darstellt. Das persistente System sollte intern
entsprechende Konvertierungen realisieren, die allerdings dazu f\"{u}hren
k\"{o}nnen, da\ss{} die Manipulationsm\"{o}glichkeiten der Datenbank nur
teilweise genutzt werden k\"{o}nnen.
%
\par{}Objektspeicher erm\"{o}glichen f\"{u}r die Realisierung eines
persistenten Systems die \"{U}bernahme der in \cl\ verwendeten
Datenmodellierung in den persistenten Speicher. Ein Objektspeicher
bietet \ia\ nur einfachste M\"{o}glichkeiten zur Datenmanipulation; ein
persistentes System sollte je nach Einsatzgebiet zus\"{a}tzliche
M\"{o}glichkeiten realisieren.
%
\par{}Um ein transientes Objekt in einem persistentem Speicher ablegen
zu k\"{o}nnen, mu\ss{} die Struktur des Objektes bekannt sein. In \cl\ sind
Strukturbeschreibungen f\"{u}r Objekte der Basisklassen statisch;
f\"{u}r \clos-Instanzen kann aus ihrer Klasse dynamisch eine
Strukturbeschreibung bestimmt werden.
%%% Local Variables: 
%%% mode: latex
%%% TeX-master: "main"
%%% buffer-file-coding-system: raw-text-unix
%%% End: 
