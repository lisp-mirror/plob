% This is -*-LaTeX-*-
%
\newcommand{\etal}{et al.}
\newcommand{\manualpagesof}[1]{%
\bibitem[man #1]{bib:#1}
\unix\ Manualseiten zu \lisp{#1}}%
%
\newcommand{\inaoods}[1]{%
K.\ R.\ Dittrich (Hrsg.):
{\em Advances in Object-Oriented Database Systems,
Bad M\"{u}nster am Stein-Ebernburg, Deutschland 1988\/},
\citepage{#1},
Springer-Verlag, Berlin, 1988}
%
\newcommand{\indtap}[1]{%
Malcolm P.\ Atkinson, Peter Buneman, Ronald Morrison (Hrsg.):
{\em Data Types and Persistence\/},
\citepage{#1},
Springer-Verlag, Berlin, 1988}
%
\newcommand{\inpos}[1]{%
John Rosenberg, David Koch (Hrsg.):
{\em Persistent Object Systems,
Newcastle, Australia 1989\/},
\citepage{#1},
Springer-Verlag, Berlin, 1989}
%
\newcommand{\inoocda}[1]{%
Won Kim, Frederick H.\ Lochovsky (Hrsg.):
{\em Object-Oriented Concepts, Databases, and Applications\/},
\citepage{#1},
ACM Press, New York, 1989}
%
\newcommand{\inpea}[1]{%
P.\ P.\ Spies (Hrsg.):
{\em Proceedings Euro-ARCH '93\/},
\citepage{#1},
Informatik Aktuell,
Springer-Verlag, Berlin, 1993}
%
\newcommand{\inroods}[1]{%
Stanley B. Zdonik, David Maier (Hrsg.):
{\em Readings in Object-Oriented Database Systems\/},
\citepage{#1},
Morgan Kaufmann Publishers, San Mateo, Ca., 1990}
%
\newcommand{\inrdoop}[1]{%
Bruce Shriver, Peter Wegner (Hrsg.):
{\em Research Directions in Object-Oriented Programming\/},
\citepage{#1},
MIT Press, Cambridge, Mass., 1987}
%
\begin{thebibliography}{Stonebraker \etal\ 86a}
\addcontentsline%
 {toc}%
 {chapter}%
 {\protect\numberline{}{\protect\bibname}}%
%
\item[]%
\vspace{-\topsep}\hspace{-\leftmargin}%
%
\begin{fortune}[11cm]
{\rglqq}I saw Hamlet last night.\ {[}\ldots{]} It's full of
quotations.{\rgrqq}
%\from{Samuel Goldwyn \cite[\citepage{20}]{ka92}}\\[\medskipamount]
\from{Samuel Goldwyn}\\[\bigskipamount]
After all, all he did was string together a lot of old, well-known
quotations.
\from{H.\ L.\ Mencken \"{u}ber Shakespeare}
\end{fortune}%
%
\vspace{-\itemsep}%
%
\bibitem[Ahmed \etal\ 91]{bib:ah91}
Shamim Ahmed, Albert Wong, Duvvuru Sriram, Robert Logcher:
A Comparision of Object-Oriented Database Management Systems for
Engineering Applications.
Bericht R91-12, % Best.\ Nr.\ IESL90-03, 91-03,
MIT, 1991
%
\bibitem[AMOP]{bib:amop91}
Gregor Kiczales, Jim des Rivi\`{e}res, Daniel G.\ Bobrow:
The Art of the Metaobject Protocol.
MIT Press, Cambridge, Mass., 1991
%
\bibitem[Atkinson \etal\ 89]{bib:at89}
Malcolm Atkinson, Ronald Morrison:
Persistent System Architectures.
\inpos{73--97}
%
\bibitem[Beech 87]{bib:be87}
David Beech:
Groundwork for an Object Database Model.
\inrdoop{317--354}
%
\bibitem[Bj\"{o}rnerstedt \etal\ 89]{bib:bj89}
Anders Bj\"{o}rnerstedt, Christer Hult\'{e}n:
Version Control in an Object-Oriented Architecture.
\inoocda{451--485}
%
\bibitem[Br\"{o}ssler \etal\ 89]{bib:broe89}
P.\ Br\"{o}ssler, B.\ Freisleben:
Transactions on Persistent Objects.
\inpos{303--318}
%
\bibitem[Brown 92]{bib:br92}
A.\ L.\ Brown:
\sh\ manual pages,
Datei \lisp{postore/""man/""stable\us{}heap.%
""3p},
25~Mai 1992
%
\bibitem[Cardelli \etal\ 88]{bib:ca88}
Luca Cardelli, David MacQueen:
Persistence and Type Abstraction.
\indtap{31--41}
%
\bibitem[Clamen 91]{bib:cl91}
Stewart M.\ Clamen:
Data Persistency in Programming Languages, A Survey.
Bericht CMU-CS-91-155,
Carnegie Mellon University, Pittsburgh,
1991
%
\bibitem[CLtLII]{bib:st90}
Guy L.\ Steele Jr.:
\cl\ the Language, Second Edition.
Digital Press, Bedford, Mass., 1990
%
\bibitem[Codd 79]{bib:co79}
E.\ F.\ Codd:
Extending The Database Relational Model to Capture More Meaning.
{\em ACM Transactions on Database Systems\/},
4(4):397--434, Dezember 1979
%
\bibitem[Fernandez \etal\ 89]{bib:fe89}
Mary F.\ Fernandez, Stanley B.\ Zdonik:
Transaction Groups:
A Model for Controlling Cooperative Transactions.
\inpos{341--350}
%
\bibitem[Ford \etal\ 88]{bib:fo88}
Steve Ford, John Joseph, David E.\ Langworthy, David F.\ Lively,
Girish Pathak, Edward R.\ Perez, Robert W.\ Peterson, Diana
M.\ Sparacin, Satish M.\ Thatte, David L.\ Wells, Sanjive Agarwala:
ZEITGEIST: Database Support for Object-Oriented Programming.
\inaoods{23--42}
%
\bibitem[Goldberg \etal\ 89]{bib:go89}
Adele Goldberg, David Robson:
\smalltalk: The Language.
Addison-Wesley, Reading, Mass., 1989
%
\bibitem[Gray \etal\ 93]{bib:gr93}
Jim Gray, Andreas Reuter:
Transaction Processing:
Concepts and Techniques.
Morgan Kaufmann Publishers, San Mateo, Ca., 1993
%
\bibitem[Heiler \etal\ 89]{bib:he89}
Sandra Heiler, Barbara Blaustein:
Generating and Manipulating Identifiers
for Heterogeneous, Distributed Objects.
\inpos{235--247}
%
\bibitem[Itasca 91]{bib:it91}
Itasca Systems, Inc.:
\itasca\ Distributed Object Database Management System,
Technical Summary.
Itasca Systems, Inc., Minneapolis, 1991
% Bibliografische Angaben aus \cite[\citepage{10}]{bib:ah91}
%
\bibitem[Jessen \etal\ 87]{bib:je87}
Eike Jessen, R\"{u}diger Valk:
Rechensysteme.
Springer-Verlag, Berlin, 1987
%
\bibitem[Kaehler \etal\ 90]{bib:ka90}
Ted Kaehler, Glenn Krasner:
LOOM - Large Object-Oriented Memory for \smalltalk\ Systems.
\inroods{298--307}
%
\bibitem[Keene 89]{bib:ke89}
Sonya E.\ Keene:
Object-oriented Programming in \cl:
A Programmer's Guide to \clos.
Addison-Wesley, Reading, Mass., 1989
%
\bibitem[Kernighan \etal\ 88]{bib:ker88}
Brian W.\ Kernighan, Dennis M.\ Ritchie:
The C Programming Language,
Second Edition.
Prentice Hall, Englewood Cliffs, N.J., 1988
%
\bibitem[Khoshafian \etal\ 90]{bib:ko90}
Setrag N.\ Khoshafian, George P.\ Copeland:
Object Identity.
\inroods{37--46}
%
\bibitem[Kim \etal\ 89]{bib:ki89}
Won Kim, Nat Ballou, Hong-Tai Chou, Jorge F.\ Garza, Darrell Woelk:
Features of the \orion\ Object-Oriented Database System.
\inoocda{251--282}
%
\ifbericht
\bibitem[Kirschke 94a]{bib:ki94a}
Heiko Kirschke:
Persistenz in objekt-orientierten Programmiersprachen am Beispiel von
\clos.
Diplomarbeit,
Fachbereich Informatik, Universit\"{a}t Hamburg,
1994
\fi
%
\ifbericht
\bibitem[Kirschke 94b]{bib:ki94b}
\else
\bibitem[Kirschke 94]{bib:ki94b}
\fi
Heiko Kirschke:
Persistent LISP Objects:
User's Guide, Reference Manual.
Fachbereich Informatik, Universit\"{a}t Hamburg,
1994
%Erh\"{a}ltlich als komprimierte \ps-Datei
%\"{u}ber die WWW Seite
%\texttt{http://%
%lki-www.informatik.uni-hamburg.de/\td{}kirschke/home.html}%
%
\bibitem[Klaus \etal\ 76]{bib:kl76}
Georg Klaus, Manfred Buhr (Hrsg.):
Philosophisches W\"{o}rterbuch,
12.~Auf\-la\-ge.
Verlag das europ\"{a}ische Buch, Berlin, 1976
%
\bibitem[Maier \etal\ 87]{bib:ma87}
David Maier, Jacob Stein:
Development and Implementation of an Object-Oriented DBMS.
\inrdoop{355--392}
%
\manualpagesof{mmap}
%
\manualpagesof{socket}
%
\bibitem[Matthes 92a]{bib:ma92a}
F.\ Matthes:
The Database Programming Language DBPL.
Rationale and Report.
Bericht FBI-HH-B-158/92,
Fachbereich Informatik, Universit\"{a}t Hamburg,
1992
%
\bibitem[Matthes \etal\ 92b]{bib:ma92b}
F.\ Matthes, A.\ Rudloff, J.\ Schmidt, K.\ Subieta:
The Database Programming Language DBPL.
User and System Manual.
Bericht FBI-HH-B-159/92,
Fachbereich Informatik, Universit\"{a}t Hamburg,
1992
%
\bibitem[Matthes \etal\ 92c]{bib:ma92c}
F.\ Matthes, J.\ Schmidt:
Definition of the Tycoon Laguage TL -- A Preliminary Report
Bericht FBI-HH-B-160/92,
Fachbereich Informatik, Universit\"{a}t Hamburg,
1992
%
\bibitem[Matthes \etal\ 93]{bib:ma93}
F.\ Matthes, J.\ Schmidt:
System Construction in the Tycoon Environment:
Architectures, Interfaces and Gateways.
\inpea{301--317}
%
\bibitem[Morrison \etal\ 89a]{bib:mo89a}
R.\ Morrison, A.L.\ Brown, R.\ Carrick, R.\ Connor, A.\ Dearle:
The Napier Type System.
\inpos{3--18}
%
\bibitem[Morrison \etal\ 89b]{bib:mo89b}
R.\ Morrison, A.L.\ Brown, R.\ Connor, A.\ Dearle:
The Napier88 Reference Manual.
Bericht PPRR 77-89, Universities of Glasgow and St.\ Andrews, 1989
%
\bibitem[M\"{u}ller 91]{bib:mu91}
Rainer M\"{u}ller:
Sprachprozessoren und Objektspeicher:
Schnittstellenentwurf und -implementierung.
Diplomarbeit,
Fachbereich Informatik, Universit\"{a}t Hamburg,
1991
%
\bibitem[Nikhil 88]{bib:ni88}
Rishiyur S.\ Nikhil:
Functional Databases, Functional Languages.
\indtap{51--67}
%
\bibitem[Norvig 92]{bib:no92}
Peter Norvig:
Paradigms of Artificial Intelligence Programming:
Case Study in Common LISP.
Morgan Kaufmann Publishers, San Mateo, Ca., 1992
%
\bibitem[O'Neil 94]{bib:on94}
Patrick O'Neil:
Database Principles, Programming, Performance.
Morgan Kaufmann Publishers, San Francisco, Ca., 1994
%
\bibitem[Paepke 91a]{bib:pa91a}
Andreas Paepke:
\pclos\ Reference Manual.
HP Laboratories, Palo Alto, Ca., 1991
%
\bibitem[Paepcke 91b]{bib:pa91b}
Andreas Paepke:
User-Level Language Crafting:
Introducing the \clos\ Metaobject Protocol.
HP Laboratories Technical Report HPL-91-169, 1991
%
\bibitem[Paton \etal\ 88]{bib:pat88}
Norman W.\ Paton, Peter M.\ D.\ Gray:
Identification of Database Objects by Key.
\inaoods{280--285}
%
\bibitem[Rhein \etal\ 93]{bib:rh93}
John Rhein, Greg Kemnitz (Hrsg.):
The \postgres\ User Manual,
Version~4.1.
Bericht, University of California, Berkeley, 1993
%
\bibitem[Richardson \etal\ 89]{bib:ri89}
Joel E.\ Richardson, Michael J.\ Carey:
Implementing Persistence in E.
\inpos{175--199}
%
\bibitem[Rowe 87]{bib:ro87}
Lawrence A.\ Rowe:
A Shared Object Hierarchy.
Bericht, University of California, Berkeley, 1987
%
\bibitem[Schank \etal\ 90]{bib:scha90}
Patricia Schank, Joe Konstan, Chung Liu, Lawrence A.\ Rowe, Steve
Seitz, Brian Smith:
\picasso\ Reference Manual,
Version~1.0.
Bericht, University of California, Berkeley, 1990
%
\bibitem[Sche\-we \etal\ 92]{bib:sche92}
Klaus-Dieter Schewe, Bernhard Thalheim, Ingrid Wetzel:
Foundations of Object Oriented Database Concepts.
Bericht FBI-HH-B-157/92,
Fachbereich Informatik, Universit\"{a}t Hamburg,
1992
%
\bibitem[Schlageter \etal\ 83]{bib:schl83}
G.\ Schlageter, W.\ Stucky:
Datenbanksysteme: Konzepte und Modelle.
2.~Auf\-la\-ge,
Teubner, Stuttgart, 1983
%
\bibitem[Schmidt 77]{bib:schm77}
Joachim W.\ Schmidt:
Some High Level Language Constructs for Data of Type Relation.
{\em ACM Transactions on Database Systems\/},
2(3):247--261, September 1977
%
\bibitem[Schmidt \etal\ 80]{bib:schm80}
Joachim W.\ Schmidt, Manuel Mall:
\pascalr\ Report.
Bericht IFI-HH-B-66/80,
Fachbereich Informatik, Universit\"{a}t Hamburg,
1980
%
\bibitem[Schmidt 82]{bib:schm82}
Heinrich Schmidt:
Philosophisches W\"{o}rterbuch,
21.~Auf\-la\-ge.
Stuttgart, 1982
%
\bibitem[Singhal \etal\ 93]{bib:skw93}
Vivek Singhal, Sheetal V.\ Kakkad, Paul R.\ Wilson:
Texas: Good, Fast, Cheap Persistence for \cpp.
{\em OOPS Messenger\/},
4(2):145--147, April 1993
%
\bibitem[Skarra \etal\ 87]{bib:sk87}
Andrea H.\ Skarra, Stanley B.\ Zdonik:
Type Evolution in an Object-Oriented Database,
\inrdoop{393--415}
%
\bibitem[Snape 93]{bib:sn93}
Pers\"{o}nliche Kommunikation \"{u}ber E-Mail mit Guy Snape, Harlequin
Limited:
Re: [kirschke: memory representation in LispWorks].
Text in Datei \lisp{plob/""mail/""binary-repr},
9.~November 1993
%
\bibitem[Snape 94]{bib:sn94}
Pers\"{o}nliche Kommunikation \"{u}ber E-Mail mit Guy Snape, Harlequin
Limited:
Re: Metaobject-Protocol support for LispWorks \clos.
Text in Datei \lisp{plob/""mail/""mop-support},
22.~April 1994
%
\bibitem[St.\ Clair 93]{bib:wo93}
Bill St.\ Clair:
\wood\ (William's Object Oriented Database),
Datei \lisp{cambridge.""apple.""com:%
""/pub/""MCL2/""contrib/%
""wood.""doc}, 1993
%
\bibitem[Stonebraker \etal\ 86a]{bib:sto86a}
Michael Stonebraker, Lawrence A.\ Rowe:
The Design of \postgres.
Bericht, University of California, Berkeley, 1986
% Siehe auch Konferenzband
% \cite[\citepage{55}, Literaturangabe (32)]{bib:at89} 
%
\bibitem[Stonebraker 86b]{bib:sto86b}
Michael Stonebraker:
Inclusion of New Types in Relational Data Base Systems.
Bericht, University of California, Berkeley, 1986
%
\bibitem[Stonebraker \etal\ 89]{bib:sto89}
Michael Stonebraker, Lawrence A.\ Rowe, Michael Hirohama:
The Implementation of \postgres.
Bericht, University of California, Berkeley, 1989
%
\bibitem[Thatte 90]{bib:th90}
Satish Thatte:
Persistent Memory: Merging AI-Knowledge and Databases.
\inroods{242--250}
%
\bibitem[Uhl \etal\ 93]{bib:uhl93}
J\"{u}rgen Uhl, Dietmar Theobald, Bernhard Schiefer, Michael Ranft,
Walter Zimmer, Jochen Alt:
The Object Management System of STONE.
Bericht aus {\tt ftp.fzi.de:/pub/OBST/OBST3-3.3},
Forschungszentrum Informatik, Karlsruhe, 1993
%
\bibitem[Valduriez \etal\ 89]{bib:vaga89}
Patrick Valduriez, Georges Gardarin:
Analysis and Comparison of Relational Database Systems.
Addison-Wesley, Reading, Mass., 1989
%
\bibitem[Wahrig 68]{bib:wa68}
Gerhard Wahrig:
Deutsches W\"{o}rterbuch.
Bertelsmann Lexikon-Verlag, G\"{u}tersloh, 1968
%
\bibitem[Wand 89]{bib:wa89}
Yair Wand:
A Proposal for a Formal Model of Objects.
\inoocda{537--559}
%
\bibitem[Wegner 90]{bib:we90}
Peter Wegner:
Concepts and Paradigms of Object-Oriented Programming.
{\em OOPS Messenger\/},
1(1):7--87, August 1990
%
\bibitem[Wensel 93]{bib:we93}
S.\ Wensel (Hrsg.):
The \postgres\ Reference Manual,
Version~4.1.
Bericht M88/20, University of California, Berkeley, 1993
%
\bibitem[Winston \etal\ 89]{bib:wi89}
Patrick Henry Winston, Berthold Klaus Paul Horn:
LISP, Third Edition.
Addison-Wesley, Reading, Mass., 1989
%
\bibitem[Wirth 83]{bib:wi83}
Niklaus Wirth:
Algorithmen und Datenstrukturen,
3.~Auf\-la\-ge.
B.\ G.\ Teubner, Stuttgart, 1983
%\think{Neuere Ausgabe?}
%
\bibitem[Zdonik \etal\ 90]{bib:zd90}
Stanley B.\ Zdonik, David Maier:
Fundamentals of Object-Oriented Databases.
\inroods{1--32}
%%
%\item[]%
%\hspace{-\leftmargin}%
%Die eingestreuten Zitate entstammen folgenden Werken:
%%
%\bibitem[Karasek 92]{ka92}
%Hellmuth Karasek:
%Billy Wilder: eine Nahaufnahme von Hellmuth Karasek.
%Hoffmann und Campe, Hamburg, 1992
%%
%\bibitem[Lichtenberg 50]{li50}
%Georg Christoph Lichtenberg, Wilhelm Jagow (Hrsg.):
%Gedanken / Eine Auslese aus seinen Sudelb\"{u}chern.
%Th\"{u}ringer Volksverlag, Weimar, 1950
%%
%\bibitem[Russell 92]{ru92}
%Bertrand Russell:
%Denker des Abendlandes / Eine Geschichte der Philosophie,
%3.~Auf\-la\-ge.
% dtv, M\"{u}nchen, 1992
%%
%\bibitem[Schwitters 87]{sc87}
%Kurt Schwitters, Christina Weiss \etal\ (Hrsg.):
%\rglqq{}Eile ist des Witzes Weile\rgrqq{}.
%Reclam, Stuttgart, 1987
%%
\end{thebibliography}
\vspace*{\fill}
\begin{fortune}
There are {\em (sic!)\/} a finite number of jokes in the universe.
% Von einem D. Byrne CD Cover (Stop Making Sense); da steht wirklich
% ,,are'', nicht das korrekte ,,is''
\from{David Byrne: Stop Making Sense}
\end{fortune}

%%% Local Variables: 
%%% mode: latex
%%% TeX-master: "main"
%%% End: 
