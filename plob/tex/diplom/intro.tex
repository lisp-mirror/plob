%       This is -*-LaTeX-*-
%
\chapter{Einleitung}\label{chap:intro}
%
Die \gande{Lebensdauer}{extent}\ von Daten und Objekten ist bei den
meisten Systemen auf die Lebensdauer des Prozesses eingeschr\"{a}nkt, der
sie erzeugt hat, d.h.\ insbesondere wenn dieser Proze\ss{} terminiert,
werden alle Daten freigegeben ({\em fl\"{u}chtige\/} oder
{\em transiente\/} Objekte).  Ein persistentes Objekt ist ein Objekt,
dessen Lebensdauer unabh\"{a}ngig von dem Proze\ss{} ist, der dieses Objekt
erzeugt hat, d.h.\ ein einmal erzeugtes Objekt kann zu einem anderen
Zeitpunkt von einem anderen, sp\"{a}teren Proze\ss{} wiederbenutzt werden
\cite[\citepage{75}]{bib:at89}. In \cite[\citepage{4}]{bib:mo89a} wird
Persistenz von Daten als der Zeitraum definiert, w\"{a}hrend der die
Daten existieren und \gande{benutzbar}{usable}\ sind; anzustreben
sind Systeme, bei denen Persistenz und Benutzung von Daten getrennt
werden.
%
\par{}Die bisher gebr\"{a}uchliche Methode zum Speichern von langlebigen
Daten ist die Benutzung einer Datenbank, die den BenutzerInnen
Datentypen und Operationen \"{u}ber die abgelegten Daten zur Verf\"{u}gung
stellt.  Viele Datenbanken verf\"{u}gen \"{u}ber eigene Anfragesprachen, die
einen vollst\"{a}ndigen Zugriff auf die M\"{o}g\-lich\-kei\-ten der
Datenbank bieten; meist sind diese Anfragesprachen aber nicht f\"{u}r das
Erstellen kompletter Applikationen geeignet. Daher wird neben der
Anfragesprache oft eine Schnittstelle zu einer Programmiersprache
angeboten. Diese Schnittstelle stellt lediglich die Funktionalit\"{a}t
der Datenbank zur Verf\"{u}gung, richtet sich \ia\ jedoch nicht nach den
Konzepten der Programmiersprache, die die Datenbank verwendet:%
\footnotemark\addtocounter{footnote}{-1}%
%
\begin{figure}[htbp]%
\ifbuch%
\centerline{\psfig{figure=twelvept/incldbmo.eps}}%
\else%
\centerline{\psfig{figure=tenpt/incldbmo.eps}}%
\fi%
\end{figure}%
\refstepcounter{figure}%
\addcontentsline%
 {lof}%
 {figure}%
 {\protect\numberline{\thefigure}{\ignorespaces%
  Benutzung von Programmiersprache und Datenbank zur Abstraktion eines
  realen Systems}}%
%
\par\noindent{}Eine L\"{o}sung dieses Problems besteht darin, die
Trennung zwischen Datenbank und Programmiersprache aufzuheben,
d.h.\ die  Datenbankkonzepte m\"{o}glichst vollst\"{a}ndig in eine
Programmiersprache zu integrieren:\footnote{Abbildungen nach
\cite[\citepage{81}]{bib:at89}} %
%
\begin{figure}[htbp]%
\ifbuch%
\centerline{\psfig{figure=twelvept/inclmo.eps}}%
\else%
\centerline{\psfig{figure=tenpt/inclmo.eps}}%
\fi%
\end{figure}%
\refstepcounter{figure}%
\addcontentsline%
 {lof}%
 {figure}%
 {\protect\numberline{\thefigure}{\ignorespaces%
  Benutzung einer persistenten Programmiersprache zur Abstraktion
  eines realen Systems}}%
%
\par\noindent{}Es bietet sich an, entweder eine v\"{o}llig neue Sprache
zu erstellen oder aber diese Integration f\"{u}r eine vorhandene
Sprache durchzuf\"{u}hren. In \ifbericht dieser Forschungsarbeit
\else\ifbuch der in diesem Buch vorgestellten Forschungsarbeit \else
dieser Arbeit \fi\fi wurde der zweite Ansatz untersucht und
realisiert:\ das \cl\ Object System (\clos) wurde um Persistenz
erweitert.
%
\iffalse
Persistente Objekte sind quasi die Vorstufe einer
Datenbank f\"{u}r Objekte. Durch Bereitstellung von Methoden, die eine
datenbank-typische Verwaltung von persistenten Objekten durchf\"{u}hren,
l\"{a}\ss{}t sich ein System, das persistente Objekte anbietet, sehr leicht
in eine objektorientierte Datenbank erweitern
\cite[\citepage{88}]{bib:at89}.
\fi
%
\ifdothink%
\section*{Kommentare}
Dieser Text ist noch in der Entstehung begriffen und enth\"{a}lt daher
noch Anmerkungen, die in der endg\"{u}ltigen Fassung nicht mehr
auftauchen (genau wie dieser Abschnitt).  Diese Anmerkungen beginnen
mit dem Wort \rglq{}\"{U}berlegen\rgrq{} und sind kursiv gesetzt und mit
einem doppelten Pfeil am Rand gekennzeichnet, wenn der Kommentar noch
nicht beantwortet ist, z.B: \think{Wenn einem Gutes widerf\"{a}hrt, das
ist doch eine \"{U}berlegung wert}. Beantwortete Kommentare sind zur
Unterscheidung zu \rglq{}offenen\rgrq{} Fragen mit einem einfachen Pfeil
am Rand gekennzeichnet:
\think{Was bedeutet der einfache Pfeil am Rand?}[Eine
beantwortete \"{U}berlegung].%
\fi%
%
\section{Hintergrund}
%
Das \oscar\ ({\em Open Skies for Conventional Arms Reduction\/})
Projekt im Fachbereich Informatik der Universit\"{a}t Hamburg befa\ss{}t
sich mit der Auswertung von Satellitenbildern zur Verifikation von
Abr\"{u}stungsvereinbarungen. Dazu werden \"{A}nderungen an R\"{u}stungsanlagen
halbautomatisch \"{u}berwacht, indem ein Modell einer solchen Anlage im
Rechner abgelegt wird und dieses Modell mit den Satellitenbildern der
gleichen Anlage verglichen wird; die Aufgabe von \oscar\ besteht
darin, abr\"{u}stungstechnisch relevante \"{A}nderungen zwischen
Rechnermodell und Satellitenbild zu detektieren und diese \"{A}nderungen
zur weiteren Verarbeitung kenntlich zu machen. Das im Rechner
abgelegte Modell entspricht einem Soll-Zustand; die Satellitenbilder
der Anlage geben den Ist-Zustand wieder. Die zu \"{u}berwachenden Anlagen
sind geografischer Natur, d.h.\ im Gel\"{a}nde feststehende Objekte mit
einer gewissen Gr\"{o}\ss{}e, wie z.B.\ Flugh\"{a}fen, Kasernen und
Hafenanlagen.
%
\par{}Die f\"{u}r die Verifikation relevanten Anlagen
werden als Objekte bestehend aus Teil\-kom\-po\-nen\-ten modelliert;
so besteht ein Flughafen z.B.\ aus ein- oder mehreren Hangars, einem
Radarturm, den Start- und Landebahnen, den Verbindungswegen, auf denen
die Flugzeuge bewegt werden, usw.  Die Ab\-h\"{a}n\-gig\-kei\-ten
zwischen den Teilkomponenten m\"{u}ssen ebenfalls be\-r\"{u}ck\-sich\-tigt
werden, z.B.\ wird man auf einem Flughafen immer einen Verbindungsweg
zwischen den Hangars und den Start- und Landebahnen finden. Das
allgemeine Konzept einer solchen Anlage wird als generisches Modell
abgelegt, welches die einzelnen Teilkomponenten spezifiziert und die
Ab\-h\"{a}n\-gig\-kei\-ten zwischen ihnen festlegt.  Im generischen
Modell wird lediglich qualitativ festgelegt, aus welchen Teilen die
gesamte Anlage besteht; erst durch Instanziierung eines generischen
Modells wird eine konkrete Anlage erzeugt.
\think{Evt. darstellen: Zusammenhang mit Tree Grammars,
Ballard/Brown S.~175 ff., kd-Trees, Ballard/Brown S.~281 ff.
Wissensrepr\"{a}sentation.}
%
\par{}In einem n\"{a}chsten Schritt lie\ss{}e sich feststellen, ob beliebige
Satellitenbilder R\"{u}\-stungs\-an\-la\-gen enthalten; dazu w\"{u}rde im
Bild nach Strukturen gesucht werden, die die Projektion einer
Instanziierung eines generischen Modells auf das Bild sein k\"{o}nnten.
%
\par{}In diesem Zusammenhang ergibt es sich, da\ss{} die generischen Modelle
sowie ihre Instanziierungen persistent, d.h.\ \"{u}ber einen l\"{a}ngeren
Zeitraum, gespeichert werden m\"{u}ssen. Aus den
\og[en]\ Ab\-h\"{a}n\-gig\-kei\-ten
zwischen den Teilkomponenten ergibt sich die Notwendigkeit, auf diese
Komponenten assoziativ zugreifen zu m\"{u}ssen. Da es sich bei den
Instanziierungen um r\"{a}umliche Daten handelt, mu\ss{} ebenfalls ein
Zugriff mit mehrdimensionalen Schl\"{u}sseln m\"{o}glich sein.
\think{Den Rest des Absatzes noch sauberer, ausf\"{u}hrlicher formulieren
und besser begr\"{u}nden, warum objekt-orientierte Programmierung:}
Wegen der vom Projekt \oscar\ betrachteten Objekte bietet sich
f\"{u}r die Realisierung die objekt-orientierte Programmierung an,
bei der die Teilkomponenten als Instanzen repr\"{a}sentiert werden.
%
\section{Aufgabenstellung}%
\label{sec:task}%
%
Die Aufgabenstellung besteht darin, f\"{u}r das im \oscar\ Projekt
eingesetzte LISP-System eine M\"{o}g\-lich\-keit zu schaffen,
Datenobjekte transparent beliebig lange speichern zu k\"{o}nnen. Dabei
m\"{u}ssen folgende Punkte beachtet werden:
\begin{itemize}
%
\item Das persistente System mu\ss{} sehr gro\ss{}e Objekte handhaben k\"{o}nnen,
wie z.B.\ Satellitenbilder mit hoher Aufl\"{o}sung.
%
\item Die Anzahl der in einem LISP-System von einem Objekt
referenzierten Instanzen kann sehr hoch sein; es sollte m\"{o}glich sein,
ihre Anzahl bez\"{u}glich Persistenz einzuschr\"{a}nken, indem sie
beispielsweise in \rglq{}interessante\rgrq\ und
\rglq{}uninteressante\rgrq\ Instanzen eingeteilt werden und lediglich
die \rglq{}interessanten\rgrq\ Objekte persistent gehalten werden.  Eine
Spei\-cher\-r\"{u}ck\-ge\-win\-nung sollte den durch nicht mehr
ben\"{o}tigte Objekte belegten Speicherplatz wieder freigeben.
%
\item Die Integration des persistenten Systems in vorhandene Systeme
soll sehr einfach sein und m\"{o}glichst wenig \"{A}nderungen nach sich
ziehen.
%
\item Gefordert sind ebenfalls Datenbankfunktionalit\"{a}ten, wie
Transaktionen und assoziativer mehrdimensionaler Zugriff auf Objekte.
%
\end{itemize}
%
\section{\"{U}bersicht}
%
In Kapitel~\ref{chap:back} werden verschiedene Ans\"{a}tze f\"{u}r
Persistenz und ihre Integration in Programmiersprachen vorgestellt.
Kapitel~\ref{chap:idty} behandelt die Identit\"{a}t von Objekten unter
dem Gesichtspunkt der Persistenz und die sich daraus ergebenen
Konsequenzen f\"{u}r die Realisierung eines persistenten Systems.
Verschiedene Architekturen von persistenten Objektsystemen werden in
Kapitel~\ref{chap:arch} erl\"{a}utert.  Kapitel~\ref{chap:cltodb}
besch\"{a}ftigt sich mit verschiedenen Programmierkonzepten, die in
Bezug auf Persistenz wichtig sind.  \ifbericht\else In
Kapitel~\ref{chap:soh} wird die Realisierung des persistenten Systems
\soh\ beschrieben und analysiert. \fi Kapitel~\ref{chap:plob}
beschreibt das von mir realisierte System f\"{u}r persistente Objekte
in \cl. Kapitel~\ref{chap:exit} enth\"{a}lt eine Zusammenfassung
\ifbericht des Berichtes \else\ifbuch des Buches \else der Arbeit
\fi\fi und gibt Hinweise f\"{u}r die Weiterentwicklung des in
Kapitel~\ref{chap:plob} vorgestellten Systems.
%
\section{Voraussetzungen}
%
Voraussetzung f\"{u}r das Verst\"{a}ndnis dieses Textes sind Grundkenntnisse
in \cl; eine gute Einf\"{u}hrung in \cl\ mit dem Schwerpunkt auf
praktische Beispiele ist \cite{bib:no92}. Eine allgemeine Darstellung
von LISP findet sich in \cite{bib:wi89}. In
\amopcite{243--255}\ werden die Konzepte des \cl\ Object System
(\clos) im Vergleich zu denen von \smalltalk\ und \cpp\ kurz
dargestellt.
%
\par{}F\"{u}r das Verst\"{a}ndnis der beschriebenen Realisierungen von
persistenten Systemen sind Kenntnisse des \clos\ Metaobjekt-Protokolls
(\mop) \"{u}berwiegend im Bereich der Beeinflussung der
Repr\"{a}sentation von \clos-Instanzen hilfreich; dies wird in
\amopcite{13--34, 47--51, 72--78, 85--90, {\em 96--106\/}, 137--139,
146--149, 154--158}\ dargestellt.
%
\section{Schreibweisen}
%
\begin{fortune}
Man spricht deutsch. Si parla Italiano. English spoken.
American understood.
%\from{Schild an einem Gesch\"{a}ft im Ernst-\lb{}Lu\-bitsch-\lb{}Film
%\rglqq{}Bluebard's Eigth Wife\rgrqq{} \cite[\citepage{171}]{ka92}}
\from{Schild an einem Gesch\"{a}ft im Ernst-\lb{}Lu\-bitsch-\lb{}Film
\rglq{}Bluebard's Eigth Wife\rgrq}
\end{fortune}
%
Der folgende Text enth\"{a}lt, soweit m\"{o}glich, die amerikanischen
Fachausdr\"{u}cke in \"{u}bersetzter Form. Zur Klarstellung, wie der
deutsche Begriff zu verstehen ist, wird bei der ersten Benutzung eines
von mir \"{u}bersetzten Begriffes dahinter der amerikanische
Originalbegriff in runde Klammern eingeschlossen angegeben,
z.B.: Nach \gande{Ablauf einer bestimmten Wartezeit}{time out}\ wird
ein Fehler signalisiert. Viele dieser Begriffe werden im Glossar
(\citepage{\pageref{chap:glossary}}) erkl\"{a}rt.
%
\par{}Begriffe, deren sinngem\"{a}\ss{}e \"{U}bersetzung nicht m\"{o}glich ist, wie
z.B.\ \Slt\/ oder \cache, wurden von mir im Original verwendet.  Bei
Zusammensetzungen mit solchen Begriffen habe ich nicht ein Wort
gebildet, sondern die Einzelw\"{o}rter durch einen Bindestrich getrennt,
z.B.\ \Slt\/-Zustand statt \rglq{}\Slt\/zustand\rgrq.
%
\section{Begriffe}
%
\vspace*{-4.5ex}\begin{fortune}[6cm]
\rm\hspace*{\fill}Altes Lautgedicht\hspace*{\fill}\\[\smallskipamount]
\sf%
H\,H\,H\hspace*{\fill}H\,H\hspace*{\fill}H\,H\hspace*{\fill}H\,H\,H\\
\hspace*{\fill}H\,H\,H\hspace*{\fill}\\
\hspace*{\fill}H\,H\,H\hspace*{\fill}H\,H\,H\hspace*{\fill}\\
\hspace*{\fill}A\,A\,A\hspace*{\fill}\\
O la la la\hspace*{\fill}O\,A\hspace*{\fill}O\,A\hspace*{\fill}la la%
\\[\smallskipamount]
\rm{}Plinius (i.J.\ 1847.)
%\from{Kurt Schwitters \cite[\citepage{42}]{sc87}}
\from{Kurt Schwitters}
\end{fortune}
%
Die in diesem Abschnitt definierten Begriffe spiegeln teilweise den
Objektbegriff von \clos\ wider; f\"{u}r eine allgemeinere Darstellung
verweise ich auf
%\cite[\citepage{8}]{bib:we90}.
\cite{bib:we90}.
%
\subsection{Objektbegriff}
%
Ein {\em Objekt\/} ist die Abstraktion einer realen Entit\"{a}t; es
besitzt eine {\em Identit\"{a}t\/} und hat einen {\em Zustand\/} und ein
{\em Verhalten\/}.
%
\par{}Die Identit\"{a}t eines Objektes zeichnet es als einzigartig unter
allen Objekten aus. {\em Werte\/} sind Objekte, deren Identit\"{a}t durch
ihren Zustand festgelegt wird; weiter einschr\"{a}nkend werden Werte als
Basisklassen-Instanzen definiert, die grunds\"{a}tzlich keine weiteren
Objekte referenzieren k\"{o}nnen, wie beispielsweise Zahlen, Zeichen,
Zeichenketten und Bitvektoren.  Bei (\rglq{}echten\rgrq) Objekten ist
die Identit\"{a}t
unabh\"{a}ngig von ihrem Zustand \cite[\citepage{3}]{bib:sche92}.  Mit
einer \representationform{} f\"{u}r Identit\"{a}t kann ein Objekt
referenziert werden. Ein Wert, dessen Zustand direkt in einer
Iden\-ti\-t\"{a}ts\-re\-pr\"{a}\-sen\-ta\-tion repr\"{a}sentiert werden kann,
wird als \immval\addglossary{Immediate}[{Ein \protect\immval\/ ist ein
Wert, dessen Zustand direkt in einer
Iden\-ti\-t\"{a}ts\-re\-pr\"{a}\-sen\-tat\-ion repr\"{a}sentiert werden
kann.}]\ bezeichnet.
%
\par{}Die Struktur des Zustands und das Verhalten eines
Objektes wird durch seine {\em Klasse\/} festgelegt. Die Struktur des
Objektzustands ist aus \Slt[s]\/ zusammengesetzt; ein
\Slt{\em-Zustand\/} ist der in einem \Slt\/ enthaltene Teilzustand des
Objektes und referenziert einen Wert oder ein Objekt.  Das Verhalten
des Objektes wird durch seine {\em\mtd[n]\/} gebildet; jede einzelne
\mtd\ eines Objektes definiert einen Teil seines Verhaltens. Klassen
werden durch {\em\clsmo[e]\/} repr\"{a}sentiert, \Slt[s]\/ durch
\Slt-{\em\mo[e]\/} und \mtd[n]\ durch {\em\mtdmo[e]\/}.
%
\par{}Eine Klasse entsteht durch {\em Spezialisierung\/} einer oder
mehrerer Superklassen, von denen sie Struktur und Verhalten erbt und
gegebenenfalls modifiziert oder erweitert. Die aus einer solchen
Modifikation hervorgegangene \mtd\ ist eine {\em\spc\ \mtd\/}.
%
\par{}Ein {\em transientes Objekt\/} ist ein Objekt, dessen Zustand in
einem transienten Speicher re\-pr\"{a}\-sen\-tiert wird; der Zustand
eines {\em persistenten Objektes\/} wird in einem persistenten
Speicher repr\"{a}sentiert.  Die {\em transiente Repr\"{a}sentation eines
persistenten Objektes\/} ist der im persistenten Speicher
repr\"{a}sentierte Zustand des Objektes, der vollst\"{a}ndig oder teilweise
in eine Re\-pr\"{a}\-sen\-ta\-tion im transienten Speicher kopiert wurde.
Das Verhalten eines persistenten Objektes wird durch das Verhalten
seiner transienten Re\-pr\"{a}\-sen\-ta\-tion bestimmt.
%
\par{}Ein {\em Objektsystem\/} ist eine Sammlung von Objekten; es
bietet Hilfsmittel zur Manipulation von Objekten an, wie Editoren zum
Erstellen von Klassen- und Methoden-Definitionen, Funktionen
zur Erzeugung von Klassen, Objekten usw. Innerhalb eines Objektsystems
gibt es {\em Basisklassen\/}%
\addglossary{Basisklasse}[built-in class][{Eine Basisklasse ist eine
in das LISP-System
\protect\gande{\protect\rglq{}eingebaute\protect\rgrq}%
{built-in}\ Klasse, zu der im Gegensatz zu allgemeinen Klassen keine
Subklassen spezialisiert werden k\"{o}nnen. Ebenso wird auf den Zustand
der Instanzen von Basisklassen nicht mit \protect\mtd[n], sondern mit
\protect\fn[en]\ zugegriffen; damit entf\"{a}llt ebenfalls die
M\"{o}g\-lich\-keit, die \protect\std-Zugriffs-\protect\mtd[n]\ weiter
spezialisieren zu k\"{o}nnen.}], die sich im Vergleich zu allgemeinen
Klassen durch eingeschr\"{a}nkte M\"{o}glichkeiten auszeichnen; so k\"{o}nnen
sie beispielsweise nicht spezialisiert werden.
%
\par{}Persistenz ist {\em orthogonal\/}, wenn sie eine grundlegende
Eigenschaft eines Datums ist und nicht von anderen Eigenschaften des
Datums, wie z.B.\ der Zugeh\"{o}rigkeit zu einem bestimmten Typ,
abh\"{a}ngt. Bei {\em transparenter\/} Persistenz ergeben sich aus der
Sicht der BenutzerInnen keine \"{A}n\-de\-run\-gen der Schnittstelle zum
Objektsystem; die Objekte erhalten unter Beibehaltung ihrer
Schnittstelle zus\"{a}tzlich die Eigenschaft der Persistenz. Intern
erfolgt die Einbindung von transparenter Persistenz \"{u}ber
\spc\ \mtd[n]\ von im  System bereits vohandenen  (generischen)
\fn[en]. Bei {\em intransparenter\/} Persistenz m\"{u}ssen die f\"{u}r
Persistenz erstellten (generischen) \fn[en]\ von den BenutzerInnen
explizit benutzt werden.
%
\par{}Die Begriffe Objekt, Instanz und Datum werden synonym verwendet;
dies gilt auch f\"{u}r die Begriffe Typ und Klasse.
%
\subsection{Begriffe des \protect\mop}%
%
Bei der Bezeichnung der Klassen, Objekte und Methoden des \mop\ habe
ich mich an den in \amopcite{74--75, 137, 140}\ definierten \Std\ gehalten.
\begin{itemize}
%
\item Die Klassen \class{class}, \class{slot-definition},
\class{generic-function}, \class{method}\ und
\class{method-combination}\ werden als {\em Basis-\mc[n]\/}
bezeichnet. Eine {\em\mc\/} ist eine Subklasse genau einer dieser
Klassen.
%
\item Die Klassen \class{standard-class},
\class{stan\-dard-di\-rect-slot-de\-fi\-ni\-tion},
\class{stan\-dard-ef\-fec\-tive-slot-de\-fi\-ni\-tion},
\class{standard-method},
\class{standard-reader-method},
\class{standard-writer-method}\ und
\class{stan\-dard-ge\-ne\-ric-func\-tion} werden als
{\em\std-\mc[n]\/} bezeichnet. Er\-g\"{a}n\-zend fasse ich Subklassen der
Klasse \class{standard-accessor-method}\ unter dem Begriff
{\em Zugriffs-\mtdmc[n]\/} zusammen. Eine von BenutzerInnen definierte
Subklasse einer \std-\mc\ wird als {\em\spc\ \mc\/} bezeichnet.
%
\item Auf \Std-\mc[n]\ spezialisierte \mtd[n]\ werden als
{\em\Std-\mtd[n]\/} bezeichnet. Auf \spc\ \mc[n]\ spezialisierte
\mtd[n]\ werden als {\em\spc\ \mtd[n]\/} bezeichnet.
%
\item Instanzen einer \mc\ werden als {\em\mo[e]\/} bezeichnet. Ein
{\em\Std-\mo\/} ist die Instanz einer \Std-\mc. Ein {\em\spc[s]\ \mo\/}
ist die Instanz einer \spc[n]\ \mc.
\begin{itemize}
%
\item Ein {\em\clsmo\/} ist eine Instanz der Klasse
\class{standard-class}\ oder einer ihrer Subklassen.
%
\item Ein \Slt-{\em\mo\/} ist eine Instanz der Klasse
\class{standard-direct-slot-definition}\ oder
\class{standard-effective-slot-definition}\ oder
einer ihrer jeweiligen Subklassen.
%
\item Ein {\em generisches \fn[s]-\mo\/} ist eine Instanz der Klasse
\class{standard-generic-function}\ oder einer ihrer Subklassen.
%
\item Ein {\em\mtdmo\/} ist eine Instanz der Klasse
\class{standard-method}\ oder einer ihrer Subklassen.
%
\end{itemize}
%
\item Der (ungenaue) Begriff \rglq{}Metaklasse\rgrq\ wird nicht
verwendet; statt dessen wird eine Klasse wie
z.B.\ \class{standard-class}\ explizit als {\em\clsmc\/} bezeichnet
(\figurename\ \ref{fig:mopcls}).
%
\end{itemize}
%
\
\begin{figure}[htbp]%
%
%\def\class#1{#1}%
\def\name#1{{\footnotesize#1}}%
%
\newlength{\indentw}\setlength{\indentw}{5mm}%
\newlength{\restwidth}\setlength{\restwidth}{\textwidth}%
\addtolength{\restwidth}{-\leftmargini}%
\def\mkline#1#2#3{%
\parbox{\restwidth}{%
\hspace*{#1}%
\class{#2}%
%\hspace*{\fill}%
\dotfill%
\name{#3}}}%
%
\begin{listing}[\small]%
\mkline{0\indentw}{T}{\cls}\\
\mkline{1\indentw}{standard-object}{\cls}\\
\mkline{2\indentw}{metaobject}{\cls}\\
\mkline{3\indentw}{class}{Basis-\mc}\\
\mkline{4\indentw}{built-in-class}{\mc}\\
\mkline{4\indentw}{forward-referenced-class}{\mc}\\
\mkline{4\indentw}{standard-class}{\Std-\clsmc}\\
\mkline{5\indentw}{\ldots}{\Spc\ \clsmc}\\
\mkline{3\indentw}{slot-definition}{Basis-\mc}\\
\mkline{4\indentw}{standard-slot-definition}{\mc}\\
\mkline{5\indentw}{standard-direct-slot-definition}{\Std-\sltmc}\\
\mkline{6\indentw}{\ldots}{\Spc\ \sltmc}\\
\mkline{5\indentw}{standard-effective-slot-definition}{\Std-\sltmc}\\
\mkline{6\indentw}{\ldots}{\Spc\ \sltmc}\\
\mkline{3\indentw}{generic-function}{Basis-\mc}\\
\mkline{4\indentw}{standard-generic-function}{\Std-Generische-\fn[s]-\mc}\\
\mkline{5\indentw}{\ldots}{\Spc\ Generische \fn[s]-\mc}\\
\mkline{3\indentw}{method}{Basis-\mc}\\
\mkline{4\indentw}{standard-method}{\Std-\mtdmc}\\
\mkline{5\indentw}{standard-accessor-method}{\Std-\mtdmc}\\
\mkline{6\indentw}{standard-reader-method}{\Std-\mtdmc}\\
\mkline{7\indentw}{\ldots}{\Spc\ \mtdmc}\\
\mkline{6\indentw}{standard-writer-method}{\Std-\mtdmc}\\
\mkline{7\indentw}{\ldots}{\Spc\ \mtdmc}\\
\mkline{3\indentw}{method-combination}{Basis-\mc}
\end{listing}%
%
\caption{Metaobjekt-Klassenhierarchie und Klassenbezeichnungen}%
\label{fig:mopcls}%
\end{figure}

%%% Local Variables: 
%%% mode: latex
%%% TeX-master: "main"
%%% End: 
