% This is -*-LaTeX-*-
%
\pfsection{Performanz}\label{chap:perf}
%
\begin{fortune}[6cm]%
Don't let the sun go down on me.
\from{Elton John}
\end{fortune}%
%
\ifx\pfsection\chapter%
Dieses Kapitel \else Dieser Abschnitt\fi\ enth\"{a}lt
verschiedene Messungen der Performanz von
\plob. Gemessen wurden die elementaren Funktionen, wie Allozieren und
Zugriff auf den Zustand von persistenten Objekten, sowie Einf\"{u}gen und
Lesen von aktiv-persistenten Instanzen, bei denen ein Index f\"{u}r einen
\Slt\/ deklariert wurde. Die Tabellen geben in der Spalte 
\rglq{}LISP\rgrq\ die Zeit an, die \lw\ f\"{u}r die Ausf\"{u}hrung der Aktion
ben\"{o}tigte; in der Spalte \rglq{}BS\rgrq\ findet sich die Zeit, die
das Betriebssystem mit der Ausf\"{u}hrung der Aktion besch\"{a}ftigt
war. Die insgesamt von \plob\ ben\"{o}tigte Zeit ergibt sich aus der
Summe der in der Spalten
\rglq{}LISP\rgrq\ und \rglq{}BS\rgrq\ aufgef\"{u}hrten Zeiten.  Die Einheiten
sind in eckigen Klammern in der
Spalten\"{u}berschrift angegeben; [s] steht f\"{u}r Sekunden und [ms] f\"{u}r
Millisekunden.  Bei Ladevorg\"{a}ngen wurde f\"{u}r die Messungen darauf
geachtet, da\ss{} die Objekte tats\"{a}chlich aus dem persistenten Speicher
und nicht aus dem \cache\/ geladen wurden; bei der Benutzung von
\plob\ wird man
\ia\ daher eine wesentlich bessere Performanz als hier angegeben
erhalten. Die f\"{u}r die Messung benutzte Rechnerkonfiguration ist in
\cite{bib:ki94b}\ im Abschnitt \rglq{}Installation\rgrq\ beschrieben.
%
\newlength{\perfw}\setlength{\perfw}{0.6\textwidth}%
\newlength{\smallperfw}\setlength{\smallperfw}{0.45\textwidth}%
\newlength{\perfh}\setlength{\perfh}{\medskipamount}%
%
\pfsubsection{Transaktionen}
%
Die ermittelte Zeit ist der Verwaltungsaufwand f\"{u}r das Starten und
Beenden einer Transaktion:\\[\perfh]%
%
\begin{tabular}{|p{\perfw}|r@{,}l|r@{,}l|}
\hline
\tabularheader{Erkl\"{a}rung}
        &\multicolumn{2}{|c}{\tabularheader{LISP [ms]}}
                &\multicolumn{2}{|c|}{\tabularheader{BS [ms]}}\\
\hline\hline
\parbox[t]{\perfw}{%
Transaktion mit leerer Anweisung:\\
{\small\tt(with-transaction nil)}}\smallskip
        & 0&250  & 0&0\\
\hline
\end{tabular}
%
\pfsubsection{Sperren}
%
Bei den nachfolgenden Messungen wurde ein Vektor mit 1000
\immval[s]\/ benutzt:
%
\begin{listing}%
(setf *v* (make-array 1000 :initial-element
                           {\rm\it\lt\rm\immval\it\gt\/}))
\end{listing}%
%
\noindent Auf die transiente \representation{} wurde mit
\swizzling\ (Zugriffsart~1) sowie der direkten Manipulation der
persistenten \representation{} (Zugriffsarten~2--4)
schreibend (1.~Tabelle) und lesend (2.~Tabelle) zugegriffen. Die
Zugriffsarten unterscheiden sich in der Anzahl der Transaktionen und
der Anzahl der gesetzten Sperren, die f\"{u}r den Zugriff ausgef\"{u}hrt
wurden; die Spalten \rglq{}Tr.\rgrq\ bzw.\ \rglq{}Sp.\rgrq\ geben die
Anzahl der durchgef\"{u}hrten Transaktionen bzw.\ Sperren
an:\\[\perfh]
%
\begin{tabular}{|c|p{\smallperfw}|r|r|r@{,}l|r@{,}l|}
\hline
\tabularheader{Nr}
 &\tabularheader{Erkl\"{a}rung}
  &\tabularheader{Tr.}
   &\tabularheader{Sp.}
    &\multicolumn{2}{|c}{\tabularheader{LISP [s]}}
     &\multicolumn{2}{|c|}{\tabularheader{BS [s]}}\\
\hline\hline
1&
\parbox[t]{\smallperfw}{%
Speichern des transienten Vektors \lisp{*v*}:\\
{\small\tt(setf *objid* (store-object *v*))}}\smallskip
        & 1 & 1
                & 0&290  & 0&0\\
\hline
2&
Schreiben eines \immval[s]\/ in jede Komponente 
von \lisp{*objid*}\ mit einer Transaktion f\"{u}r alle
Schreibzugriffe und einer Sperre auf Vektorebene f\"{u}r
alle Schreibzugriffe
        & 1 & 1
                & 0&320  & 0&010\\
\hline
3&
Schreiben eines \immval[s]\/ in jede Komponente
von \lisp{*objid*}\ mit einer Transaktion f\"{u}r alle
Schreibzugriffe und einer Sperre auf Elementebene pro
Schreibzugriff
        & 1 & 1000
                & 4&020  & 0&020\\
\hline
4&
Schreiben eines \immval[s]\/ in jede Komponente
von \lisp{*objid*}\ mit einer Transaktion pro
Schreibzugriff und einer Sperre auf Elementebene pro
Schreibzugriff
        & 1000 & 1000
                & 15&470 & 0&050\\
\hline
\end{tabular}\\[\perfh]%
%
Die Me\ss{}werte zeigen, da\ss{} die Kosten des Zugriffs auf den Zustand
eines persistenten Objektes \"{u}berwiegend bei der
Transaktionsverarbeitung und beim Sperrprotokoll liegen.  Die
Zugriffsart~2 entspricht vom Zeitaufwand her der Zugriffsart~1;
Zugriffsart~3 lohnt sich f\"{u}r den schreibenden Zugriff bei bis zu
$\frac{\mbox{0,290}}{\mbox{4,020}/\mbox{1000}}$\,$\approx$\,72
Elementen, Zugriffsart~4 bei bis zu
$\frac{\mbox{0,290}}{\mbox{15,470}/\mbox{1000}}$\,$\approx$\,19
Elementen, d.h.\ ab der jeweils genannten Zahl wird der Aufwand f\"{u}r
das Sperrprotokoll so gro\ss{}, da\ss{} es sich lohnt, auf den Vektor nur
mit
\swizzling\ zuzugreifen.
\\[\perfh]
%
\begin{tabular}{|c|p{\smallperfw}|r|r|r@{,}l|r@{,}l|}
\hline
\tabularheader{Nr}
 &\tabularheader{Erkl\"{a}rung}
  &\tabularheader{Tr.}
   &\tabularheader{Sp.}
    &\multicolumn{2}{|c}{\tabularheader{LISP [s]}}
     &\multicolumn{2}{|c|}{\tabularheader{BS [s]}}\\
\hline\hline
1&
\parbox[t]{\smallperfw}{%
Laden des persistenten Vektors \lisp{*objid*}:\\
{\small\tt(setf *v* (load-object *objid*))}}\smallskip
        & 1 & 1
                & 0&280  & 0&0\\
\hline
2&
Lesen jeder Komponente von \lisp{*objid*}\ mit einer Transaktion f\"{u}r
alle Lesezugriffe und einer Sperre auf Vektorebene f\"{u}r
alle Lesezugriffe
        & 1 & 1
                & 0&280  & 0&030\\
\hline
3&
Lesen jeder Komponente von \lisp{*objid*}\ mit einer Transaktion f\"{u}r
alle Lesezugriffe und einer Sperre auf Elementebene pro
Lesezugriff
        & 1 & 1000
                & 4&230  & 0&010\\
\hline
4&
Lesen jeder Komponente von \lisp{*objid*}\ mit einer Transaktion pro
Lesezugriff und einer Sperre auf Elementebene pro Lesezugriff
        & 1000 & 1000
                & 15&580 & 0&250\\
\hline
\end{tabular}\\[\perfh]%
%
Auch hier wird der Gesamtzeitbedarf in erster Linie durch das
Sperrprotokoll bestimmt; die eigentliche \"{U}bertragung des Zustands
f\"{a}llt kaum ins Gewicht. Die Zugriffsarten~2 und 1 sind vom
Zeitaufwand her gleich; Zugriffsart~3 lohnt sich f\"{u}r den
lesenden Zugriff bei bis zu
$\frac{\mbox{0,280}}{\mbox{4,230}/\mbox{1000}}$\,$\approx$\,66
Elementen, Zugriffsart~4 bei bis zu
$\frac{\mbox{0,280}}{\mbox{15,580}/\mbox{1000}}$\,$\approx$\,18
Elementen.
%
\pfsubsection{Listen}
%
\begin{tabular}{|p{\perfw}|r@{,}l|r@{,}l|}
\hline
\tabularheader{Erkl\"{a}rung}
        &\multicolumn{2}{|c}{\tabularheader{LISP [ms]}}
                &\multicolumn{2}{|c|}{\tabularheader{BS [ms]}}\\
\hline\hline
Speichern einer einelementigen Liste
\lisp{'({\rm\it\lt\rm\immval\it\gt\/})}
        & 14&890  & 0&020\\
\hline
Laden einer einelementigen Liste
\lisp{'({\rm\it\lt\rm\immval\it\gt\/})}
        & 15&200  & 0&020\\
\hline
\end{tabular}\\[\perfh]%
%
Auch beim Speichern und Laden von Listen wird die meiste Zeit
f\"{u}r das Sperren der persistenten {\em cons\/}-Zellen ben\"{o}tigt.
%
\pfsubsection{Felder}
%
Bei den nachfolgenden Messungen wurde ein Feld benutzt, dessen
Dimensionen in etwa denen eines RGB-Farbbildes entsprechen:
%
\begin{listing}%
(setf *array* \=(make-array \='(256 256 3)\\
              \>            \>:element-type 'single-float\\
              \>            \>:initial-element 1.0e0))
\end{listing}%
%
\noindent\begin{tabular}{|p{\perfw}|r@{,}l|r@{,}l|}
\hline
\tabularheader{Erkl\"{a}rung}
        &\multicolumn{2}{|c}{\tabularheader{LISP [ms]}}
                &\multicolumn{2}{|c|}{\tabularheader{BS [ms]}}\\
\hline\hline
\parbox[t]{\perfw}{%
Erzeugen des transienten Feldes \lisp{*array*}:\\
{\small\tt(setf *array*
                (make-array {\rm\it\lt{}siehe
                oben\gt\/}))}}\smallskip
        &  35&0  &   6&0\\
\hline
\parbox[t]{\perfw}{%
Speichern des transienten Feldes \lisp{*array*}:\\
{\small\tt(setf *objid* (store-object *array*))}}\smallskip
        & 222&0  & 366&0\\
\hline
\parbox[t]{\perfw}{%
Erzeugen der transienten \representation{} und Laden des persistenten
Feldes \lisp{*objid*}:\\
{\small\tt(setf *array* (load-object *objid*))}}\smallskip
        &  97&0  &  59&0\\
\hline
\end{tabular}\\[\perfh]%
%
Das Feld besteht aus \immval[s]\/ des Typs \class{single-float}; daher
wird der Zustand der Feldelemente sowohl in der transienten als auch
in der persistenten \representation{} in einem {\em ivector\/}
abgelegt (\figurename~\ref{fig:plexarr},
\citepage{\pageref{fig:plexarr}}). Bei der \"{U}bertragung des Feldes
zwischen transientem und persistentem Speicher wird der Zustand des
{\em ivectors\/} unkonvertiert zwischen den Umgebungen kopiert; damit
erkl\"{a}rt sich die relativ hohe Betriebssystemzeit bei der \"{U}bertragung.
F\"{u}r das Kopieren des Feldzustands wird die
\clogo\ \fn\ \stfn{memcpy}\ benutzt, die wahrscheinlich byteweise
kopiert; in diesem Fall w\"{u}rde ein Ersetzen der
\Std-\clogo-\fn\ \stfn{memcpy}\ durch eine selbst realisierte
Kopierfunktion, die wortweise arbeitet,
einen erheblichen Effizienzgewinn bringen.
%
\pfsubsection{Aktiv-persistente Instanzen}
%
Die \Slt\/-Namen und -Lebensdauern in der Definition der zur Messung
verwendeten ak\-tiv-""per\-si\-sten\-ten Klasse
\class{p-example-class}\ wurden so gew\"{a}hlt, da\ss{} sie zu
\ifbericht der in \cite[\citepage{120}]{bib:ki94a} gezeigten
\figurename \else \figurename~\ref{fig:extent}
(\citepage{\pageref{fig:extent}})\fi\ passen.  Transiente Instanzen
der Klasse \class{t-example-class}\ werden zu Vergleichsmessungen
benutzt.
%
\begin{listing}%
(d\=efclass p-example-class ()\\
  \>(\=(slot-4\ \ :initform nil :extent :transient)\\
  \> \>(slot-6\ \  :initform nil :extent :cached)\\
  \> \>(slot-8\ \  :initform nil :extent :cached-write-through)\\
  \> \>(slot-10\  :initform nil :extent :persistent))\\
  \>(:metaclass persistent-metaclass))\\
\pagebreak[3]\\
(d\=efclass t-example-class ()\\
  \>(\=(slot-3 :initform nil) (slot-5 :initform nil)\\
  \> \>(slot-7 :initform nil) (slot-9 :initform nil)))
\end{listing}
%
\pfsubsubsection{Erzeugung von aktiv-persistenten Instanzen}
%
\noindent\begin{tabular}{|p{\perfw}|r@{,}l|r@{,}l|}
\hline
\tabularheader{Erkl\"{a}rung}
        &\multicolumn{2}{|c}{\tabularheader{LISP [ms]}}
                &\multicolumn{2}{|c|}{\tabularheader{BS [ms]}}\\
\hline\hline
Erzeugen einer (rein transienten) Instanz der Klasse
\class{t-example-class}
        & 0&090 & 0&0\\
\hline
Erzeugen einer persistenten Instanz der Klasse
\class{p-example-class}
        & 45&810 & 0&770\\
\hline
Laden einer persistenten Instanz der Klasse
\class{p-example-class}
        & 38&610 & 0&240\\
\hline
\end{tabular}\\[\perfh]%
%
Die Werte in der ersten Zeile zeigen zum Vergleich die Werte der
Erzeugung einer Instanz f\"{u}r eine rein transiente Klasse.
%
\pfsubsubsection{Zugriff auf den Zustand von aktiv-persistenten
Instanzen}
%
Bei der folgenden Messung wurde jeder \Slt\/-Zustand einmal auf ein
\immval\/ gesetzt.
%
\begin{listing}
(setf *objid* (make-instance 'p-example-class))\\
(setf (slot-value *objid* {\rm\it\lt{}Slot-Name\gt\/})
      {\rm\it\lt{}Immediate\gt\/})
\end{listing}
%
\noindent\begin{tabular}{|l|l|r@{,}l|r@{,}l|}
\hline
\tabularheader{\Slt\/-Name}
 &\tabularheader{\Slt\/-Lebensdauer}
  &\multicolumn{2}{|c}{\tabularheader{LISP [ms]}}
   &\multicolumn{2}{|c|}{\tabularheader{BS [ms]}}\\
\hline\hline
\lisp{slot-3} & \Slt\/ aus \class{t-example-class}
        & 0&007 & 0&0\\
\lisp{slot-4} & \lisp{:transient}
        & 0&430 & 0&0\\
\lisp{slot-6} & \lisp{:cached}
        & 0&440 & 0&0\\
\lisp{slot-8} & \lisp{:cached-write-through}
        & 1&340 & 0&0\\
\lisp{slot-10} & \lisp{:persistent}
        & 1&280 & 0&0\\
\hline
\end{tabular}\\[\perfh]%
%
Wie sich zeigt, ist die Performanz schlechter als bei rein
transienten Klassen, auch bei der \Slt\/-Lebensdauer
\lisp{:transient}\ des \Slt[s]\/ \stslt{slot-4}\ der aktiv-persistenten
Klasse \class{p-example-class}. Dies
ist darin begr\"{u}ndet, da\ss{} die entsprechenden \Slt\/-Lebensdauern aus
dem im \clsdo\ enthaltenen \sltdo\ gelesen werden; offensichtlich ist
die Ermittlung des \sltdo[es]\ relativ zeitintensiv.
%
\par{}F\"{u}r die \Slt\/-Lebensdauer \lisp{:cached-write-through}\ wird
der \Slt\/-Zustand auch in die transiente \representation{} propagiert;
damit ist der schreibende Zugriff langsamer als bei der
\Slt\/-Lebensdauer \lisp{:persistent}.
%
\par{}Bei der folgenden Messung wurde jeder \Slt\/-Zustand 
einmal gelesen.
%
\begin{listing}
(slot-value *objid* {\rm\it\lt{}Slot-Name\gt\/})
\end{listing}
%
\noindent\begin{tabular}{|l|l|r@{,}l|r@{,}l|}
\hline
\tabularheader{\Slt\/-Name}
 &\tabularheader{\Slt\/-Lebensdauer}
  &\multicolumn{2}{|c}{\tabularheader{LISP [ms]}}
   &\multicolumn{2}{|c|}{\tabularheader{BS [ms]}}\\
\hline\hline
\lisp{slot-3} & \Slt\/ aus \class{t-example-class}
        & 0&006 & 0&0\\
\lisp{slot-4} & \lisp{:transient}
        & 0&390 & 0&0\\
\lisp{slot-6} & \lisp{:cached}
        & 0&400 & 0&0\\
\lisp{slot-8} & \lisp{:cached-write-through}
        & 0&400 & 0&0\\
\lisp{slot-10} & \lisp{:persistent}
        & 1&170 & 0&0\\
\hline
\end{tabular}\\[\perfh]%
%
Erwartungsgem\"{a}\ss{} unterscheiden sich die Zeiten f\"{u}r das \Slt\/-Lesen
und -Schreiben signifikant bei der \Slt\/-Lebensdauer
\lisp{:cached-write-through}\ des \Slt[s]\/ \stslt{slot-8}\ der
aktiv-persistenten Klasse \class{p-example-class}; Schreibzugriffe auf
den \Slt\/ werden sowohl in die transiente als auch in die persistente
\representation{} des persistenten Objektes propagiert, w\"{a}hrend
Lesezugriffe immer die transiente \representation{} referenzieren.
%
\pfsubsubsection{Indexdeklarationen f\"{u}r aktiv-persistente Klassen}
%
F\"{u}r die letzte Messung wurde eine aktiv-persistente Klasse mit einem
Index benutzt. Der \Slt\/ \stslt{slot-i}, f\"{u}r den der Index
\lisp{(btree :test equal)}\ deklariert wurde, wurde bei der
Objekterzeugung initialisiert; damit wurde die Assoziation vom
\Slt\/-Zustand auf das Objekt im Verlauf der Erzeugung der
aktiv-persistenten Instanz aufgebaut:%
%
\begin{listing}%
(d\=efclass p-example-index-class ()\\
  \>(\=(slot-i \=:initarg :slot-i\\
  \> \>        \>:index (btree :test equal)\\
  \> \>        \>:extent :persistent))\\
  \>(:metaclass persistent-metaclass))
\end{listing}%
%
\newlength{\onespace}%
\noindent\begin{tabular}{|p{\perfw}|r@{,}l|r@{,}l|}
\hline
\tabularheader{Erkl\"{a}rung}
        &\multicolumn{2}{|c}{\tabularheader{LISP [ms]}}
                &\multicolumn{2}{|c|}{\tabularheader{BS [ms]}}\\
\hline\hline
\parbox[t]{\perfw}{%
Erzeugung einer Instanz der Klasse \class{p-example-index-class}:\\
{\tt\small\settowidth{\onespace}{\ }%
(make-instance\\
\hspace*{\onespace}'p-example-index-class\\
\hspace*{\onespace}:slot-i \dq{}Slot-I with string
                               {\rm\it\lt{}Nummer\gt\/}\dq)}}\smallskip
        & 114&800 & 0&970\\
\hline
\parbox[t]{\perfw}{%
Laden einer Instanz  der Klasse \class{p-example-index-class}:\\
{\tt\small\settowidth{\onespace}{\ }%
(p-select 'p-example-index-class\\
\hspace*{10\onespace}:where 'slot-i)}}\smallskip
        & 64&490 & 0&250\\
\hline
\end{tabular}\\[\perfh]%
%
Im Verlauf der Objektinitialisierung wird die Zeichenkette
abgespeichert und die Assoziation vom \Slt\/-Zustand auf das Objekt
in die Indextabelle eingetragen; damit ergibt sich ein h\"{o}herer
Zeitbedarf.
%
\pfsubsection{Bewertung der Performanz}
%
Wie insbesondere die erste Messung zeigt, ist der Zeitbedarf f\"{u}r das
Sperrprotokoll in Relation zur eigentlichen Zustands\"{u}bertragung recht
hoch. In der 2.~Schicht  k\"{o}nnten noch Optimierungen durchgef\"{u}hrt
werden, da dort im Moment das Sichern des Zustands der
\postore-Vektoren in die UNDO-{\em Log\/}-Datei mit den relativ
ineffizienten {\em Stream\/}-Funktionen der \std-\clogo-Bibliothek
erfolgt.
%
\par{}In der 3.~Schicht lie\ss{}e sich eine Verbesserung der Performanz
beim Zugriff auf \Slt[s]\/ von aktiv-persistenten Instanzen durch
Optimierung der \spc[n]\ \mtd[n]\ der \gfn[n]\ 
\stfn{make-accessor-lambda}\ \ifbericht
\cite[\citepage{111}]{bib:ki94a} \else (Protkoll~\ref{pro:defclass}
\stfn{(defclass)}, Schritt~\ref{enu:dcmcr},
\citepage{\pageref{enu:dcmcr}})\fi\ erreichen. Sofern Wert auf
Performanz gelegt wird, sollten in der 4.~Schicht Transaktionen
m\"{o}glichst gro\ss{}e Bl\"{o}cke umfassen.
%

%%% Local Variables: 
%%% mode: latex
%%% TeX-master: "main"
%%% End: 
