% This is -*-LaTeX-*-
%
% install.tex
% HK 6.7.94
%
\def\WideOption{}%
\def\ClassName{article}
\def\thistitle{Installation Guide}
\def\thispartnumber{IG}
%
% 1998/01/15 HK
%
\ifx\NoOfSides\undefined%
\def\NoOfSides{twoside}%
\fi%
\ifx\WideOption\undefined%
\def\WideOption{narrow}%
\fi%
\ifx\ClassName\undefined
\def\ClassName{pretty}%
\fi%
%
\ifx\Path\undefined%
\def\Path{.}%
\fi%
%
\def\OneSideToken{oneside}
\def\TwoSideToken{twoside}
\def\PrettyToken{pretty}
%
\documentclass[a4paper,\NoOfSides,\WideOption]{\ClassName}%
%
% This is -*-LaTeX-*-
%
 



































% This file was generated from plobversion.h;
% changes done here will be lost!


 








 
 























 







































 





































   \def\thisversion{2.09}            
   \def\thisday{22}            
   \def\thismonth{May}            
   \def\thisyear{2000}            

   \def\thisauthor{Heiko Kirschke}            
   \def\thisemail{Heiko.Kirschke@acm.org}            
   \def\thisurl{www.lisp.de/software/plob/}            

 







 





\author{\thisauthor}%
%
\def\thissubject{Persistent LISP Objects %
\thisversion\space of \thisday.\space\thismonth\space\thisyear}%
%
\def\thiskeywords{persistency,database,persistent objects,LISP,CLOS,MOP}%
%
%\date{6.\ September 1994}%
%\date{30.\ October 1996}%
%\date{18.\ March 1997}%
%\date{14.\ April 1997}%
%\date{9.\ July 1997}%
%\date{5.\ August 1997}%
%\date{27.\ January 1998}%
%\date{6.\ February 1998}%
%\date{9.\ March 1998}%
%\date{12.\ March 1998}% 2.04
%
\def\thisdate{\thisday.\ \thismonth\ \thisyear}%
\date{\thisdate}%
\def\thiswww{http://\thisurl}
%
%
%
\usepackage[english]{babel}
\usepackage{epsfig}
\usepackage{deflist}
\usepackage{avb}
\usepackage{longtable}
\usepackage[T1]{fontenc}
\ifx\ClassName\PrettyToken%
\def\usepackagelispdoc{\usepackage{lispdoc}}%
\else%
\let\usepackagelispdoc\relax%
\fi%
\usepackagelispdoc%
\let\usepackagelispdoc\undefined%
\usepackage{crossref}%
\usepackage{dipldefs}
\usepackage{iconpar}
\usepackage{readaux}
\usepackage{cdnamed}
\usepackage{plobbib}
%\usepackage{think}
\usepackage[nothink]{think}
%
\ifx\NoOfSides\TwoSideToken%
%
\def\usepackagetimes{\usepackage{timestt}}%
%
\def\usepackagehyperref{\usepackage[breaklinks,%
%draft,%
plainpages=false,%
pdftitle={PLOB \thistitle},%
pdfauthor={\thisauthor},%
pdfsubject={\thissubject},%
pdfkeywords={\thiskeywords}%
]{hyperref}}
%
\let\BetterSloppy\relax
%
\else% one sided printing
%
\def\usepackagetimes{\usepackage{times}}%
%
\def\usepackagehyperref{\usepackage[%
%ps2pdf,%
%pdftex,%
%% breaklinks does not work with ps2pdf:
%breaklinks,%
plainpages=false,%
pdftitle={PLOB \thistitle},%
pdfauthor={\thisauthor},%
pdfsubject={\thissubject},%
pdfkeywords={\thiskeywords}%
]{hyperref}}
%
\let\BetterSloppy\sloppy
%
\fi%
%
\usepackagetimes%
\let\usepackagetimes\undefined%
\usepackagehyperref%
\let\usepackagehyperref\undefined%
%
\defattr{objid}{\sl}%
\defattr{typetag}{\sl}%
\defattr{Typetag}{\sl}%
\defattr{TypeTag}{\sl}%
%
\def\figurefontsize{small}%
%
\providecommand\td{\ensuremath{\sim}}% ASCII-Tilde
%
\readaux{\Path/../userg/\NoOfSides/manuadm}{bib:PLOB-UsersGuide}{userg.pdf}
\readaux{\Path/../userg/\NoOfSides/manudis}{bib:PLOB-UsersGuide}{userg.pdf}
%
%
\addtolength\textwidth      {20mm}%
\addtolength\evensidemargin {-20mm}%
\addtolength\textheight     {16.10072mm}%
\let\CodeSize\small
%
\begin{document}

\newcommand{\lb}{\linebreak[0]}                 % Linebreaks in \tt
\def\osname{\textrm{\textit{\lt{}operating system name\gt}}}
\def\shortosname{\textrm{\textit{\lt{}opsys\gt}}}

\title{\ifx\avb\undefined\relax\else\avb\fi%
Persistent LISP Objects\\
\plob\ \thistitle}
\author{\thisauthor}%
\maketitle

This document describes how to get \plob\ version \thisversion\ as of
\thisdate\ up and running.  \cite{bib:PLOB-UsersGuide} contains a
user's guide telling how to use \plob.

\section{Distribution}

See \fcite{WWW and email addresses}\ on how \plob\ is distributed.
This software is distributed as free software, see \fcite{License
  Terms}\ for details. The author's email address is
\url{mailto:\thisemail}

\section{Installation}

\begin{fortune}[0.9\textwidth]
Congratulations!  You have purchased an extremely fine device that
would give you thousands of years of trouble-free service, except that
you undoubtably will destroy it via some typical bonehead consumer
maneuver.  Which is why we ask you to {\sc please for god's sake read
this owner's manual carefully before you unpack the device.  You
already unpacked it, didn't you?  You unpacked it and plugged it in
and turned it on and fiddled with the knobs, and now your child, the
same child who once shoved a polish sausage into your videocassette
recorder and set it on "fast forward", this child also is fiddling
with he knobs, right?  And you're just now starting to read the
instructions, right???  We might as well just break these devices
right at the factory before we ship them out, you know that?}
\from{Dave Barry: "Read This First!"}
\end{fortune}%

Since the archive may contain not yet fixed bugs, please consult the
bug tracking utility at \url{http://\thisproject}\ before starting an
installation.  This document describes the requirements and all steps
in detail which are necessary to install \plob. Also, when there are
errors with getting \plob\ running, look into the user's guide into
\fcite{Common error messages}.

\subsection{Hardware requirements}

This is the hardware which was used for implementing \plob:
\begin{itemize}
  
\item SPARCstation~4, 5, 10, 20 or UltraSPARC with at least 32~MB RAM
  and Solaris 2.x or a Linux workstation running at least kernel 2.x
  or a Windows/NT 4.0/2000/XP PC or an equivalent Silicon Graphics
  workstation running at least IRIX 6.2 (IRIX support on request).

\item Approximately 80~MB disk space for the \plob\ source files and
  object codes.  Additional disk space is used for the \postore\ file
  which contains the persistent objects; this file grows the more
  objects it holds.

\item Access to Internet \lisp{ftp}\ service\footnote{This should be
    fulfilled when you read these lines ;-)}.

\end{itemize}

\subsection{Software requirements}

This is the software which was used for implementing \plob\ and which
is at least necessary to install it:

\begin{itemize}

\item Solaris~2.5 or above

\item or Silicon Graphics running IRIX 6.2 or above

\item or Linux kernel version 2.0.0 or above

\item or Windows/NT 4.0/2000/XP.

\item Standard \unix\ software development tools, like \lisp{make},
  \lisp{sed}\ (not needed for the NT installation).

\item \lwcl\ up to and including version 4.2

\item or \allegrocl\ up to and including version 6.2

\item \emph{If the sources should be recompiled:}
  GNU C version 2.7.2, GNU C version 2.8 for Solaris 2.6. \plob\ 
  exercises the C~preprocessor really heavy w.r.t.\ ANSI
  compatibility, so using another C~compiler may fail to compile
  \plob. It will always fail with a non-ANSI C~compiler. SGI's
  \lisp{cc}\ will be able to compile \plobwoexcl.

\end{itemize}

\subsubsection[Generating the documentation]
{Software needed for generating the documentation}

This additional software was used for generating the documentation; it
is only necessary if the documentation should be changed and the \ps\ 
files should be recreated from the changed documentation:

\begin{itemize}

\item \TeX\ version 3.141, \LaTeX\ version~2.09$\epsilon$

\item \TeX\ macro package \lisp{epsfig}\ version~1.6 or higher

\item \TeX\ macro package \lisp{xypic}\ version~3.1

\item \lisp{.dvi}\ to \ps\ converter \lisp{dvips}\ version~5.58

\item \ps\ interpreter Ghostscript version~3.5.3 (\unix\ shell
command \lisp{gs})

\end{itemize}

\subsection[Installing POSTORE]%
{Installing \protect\postore}\label{sec:postore}

\plob's server uses \postore\ (\textit{P}ersistent \textit{O}bject
\textit{Store}) as a low-level persistent memory; it is provided in
binary form in this distribution with the permission of the University
of St.\ Andrews.  \postore\ is the low-level persistent memory used by
the persistent programming language
\href{http://www-ppg.dcs.st-andrews.ac.uk}{Napier88}.

The \postore\ library found with \plob\ has been slightly adapted to
Solaris, IRIX, Linux and Windows/NT for \plob's needs. The Stable Heap
administered by the \postore\ library can hold persistent objects
which sum up to a maximum of 384~MB (1~GB on Windows/NT).

\subsection[Installing PLOB on Unix]%
{Installing \protect\plob\ on \protect\unix}%
\label{sec:install}

This section describes how to install \plob\ on a \unix\ system.
\begin{enumerate}

\item\emph{For Linux:} Check if the portmap daemon is installed on the
  machine. In a standard Linux installation, the portmap daemon is not
  installed; it should be found on one of the Linux installation CD
  ROMs.

\item Unzip and untar the archive file.\\[\smallskipamount]
\begin{tt}\CodeSize
\td>gunzip plob-\thisversion.tar.gz\\
\td>tar fx plob-\thisversion.tar
\end{tt}

\item Change to the \lisp{plob-\thisversion/}\ subdirectory (this is
  found in the directory where the \lisp{plob-\thisversion.\lb{}tar}\ 
  archive has been unpacked).\\[\smallskipamount]
\begin{tt}\CodeSize
\td>cd plob-\thisversion\\
\td/plob-\thisversion>
\end{tt}

\item\label{itm:config}Call \lisp{make config}.\\[\smallskipamount]
\begin{tt}\CodeSize
\td/plob-\thisversion>make config
\end{tt}\\[\smallskipamount]
  This will ask some questions about your local configuration. For
  many of the questions, default answers are provided. If this script
  fails to execute properly, see
  section~\Nameref{sec:ManualInstallation} on what you can do to
  proceed.

\item \emph{This step is only necessary when client and server will
    run on different machines.} If \lisp{make}\ succeeded with
  installing, the \plob\ server should have been built and should now
  be installed on the database server host. If the directory
  containing the \plob\ distribution is accessible on the database
  server host (e.g.\ mounted by NFS), this is quite simple: Login to
  the database server host, \lisp{cd}\ to the
  \lisp{plob-\thisversion/}\ subdirectory and call \lisp{make server}\ 
  from the shell. If the directory containing the \plob\ distribution
  is \emph{not} directly accessible on the server host, do the
  following:
  \begin{itemize}
  \item Login to the database server host specified by you in
    step~\ref{itm:config}
  \item Create the \plob\ database root directory specified by you in
    step~\ref{itm:config}
  \item Copy files
    \lisp{plob-\thisversion/\lb{}bin/\lb\osname/\lb{}plobd[.exe]},
    \lisp{plob-\thisversion/\lb{}bin/\lb\osname/\lb{}plobdadmin[.exe]},
    \lisp{plob-\thisversion/\lb{}bin/\lb\osname/\lb{}librpclientplob.[so,dll]}\ 
    and \lisp{plob-\thisversion/\lb{}bin/\lb{}plobdmon}\ into the
    database root directory.
  \item Call program \lisp{plobdadmin}\ now located in the
    database root directory without any arguments; this will start the
    program in interactive mode. At the
    \lisp{tcp://\lb{}localhost/\lb{}database}\ prompt,
    enter:\\[\smallskipamount]
    \begin{tt}\CodeSize
      tcp://localhost/database\gt{}create
      \textrm{\textit{\lt{}the default database specified in
          step~\ref{itm:config}\gt}}\\ 
    \end{tt}\\[\smallskipamount]
    (Entering \lisp{help}\ at the prompt will show the help text of
    \lisp{plobdadmin}.)
  \end{itemize}

\item For using the \lisp{plobdadmin}\ program properly, the PATH and
  LD\us{}LIBRARY\us{}PATH should be extended to include the database
  root directory. Put the following startup code into file
  \lisp{\$\{HOME\}/.plobdrc}\\[\smallskipamount]
  \begin{tt}\CodeSize
    root \textrm{\emph{\lt{}database root directory\gt{}}}\\
    start
  \end{tt}

\item\label{itm:lisp}For installation under \allegrocl: Change to the
  \lisp{plob/\lb{}src/\lb{}allegro/}\ subdirectory and start
  \allegro.\\[\smallskipamount]
\begin{tt}\CodeSize
\td/plob>cd src/allegro\\
\td/plob/src/allegro>cl
\end{tt}

\item For installation under \lwcl: Change to the
  \lisp{plob/\lb{}src/\lb{}lispworks/}\ subdirectory and start
  \lw.\\[\smallskipamount]
\begin{tt}\CodeSize
\td/plob>cd src/lispworks\\
\td/plob/src/lispworks>lispworks
\end{tt}

\item\label{itm:compile} Compile and load file
  \lisp{defsystem-plob.lisp}\ (this file should be in the current
  directory because of the \lisp{cd}\ done in the previous step).
  Eventually it is necessary to compile and load
  \lisp{define-system.lisp}\ from the same directory before this step.

\item Start a listener and evaluate \lisp{(compile-plob)}; this
  will compile all source modules.  After compiling, evaluate
  \lisp{(load-plob)}; this will [re]load the out-of-date module[s].

\item Evaluate \lisp{(open-my-session)}; this will open a connection
  to the server and format its LISP root. Now \plob\ is ready for
  using.  Try compiling and loading \lisp{plob-example.lisp}

\item\label{itm:last} \emph{This step is only necessary if you've
    changed any documentation strings and want to update the
    documentation.} If you want to extract the \TeX\ documentation
  from the LISP source files anew, find-file
  \lisp{plob/\lb{}src/\lb{}lisp-doc/\lb{}lisp-doc.\lb{}lisp}\ into a
  editor window; compile \& load it and evaluate
  \lisp{(scan-plob-files)}.

\end{enumerate}

\subsection[Installing PLOB on NT/2000/XP]%
{Installing \protect\plob\ on Windows/NT/2000/XP}%
\label{sec:installnt}

The binaries for a Windows/NT/2000/XP server and the sources for a
\lwcl\ and a \allegrocl\ client installation are contained in this
distribution.  For using \plobwoexcl's Windows/NT/2000/XP server with
a \lwcl\ or \allegrocl\ client, do the following.
%% \sloppy
\begin{enumerate}

\item Unzip the archive file, for example with WinZip
(\url{http://www.winzip.com} [the `evaluation version' offered as free
download will do the job], an ftp'able version is available at
\url{ftp://ftp.winzip.com/winzip95.exe}).

\item Start a shell and change to the \lisp{plob-\thisversion/}\ 
  subdirectory (this is found in the directory where the
  \lisp{plob-\thisversion.\lb{}tar.gz}\ archive has been
  unpacked).\\[\smallskipamount]
\begin{tt}\CodeSize
\td>cd plob-\thisversion\\
\td/plob-\thisversion>
\end{tt}

\item\label{itm:ntconfig} Call \lisp{make config}.\\[\smallskipamount]
\begin{tt}\CodeSize
\td/plob-\thisversion>make config
\end{tt}\\[\smallskipamount]
 Follow the instructions given by the script's output.  If this script
 fails to execute properly, see
 section~\Nameref{sec:ManualInstallation} on what you can do to
 proceed.
 
\item\label{itm:ntcompile} For both \lwcl\ and \allegrocl, change into
  the \lisp{\td/plob-\thisversion/\lb{}src/\lb{}allegro}\ subdirectory
  containing the LISP source code and start the LISP
  system.\\[\smallskipamount]
\begin{tt}\CodeSize
\td/plob-\thisversion>cd src/allegro\\
\td/plob-\thisversion/src/allegro>c:/opt/Harlequin/LispWorks/lispworks
\end{tt}\\[\smallskipamount]
Edit file \lisp{defsystem-plob.lisp}\ and change the constant
\lisp{+plob-\lb{}dir+}\ to point to the absolute directory containing
the local \plob\ installation, for the example code shown so far this
would be:\\[\smallskipamount]
\begin{tt}\CodeSize
(defconstant +plob-dir+ "c:/home/kirschke/plob-\thisversion"\\
\hspace*{2em}\#+:Lisp-Doc "\plob\ installation directory.")
\end{tt}\\[\smallskipamount]
Compile and load file \lisp{defsystem-plob.lisp}.

\item Edit file \lisp{plob-\lb{}defaults.\lb{}lisp}. Make sure that
  \lisp{*default-\lb{}database-\lb{}url*}\ has a value of
  \lisp{"tcp://\lb{}localhost/\lb{}database"}\\[\smallskipamount]
\begin{tt}\CodeSize
(defparameter *default-database-url* "tcp://localhost/database"\\
\hspace*{2em}\#+:Lisp-Doc "\textrm{\emph{\ldots\ documentation omitted here \ldots}}")
\end{tt}

\item For using the \lisp{plobdadmin}\ program properly, the PATH
  should be extended to include the database root directory. Set the
  \lisp{HOME}\ environment vaiable to point to a home directory, and
  put the following startup code into file
  \lisp{\$\{HOME\}/.plobdrc}\\[\smallskipamount]
  \begin{tt}\CodeSize
    root \textrm{\emph{\lt{}database root directory\gt{}}}\\
    start
  \end{tt}

\item Continue as described in section~\Nameref{sec:install},
  steps~\ref{itm:compile}--\ref{itm:last}.

\item To use the Windows/NT server, evaluate for example\\[\smallskipamount]
\begin{tt}\CodeSize
(open-my-session "//ntserver")
\end{tt}\\[\smallskipamount]
with \lisp{ntserver}\ being the name of the Windows/NT host running
the server.

\item To stop the server, open a task manager and end the
  \lisp{plobd.exe}\ process. Another way would be to evaluate
  \lisp{(p-exit)}\ in a LISP listener connected to an open database.

\end{enumerate}

\subsection[Installing PLOB manually]%
{Installing \protect\plob\ manually}\label{sec:ManualInstallation}

For each offered platform of \plob\ at least one check has been done
if \plob\ can be installed and run succesfully. But, there might be
unforeseeable problems, for example because of an unexpected local
host's shell configuration or file system configuration. The
installation script tries to cope with these unexpected settings, but
can have problems in doing a successful installation. To get \plob\
installed in such a case without using the installation script, the
following proceeding can be applied as a replacement for the
installation script called in the UNIX installation,
step~\ref{itm:config} (p.\ \pageref{itm:config}) resp.\ in the NT
installation, step~\ref{itm:ntconfig} (p.\
\pageref{itm:ntconfig}). The instructions assume that the \plob\
archive has been unpacked already. In the following text,
\shortosname\ should be one of \lisp{irix}, \lisp{linux},
\lisp{solaris}\ or \lisp{win32}, whatever matches the opearating
system \plob\ should be installed at.

\begin{enumerate}

\item Make the directories \lisp{lib/\lb\shortosname},
  \lisp{bin/\lb\shortosname}, \lisp{src/\lb{}allegro/\lb{}allegro4},
  \lisp{src/\lb{}allegro/\lb{}allegro5},
  \lisp{src/\lb{}util/\lb{}allegro4}\ and
  \lisp{src/\lb{}util/\lb{}allegro5}.\\[\smallskipamount]
\begin{tt}\CodeSize
\td/plob-\thisversion>mkdir lib/\shortosname\\
\td/plob-\thisversion>mkdir bin/\shortosname\\
\td/plob-\thisversion>mkdir src/allegro/allegro4\\
\td/plob-\thisversion>mkdir src/allegro/allegro5\\
\td/plob-\thisversion>mkdir src/util/allegro4\\
\td/plob-\thisversion>mkdir src/util/allegro5
\end{tt}

\item From \lisp{conf/\lb\shortosname/}, copy all library files to
\lisp{lib/\lb\shortosname}\\[\smallskipamount]
\begin{tt}\CodeSize
  \td/plob-\thisversion>cp conf/\shortosname/*.so
  lib/\shortosname\hspace*{\fill}\textrm{\emph{(for UNIX)}}\\ 
  \td/plob-\thisversion>cp conf/\shortosname/*.dll
  lib/\shortosname\hspace*{\fill}\textrm{\emph{(for NT)}}
\end{tt}

\item Copy \lisp{conf/\lb\shortosname/plobd[.exe]}\ and
  \lisp{conf/\lb\shortosname/plobdadmin[.exe]} \ to
  \lisp{bin/\lb\shortosname}\\[\smallskipamount]
\begin{tt}\CodeSize
  \td/plob-\thisversion>cp conf/\shortosname/plobd bin/\shortosname
  \hspace*{\fill}\textrm{\emph{(for UNIX)}}\\ \td/plob-\thisversion>cp
  conf/\shortosname/plobdadmin bin/\shortosname
  \hspace*{\fill}\textrm{\emph{(for UNIX)}}\\ \td/plob-\thisversion>cp
  conf/\shortosname/plobd.exe bin/\shortosname
  \hspace*{\fill}\textrm{\emph{(for NT)}}\\ \td/plob-\thisversion>cp
  conf/\shortosname/plobdadmin.exe bin/\shortosname
  \hspace*{\fill}\textrm{\emph{(for NT)}}
\end{tt}

\item Choose a database root directory. In principle, this directory
  can be placed anywhere, but it should not be placed into
  \lisp{/tmp}\ or -- for performance reasons -- into a NFS or Novell
  mounted directory or drive. This text assumes a database root
  directory at \lisp{/opt/\lb{}data/\lb{}plob/}. Make this directory,
  and copy the executable daemon, the administration tool and the
  shared library needed by the administration tool into this
  directory.\\[\smallskipamount]
\begin{tt}\CodeSize
\td/plob-\thisversion>mkdir /opt/data/plob\\
\td/plob-\thisversion>cp conf/\shortosname/plobd /opt/data/plob
\hspace*{\fill}\textrm{\emph{(for UNIX)}}\\
\td/plob-\thisversion>cp conf/\shortosname/plobdadmin /opt/data/plob
\hspace*{\fill}\textrm{\emph{(for UNIX)}}\\
\td/plob-\thisversion>cp conf/\shortosname/librpclientplob.so /opt/data/plob
\hspace*{\fill}\textrm{\emph{(for UNIX)}}\\
\td/plob-\thisversion>cp conf/\shortosname/plobd.exe /opt/data/plob
\hspace*{\fill}\textrm{\emph{(for NT)}}\\
\td/plob-\thisversion>cp conf/\shortosname/plobdadmin.exe /opt/data/plob
\hspace*{\fill}\textrm{\emph{(for NT)}}\\
\td/plob-\thisversion>cp conf/\shortosname/librpclientplob.dll /opt/data/plob
\hspace*{\fill}\textrm{\emph{(for NT)}}
\end{tt}

\item \emph{For NT:} Install the ONC RPC library as described in file
\href{../oncrpc-1.12/doc/usage.htm}{\lisp{plob-\thisversion/\lb{}oncrpc-\lb{}1.12/\lb{}doc/\lb{}usage.htm}}.

\item Change to the database root directory and call \lisp{plobd}\ in
that directory. It is important to do a real directory change, and not
to call \lisp{plobd}\ with a prefixed path to the database root
directory.\\[\smallskipamount]
\begin{tt}\CodeSize
\td/plob-\thisversion>cd /opt/data/plob\\
/opt/data/plob>./plobd
\end{tt}\\[\smallskipamount]
Calling \lisp{./plobd}\ with option \lisp{-h}\ will echo a help text
about the daemon.

\item For both \lwcl\ and \allegrocl, change into the
  \lisp{\td/plob-\thisversion/\lb{}src/\lb{}allegro}\ subdirectory
  containing the LISP source code.\\[\smallskipamount]
\begin{tt}\CodeSize
  /opt/data/plob>cd \td/plob-\thisversion/src/allegro
\end{tt}\\[\smallskipamount]
Edit file \lisp{defsystem-\lb{}plob.\lb{}lisp}\ and change the
constant \lisp{+plob-\lb{}dir+}\ to point to the absolute directory
containing the local \plob\ installation. For the example code shown
so far this would be:\\[\smallskipamount]
\begin{tt}\CodeSize
  (defconstant +plob-dir+ "/home/kirschke/plob-\thisversion"\\ 
  \hspace*{2em}\#+:lisp-doc "\plob\ installation directory.")
\end{tt}\\[\smallskipamount]
Compile and load file \lisp{defsystem-\lb{}plob.\lb{}lisp}.

\item Edit file \lisp{plob-\lb{}defaults.\lb{}lisp}.  Make sure that
  \lisp{*default-database-url*}\ has a value of
  \lisp{"tcp://\lb{}localhost/\lb{}database"}:\\[\smallskipamount]
\begin{tt}\CodeSize
  (defparameter *default-database-url* "tcp://localhost/database"\\ 
  \hspace*{2em}\#+:lisp-doc "\textrm{\emph{\ldots\ documentation
      omitted here \ldots}}")
\end{tt}

\end{enumerate}
Now, the installation can be completed by continuing with
step~\ref{itm:compile} (p.\ \pageref{itm:compile}) for UNIX resp.\ 
step~\ref{itm:ntcompile} (p.\ \pageref{itm:ntcompile}) for NT.

\subsection[Installing a new version]%
{Installing a new version of \protect\plob}

To install a new version of \plob, stop the server process by calling
\lisp{plobdadmin -exit}\ (see \cite{bib:PLOB-UsersGuide}\ for details
on program \lisp{plobdadmin}) and repeat all steps described in
section~\ref{sec:install}. All configuration data which was given in
the very first installation is stored within file
\lisp{plob/\lb{}conf/\lb{}make.\lb{}vars.\lb{}in}; this data will be
used for default values when installing a new version on top of an old
version.

\section[Starting PLOB]%
{Starting \protect\plob}

For starting \plob, both server and client must be started. More
details on administration can be found in \cite{bib:PLOB-UsersGuide}.

\subsection[Server side]%
{Starting \protect\plob: Server side}

Normally, the server needs only be started when the machine running
the server was rebooted, since the server does not terminate itself
when the last client disconnects.  If a restart is necessary, login to
the server host (if the server host is not \lisp{localhost}) and
change to the database root directory. Call program \lisp{plobdadmin
-start}, this will start the server process.

\subsection[Client side]%
{Starting \protect\plob: Client side}

\noindent To restart \plob, load file \lisp{defsystem-plob}\ and evaluate
\lisp{(load-plob)}.\\[\smallskipamount]
\begin{tt}\CodeSize
\td/plob/src/allegro>cl\\
USER(1): (load "defsystem-plob")\\
; Fast loading ./defsystem-plob.fasl\\
T\\
USER(2): (load-plob)\\
...\\
USER(3): (open-my-session)\\
...\\
\end{tt}

\bibliographystyle{cdnamed}
\bibliography{bibdefs,bibstrings,plob}

\end{document}

%%% Local Variables: 
%%% buffer-file-coding-system: iso-latin-1-unix
%%% mode: latex
%%% TeX-master: "install"
%%% End: 
