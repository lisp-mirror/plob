
\begin{CompactCode}
(defclass \comment{\lt{}class name\gt}\ (\comment{\,\{\,\lt{}superclasses\gt\,\}\,})
  (\comment{\,\{\,\lt{}slots\gt\,\}\,})
  (:metaclass persistent-metaclass)
  \underline{(:constructor \comment{\lt{}funcallable object\gt})}  \comment{;; usage deprecated}
  \underline{(:dependent \comment{\lt{}boolean\gt})}
  \underline{(:extent \comment{\lt{}keyword\gt})}
  \underline{(:schema-evolution \comment{\lt{}keyword\gt})}
  \comment{\{\,\lt{}other class options\gt\,\}\,})
\end{CompactCode}

\subsubsection{Class option
\protect\lisp{:deferred}\ \protect\emph{(not implemented)}}%
\label{sec:ClassDeferred}

The additional class option \lisp{:deferred}\ is neither used nor
further implemented. It was planned to use it to support more
efficient object storing and loading by improved support for forward
referenced objects, as a replacement for the missing store/load
plan generator. The idea was to use the value of this option as a hint
to store instances of classes being declared \lisp{:deferred} before
instances of other classes not declared as being \lisp{:deferred} and
to load instances of such a class after instances of other classes.
This option would take a priority number as argument, the higher the
number, the earlier the storing / the later the loading of the
class' instances.


