% This is -*-LaTeX-*-
%
% userg.tex
% HK 26.5.94
% Rewritten 1997/09

\chapter{Administration}

This chapter describes how to startup \plob\ and how the \plob\ server
is administrated.

\section{Administrating the client}

\subsection{Starting on the client side}%
\label{sec:StartClient}

For an example of a startup file see
\lisp{plob-\thisversion/}\lisp{src/}\lisp{example/}\lisp{plob-toplevel.lisp}.
To load all \plob\ files at once after installation is complete, load
the file \lisp{defsystem-plob}\ and evaluate \lisp{(load-plob)}.
Unless a session to the server is explicitly opened by a call to
\fcite{open-my-session}, a session will be opened when necessary.

If \plob\ should write more or fewer messages about its proceeding to
stream \textbf{t}, set \fcite{*verbose*}\ to an appropiate value.

\subsection{Finishing on the client side}%
\label{sec:FinishClient}

All sessions opened to the server will be closed when the \lw\ or
\allegro\ process will terminate. Calling \fcite{close-heap}\ will
force a close of all sessions at runtime. If there are no other
clients connected to the server, closing the last client will trigger
the garbage collection.

\subsection{Creating a new database}

New databases on the server can be created from the client's side
either by calling \fcite{p-create-database}, or by calling
\fcite{open-my-session}\ and passing the URL of the new database as
argument. If the second approach is used, the user is asked to confirm
that a new database should be created. For example, this will create
and open a database on host \lisp{ki8}\ named \lisp{my-base}:
\begin{CompactCode}
\listener{}(open-my-session "//ki8/my-base")
;;;; Bootstrap   : Opening tcp://ki8/my-base
Error: cplob.c(152): fnOpen:
       Could not open database tcp://ki8/my-base
  [condition type: POSTORE-CERROR]
Restart actions (select using :continue):
 0: Try to create it.
\OmitUnimportant
\errlistener{}:cont
\end{CompactCode}

\section{Administrating the server}

For administrating the server process, the \lisp{plobdadmin}\ program
is provided within the distribution. The \plob\ administrator should
either change her/his \lisp{PATH}\ environment variable to include the
\lisp{plobdadmin}\ program found in the operating system specific
configuration directory
\lisp{plob-\thisversion/}\lisp{conf/}\lisp{\emph{\lt{}operating
    system\gt}/}\lisp{plobdadmin[.exe]} or put a (symbolic) link to it
into one of the directories found in the \lisp{PATH}\ variable. For
doing the administration, the \plob\ administrator must execute the
\lisp{plobdadmin}\ program on the machine which runs the server
process. The server process is started with the effective user-ID of
the \plob\ administrator. If at configuration the client
authentication is set to \lisp{AUTH\us{}DES}, the server must be
started with root rights.

For usage of \lisp{plobdadmin}, see its help text obtained by calling
\lisp{plobdadmin -h}. When \lisp{plobdadmin}\ is called without
options, it starts up in interactive mode, that means it prints a
prompt and awaits input from \lisp{stdin}. Since \lisp{plobdadmin}\ is
linked against the RPC client shared library of \plobwoexcl,
\lisp{librpclientplob.[so,dll]}\ must be found in the
\lisp{LD\us{}LIBRARY\us{}PATH} (for UNIX) or \lisp{PATH} (for
Windows/NT).

\subsection{Starting and stopping the server}

Normally, the server needs only be started when the machine running
the server was rebooted, since the server does not terminate itself
when the last client disconnects.  If a restart is necessary, login to
the server host (if the server host is not \lisp{localhost}) and
change to the database root directory. Call program \lisp{plobdadmin
-start}, this will start the server process.

Calling program \lisp{plobdadmin -stop}\ will stop the server
process. To get rid of orphan sessions (that means, sessions whose
client process has been lost for some reasons), do a \lisp{plobdadmin
-restart}.

\subsection[The administrator]{The \protect\plob\ administrator}

The user which opens the stable heap the very first time is made the
\plob\ system administrator. The administrator is allowed to reset
(\fcite{p-reset}), restart (\fcite{p-restart}) or exit
(\fcite{p-exit}) the daemon process from her or his LISP listener and
-- a little bit more important -- may allow or deny logins from client
machines (\fcite{(setf p-machine-loginp)}). The administrator is
identified by her or his login name and the machine on which the
client process runs; the administator which is accepted by the server
can be resetted either by calling the \fcite{p-reset}\ by the
(current) administrator or by using the \lisp{reset}\ subcommand of
the \lisp{plobdadmin}\ script. If the adiministrator is resetted, the
next user which opens the heap will become the new administrator.

\subsection{Remote daemon control}%
\label{sec:RemoteDaemonControl}

As mentioned already, the administrator has remote access for
administrating the daemon. \Fcite{p-reset}, \fcite{p-restart}\ and
\fcite{p-exit}\ may be used by the administrator in a listener to
administrate the daemon. The name and host of the administrator of a
persistent heap is stored within the persistent heap itself; in
consequence, for calling \fcite{p-reset}, the administrator's listener
process must be connected to the persistent heap which should be
resetted. After the daemon has been terminated by a call to
\fcite{p-exit}, it can only be restarted from a shell by calling
program \lisp{plobdadmin}\ with a \lisp{-start}\ argument.

When configured correctly, \fcite{p-admin}\ can be used to call
script \lisp{plobdadmin}\ from within a listener. Prerequisite is that
\lisp{rsh}\ will work on the host running the LISP system.

\refpar \Fcite{p-restart}, \fcite{p-exit}, \fcite{p-reset}.

\subsection{Login administration}

\refpar \Fcite{(setf p-machine-loginp)}, \fcite{show-machines}.

\section[Serverless PLOB]{Running \protect\plob\ serverless}

It is possible to run \plob\ with no server at all. In serverless
mode, the server's code is directly loaded into the current LISP
client image. The following sections describe how to start \plob\ in
serverless mode. The sections after the next section describe the main
differences between using \plob\ in server mode and in serverless
mode.

\subsection[Starting PLOB in serverless mode]
{Starting \protect\plob\ in serverless mode}

To start \plob\ in serverless mode, specify a protocol name of
\lisp{local}\ followed by a database directory name within the database
URL to open. For example:
\begin{CompactCode}
\listener{}(open-my-session "local:c:/home/kirschke/database")
;;;; Bootstrap   : Opening local:c:/home/kirschke/database
\OmitUnimportant
\end{CompactCode}
This will open the database located in directory
\lisp{c:/home/kirschke/database}; if this database does not exist, you
are asked if it should be created. Please note when using a Windows/NT
filename, the protocol name \lisp{local}\ must be specified if
the database to open contains a drive letter; otherwise, \plob\ will
mistakenly interpret the drive letter as a protocol name.

After the very first database has been opened in server mode or
serverless mode, it is not possible to switch from one mode to the
other. The LISP system must be restarted to switch between both
modes.\footnote{This has the technical reason that a shared library
  cannot be unloaded safely from a running LISP image.}

\emph{Up to now (1998/09/17), the serverless version runs stable only
  with \lwcl\ 4.0.1\ and \allegrocl\ 5.0 beta on Windows/NT and with
  ACL 4.3 on Solaris; I was not able to get a stable version running
  on Linux. It does not work with \lwcl\ 3.2.2 on Solaris and
  \allegrocl\ 4.3.1 on Irix. All other configurations have not yet
  been tested.}

\subsection[Server and serverless mode]
{Differences between server mode and serverless mode}

To understand the differences, some technical background is given. In
server mode, the LISP client process communicates with the server
process by RPCs; this is a request-reply serial communication channel
resulting in a rather weak coupling. This means for example, if one of
these processes abends, the other process will live on; so, the other
process still has a chance to save its data.

In serverless mode, \plobwoexcl's server functionality is directly
linked into the current LISP image and running within the same
process, resulting in a rather tight binding. All resources are shared
between LISP and \plobwoexcl; this poses a problem e.g.\ on Linux,
since LISP and \plob\ can not agree on how to share the address space
between them. Also, failures of one of both subsystems leading to a
process abend will abend the other subsystem, too.

One reason for using the serverless mode is performance, since
serverless mode incures much fewer communication overhead than server
mode.  In serverless mode, \plobwoexcl's functions are running about
twice as fast as in server mode, some functions are even more
faster. Another advantage is that the locking algorithm of \plob\ now
will truly suspend LISP threads waiting for a lock to be granted. In
server mode, this is done by polling since all RPC communication can
only be initiated by the client.

Of course, the serverless mode needs no server administration at all,
especially no active \plob\ server is needed.  For \plobwoexcl's API,
there are almost no differences between both modes; only some
administration functions (like \fcite{p-exit}) have been changed,
since they make no sense in serverless mode.

\flabel{\crfsection}{Common error messages}{}
{\section{Common error messages}}%
\label{sec:CommonErrorMessages}

This section explains the most common error messages and what to do
when receiving such a message.

\begin{description}
\item[Error message]
In the LISP listener, following message is raised:
\begin{CompactCode}
Error: cplob.c(383): fnClientCreate:
       Connect to host ki8.kogs.hh failed:
       localhost: RPC: Remote system error - Connection refused
       (Check if the PLOB! daemon is running on the host)
\end{CompactCode}

\item[What happened?] The daemon has not yet been started.
\item[To Do] Ask the \plob\ administrator to start the daemon on the
  server machine by using program \lisp{plobdadmin}, subcommand
  \lisp{open}; meanwhile (1998/05/07), it is also possible to start
  the daemon directly without program \lisp{plobdadmin}. A start can
  also be done by using \fcite{p-admin}\ (section
  \Nameref{sec:RemoteDaemonControl}) at the debugger's prompt, for
  example:
\begin{IndentedCompactCode}
;;;; Bootstrap   : Opening tcp://localhost/plob
Error: cplob.c(383): fnClientCreate:
       Connect to host ki8.kogs.hh failed:
       localhost: RPC: Remote system error - Connection refused
       (Check if the PLOB! daemon is running on the host)
Restart actions (select using :continue):
\OmitUnimportant
\errlistener{}(p-admin :open)
0       \comment{;; Error code returned from remote called shell script \lisp{\CompactCodeSize{}plobdadmin}}
\errlistener{}:cont
\end{IndentedCompactCode}
\end{description}

\begin{description}
\item[Error message]
In the LISP listener, following message is raised:
\begin{CompactCode}
Error: cplob.c(371): fnClientCreate:
       Connect to host ki8.kogs.hh failed:
       ki8.kogs.hh: RPC: Miscellaneous tli error -
       An event requires attention
\end{CompactCode}

\item[What happened?] The daemon is not running, although the client
  expected a running daemon. Probably, there is a communication
  problem, or the daemon terminated due to an internal failure.
\item[To Do] Ask the \plob\ administrator to look for the failure
  condition. The daemon can be started again on the
  server machine by using the script \lisp{plobdadmin}, subcommand
  \lisp{open}. A start can also be done by using \fcite{p-admin}\
  (section \Nameref{sec:RemoteDaemonControl}).
\end{description}

\begin{description}
\item[Error message]
In the LISP listener, following message is raised:
\begin{CompactCode}
Error: It looks as if the LISP PLOB root object should be formatted.
\end{CompactCode}

\item[What happened?] You opened the stable heap the very first time;
  the LISP root object was not found in the stable heap.
\item[To Do] Acutally, this is no error but an emergency break to
  escape from a maybe unwanted formatting of the stable heap.
  Normally, do a \lisp{:continue}\ and everything should work fine.
\end{description}

\begin{description}
\item[Error message] When starting the \lisp{plobd}\ server process
using \lisp{plobdadmin -start}\ from the shell, the following error
message is shown:
\begin{CompactCode}
Cannot register service: RPC: Unable to receive; errno = Connection refused
unable to register (PLOBD, PLOBDVERS, udp).
\end{CompactCode}

\item[What happened?] The \lisp{portmap}\ (for Solaris: \lisp{rpcbind})
daemon is not running;
this daemon is part of the operating system and no part of \plob
\item[To Do] Start the \lisp{portmap}\ (for Solaris: \lisp{rpcbind})
daemon on your machine as root user. The
\lisp{portmap}\ (for Solaris: \lisp{rpcbind}) daemon should be
located in \lisp{/usr/sbin}.
\end{description}

\begin{description}
\item[Error message]
In the LISP listener, following message is raised:
\begin{CompactCode}
Error: cplob.c(147): fnCreateAuth:
       Creating the AUTH_DES authentication failed. Check if
       `ki8.informatik.uni-hamburg.de' names a host in your
       net domain and that keyserv is running on your local
       client machine. Check if calling the shell command
       `keylogin' might help.
\end{CompactCode}

\item[What happened?] The client could not create its authentication
  data.
\item[To Do] If issuing a `keylogin' on the shell prompt does not
  help, the server name in \fcite{*database-url*}\ is perhaps not what
  you mean: If \plob\ was compiled with \lisp{AUTH\us{}DES}
  authentication, the \fcite{*database-url*}\ must contain a host name
  and a domain name (for example \lisp{"ki8.kogs.hh"}) of your local
  net domain and \emph{not} a fully qualified Internet address.
\end{description}

\begin{description}
\item[Error message]
In the LISP listener, following message is raised:
\begin{CompactCode}
Error: From server at executing client:fnSHopen:
       Login to stable heap failed.
\end{CompactCode}

\item[What happened?] The daemon process is running, but denied to log
  you into the stable heap.
\item[To Do] Ask the \plob\ administrator to view into the
  \lisp{messages.log}\ file associated with the daemon to see the
  reason why the login was denied. This file is located on the server
  in the database directory. Perhaps the \plob\ administrator
  has denied the login for your machine.
\end{description}
%%
%\begin{description}
%\item[Error message]
%\begin{CompactCode}
%\end{CompactCode}
%\item[What happened?] 
%\item[To Do] 
%\end{description}
%%


%
%%% Local Variables: 
%%% mode: latex
%%% TeX-master: "userg"
%%% End: 

