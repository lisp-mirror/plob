% This is -*-LaTeX-*-
%
% userg.tex
% HK 26.5.94
% Rewritten 1997/09

\chapter[Using PLOB]%
{Using \protect\plob}
%
\begin{fortune}%
A rose is a rose is a rose is a rose.
\from{Gertrude Stein}
\begin{tt}\CompactCodeSize\begin{tabbing}%
(d\={}efun is-a (the-object the-class-object)\\
  \>{}(w\={}hen (eq (class-of the-object) the-class-object)\\
  \>    \>{}the-class-object))\\[\smallskipamount]
(is-a \={}(is-a \={}(is-a \={}(find-class 'standard-class)\\
      \>{}      \>{}      \>{}(find-class 'standard-class))\\
      \>{}      \>{}(find-class 'standard-class))\\
      \>{}(find-class 'standard-class))
\end{tabbing}\end{tt}%
\fromwonl{\clos}%
\end{fortune}%
%
\ResetListener%
%  
\noindent{}This chapter focuses on \plobwoexcl's API, starting from the
mostly transparent high-level \clos\ interface going towards the LISP
objects interface and the direct persistent object interface.
Persistency is supported with three kinds of operation:
\begin{description}

\item[Tight binding] The tight binding is available for instances of
  \clos\ classes intended to use persistency explicitely
  (section \Nameref{sec:TightBinding}). This interface is the most
  transparent one.

\item[Loose binding] The loose binding gives the possibility of making
  any LISP object persistent where either the class definition is out
  of scope or not available at all
  (section \Nameref{sec:LooseBinding}).  It can also be used for
  instances of \clos\ classes where persistency is not a so important
  feature to pay the (performance) price of the tight binding.

\item[Direct representation] It is also possible to work on persistent
  objects directly (section \Nameref{sec:DirectBinding}).  Since
  \plobwoexcl's typing is adapted to LISP, this interface looks
  similar to the interface defined in \cite{bib:CLtLII}. Also,
  instances of database specific types can be manipulated
  directly, for example persistent btrees (section
  \Nameref{sec:PersistentBTree}).

\end{description}
All these interfaces can be mixed; a decision for using one of those
interfaces at a certain place does not exclude usage of another (but
lower-level) interface at another place.

Many of the examples given in the following text correspond to the
code found in module
\lisp{plob-\thisversion/}\lisp{src/}\lisp{example/}\lisp{plob-example}, so it is a
completement to the more low-level example module documentation of the
example code in \fcite{Example Module}.

\section{Objects and identity}

A reference to an object can be achieved by thinking of an object's
\emph{identity}\marginlabel{An object's represented identity makes it
  possible to refer it.}: The identity of an object is something
unique to it, making it distingushable from all other objects in a
set. So, a representation of identity can be used for establishing an
unambigous reference to an object. In transient LISP systems, the
memory address of a non-immediate object is used for representing its
identity (for \allegrocl, the hexadecimal number behind the `@'
character in the output of the standard print-method).  In a LISP
system, these identity representations are almost never visible at top
level, since references on top level are handled by symbols; anyway,
LISP has a defined notion of identity for objects.

\subsection[Identification of objects]%
{Identification of persistent objects by their
\protect\objid}%
\label{sec:IdByObjId}

\begin{fortune}%
We are using tokens or tokens of tokens as long as we do not have
access to the things themselves.
\from{Umberto Eco: The Name of the Rose}
\end{fortune}%

In \plobwoexcl, the representation of a persistent object's identity
is not hidden in the internals, but visible at the top-level API, for
the purpose of working directly on the persistent object's
representation.  This representation is called \objid\ (\emph{obj}ect
\emph{id}entifier), so an \objid\ can be used to reference a
persistent object. As explained later in
section \Nameref{sec:PersistentSymbol}), the LISP way-of-working
with persistent symbols referencing persistent objects is possible,
too.

An \objid\ itself is represented by a fixnum\marginlabel{In
  \protect\plobwoexcl, {\protect\objid[s]}\ are represented by
  fixnums.}; sometimes it is necessary to distinguish an \objid\ from
a fixnum.  This is accomplished by wrapping the fixnum into an
instance of \fcite{persistent-object}\ for non-immediate persistent
objects and, if necessary, into an instance of
\fcite{persistent-immediate-object}\ for immediate objects. The
correct class is choosen by \fcite{make-persistent-object}.

\refpar \Fcite{make-persistent-object}

\subsection{Identity, persistent and transient objects}

Object access in \plob\ is done by \emph{swizzling}: A persistent
object's state is transferred from the server's objectspace
\marginlabel{Objects are represented twice, in persistent and
  transient memory.}into a transient representation of the client's
objectspace, and the state manipulating functions of the client work
on such a transient counterpart. At appropiate moments, both
representations are equalized (for example, when calling
\fcite*{store-object} or \fcite*{load-object}, see section
\Nameref{sec:LooseBinding}). For doing such an equalization, the
reference of a transient representation to its persistent object must
be known, to transfer the transient state into the `correct'
persistent object, that means, the persistent object the transient
representation was filled from. This reference itself is represented
by an \objid. For transient representations generated using the tight
binding (section \Nameref{sec:TightBinding}), this \objid\ is stored
within each transient representation and can be requested when
necessary. For \clos\ instances not using the tight binding and
instances of all other LISP classes, a table is maintained mapping a
transient representation onto its \objid. The \objid\ of a transient
representation can be requested by a call to
\fcite{persistent-object-objid}.

\refpar \Fcite{persistent-object-objid}.

\subsection{Identity and caching}%
\label{sec:IdAndCache}

\begin{fortune}%
  \hspace*{\fill}Entia non sunt multiplicanda praeter necessitatem.\\ 
  \hspace*{\fill}[Entities must not be copied unnecessarily.]
\from{William of Occam}%
\end{fortune}%

Since this table references \objid[s]\ and transient representations,
it serves as an object cache, too.\think{Perhaps it is not a so good
  idea to couple an object's identity with caching its state.  Since
  there is a difference between identity and caching, both should be
  kept seperately.} At the moment, there is the unsatisfying situation
that either the cache has to be cleared sometimes explicitely (section
\Nameref{sec:CacheLitteringMemory}) or is cleared by \plob\ itself
(section \Nameref{sec:TransactionAbort}).  This clearing will break
the binding between all transient representations contained in this
table to their persistent counterparts.

For strings and numbers, \fcite{*cache-strings*}\ and
\fcite{*cache-numbers*}\ decide if strings and numbers should be
cached at all. Especially numbers often have a more `immediate'
character, and so it is desirable not to maintain the persistent
identity of each of such instances, since this would fill the cache
with many small objects.

\refpar \Fcite{clear-cache};
\fcite{*cache-strings*}, \fcite{*cache-numbers*}.

\section[Tight binding]%
{Making \protect\clos\ objects persistent: Tight binding}%
\label{sec:TightBinding}

A tight binding means \marginlabel{Within the tight binding, \plob\ 
  takes control over \clos\ instances.}that \plob\ takes control over
the creation and state of a persistent \clos\ object. All creation,
state requesting and state changing calls to a persistent object with
a tight bounded class are bypassed to \plobwoexcl. The tight binding
itself is achieved with the help of the Meta Object Protocol (MOP), an
integral part of \clos\ and is described in detail in \cite{bib:AMOP}.
Because of technical restrictions, the tight binding can only be
offered for \clos\ classes. The binding is obtained by adding the
class option
\lisp{:metaclass}\hspace{\lispblank}\lisp{persistent-metaclass}\ to a
class declaration:
\begin{IndentedCompactCode}
(defclass \comment{\lt{}class name\gt}\ (\comment{\,\{\,\lt{}superclasses\gt\,\}\,})
  (\comment{\,\{\,\lt{}slots\gt\,\}\,})
  \underline{(:metaclass persistent-metaclass)})
\end{IndentedCompactCode}

Using this class option, additional slot and class options are
enabled, which can be used to steer persistency for the class
declared. Section \Nameref{sec:ToplevelDeclarations} contains further
details how persistency can be influenced by top level declarations.

\subsection{Additional slot options}%
\label{sec:AdditionalSlotOptions}

Within the tight binding, additional slot options \marginlabel{The
  tight binding adds the slot options \lisp{:extent}, \lisp{:index}\ 
  and \lisp{:location}.} can be used to steer persistency on a
per-slot level. Here is an overview of the additional slot options:
\begin{IndentedCompactCode}
(defclass \comment{\lt{}class name\gt}\ (\comment{\,\{\,\lt{}superclasses\gt\,\}\,})
  ((\comment{\lt{}slot name\gt}
    \underline{:extent \comment{\lt{}keyword\gt}}
    \underline{:index \comment{\lt{}index definition\gt}} \comment{;; Explained in section \ref{sec:SlotIndex}, \citepage{\pageref{sec:SlotIndex}}}
    \underline{:location \comment{\lt{}fixnum constant\gt}}
    \comment{\{\,\lt{}other slot options\gt\,\}\,})
   \comment{\{\,\lt{}more slots\gt\,\}\,})
  (:metaclass persistent-metaclass))
\end{IndentedCompactCode}

The slot option \lisp{:allocation}\ is obeyed by \plob\ as defined in
\stcite{776, 779}; in other words, shared slots among instances of
a class are made persistent, too, and keep their property of being
shared slots.

\subsubsection{Slot option \protect\lisp{:extent}}%
\label{sec:SlotExtent}

Declare the extent of the slot\marginlabel{The slot option
  \lisp{:extent}\ tells how a slot is represented.}. Normally, each
slot of a persistent object is represented twice, in persistent and
transient memory (the second representation being for performance
reasons).  The \lisp{:extent}\ slot option influences when and how the
slot's state is transferred between the transient and persistent
memory (figures~\ref{tab:usr:SlotExtent} and
\ref{fig:usr:SlotExtent}).
%%%
\def\labeltabextent{tab:usr:SlotExtent}
\def\labelfigextent{fig:usr:SlotExtent}
% -*-LaTeX-*-
\ifx\Path\undefined%
\def\Path{.}%
\fi%
\ifx\labeltabextent\undefined\def\labeltabextent{tab:extent}\fi
\begin{figure}[htbp]\centering%
\begin{small}%
\def\coli{25ex}%
\def\colii{0.95\textwidth}%
\begin{tabular}{|p{\coli}ccl|}
\hline
\textbf{Value of slot}
        & \multicolumn{2}{|c|}{\textbf{Represent.}}&\\
\textbf{option \lisp{:extent}}
        & \multicolumn{1}{|c|}{\textbf{Tr.}}
        & \multicolumn{1}{|c|}{\textbf{Pe.}}
        & \textbf{Explanation}\\

\hline\hline
{\lisp{:transient}} & Yes & No &\\
\multicolumn{4}{|p{\colii}|}{%
The slot will only be represented in transient memory but will not be
represented in persistent memory. This slot is `ignored' by
\plob\ w.r.t.\ persistency.}\\

\hline
\lisp{:cached} & Yes & Yes &\\
\multicolumn{4}{|p{\colii}|}{%
The slot will be represented both in transient and persistent memory.
Slot reading is done from the transient slot's state.
Slot writing only affects the transient slot's state;
no changes of the transient representation will be propagated
automatically to persistent memory. The slot's state will only be
promoted to persistent memory by a call to \fcite{store-object}\ on
the \clos\ instance the slot belongs to.}\\

\hline
\lisp{:cached-write-through} & Yes & Yes &\\
\multicolumn{4}{|p{\colii}|}{%
The slot will be represented in transient and persistent memory.
Slot reading is done from the transient slot's state.
Changes of the transient representation will be propagated
automatically and immediately to persistent memory.}\\

\hline
\lisp{:persistent} & No & Yes &\\
\multicolumn{4}{|p{\colii}|}{%
The slot will only be represented in persistent memory but will not be
represented in transient memory. Reading is done directly from the
persistent memory. For writing, this slot extent implies that changes
to the slot's state will be propagated automatically to persistent
memory. Use of this extent is depreciated, since it is very resource
consuming.}\\

\hline
\lisp{:object} & Yes & Yes &\\
\multicolumn{4}{|p{\colii}|}{%
This extent indicates that a direct persistent representation should
be used for the slot's state: The slot will not contain a transient
representation of a persistent object, but a forward object
referencing the persistent object.}\\

\hline
\lisp{:objid} & Yes & Yes &\\
\multicolumn{4}{|p{\colii}|}{%
This extent indicates that a direct persistent representation should
be used for the slot's state: The slot will not contain a
transient representation of a persistent object, but a numeric \objid\
referencing the persistent object.}\\

\hline
\end{tabular}\end{small}%
\caption{Slot extent, transient and persistent object representation}%
\label{\labeltabextent}
\end{figure}%

\ifx\labelfigextent\undefined\def\labelfigextent{fig:extent}\fi
\vspace*{\bigskipamount}\noindent Figure~\ref{\labelfigextent} shows
how the slot extent affects the slot representation and promoting of
the slot's state between the transient and persistent representation
of an object; the structure and class definitions of this example are:
\begin{CompactCode}
(defstruct persistent-example-structure slot-1 slot-2)
(setf (class-extent (find-class 'persistent-example-structure))
      :cached-demand-load)

(defclass persistent-example-class ()
  ((slot-3  :extent :transient)
   (slot-4  :extent :transient)
   (slot-5  :extent :cached)
   (slot-6  :extent :cached)
   (slot-7  :extent :cached-write-through)
   (slot-8  :extent :cached-write-through)
   (slot-9  :extent :persistent)
   (slot-10 :extent :persistent))
  (:metaclass persistent-metaclass))
\end{CompactCode}
\begin{figure}[htbp]
\centerline{\psfig{figure=\Path/../manual/manext.eps}}
\caption{Transient object, persistent object and slot extent}%
\label{\labelfigextent}
\end{figure}%

%%%
If a slot has no slot option \lisp{:extent}, the value of class option
\lisp{:extent}\ will be used by default.

\subsubsection{Slot option \protect\lisp{:location}}%
\label{sec:SlotLocation}

The slot option \lisp{:location}\ \marginlabel{The slot option
  \lisp{:location}\ tells where a slot is represented.} is used to
enforce a user-defined ordering for the slots of a class. This is
important if persistent instances are to be used as keys for an index
or a btree, since the slot ordering represents the significance of the
values contained in the object's slots.

Consider for example the following class declaration:
\begin{CompactCode}
\listener{}(defclass ordered-instance (\comment{\,\{\,\lt{}superclasses\gt\,\}\,})
              ((most-specific-slot \underline{:location 10}
                                   \comment{\,\{\,\lt{}other slot options\gt\,\}\,})
               (medium-specific-slot \underline{:location 20}
                                     \comment{\,\{\,\lt{}other slot options\gt\,\}\,})
               (least-specific-slot \underline{:location 30}
                                    \comment{\,\{\,\lt{}other slot options\gt\,\}\,})
               \comment{\{\,\lt{}more slots\gt\,\}\,})
              (:metaclass persistent-metaclass))
\end{CompactCode}

By default, the slot locations are computed after the slots have been
ordered alphabetically by name.  Specifying each slot with a
\lisp{:location}\ option will enforce that the slots of instances of
\class{ordered-instance}\ will be arranged in the order as given by
the numeric argument passed to the \lisp{:location}\ option.  The
numbers passed with the \lisp{:location}\ option are used only for
changing the default ordering, they are not directly used as slot
locations in persistent objects representations.  Now, when instances
of \class{ordered-instance}\ are used as keys for btrees with an
\lisp{equal}\ test mode, the instances will be inserted into the btree
according to the specified slot ordering, that means, the value of
slot \textbf{most-specific-slot}\ will be the primary key, the value
of slot \textbf{medium-specific-slot}\ will be the secondary key, and
the value of slot \textbf{lowest-specific-slot}\ will be used as
ternary key.

\refpar[for all additional slot options] \Fcite{persistent-metaclass},
\fcite{(setf slot-extent)}.

\subsection{Additional class options}%
\label{sec:AdditionalClassOptions}

Within the tight binding, additional class options \marginlabel{The
  tight binding adds the class options \lisp{:constructor},
  \lisp{:dependent}, \lisp{:extent}\ and \lisp{:schema-evolution}.} can
  be used to steer persistency. Here is an overview of the additional
  class options:

\begin{IndentedCompactCode}
(defclass \comment{\lt{}class name\gt}\ (\comment{\,\{\,\lt{}superclasses\gt\,\}\,})
  (\comment{\,\{\,\lt{}slots\gt\,\}\,})
  (:metaclass persistent-metaclass)
  \underline{(:constructor \comment{\lt{}funcallable object\gt})}  \comment{;; usage deprecated}
  \underline{(:dependent \comment{\lt{}boolean\gt})}
  \underline{(:extent \comment{\lt{}keyword\gt})}
  \underline{(:schema-evolution \comment{\lt{}keyword\gt})}
  \comment{\{\,\lt{}other class options\gt\,\}\,})
\end{IndentedCompactCode}

\subsubsection[Class option \protect\lisp{:constructor}]%
{Class option \protect\lisp{:constructor}\ \protect\emph{(usage
  deprecated)}}%
\label{sec:ClassConstructor}

Set the constructor function for instances of the class to the
option's argument\marginlabel{The class option \lisp{:constructor}\ 
  declares a constructor for transient representations of a persistent
  class.}, which is a function being declared as \lisp{(defun}\ 
\textrm{\emph{\lt{}constructor\gt}}\ \lisp{(p-objid depth p-heap)
  \ldots)}.  The constructor is called for creating transient
instances of the class being loaded from the persistent heap into
transient memory. For persistent \clos\ objects, the constructor is
called with 3 arguments:
\begin{description}

\item[\funarg{p-objid}] The \objid\ of the object to load from the
  persistent heap (section \Nameref{sec:IdByObjId}).

\item[\funarg{depth}] The depth the object should be loaded to.

\item[\funarg{p-heap}] The persistent heap the object should be loaded
  from.

\end{description}
The constructor is responsible for creating the transient instance and
filling the transient instance from its persistent representation.  If
this class option is not passed, \plob\ will create the transient
representation automatically.  This class option is meant for
instances which can not be created the standard way \plob\ does it;
for establishing side effects on storing or loading of objects, use
the mechanisms described in section \Nameref{sec:HookIntoTransfer}.
To make things clear, \emph{usage of this class option is strongly
  deprecated}.

\subsubsection{Class option \protect\lisp{:dependent}}%
\label{sec:ClassDependent}

Declaring a class as being \lisp{:dependent}\ \marginlabel{The class
  option \lisp{:dependent}\ tells if instances of a persistent class
  should be transferred from server to client among with objects
  referencing those instances.}with an option argument of \nonnil\ 
will result in instances of that class to be transferred from the
server to the client together with the object referencing such
instances. Use this option if you have classes whose instances are
more or less `embedded' into the referencing instance, and are only
referenced from this `embedding' instance. This is the case when it is
known in advance that when the client loads the referencing instance,
the referenced instance will be loaded in the same transaction in
almost all cases, too.  For example, consider a class for geometric
lines ended by two points.
\begin{CompactCode}
\listener{}(defclass point ()
              ((x :type float :initarg :x :initform 0.0)
               (y :type float :initarg :y :initform 0.0))
              (:metaclass persistent-metaclass)
              \comment{;; Transfer a \class{point}\ from server to client}
              \comment{;; when its referencing object is to be transferred.}
              \underline{(:dependent t)})

\listener{}(defclass line ()
              (\comment{;; Instances of class \class{point}\ will almost ever be referenced}
               \comment{;; by exactly one instance of class \class{line}.}
               (point-1 :type point :initform nil)
               (point-2 :type point :initform nil))
              (:metaclass persistent-metaclass))
\end{CompactCode}

If it is known in advance that the end points are always needed when a
\class{line}\ instance is loaded into transient memory, it would make
sense to mark the class \class{point}\ as being \lisp{:dependent}.  A
rule of thumb is if the slot of a class (\class{line}\ in the example
above) is typed by a \lisp{:type}\ slot option (like slots
\textbf{point-1}\ and \textbf{point-2}), a \lisp{:dependent}\ class
option should be given with the class the slot is typed to
(\class{point}\ in this example).

Please note that the \lisp{:dependent}\ flag is placed onto the
referenced object, and not on the referencing object.  Clever usage of
this option will result in much faster access to instances of
\lisp{:dependent}\ marked classes being referenced by other instances,
since communication overhead is dramatically reduced for such
instances.  Exhaustive use of this option may result in big, maybe
redundant data packages transferred from the server to the client.

A \lisp{:dependent}\ flag can be placed on individual objects, too
(see \fcite{(setf p-dependent)}).  Consider for example a cons cell
which should be loaded from persistent memory. If it is known that the
object referenced by the cdr of the cons cell should be loaded too,
marking the cdr object as being \lisp{:dependend}\ will transfer the
cdr object together with the cons cell.

\subsubsection{Class option \protect\lisp{:extent}}%
\label{sec:ClassExtent}

This option's argument will be used as default for the slot extent of
each slot of the class\marginlabel{The class option \lisp{:extent}\ 
  declares a default slot extent for a class.}. If a slot has an
\lisp{:extent}\ slot option specified, the slot option's value takes
precendence over the class option's value. The default class extent
for classes with tight binding (that means, with a \lisp{:metaclass}\ 
of \lisp{persistent-metaclass}) is specified by the value of
\fcite{*default-plob-slot-extent*}.

\subsubsection{Class option \protect\lisp{:schema-evolution}}%
\label{sec:ClassSchemaEvolution}

The value of the \lisp{:schema-evolution}\ class option
\marginlabel{The class option \lisp{:schema-evolution}\ tells how a
  change of a transient class definition should be promoted to its
  persistent \clsdc.}decides on how transient class redefintions
should be promoted to the instances of the class in persistent memory
(section \Nameref{sec:SchemaEvolution}).

\refpar[for all additional class options]
\Fcite{persistent-metaclass}, \fcite{(setf class-constructor)},
\fcite{(setf class-dependent)}, \fcite{(setf class-extent)},
\fcite{(setf class-dependent)}, \fcite{(setf p-dependent)},
\fcite{(setf schema-evolution)}.

\section{Declaring an index on a slot}%
\label{sec:SlotIndex}

For the tight binding \clos\ interface, an index can be declared on a
slot\marginlabel{The slot otion \lisp{:index}\ declares an index to be
  maintained on the slot values.}: The additional slot option
\lisp{:index}\ declares an index to be maintained for the slot's
values.  An index maps the slot value of the slot it is declared for
of all instances to the instance having the corresponding slot value.
Currently, indexes are represented by persistent btrees; those will be
explained in detail in section \Nameref{sec:PersistentBTree}. For the
moment, it is sufficient to assume that a btree is like a hash table
having its key values arranged always sorted.  An index is always a
1:1 mapping from a slot value to the instance having that slot value.
For example, this class will maintain an index on its slot named
\lisp{soc-sec-\#}:
\begin{CompactCode}
\listener{}(defclass person (\comment{\,\{\,\lt{}superclasses\gt\,\}\,})
              ((soc-sec-# :initarg :soc-sec-#
                          :accessor person-soc-sec-#
                          :extent :cached-write-through
                          \underline{:index (btree :test equal)})
               \comment{\{\,\lt{}more slots\gt\,\}\,})
              (:metaclass persistent-metaclass))
\end{CompactCode}

Since an index has to be unique in its key, either no
\lisp{:initform}\ at all should be declared for the slot or the
\lisp{:initform}\ must not return the same value for two non-identical
instances; it must always return a different value. Indexes are
inherited to subclasses.

\subsection{Inserting objects into an index}

Objects will be inserted into the index \marginlabel{Setting a slot's
  value of an object will insert the object into the index.}as soon as
the slot for which an index has been declared is set to a value,
either by a call to \fcite{(setf slot-value)}\ or by a call to a
writer method defined on the slot. Since \plob\ cares for calling
\fcite{(setf slot-value)}\ when initializing an instance, it is
possible to pass the slot value to be maintained in an index as an
\lisp{:init-arg}\ to \fcite{make-instance}.
\begin{CompactCode}
\listener{}(setf *ps-1* (make-instance 'person :soc-sec-# 1))
;;;;; Stored #<instance class-description person 1.00
;;;;;                   slots=1/2 short-objid=50327387>.
;;;;; Stored #<structure structure-description persistent-btree
;;;;;                    1.00 slots=0/2 short-objid=50326860>.
#<instance person soc-sec-#=1 short-objid=50326730>
\listener{}(setf *ps-2* (make-instance 'person :soc-sec-# 2))
#<instance person soc-sec-#=2 short-objid=50326732>
\end{CompactCode}

This example abstracts for now from the problem of referencing the
created persistent instance in a later LISP session, where the binding
of the transient symbol \lisp{*ps-1*}\ to the persistent object does
no longer exist; this problem and its solution is handled in
section \Nameref{sec:PersistentSymbol}.

The requirement of an index being unique \marginlabel{Index keys have
  to be unique.}in its key will result in an error message if the
passed \lisp{:soc-sec-\#}\ is not unique in all of the index' current
keys.
\begin{CompactCode}
\listener{}(setf *ps-b* (make-instance 'person :soc-sec-# 2))
Error: Trying to set duplicate slot value 2 for index
       #<btree equal 2/678 short-objid=50327130> on slot
       #<instance effective-slot-description soc-sec-#
                  extent=cached-write-through
                  short-objid=50327645>
       of object #<instance person soc-sec-#=#<unbound-marker=0x104>
                            short-objid=50326734>;
       object which captures the value already is
       #<instance person soc-sec-#=2 short-objid=50326732>.
\end{CompactCode}

Changing the slot's value of an instance contained in an index will
remove the association from the slot's old value to the instance and
will set up an association from the slot's new value to the instance.
\begin{CompactCode}
\listener{}(setf (slot-value *ps-2* 'soc-sec-#) 3)
3
\end{CompactCode}
Now, the index key of \lisp{2}\ is not bound any more and can be used
for another instance:
\begin{CompactCode}
\listener{}(setf *ps-b* (make-instance 'person :soc-sec-# 2))
#<instance person soc-sec-#=2 short-objid=50326736>
\end{CompactCode}

As explained in section \Nameref{sec:PersistentBTree}, it is
also possible for an index represented by a btree to hold persistent
vectors as keys; this makes it possible to work with primary,
secondary, \ldots, \emph{n}ary keys on slots, where \emph{n} is the
(\emph{n\,-\,1})\,th componenent of the persistent vector. Also, this
can be used to enforce uniqueness for btree keys by using a vector
with an additional, unique-guaranteeing component compared to the
non-unique, non-vector key.

\subsection{Removing objects from an index}
%
\begin{fortune}[11cm]%
\hfill\begin{tabular}{lp{9cm}}
Don:    & I didn't know you had a cousin Penelope, Bill!  Was she
          pretty?\\
W.\ C.: & Well, her face was so wrinkled it looked like seven miles of
          bad road.  She had so many gold teeth, Don, she
% Aus /usr/games/lib/fortunes.dat; da steht wirklich ,,use'', nicht
% das korrekte ,,used''
          % use {\em (sic!)\/} %
          used
          to have to sleep with her head in a safe.
          She died in Bolivia.\\
Don:    & Oh Bill, it must be hard to lose a relative.\\
W.\ C.: & It's almost impossible.
\end{tabular}%
\from{W.\ C.\ Fields}%
\end{fortune}%
\noindent{}An instance will be removed from an index
\marginlabel{Unbinding the slot of an object will remove the object
  from the index.} by making the slot's value unbound.
\begin{CompactCode}
\listener{}(slot-makunbound *ps-1* 'soc-sec-#)
#<instance person soc-sec-#=#<unbound-marker=0x104> short-objid=50326730>
\end{CompactCode}

Unbounding the slot's value of an instance contained in an index will
remove the association from the slot's value to the instance. Except
that the slot's state is set to be unbound, the instance itself is not
changed or destroyed in any way. Unless there are other references to
the instance, it now may disappear with the next garbage collection.

\subsection{Querying an index}%
\label{sec:IndexQuery}

The \fcite{p-select}\ will query for all instances having specific
values in a slot with a declared index.\marginlabel{\Fcite{p-select}\ 
  is used to query an index.} It is possible to ask for a single value
or for a range of values. For example:
\begin{CompactCode}
\listener{}(p-select 'person :where 'soc-sec-# := 2)
#<instance person soc-sec-#=2 short-objid=50326732>
\listener{}(soc-sec-# (p-select 'person :where 'soc-sec-# := 2))
2
\listener{}(p-select 'person :where 'soc-sec-# :< 3)
(#<instance person soc-sec-#=2 short-objid=50326732>)
\end{CompactCode}

An index can also be used to retrieve all objects of a class:
\begin{CompactCode}
\listener{}(p-select 'person :where 'soc-sec-#)
(#<instance person soc-sec-#=2 short-objid=50326732>
 #<instance person soc-sec-#=3 short-objid=50326734>)
\end{CompactCode}

At the moment, no SQL- or OQL-like interface is available.

\subsection{Indexes and reachability}%
\label{sec:SlotIndexAndReachability}%

An index-representing object has references to all objects it
contains\marginlabel{Objects in an index are always reachable and
  therefore persistent.}. \plob\ implements persistency by
reachability: As long as a persistent object is reachable from a
persistent root object, it will remain persistent. All persistent
classes (represented by \clsdo[s]) known to \plob\ are implicit
reachable. Since an index-representing object is bound to the
persistent \sltdo\ holding the slot's description, and this \sltdo\ is
bound to the persistent \clsdo, all instances contained in an
index-representing object are reachable and therefore
persistent.\footnote{In some persistent systems, such an object
  referencing all objects of a class is called an \emph{allset}.} Here
is the complete reference chain for the example class \class{person}\ 
given above:
\begin{quote}
  Persistent root $\rightarrow$ class table $\rightarrow$ \clsdo[{ of
    class \class{person}}] $\rightarrow$ \sltdo[ of slot
  {\textbf{soc-sec-\#}}] $\rightarrow$ index-representing object (a
  btree) $\stackrel{*}{\rightarrow}$ instance of class
  {\class{person}}
\end{quote}
So, the only possibility to get rid of an object of a class with a
declared index is to remove the object from the index.

\refpar Section~\Nameref{sec:PersistentBTree},
\fcite{p-select}, \fcite{+plob-slot-write-through-extents+},
\fcite{slot-makunbound}.

\section[Loose Binding]%
{Making objects persistent: Loose binding}%
\label{sec:LooseBinding}

The loose binding is established by a store/load interface made up by
the two functions \fcite*{store-object}\ and
\fcite*{load-object}\marginlabel{`Loose binding' means explicite
  equalization of an object's state between its transient and
  persistent representation.}.  Their purpose is to destructively
equalize between the state of an object in persistent memory and the
state of its counterpart in transient memory.  It lacks some of the
features of the tight binding, for example no automatic index
administration is possible for instances handled this way. Some other
functionality is brought back by the top-level declarations which can
be used to influence persistency for instances of loose-bound classes,
too, see section \Nameref{sec:ToplevelDeclarations}. The loose binding is
done by storing an instance by a call to \fcite{store-object}:
\begin{CompactCode}
\listener{}(setf *objid* (store-object \textrm{\emph{\lt{}object\gt}}))
\comment{;; To be more specific, for a call like}
\comment{;; \texttt{\listener{}(store-object "A string.")}}
\comment{;; \texttt{#<simple-string `A string' short-objid=50327649>}}
\comment{;; is returned.}
\end{CompactCode}

Any LISP object can be stored this way, \emph{\lt{}object\gt} is not
limited to be a \clos\ instance. The term `any' is to be understood
literally here, the only exception is binary function code (see
section \Nameref{sec:BinaryCode}).  The reason for offering a loose
binding for general LISP and \clos\ objects is that getting control
over the access to the state of non-\clos\ and general \clos\ 
instances would mean to overwrite many system-supplied functions
working on those objects, with unforeseeable side effects, like system
instability, loss of overall system performance etc.

The example call above returned an \objid\ which acts as a proxy for
the persistent representation of the stored transient object. This
proxy can be used in turn to load the object:
\begin{CompactCode}
\listener{}(load-object *objid*) \comment{;; returns \textrm{\emph{\lt{}object\gt}}}
\comment{;; For the string stored above,}
\comment{;; \texttt{"A string."}}
\comment{;; is returned.}
\end{CompactCode}
Besides loading the whole persistent representation into a transient
object, the \objid\ (proxy) can be used to work directly on the
persistent representation; this is explained in
section \Nameref{sec:DirectBinding}.

This call will return the object currently stored. As mentioned
already in an example above, this one abstracts too from the problem
of referencing the created persistent instance in a later LISP
session, where the binding of the transient symbol \lisp{*objid*}\ to
the \objid\ does no longer exist; this problem and its solution is
handled in section \Nameref{sec:PersistentSymbol}.

Though it is also possible to pass the numeric \objid\ as shown in the
output behind the \textbf{store-object}\ call directly to
\fcite{load-object}\ for loading the object, this is no good way to
reference a persistent object. This numeric \objid\ is temporary,
since it will change its value with the next garbage collection on the
persistent heap; using persistent symbols (section
\Nameref{sec:PersistentSymbol}) for addressing is safe.

\refpar \Fcite{store-object}, \fcite{load-object}.

\subsection{Setting the depth of storing or loading an object}

Both functions \fcite*{store-object}\ and \fcite*{load-object}\ have a
\funarg{depth}\ argument telling how `deep' the object's state should
be transferred between transient and persistent memory. The
\funarg{depth}\ argument is an optional argument with its default
being the value of \fcite{*default-depth*}.

\subsubsection{Specifying a \protect\funarg{depth}\ of
  \protect\lisp{:cached}}

A depth of \lisp{:cached}\ will transfer an object's state and all
objects referenced in its slots if and only if the objects are not
found in the cache. Some calls do an object storing and implicit
caching. For example, if an instance of a tight bounded persistent
\clos\ class is created by a call to \fcite{make-instance}\ and at
least one slot of the class has an \lisp{:initform}, the created
object will be stored and put into the cache. Depending on the system
configuration, strings and numbers might not be cached at all (section
\Nameref{sec:IdAndCache}) and so will always be stored with a depth of
\lisp{:cached}.

\subsubsection{Specifying a \protect\funarg{depth}\ of
  \protect\lisp{:flat}}

A depth of \lisp{:flat}\ will transfer always the top-level of an
object's state.  The objects referenced in its slots will be
transferred if and only if they are not found in the cache.

\subsubsection{Specifying a \protect\funarg{depth}\ of
  \protect\lisp{:deep}}

A depth of \lisp{:deep}\ will transfer always the top-level of an
object's state and all the objects referenced in its slots.

\subsubsection{Specifying a \protect\funarg{depth}\ of
  \protect\lisp{:objid}\ or \protect\lisp{:object}}

This will result in a do-nothing in both functions, since this depth
indicates that the caller is working directly on the persistent
representation of the object passed, so no transfer is necessary.

\subsection[Tight vs. loose binding]{Tight vs.\ loose binding:
  Benefits and drawbacks}

The benefits of using a tight binding is a very high degree of
transparency for persistency: Besides knowing of the above mentioned
additional slot and class options, the user of the tight-binding
interface can use persistent objects with few knowledge of what is
happening back-stage in \plobwoexcl. Also, the tight binding offers
more functionality than the loose binding interface, for example the
automatic administration of indexes.  The drawback is a possible loss
in overall performance on instances of classes with tight binding to
persistency, since the tight binding is very generic. Also, many LISP
systems (both \lwcl\ and \allegrocl\ belonging to them) do heavily
optimizations to slot accesses of instances whose class is a direct
instance of the \clsmc\ \class{standard-class}; this advantage is lost
for exactly the classes using the tight binding.

A further benefit on using the loose binding besides keeping the
optimized slot access is that instances whose class declaration is
unknown or out of scope can be stored and loaded, too. A drawback is
that some functionality is lost w.r.t.\ the tight binding interface.

\section[Direct Interface]%
{Working directly on persistent objects}%
\label{sec:DirectBinding}

This section explains how to work directly on the persistent
representations of persistent objects. This way of working is
sometimes needed for efficiently establishing side
effects.\footnote{This kind of working would be completely unnecessary
  when designing a persistent LISP system from scratch on, but this
  was beyond the scope of this work.} For example, when working
destructively on lists which should be kept persistent, there are at
least two methods how to do this:
\begin{enumerate}

\item\label{enu:WorkTransient} Do destructive changes to the transient
  representation of the list and store the list anew (for example, by
  a call to \fcite{store-object}) after all modifications have been
  done.

\item\label{enu:WorkDirect} Do destructive changes directly to the
  persistent list and, if necessary, load the list anew after all
  modifications have been done (for example, by a call to
  \fcite{load-object}) .

\end{enumerate}
Solution~\ref{enu:WorkTransient} has the benefit of being simple to
code, but will store not only the list itself anew, but eventually all
objects referenced by the list. \plob\ does not distinguish on a
per-class basis between objects to be stored and objects not to be
stored during a store operation. Instead, all objects referenced by
the object to store are written to the persistent heap.

By working directly on persistent objects,
solution~\ref{enu:WorkDirect} does not suffer from this problem, but
another application interface has to be used. At a first glance, this
solution sounds similar to the first one the other way round. This
worked in practice when building up a database in an offline phase
(\emph{not} [re]loading the [modified] persistent objects) and using
it (more or less read-only) in an online phase. The lists have been
referenced by slots of persistent \clos\ instances representing nodes
of a graph. In that application, the lists have not been loaded
explicit by calls to \fcite{load-object}, but when a persistent
\clos\ instance was loaded on-demand and its slot value containing the
list has been accessed.

\subsection{Immediates}

An \hyperlink{link:Immediate}{immediate}\ is an object whose
representation of state captures fewer bits than an \objid\ 
identifying the object, so it is much more efficient to represent the
immediate object's state directly in a
(type marked) \objid. Table~\ref{tab:Immediates} %
\begin{figure}[htbp]\centering
\begin{tabular}{|l|r|l|}
\hline
\multicolumn{1}{|c|}{\textbf{Type}}
        &\multicolumn{1}{c|}{\textbf{Width [bits]}}
        &\textbf{See also}\\
\hline
\hline
character       &    8 & \\
bitmask         & 24+5 & reserved, but not yet implemented\\
short-float     &   28 & \\
fixnum          &   30 & \\
marker          &   29 & section \Nameref{sec:Markers}\\
\hline
\end{tabular}
\caption{Immediate represented types and their data width in bits}%
\label{tab:Immediates}
\end{figure}
shows all types whose instances are represented as immediates in
\plobwoexcl. Therefore, instances of these types have no functions for
direct access. True \class{short-float}s are only represented in
\lwcl, in \allegrocl\ they are always coerced to type
\class{single-float}.

\subsection{Establishing a context for direct working}

For establishing a context for direct working, there are the following
two posibilities.

\subsubsection{Macro \textbf{with-direct-representation}}

\Fcite{with-direct-representation}\ establishes a context on its
\funarg{forms}\ argument where direct representation is used for all
following direct interface functions. Since all what
\fcite{with-direct-representation}\ does is binding a special variable
to a new value, the scope of this context is dymnamic and not
lexical.
Consider for example:
\begin{IndentedCompactCode}
\listener{}(p-cons "The CAR" "The CDR")\marginnumber{\smalloi}
("The CAR" . "The CDR")
\listener{}(with-direct-representation () (p-cons "The CAR" "The CDR"))\marginnumber{\smalloii}
#<cons (#<simple-string `The CAR' short-objid=50327142> .
        #<simple-string `The CDR' short-objid=50327141>)
        short-objid=50327160>
\end{IndentedCompactCode}

The first example call \oi\ returns a transient representation for the
created persistent cons cell, whereas the second call \oii\ returns an
object referencing the persistent cons cell. In \oii, no transient
cons cell is created at all.

\Fcite{with-transient-representation}\ can be used to switch to a
context where transient representations are generated for persistent
objects; the transient representation is the default representation
selected by \plobwoexcl.

\subsubsection{Overwrite optional \protect\funarg{depth}\ argument}

Many of the direct interface functions have an optional
\funarg{depth}\ argument. When this optional argument is set to
\lisp{:object}, the function will work directly on a persistent
representation. Consider for example:
\begin{IndentedCompactCode}
\listener{}(p-cons "The CAR" "The CDR" :object)\marginnumber{\smalloii}
#<cons (#<simple-string `The CAR' short-objid=50327145> .
        #<simple-string `The CDR' short-objid=50327144>)
        short-objid=50327160>
\end{IndentedCompactCode}

[See the explanations given with the example code before.]

\subsection{Common issues for the following sections}

In the following sections, many classes have a function
\textbf{p-allocate-}\emph{\lt{}class name\gt}. These functions
allocate an instance of the class named \emph{\lt{}class name\gt} in
the persistent heap and initialize it to some specific state
(normally, all slots are set to be unbound). In the following
sections, they are marked by an \textbf{\AllocatorLabel} label.

The functions named \textbf{p-make-}\emph{\lt{}class name\gt} allocate
an instance of the class named \emph{\lt{}class name\gt} in the
persistent heap and write the state of its \funarg{t-from}\ argument
into the fresh allocated persistent object.  In the following
sections, they are marked by a \textbf{\CreatorLabel} label.

The functions named \textbf{p-}\emph{\lt{}class name\gt}\textbf{p} or
\textbf{p-}\emph{\lt{}class name\gt}\textbf{-p} are predicates taking
an \objid\ as argument. They check if the persistent object referenced
by the \objid\ is an instance of the class named \emph{\lt{}class
  name\gt}. In the following sections, those functions are maked by a
\textbf{\TypePredicateLabel} label.

With the direct interface, it is possible to work either on single
slots of a persistent object or on the state of a persistent object as
a whole.

The functions working on single slots are named similar to the ones
defined in \cite{bib:CLtLII}\ with a prefix of \textbf{p-}, for
example, \fcite{p-car}\ works on the car of a persistent cons cell. In
the following sections, those functions are marked marked by a
\textbf{\SlotAccessorLabel} label in general or by a
\textbf{\SlotReaderLabel} or \textbf{\SlotWriterLabel} label if
appropiate.

All functions whose name is matching \textbf{(setf p-}\emph{\lt{}class
  name\gt}\textbf{)} store a transient instance of the class named
\emph{\lt{}class name\gt} to the persistent heap as a whole. A
persistent instance is allocated, if appropiate, and its state is set
to the state of the transient object passed as the \funarg{new-value}\ 
function argument.  In the following sections, those functions are
marked by an \textbf{\ObjectStoreLabel} label.

All functions whose name is matching \textbf{p-}\emph{\lt{}class
  name\gt} load a persistent instance of the class named
\emph{\lt{}class name\gt} from the persistent heap. A transient
instance is allocated, if appropiate, and its state is set to the
state of the persistent object referenced by the passed \objid\ 
function argument.  In the following sections, those functions are
marked by an \textbf{\ObjectLoadLabel} label.

Some persistent classes have functions returning some kind of
information about its instances. In the following sections, those
functions are marked by an \textbf{\InformationLabel} label.

\refpar All functions named \textbf{p-allocate-}\emph{\lt{}class
  name\gt}; all functions named \textbf{p-}\emph{\lt{}class
  name\gt}\textbf{p} or \textbf{p-}\emph{\lt{}class
  name\gt}\textbf{-p}; all \textbf{p-\hbox{\lt\emph{fn}\gt}}\ 
functions with \lt\emph{fn}\gt\ being one function defined in
\cite{bib:CLtLII}; all functions named \textbf{(setf p-}\emph{\lt{}class
  name\gt}\textbf{)} and \textbf{p-}\emph{\lt{}class name\gt}.

\subsection[Persistent numbers]{Working directly on persistent
  numbers}

Figure~\ref{fig:NumericClasses} %
\begin{figure}[htbp]
\centerline{\psfig{figure=\Path/numtax.eps}}
\caption{Numeric classes}%
\label{fig:NumericClasses}
\end{figure}%
shows all numeric classes implemented as built in types in
\plobwoexcl. Not all of these classes might be available in the actual
LISP system used, for example, none of the \allegrocl\ versions and
only pre-4.0.1 versions of \lwcl\ do support short floats. \lwcl\ 
4.0.1 does not support single floats, too; all floats are represented
by double floats. Class fixnum is the class of all 30 bit wide
persistent fixnums; transient integers of the actual LISP system used
will be transformed into persistent 30 bit fixnums, if possible.

\refpar[for type bignum] 
\begin{NarrowRefList}
\Allocator \Fcite{p-allocate-bignum}
\Creator \Fcite{p-make-bignum}
\TypePredicate \Fcite{p-bignum-p}
\ObjectStore \Fcite{(setf p-bignum)}
\ObjectLoad \Fcite{p-bignum}
\end{NarrowRefList}

\refpar[for type single float]
\begin{NarrowRefList}
\Allocator \Fcite{p-allocate-single-float}
\Creator \Fcite{p-make-single-float}
\TypePredicate \Fcite{p-single-float-p}
\ObjectStore \Fcite{(setf p-single-float)}
\ObjectLoad \Fcite{p-single-float}
\end{NarrowRefList}

\refpar[for type double float]
\begin{NarrowRefList}
\Allocator \Fcite{p-allocate-double-float}
\Creator \Fcite{p-make-double-float}
\TypePredicate \Fcite{p-double-float-p}
\ObjectStore \Fcite{(setf p-double-float)}
\ObjectLoad \Fcite{p-double-float}
\end{NarrowRefList}

\refpar[for type ratio]
\begin{NarrowRefList}
\Allocator \Fcite{p-allocate-ratio}
%\Creator       \Fcite{p-make-}
\TypePredicate \Fcite{p-ratiop}
\ObjectStore \Fcite{(setf p-ratio)}
\ObjectLoad \Fcite{p-ratio}
\SlotReader \Fcite{p-numerator}, \fcite{p-denominator}
\end{NarrowRefList}

\refpar[for type complex]
\begin{NarrowRefList}
\Allocator \Fcite{p-allocate-complex}
%\Creator        \Fcite{p-make-}
\TypePredicate \Fcite{p-complexp}
\ObjectStore \Fcite{(setf p-complex)}
\ObjectLoad \Fcite{p-complex}
\SlotReader \Fcite{p-realpart}, \fcite{p-imagpart}
\end{NarrowRefList}

\subsection[Persistent lists]{Working directly on persistent lists}

Lists are handled as sequences, that means, a load or store operation
will be done for the cdr of each cons cell, too.

\refpar[for type cons]
\begin{NarrowRefList}
\Allocator \Fcite{p-allocate-cons}
\Creator \Fcite{p-cons}
\TypePredicate \Fcite{p-consp}
\ObjectStore \Fcite{(setf p-list)}
\ObjectLoad \Fcite{p-list}
\SlotReader \Fcite{p-car}, \fcite{p-cdr}
\SlotWriter \Fcite{(setf p-car)}, \fcite{(setf p-cdr)}
\end{NarrowRefList}
Addtionally, \fcite{p-null}\ is a predicate which checks for its
argument being either a transient or persistent \lispnil.

\subsection[Persistent arrays]{Working directly on persistent arrays}

If possible, specialized arrays, that means, arrays with an
\lisp{:element-type}\ $\neq$ \class{t}, are stored with the element
type they are specialized to:
\begin{CompactCode}
\listener{}(store-object (make-array '(2 3)
                                       :element-type '(unsigned-byte 8)
                                       :initial-element 0))
#<array (unsigned-byte 8) (2 3) short-objid=50327320>
\end{CompactCode}

This is only done if and only if \lwcl\ or \allegrocl\ do not
downcast the element type to type \class{t}; no downcast to type
\class{t}\ is induced by \plob\ itself on array elements:
\begin{CompactCode}
\listener{}(array-element-type (make-array '(2 3)
                                             :element-type 'float
                                             :initial-element 0.0))
T
\end{CompactCode}

In this case, a downcast to type \class{t} is done by \allegrocl, so
the array elements will also be stored as non-specialized instances.

\refpar[for type array]
\begin{NarrowRefList}
\Allocator \Fcite{p-allocate-array}
%\Creator        \Fcite{p-make-}
\TypePredicate \Fcite{p-arrayp}
\ObjectStore \Fcite{(setf p-array)}
\ObjectLoad \Fcite{p-array}
\SlotAccessor \Fcite{p-array-data-vector}\ (direct usage
  deprecated)
\Information \Fcite{(setf p-array-fill-pointer)},
  \fcite{p-array-fill-pointer}, \fcite{p-array-displaced-offset},
  \fcite{p-array-adjustable}, \fcite{p-array-rank},
  \fcite{p-array-dimensions}, \fcite{p-array-element-type}
\end{NarrowRefList}
For the moment, there is no \textbf{p-aref} function for direct access
to persistent array elements. The workaround is that a persistent
array must be stored or loaded as a whole for accessing its
elements.

\subsection[Persistent vectors]{Working directly on persistent
  vectors}

The following functions work on simple vectors, that means, vectors
without a fill pointer and their element type specialized to type
\class{t}. Non simple vectors are handled as 1-dimensional arrays.

\refpar[for type vector]
\begin{NarrowRefList}
\Allocator \Fcite{p-allocate-vector}
%\Creator        \Fcite{p-make-}
\TypePredicate \Fcite{p-vectorp}
\ObjectStore \Fcite{(setf p-vector)}
\ObjectLoad \Fcite{p-vector}
\SlotReader \Fcite{p-svref}
\SlotWriter \Fcite{(setf p-svref)}
\Information \Fcite{p-vector-length}
\end{NarrowRefList}

\subsection[Persistent bit vectors]{Working directly on persistent bit
  vectors}

\refpar[for type bit-vector]
\begin{NarrowRefList}
\Allocator \Fcite{p-allocate-bit-vector}
%\Creator        \Fcite{p-make-}
\TypePredicate \Fcite{p-bit-vector-p}
\ObjectStore \Fcite{(setf p-bit-vector)}
\ObjectLoad \Fcite{p-bit-vector}
%\SlotReader  \Fcite{p-}
%\SlotWriter  \Fcite{(setf p-)}
\Information \Fcite{p-bit-vector-length},
  \fcite{p-bit-vector-max-length}
\end{NarrowRefList}
Because a bit vector has single bit elements and all allocations are
done to word boundaries, there are \lisp{(- (p-bit-vector-max-length
  \emph{\lt{}bit-vector\gt}) (p-bit-vector-length
  \emph{\lt{}bit-vector\gt}))} overhead bits allocated with each bit
vector.

For the moment, there is no \textbf{p-aref} function for direct access
to persistent bit vector elements. The workaround is that a persistent
bit vector must be stored or loaded as a whole for accessing its
elements. In the future, persistent bit vectors with their number of
elements $\leq$ 24 may be represented as immediates.

\subsection[Persistent strings]{Working directly on persistent
  strings}

\refpar[for type string]
\begin{NarrowRefList}
\Allocator \Fcite{p-allocate-string}
%\Creator        \Fcite{p-make-}
\TypePredicate \Fcite{p-stringp}
\ObjectStore \Fcite{(setf p-string)}
\ObjectLoad \Fcite{p-string}
\Information \Fcite{p-string-length},
  \fcite{p-string-max-length}
\end{NarrowRefList}
Because a string has single byte elements and all allocations are done
to word boundaries, there are \lisp{(- (p-string-max-length
\emph{\lt{}string\gt}) (p-string-length \emph{\lt{}string\gt}))}
overhead bytes allocated with each string.

Meta bits and the like are not supported for characters contained in a
persistent string.

In the future, persistent strings with their number of characters
$\leq$ 3 may be represented as immediates.

\subsection[Persistent functions]{Working directly on persistent
  functions}

\refpar[for type function]
\begin{NarrowRefList}

  \Allocator \Fcite{p-allocate-function}

%\Creator        \Fcite{p-make-}

  \TypePredicate \Fcite{p-functionp}

  \ObjectStore \Fcite{(setf p-function)}

  \ObjectLoad \Fcite{p-function}

  \SlotReader \Fcite{p-function-code}, \fcite{p-function-language},
  \fcite{p-function-name}

  \SlotWriter \Fcite{(setf p-function-code)}, \fcite{(setf
    p-function-language)}, \fcite{(setf p-function-name)}

\end{NarrowRefList}
Direct storing and loading of binary function code is not supported,
see section \Nameref{sec:BinaryCode} for details and the implemented
solution.

\subsection[Persistent hash tables]{Working directly on persistent
  hash tables}

In many LISP systems, hash tables are stored internally as instances
of a \lisp{defstruct}\ class, so \plob\ would be able to store hash
table instances without any special intervention. Since the names
choosen for the class whose instances represent hash tables and its
slots vary among the LISP system implementators, and for efficiency
reasons, hash tables are explicit represented in persistent memory by
instances of \fcite{persistent-hash-table}.  \refpar[for type
hash-table]
\begin{NarrowRefList}
%\Allocator      \Fcite{p-allocate-hash-table}
%\Creator        \Fcite{p-make-}
%\TypePredicate  \Fcite{p-hash-table-p}
\ObjectStore \Fcite{(setf p-hash-table)}
\ObjectLoad \Fcite{p-hash-table}
%\Information \Fcite{p-hash-table-}
\end{NarrowRefList}
Using a persistent btree could be more suitable than using a
persistent hash table (section \Nameref{sec:PersistentBTree}, and
section \Nameref{sec:DataModeling}).

\subsection[Persistent CLOS instances]{Working directly on persistent
  \protect\clos\ instances}

The direct way of working on persistent structures and \clos\ 
instances is business as usual, \marginlabel{\Fcite*{slot-value}\ and
  companions work on both structure and \clos \ objects.}this can be
done by calling the functions \fcite*{(setf slot-value)},
\fcite*{slot-value}, \fcite*{slot-boundp}\ and
\fcite*{slot-makunbound}. In \allegrocl, there is a problem with those
functions working directly on persistent objects: They expect their
\funarg{object}\ argument to be an instance of a \clos\ class
effectively having the slot whose \funarg{slot-name}\ was passed; this
is not the case for the direct access interface, since this interface
works on \emph{references} to persistent objects but not on true
\clos\ objects themselves. As a workaround, when using the direct
interface on \clos\ instances in \allegrocl, the functions
\fcite*{(setf p-slot-value)}, \fcite*{p-slot-value},
\fcite*{p-slot-boundp}\ and \fcite*{p-slot-makunbound}\ must be used.
Please note that \emph{this workaround must only be used under these
  described circumstances.}
\begin{RefList}\itemsep0pt

  \Allocator \Fcite{p-allocate-instance}. Direct usage is deprecated;
  instead, the proceeding described at \textbf{\CreatorLabel} should
  be used for creating objects.

  \Creator \Fcite{make-instance}\ should be called, and, if necessary,
  the instance should be stored by a call to \fcite{store-object}.

  \TypePredicate \Fcite{p-instancep}

%\ObjectStore \Fcite{p-}
%\ObjectLoad \Fcite{p-}

  \SlotReader For \lwcl: \fcite{slot-value}. For \allegrocl:
  \fcite{p-slot-value}

  \SlotWriter For \lwcl: \Fcite{(setf slot-value)},
  \fcite{slot-makunbound}.  For \allegrocl: \Fcite{(setf
    p-slot-value)}, \fcite{p-slot-makunbound}

 \Information For \lwcl: \Fcite{slot-boundp}. For \allegrocl:
 \Fcite{p-slot-boundp}

\end{RefList}

\subsection[Persistent structure instances]{Working directly on
 persistent structure instances}

\begin{RefList}\itemsep0pt

  \Allocator \Fcite{p-allocate-structure}. Direct usage is deprecated;
  instead, the proceeding described at \textbf{\CreatorLabel} should
  be used for creating objects.

  \Creator The system generated structure constructor should be called
  and the object should be stored by a call to \fcite{store-object}.

  \TypePredicate \Fcite{p-structurep}

  \SlotReader For \lwcl: \fcite{slot-value}. For \allegrocl:
  \fcite{p-slot-value}

  \SlotWriter For \lwcl: \Fcite{(setf slot-value)},
  \fcite{slot-makunbound}.  For \allegrocl: \Fcite{(setf
    p-slot-value)}, \fcite{p-slot-makunbound}

 \Information For \lwcl: \Fcite{slot-boundp}. For \allegrocl:
 \Fcite{p-slot-boundp}

\end{RefList}

\refpar %
%
\Fcite{(setf slot-value-using-class) (t structure-class persistent-object symbol)},
%
\fcite{slot-value-using-class (structure-class persistent-object t)},
%
\fcite{slot-boundp-using-class (structure-class persistent-object t)},
%
\fcite{slot-makunbound-using-class (structure-class persistent-object symbol)}.

\think{Some of the methods above are specialized with their
  slot-name argument to \class{t}, others to \class{symbol}. Make
  this uniform for the different methods.}

\section{Binary function code}%
\label{sec:BinaryCode}

Binary function code \marginlabel{Binary function code can not be
  stored directly.}is the only exception to the orthogonal persistency
principle obeyed elsewhere in \plobwoexcl. Because of the expected
relocation problems and highly system dependent memory layout without
a lacking norm between different LISP systems, storing and loading of
binary function code is not supported.  Because no binary function
code can be stored, no methods are stored too. Future versions of
\plob\ may support direct storing of binary function code. Currently
implemented are two workarounds for this situation.

\refpar \Fcite{persistent-s-expression}.

\subsection{Storing functions by name}

By the standard set in \plob\marginlabel{Functions are stored by name,
  if the name can be retrieved.}, `storing' of a function code object
is done by trying to retrieve the function's symbolic name and marking
the persistent symbol as being fbound'ed; `loading' means just to look
for this symbolic name in the current LISP image and return the
function object bound to it.
\begin{CompactCode}
\listener{}(store-object #'car)
#<function CAR short-objid=50326928>
\listener{}(setf #!*persistent-car* *)
#<function CAR short-objid=50326928>
\listener{}#!*persistent-car*
#<Function CAR>
\listener{}(funcall #!*persistent-car* '(1 2))
1
\end{CompactCode}

If the symbol found at load time has no function binding, a dummy
function object is generated which signals an error when being called.
This way functions found in the LISP image can be `stored', and, for
standard functions as defined in \cite{bib:CLtLII}, even in a
system-independent manner. Persistent function objects are explicitely
represented in \plobwoexcl.

\subsection{Persistent s-expressions}

\plob\ has a \fcite{persistent-s-expression}\marginlabel{Given access
  to a function's source code, it can be stored as a persistent
  s-expression.}, which copes with persistent s-expressions. As a
special case, when the s-expression has a \lisp{car}\ of \lisp{defun},
the s-expression is handled as a function definition. Usage of
s-expressions requires access to a function's $\lambda$
expression's source code.  Considering functions and $\lambda$
expressions generated at application runtime, this surely is not a
very strong restriction.  Since \fcite{persistent-s-expression}\ is a
straight-forward defined persistent CLOS class, persistent
s-expressions are represented explicitely, too.
\begin{CompactCode}
\listener{}(setf #!*code*
               (make-instance 'persistent-s-expression
                              :source '(lambda (x)
                                        (format t "X is ~A~%" x)
                                        x)))
#<instance persistent-s-expression short-objid=50326545>
\listener{}(funcall (sexpr-code #!*code*) 1)
X is 1
1
\end{CompactCode}

The expression given with the \lisp{:source}\ initarg has to be a
$\lambda$ list and \emph{not} a function-quoted expression
(\lisp{\#'(\ldots)}), since function quotes might be compiled into
binary function code; this binary function code cannot be stored by
\plobwoexcl.

\section{Hooking into object storing or loading}%
\label{sec:HookIntoTransfer}

For objects of some classes, it might be necessary to have side
effects coupled to store and load operations on the object. This
section describes how objects are notified if they are about to be
stored or if they have been loaded from a persistent heap.

\subsection{Object storing}

\think{}Application interface is still insufficient, for now needs
methods specialized to \fcite{t-object-to-p-objid}. Introduce generic
function \lisp{passivate}?

\subsection{Object loading}

\think{}Application interface is still insufficient, for now needs
methods specialized to \fcite{initialize-instance}\ and looking for
\lisp{\&key}\ argument \lisp{:objid}. Introduce generic function
\lisp{activate}? Problem: Mangled handling of creating and
initializing a \clos\ instance in \fcite{make-instance}.

\section{Top level declarations}%
\label{sec:ToplevelDeclarations}

Instances of built in classes, structure classes and \clos\ classes
\marginlabel{Top level declarations as a substitute for missing slot
  and class options.}with their definition being not available at all
can not use the tight binding described in
section \Nameref{sec:TightBinding}.  For built in and structure
classes, there is no principal way of achieving a tight binding.  For
\clos\ classes, this would require a change to the original class
definition (that means, adding a class option \lisp{:metaclass
  persistent-metaclass}); this is possible in principle, but doing so
to system defined classes would be a rather dangerous attempt, since a
loss in overall system performance or stability could be the result.
So, no additional class and slot options can be given to those classes
within their definition. To bypass this situation, top level
declarations can be used to give \plob\ the hints necessary for
handling persistency for those classes. When \plob\ encounters a class
not being made persistent up to that moment, the top level
declarations for the class are evaluated and stored within the
persistent \clsdo. This means that the top level declarations for a
class must be given \emph{before} the first instance of the class is
stored, since this will store the \clsdo, too. As with the additional
slot (section \Nameref{sec:AdditionalSlotOptions}) and class options
(section \Nameref{sec:AdditionalClassOptions}), these declarations
determine a class' or a slot's extent
(section \Nameref{sec:SlotExtent}, and
section \Nameref{sec:ClassExtent}, its constructor
(section \Nameref{sec:ClassConstructor}), dependent flag
(section \Nameref{sec:ClassDependent}), and its schema evolution
(section \Nameref{sec:ClassSchemaEvolution}). For other additional
slot and class options not mentioned here there are no corresponding
top level declarations, since their functionality can not be achieved
for non tight bound classes.

\subsection{Declaring a slot's extent}

The extent declaration is the most important one and is evaluated on
following levels, going from the least to the most significant
level. If a declaration is found on a more siginificant level, it will
overwrite a declaration from the less significant level.
\begin{description}

\item[Default value] If none of the following levels package, class or
  slot specifies an extent, a class specific default value is used.
  For structure classes, this is the value of
  \fcite{*default-structure-slot-extent*}. For \clos\ classes
  \emph{not} using the tight binding, this is the value of
  \fcite{*default-clos-slot-extent*}. For \clos\ classes using the
  tight binding, this is the value of
  \fcite{*default-plob-slot-extent*}.

\item[Package level] An extent can be declared on package level by
  calling \fcite{(setf package-extent)}, for example:
\begin{IndentedCompactCode}
(setf (package-extent (find-package \comment{\lt{}package name\gt})) :transient)
\end{IndentedCompactCode}

  This extent is used for all classes whose naming symbol is in the
  package the declaration was made for; in the above example, the
  slots of all classes defined in package \emph{\lt{}package name\gt}
  belonging to that package would become transient by default.

\item[Class level] An extent can be declared on class level by calling
  \fcite{(setf class-extent)}. For example, to make all slots of the
  class named \emph{\lt{}class name\gt} transient:
\begin{IndentedCompactCode}
(setf (class-extent (find-class \comment{\lt{}class name\gt})) :transient)
\end{IndentedCompactCode}

\item[Slot level] An extent can be declared for a slot in a class
  by calling \fcite{(setf slot-extent)}. For example, to make
  the slot named \emph{\lt{}slot name\gt} of the class named
  \emph{\lt{}class name\gt} transient:
\begin{IndentedCompactCode}
(setf (slot-extent \comment{\lt{}slot name\gt} (find-class \comment{\lt{}class name\gt})) :transient)
\end{IndentedCompactCode}
\end{description}

\refpar Section \Nameref{sec:SlotExtent};
\fcite{*default-structure-slot-extent*},
\fcite{*default-clos-slot-extent*},
\fcite{*default-plob-slot-extent*}; \fcite{(setf package-extent)},
\fcite{(setf class-extent)}, \fcite{(setf slot-extent)}.

\subsection{Declaring a class' constructor}

A class' constructor can be declared for a class by calling
\fcite{(setf class-constructor)}, for example:
\begin{IndentedCompactCode}
(setf (class-constructor (find-class \comment{\lt{}class name\gt})) \comment{\lt{}funcallable object\gt})
\end{IndentedCompactCode}

\refpar Section \Nameref{sec:ClassConstructor}; \fcite{(setf
  class-constructor)}.

\subsection{Declaring a class' dependent flag}

A class' dependent flag can be declared for a class by calling
\fcite{(setf class-dependent)}, for example:
\begin{IndentedCompactCode}
(setf (class-dependent (find-class \comment{\lt{}class name\gt})) \comment{\lt{}boolean\gt})
\end{IndentedCompactCode}

\refpar Section \Nameref{sec:ClassDependent}; \fcite{(setf
  class-dependent)}.

\subsection{Declaring a class' schema evolution}

A class' schema evolution can be declared for a class by calling
\fcite{(setf schema-evolution)}, for example:
\begin{IndentedCompactCode}
(setf (schema-evolution (find-class \comment{\lt{}class name\gt})) \comment{\lt{}keyword\gt})
\end{IndentedCompactCode}

\refpar Section \Nameref{sec:ClassSchemaEvolution}; \fcite{(setf
  schema-evolution)}.
%
% \fcite{(setf slot-initarg)}, %
% Overwrites the \lisp{:initarg}\ of a slot. Not used.
%
% \fcite{(setf slot-location)}
% Set a slot's location, used only internally.
%
\subsection{Making objects read-only}

Marking objects as read-only will raise an error on requesting a write
lock onto the object or onto one of its slots:
\begin{CompactCode}
\listener{}(setf #!*precious* "A very precious string.")
"A very precious string."
\listener{}(setf (p-read-only (p-intern '*precious*)) t)\marginnumber{\smalloi}
T
\listener{}(setf #!*precious* "A not-so-precious string.")\marginnumber{\smalloii}
;;; Info from server at executing client:fnSHwriteIndex:
;;; Waiting for lock level vector, mode wi on object
;;; #<symbol common-lisp-user::*precious* short-objid=50328919> by
;;; #<heap kirschke@localhost `Initial Lisp Listener'
;;;        transaction=2466 short-objid=50327650>.
Error: Locking read-only object #<symbol common-lisp-user::*precious*
       short-objid=50328919> by #<heap kirschke@localhost `Initial
       Lisp Listener' transaction=2466 short-objid=50327650> failed.
  [condition type: SIMPLE-ERROR]
Restart actions (select using :continue):
 0: Remove all other locks brute-force from object & retry locking.
 1: Retry to lock the object.
[1] \listener{}:pop\marginnumber{\smalloiii}
\listener{}(setf (p-read-only (p-intern '*precious*)) nil)\marginnumber{\smalloiv}
NIL
\listener{}(setf #!*precious* "A not-so-precious string.")\marginnumber{\smallov}
"A not-so-precious string."
\end{CompactCode}

The persistent symbol is marked as read-only \oi. The attempt to set
the persistent symbol to a new value raises a continuable error \oii.
Here, the error is aborted \oiii; choosing continue instead of
aborting would remove the read-only lock and would set the symbol to
the new value. After the read-ony lock has been removed \oiv, the
symbol can be set to a new value \ov.

\refpar \Fcite{(setf p-read-only)},
\fcite{p-read-only}.

\section{Programming with persistent data}
%
\begin{fortune}%
Time is not the main point. Time is the one and only point.
\from{Miles Davis}
\end{fortune}%

This section contains some hints on how to program with using
persistent objects. Persistency in \plob\ is rather transparent, and
this is paid with a certain performance penalty compared to a pure
transient system. By doing careful data modeling and coding this
penalty can be kept to a minimum.

\subsection{Overhead associated with an object transfer}

Because the memory overhead with an object's transfer is rather low,
it is neglected here; considered is only the time needed for an
object's state transfer between client and server.

The total overhead $O_{total}(slots,mode)$ associated with an object
transfer can be splitted into the communication overhead
$O_{comm}(slots)$ needed by TCP/IP\marginlabel{TCP/IP is the transfer
  protocol used by the operating system for transfer of data.} for
transferring an object's state between client and server and the
overhead $O_{server}(slots,mode)$ needed by the server for working on
an object.
%
\begin{displaymath}\begin{array}{rcl}
%
O_{total}(slots,mode)  &=
        & O_{comm}(slots) + O_{server}(slots,mode)\\[\smallskipamount]
%
slots &\equiv& \textrm{number of object slots}\\
mode &\in& \{ \texttt{read}, \texttt{write} \}
%
\end{array}\end{displaymath}
\marginlabel{The term $slots$ is valid here also for objects of built
  in classes; for example, a cons cell has 2 $slots$, a symbol 5
  $slots$ etc.}
%
The server overhead can be splitted into $O_{lock}$ needed for
locking the object and $O_{trans}(slots,mode)$ for saving the
object's lock and its state into the transaction log file.
%
\begin{displaymath}\begin{array}{rcl}
O_{server}(slots,mode) &=& O_{lock} + O_{trans}(slots,mode)
\end{array}\end{displaymath}%
%
These overheads can be modeled by:
%
\begin{displaymath}\begin{array}{rcl}
%
O_{comm}(slots) &\sim& \lceil \frac{slots}{blocksize} \rceil\\
%
blocksize &\equiv
        &\textrm{TCP/IP block size of appr.\ 1500 bytes = 375
        $slots$}\\[\smallskipamount]
%
O_{lock} &=& constant\\[\smallskipamount]
%
O_{trans}(slots,mode) &\sim
        & \left\{%
\begin{array}{ll}%
constant & \textrm{if}\ mode = \texttt{read}\\
slots    & \textrm{if}\ mode = \texttt{write}
\end{array}\right .\\[\smallskipamount]
%
\end{array}\end{displaymath}
%
For the most common cases, that means, for objects with fewer than 375
slots, the following inequality holds:
%
\begin{displaymath}
O_{lock} \ll O_{trans} \ll O_{comm}
\end{displaymath}
%
The communication overhead $O_{comm}$ will contribute at most to the
total overhead $O_{total}$, and so should be kept to a minimum.
%
\subsection{Consequences for data modeling}%
\label{sec:DataModeling}

Because $O_{comm}(slots)$ is proportional to the ceiling function on the
number of slots, the overhead up to a total of around 375 slots will
always be the same. A consequence which can be drawn from this is that
sequences represented by vectors should be used instead of general
lists whereever possible.

Also, when needing associative access to data, it should be thought of
using a btree instead of a hash table, since the access to btrees can
be done on a per-element basis, whereas persistent hash tables are
always transferred as a whole.

\subsection{Consequences for transaction handling}

The transaction handling incures much more overhead on the server than
on the client's side. The server will write at least the lock state of
each object into a transaction log file; if a write lock is placed
onto an object, the object's state will be written to the transaction
log file, too. So, the consequences for transaction handling in the
client's code are not so stringent as the consequences with the data
modeling: Transactions should not last too long, since this might
overflow the server's disk file space.

\subsection{Storing or loading lots of data}%
\label{sec:LotsOfData}

\plob\ has implemented fine-granularity locking and transactions, to
ensure a high level of isolation on database operations. For this
feature, a performance price is paid, which does not shine up very
stringent in doing online work with the database, but when lots of
data should be written to or read from the database. One of these
cases is the initial populating of a database, that means when a
database should be filled with objects drawn from other, external
sources. The proposal is to do these operations offline from the
application's point of view, in an extra process reserving the whole
database for the operation as a whole. This is no performance trick to
bypass the resource-intensive locking; the proposal's intent is to
lock the whole database on the coarsest level instead of locking each
object seperately, which will gain a performance improvement of about
a magnitude.\footnote{The performance improvement at storing objects
  is reached because \plobwoexcl's client C code checks if the client
  process holds a write lock on store level. When such a lock is
  found, the server is not asked at all to lock fresh created objects.
  This reduces the communication overhead dramaticaly. Instead, the
  objects are created into the C code client's cache, filled by the
  LISP level, and flushed to the server when they have been filled
  completely.}

To cope with circular references, \plob\ uses the object-level locking
during a transaction for store or load operations to check if an
object has already been stored or loaded within that transaction; so,
when abonding object-level locking in favor of locking the whole
store, \plob\ must be instructed to use another algorithm to look for
already stored or loaded objects. This can be done by telling \plob\ 
not to store or load objects found in its internal cache.  To be more
concrete, here are two code examples.

\subsubsection{Populating a database}

Here, the term `populating' means reading objects from an external
source and write them into a database.

\begin{IndentedCompactCode}
(defun populate-database ()
  (let (\comment{;; Make sure storing ends at the first object found in the cache:}
        (plob::*default-depth* :cached))
    (clear-cache) \comment{;; make sure the cache is empty}
    \comment{;; All locking has to be done in a transaction, so start one now: }
    (with-transaction ()
      \comment{;; Write-lock the whole store; the store lock will extend to the}
      \comment{;; end of the transaction opened in the \lisp{{\CompactCodeSize{}(with-transaction \ldots)}}\ block:}
      (write-lock-store)
      \comment{;; The `effective' populating function:}
      (read-external-objects-and-write-them-to-database))))
\end{IndentedCompactCode}

\refpar \Fcite{*default-depth*}, \fcite{clear-cache},
\fcite{with-transaction}, \fcite{write-lock-store}.

\subsubsection{Dumping a database}

Here, the term `dumping' means reading objects from a database and
write them to an external sink.

\begin{IndentedCompactCode}
(defun dump-database ()
  (let (\comment{;; Make sure loading ends at the first object found in the cache:}
        (plob::*default-depth* :cached))
    (clear-cache) \comment{;; make sure the cache is empty}
    \comment{;; All locking has to be done in a transaction, so start one now:}
    (with-transaction ()
      \comment{;; Read-lock the whole store; the store lock will extend to the}
      \comment{;; end of the transaction opened in the \lisp{{\CompactCodeSize{}(with-transaction \ldots)}}\ block:}
      (read-lock-store)
      \comment{;; The `effective' dump function:}
      (read-database-objects-and-write-them-to-external-sink))))
\end{IndentedCompactCode}

\refpar \Fcite{*default-depth*}, \fcite{clear-cache},
\fcite{with-transaction}, \fcite{read-lock-store}.

\section[PLOB in existing applications]%
{Using \protect\plob\ with existing applications}

It is possible to use \plob\ for making instances of existing
applications persistent. This proceeding has been used already in
\plob\ internally for integrating some LISP built in types as
persistent classes.

\subsection{Example: Persistent logical pathnames}

For example, following steps have been done to create a persistent
class for the built in class \class{logical-pathname}\ of \lwcl:
\begin{enumerate}

\item A constructor \fcite{make-logical-pathname-by-plob}\ is defined,
  which creates a transient instance of class
  \class{logical-pathname}:
\begin{CompactCode}
#+LispWorks
(defun make-logical-pathname-by-plob
     (&key host directory name type version)
  "Create a LispWorks logical pathname which is loaded by PLOB"
  (system::make-logical-pathname-from-components
   host directory name type version))
\end{CompactCode}

Since instances of class \class{logical-pathname}\ in \lwcl\ are
represented as structure instances, the arguments of this constructor
resemble the \lisp{\&key}\ arguments given to a system generated
structure constructor function for a `normal' \lisp{defstruct}\ class.

\item This instance constructor is registered as constructor function
  for instances of class \class{logical-pathname}\ into \plobwoexcl:
\begin{CompactCode}
#+LispWorks
(setf (class-constructor (find-class 'logical-pathname))
      'make-logical-pathname-by-plob)
\end{CompactCode}

\item The class extent of class \class{logical-pathname}\ is set to
  \lisp{:cached}:
\begin{CompactCode}
#+LispWorks
(setf (class-extent (find-class 'logical-pathname)) :cached)
\end{CompactCode}

\item The slot \textbf{system::device}\ of class
  \class{logical-pathname}\ is made transient, since it is not needed
  in the persistent representation:
\begin{CompactCode}
#+LispWorks
(setf (slot-extent 'system::device (find-class 'logical-pathname))
      :transient)
\end{CompactCode}

\end{enumerate}
Now, instances of the \lwcl\ built in class \class{logical-pathname}\ 
are handled as persistent structure instances. The approach for
\allegrocl\ would be similar as shown here.

\subsection{General proceeding for legacy systems}

\begin{fortune}
\hspace*{\fill}Give me all your lupines, please!
\from{Monty Python}
\end{fortune}

Using \plob\ with existing applications depends of course very much on
the application in consideration, so more general rules and hints are
given here. One of the following two proposals or a combination of
them should be feasible to make instances of existing applications
persistent.

\subsubsection{Using the loose binding}

The loose binding (section \Nameref{sec:LooseBinding}) could be used
for example by `hooking' into the application at appropiate places,
calling functions \fcite*{store-object}\ and \fcite*{load-object}\ on
the objects to make persistent. A load could be done at application
startup and a store at application shutdown. It should be checked if
the application has generic functions for doing this.

Top-level declarations (section \Nameref{sec:ToplevelDeclarations})
can be used to accomplish or optimize this binding. It is important to
declare those slots as being \lisp{:transient}\ which reference
pure-transient data, because the transitive closure of such data could
become very large.

\subsubsection{Using the tight binding}

The tight binding (section \Nameref{sec:TightBinding}) could be used
by trying at first to retrieve the original class definitions of the
classes whose instance should be made persistent, for example with the
help of the LISP inspector. After having obtained the original class
definition, there are two possibilities to continue, either to
redefine the `old' class definition or to create a new one.

\paragraph{Redefine old definition} This is accomplished by adding the
class option \lisp{(:metaclass persistent-metaclass)}\ to the
original class definition, and the modified definition is evaluated.
This would add a true tight binding, but might slow down the
application considerably or might cause other problems, like
application instability.

\paragraph{Create a new class definition} This is accomplished by
creating a new class with another name and a class option
\lisp{(:metaclass persistent-metaclass)}, having more or less the same
slots as the original class definition.  This would create a new,
persistent class tree besides the existing, transient one. The
way-of-working is similar to the solution proposed by employing the
loose binding: There must be some equalization between instances of
both trees. Or, \lisp{(change-class)}\ could be used on those (single)
transient instances which should be held persistent.

%%% Local Variables: 
%%% mode: latex
%%% TeX-master: "userg"
%%% End: 
