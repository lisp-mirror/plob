
\chapter{Distribution}

This chapter contains miscellaneous topics having to do with the
distribution of \plobwoexcl.

\flabel{\crfsection}{WWW and email addresses}{}
{\section{WWW and email addresses}}%
\label{sec:ContactAddress}

The official distribution site is \url{\thiswww}.  The author's email
address is \url{mailto:\thisemail}.  New versions will be announced to
the mailing list \url{mailto:\thislist}\ (for subscribing, either use
the mailing list Web interface at
\url{http://lists.sourceforge.net/lists/listinfo/plob-discussion}\ or
send an email to \url{mailto:\thislistreq}\ with \lisp{subscribe}\ in
the email's text). The list is intended for getting support on
problems with the installation or usage of \plobwoexcl.

\flabel{\crfsection}{Submitting a bug}{}
{\section{Submitting a bug}}%
\label{sec:BugSubmit}

Submit bugs to the bug tracking utility at \url{http://\thisproject}.
If you submit a bug, please supply the following informations:

\begin{enumerate}

\item The machine and operating system used, for example as returned
  by calling \lisp{uname -a}\ in a shell.

\item The LISP system and version used.

\item The \plob\ version used.

\item If applicable, the output shown in the LISP listener which led
  to the error. If the LISP debugger is raised, please add a stack
  backtrace, namely both \lisp{:zoom}\ and \lisp{:bt}\ for \allegrocl\ 
  or \lisp{:bq}\ and \lisp{:bb}\ for \lwcl.

\item If applicable, take a look into the server's logfile
  \lisp{messages.log}\ located in the database directory and try to
  identify any error output which might have to do with the error
  shown at the LISP listener's error prompt. If in doubt, send the
  whole server's logfile.

\end{enumerate}

\section{Directory structure}

This section explains the structure of the directories delivered with
\plobwoexcl.

\subsection*{Subdirectory \protect\lisp{plob-\protect\thisversion/bin/}}

Find here the server process \lisp{plobd}\ and some shell scripts used
for compiling, generating documentation etc.

\subsection*{Subdirectory \protect\lisp{plob-\protect\thisversion/conf/}}

\unix\ \lisp{make}\ configuration and rule files.

\subsection*{Subdirectory \protect\lisp{plob-\protect\thisversion/database/}}

This conatains an empty database file \lisp{stablestore.empty}\ used
by the \plob\ server for storing the persistent objects.

\subsection*{Subdirectory \protect\lisp{plob-\protect\thisversion/lib/}}

C libraries: The library \lisp{libplob.a}\ with object code common to
the server and client, the server library \lisp{libsplob.a}, the
client library \lisp{libcplob.a}\ and the \postore\ library
\lisp{libpostore.a}.

\subsection*{Subdirectory \protect\lisp{plob-\protect\thisversion/mail/}}

Some mailings done to different sites.

\subsection*{Subdirectory \protect\lisp{plob-\protect\thisversion/ps/}}

The \ps\ files with the documentation.

\subsection*{Subdirectory \protect\lisp{plob-\protect\thisversion/src/server/}}

The C code for the \plob\ server.

\subsection*{Subdirectory \protect\lisp{plob-\protect\thisversion/src/client/}}

The C code for the \plob\ client.

\subsection*{Subdirectory \protect\lisp{plob-\protect\thisversion/src/allegro/}}

The LISP code for the \plob\ \allegrocl\ client.

\subsection*{Subdirectory \protect\lisp{plob-\protect\thisversion/src/lispworks/}}

The LISP code for the \plob\ \lwcl\ client.

\subsection*{Subdirectory \protect\lisp{plob-\protect\thisversion/src/common/}}

The C code used by both the \plob\ server and client.

\subsection*{Subdirectory \protect\lisp{plob-\protect\thisversion/src/include/}} C header files.

\subsection*{Subdirectory \protect\lisp{plob-\protect\thisversion/src/include/allegro/}, 
\lisp{plob-\protect\thisversion/src/include/lispworks/}}

The LISP files generated from C header files for the foreign function
interface.

\subsection*{Subdirectory \protect\lisp{plob-\protect\thisversion/src/lisp-doc/}}

The LISP module and %
\tracingmacros2
\TeX\ %
\tracingmacros0
styles used for preparing the reference
manual \cite{bib:PLOB-manual} by extracting the documentation strings from
the LISP sources.

\subsection*{Subdirectory
  \protect\lisp{plob-\protect\thisversion/src/lispworks/memory\protect\us{}representation/}}

Some early experiments about the memory representation of transient
objects in \lw.

\subsection*{Subdirectories \protect\lisp{plob-\protect\thisversion/tex/inputs/},
 \protect\lisp{plob-\protect\thisversion/tex/manual/}}

The documentation style and \TeX\ source files.

\section[Make targets]%
{Standard \protect\lisp{make}\ targets}

Each \lisp{Makefile}\ found in one subdirectory of
\lisp{plob-\thisversion/}\ has following standard targets. One or some
of these standard targets can be given as arguments to a call of
\lisp{make}\ as additional command line arguments.  Some
\lisp{Makefiles}\ have more targets than described here; consult the
\lisp{Makefile}\ comments to find out which additional targets are
defined.  It is not necessary to use always the `top-level' makefile
\lisp{plob-\thisversion/\lb{}Makefile}\ when only files in one
subdirectory have been changed; calling \lisp{make}\ in each
subdirectory will re-build only that subdirectory.

\subsection*{Standard \protect\lisp{make}\ target \protect\lisp{all}}

This is the default target; all sources are compiled with default
compiler settings.

\subsection*{Standard \protect\lisp{make}\ target \protect\lisp{initial}}

Calling this target `initializes' the subdirectory, for example by
setting up symbolic links, creating directories etc. This target
should only be called exactly once, but no damage will result from
calling it more than once.

\subsection*{Standard \protect\lisp{make}\ target \protect\lisp{clean}}

All `garbage' files are removed; these are files with their names
matching \lisp{*\td}\ (emacs backup files), \lisp{core}, \lisp{a.out}\
etc. No object files are removed.

\subsection*{Standard \protect\lisp{make}\ target \protect\lisp{dist-clean}}

Like standard target `clean' plus all object files, libraries and
otherwise generated files are removed. `dist-clean' is an abbrevation
for `distribution-clean'. All files not belonging to a distribution
are removed.

\subsection*{Standard \protect\lisp{make}\ target \protect\lisp{toc}}

This target is only found in makefiles which generate \ps\ files.
It calls \TeX\ up to three times to make sure that the table of
contents of the documentation will be correct.

\section{Disclaimer}

The work on \plob\ in its entire contents, that means all its software
and its documentation, has been carried out privately. My employer,
CSC PLOENZKE AG has nothing to do with the work presented here. CSC
PLOENZKE AG neither develops nor supports \plob\ nor will do so in the
future in any way.

\flabel{\crfsection}{License Terms}{}
{\section{License Terms}}%
\label{sec:LicenseTerms}

\edef\THISAUTHOR{\expandafter\uppercase{\thisauthor}}

\plob\ Copyright \copyright\ 1994--\thisyear\ \thisauthor. All
rights reserved.

Unlimited use, reproduction, modification and distribution of this
software is permitted.  Any copy or modified version of this software
must include both the above copyright notice of \thisauthor\ and this
paragraph; for each modified version, an additional statement must be
added telling the year of modification and quoting the author of the
modification.  Any distribution of this software must comply with all
applicable German export control laws.  This software is made
available AS IS, and \THISAUTHOR\ DISCLAIMS ALL
WARRANTIES, EXPRESS OR IMPLIED, INCLUDING WITHOUT LIMITATION THE
IMPLIED WARRANTIES OF MERCHANTABILITY AND FITNESS FOR A PARTICULAR
PURPOSE, AND NOTWITHSTANDING ANY OTHER PROVISION CONTAINED HEREIN, ANY
LIABILITY FOR DAMAGES RESULTING FROM THE SOFTWARE OR ITS USE IS
EXPRESSLY DISCLAIMED, WHETHER ARISING IN CONTRACT, TORT (INCLUDING
NEGLIGENCE) OR STRICT LIABILITY, EVEN IF \THISAUTHOR\ IS
ADVISED OF THE POSSIBILITY OF SUCH DAMAGES.

Please note that these license terms adhere only to the code of \plob\ 
itself. \plob\ uses \postore\ (\textit{P}ersistent \textit{O}bject
\textit{Store}) as a low-level persistent memory; it is provided in
binary form within \plob\ with the permission of the
\href{http://www-ppg.dcs.st-andrews.ac.uk}{University of St.\ 
  Andrews}. Contact the University of St.\ Andrews for getting their
license terms on \postore.

Permission is given to redistribute \plob\ within other software
packages, given that the redistribution contains exactly the original
\plob\ distribution files, with all source and binary code and all
documentation files. For redistributed versions of \plob, the license
terms remain the same and must not be changed by the distribution
containing the \plob\ distribution.

\vspace*{\fill}
\begin{fortune}
  \hspace*{\fill}There are \emph{(sic!)} a finite number of jokes in
  the universe.
% Von einem D. Byrne CD Cover (Stop Making Sense); da steht wirklich
% ,,are'', nicht das korrekte ,,is''
\from{David Byrne: Stop Making Sense}
\end{fortune}

%%% Local Variables: 
%%% buffer-file-coding-system: iso-latin-1-unix
%%% mode: latex
%%% TeX-master: "userg"
%%% End: 
