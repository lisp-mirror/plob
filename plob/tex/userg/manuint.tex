% This is -*-LaTeX-*-
%
% userg.tex
% HK 26.5.94
% Rewritten 1997/09

\chapter{Introduction}

This chapter gives an overview on persistency and persistent objects
in common.  It explains the fundamental properties for persistent
objects from a LISP perspective, and presents an overview on the the
approach used in \plob\ (\emph{P}ersistent \emph{L}isp \emph{OB}jects)
for implementing persistent objects. For technical data like supported
operating systems, supported LISP systems etc.\ see
section \Nameref{sec:TechnicalData}.

\section{Ways to persistency}

Focusing on application programming, persistency is an issue in many
applications. There are different ways to achieve persistency, each
having its benefits and drawbacks:

\begin{description}

\item[File oriented] In file-oriented persistency, \marginlabel{File
    oriented persistency: Save and load objects states to and from a
    file.}the state of each object to make persistent is written into
  a file and read in again later. The only benefit of this approach is
  that it is simple to implement. One of its drawbacks is that it is
  data- and system-dependent.  Also, relocation and establishing
  references between objects is hard to achieve. Support for changes
  on an object's structure is hard to implement. Database-like
  features, like selecting single objects out of the set of all
  objects, are also hard to implement.

  In the sense of LISP, constructs using methods specialized to
  \fcite{make-load-form}\ belong into this category of persistency.

\item[Database oriented] In database-oriented persistency,
  \marginlabel{Database oriented persistency: Save and load objects
    states to and from a (relational) database.}the objects are stored
  in a (relational) database. In relational databases, there is a
  different paradigm of viewing `objects' (that means, tuples) as
  \hyperlink{link:Value}{values}\footnote{Values will be defined in
    section \Nameref{sec:DefinitionOfObjects}.}\ having no identity
  independent from their state.  Because of the different programming
  paradigms, the interfacing between a relational database and a
  programming language can get quite complex and obscure (see for
  example ESQL for a binding between SQL and C); this is often called
  `impedance mismatch'.  Because there are certain means of
  establishing a mapping between object-oriented and database concepts
  (figure~\ref{tab:MapObjectsToRelations}), this approach can make
  sometimes usage of database features, like complex queries,
  transactions etc. Prerequisite is that this mapping from
  object-oriented into database concepts is simple and straight
  forward enough so that the database manipulation functions can work
  on the mapped data at all. For example, to administrate data
  efficiently, attributes must be typed statically in most relational
  databases; so, this approach results either in static typed slots of
  objects (which is inappropiate for a dynamic type language as LISP
  is) or this mismatch requires rather expensive mapping operations.
  In object-oriented systems, there is a notion of objects having an
  identity, making each one unique among the set of all objects
  considered; a (single directed) reference between two objects is
  established by using an appropiate representation of an object's
  identity, for example by using its memory address.  In a relational
  database, this has to be mimiced by using (bi directional) expensive
  joins. In databases, aggregate types are represented implicit by
  relations; especially in LISP, an explicit type representation by
  \clsmo[s]\ is preferred.

  Because LISP has no standardized relational database interface, only
  single examples can be quoted, as the SQL interface offerred by \lw\ 
  and \allegro\ \cl.

\item[Object-oriented] In object-oriented persistency,
  \marginlabel{Object-oriented persistency: Persistency as an
    orthogonal feature of each object.}persistent objects are viewed
  as usual in an object-oriented sense, and to be more specific in
  this context, as LISP or \clos\ objects. This has some consequences
  on the design of an object-oriented database. For example, each LISP
  object has a (more or less) explicit represented class, and since
  objects are typed dynamically, each object knows of its type. Also,
  LISP has a notion of identity of objects for establishing references
  between them, and this identity should be represented for persistent
  objects, too.

  Besides the system described here,
  \href{http://www.franz.com/products/allegrostore.html}{\allegro{}{\sc
      Store}}\ and \href{http://www.ibex.ch/}{Itasca}\ can be
  mentioned as examples for object-oriented databases.

\end{description}

The object-oriented approach has been favored in \plobwoexcl, although
there are some theoretical and some practical drawbacks on this
approach.  The theoretical drawback \marginlabel{There is no common
  calculus for object-oriented databases.}is that there is no calculus
on object-oriented persistency as the relational calculus for
relational databases. Relying on the relational calculus, some
mathematical attractive properties can be shown, like the existence of
normal forms to avoid redundancies and anomalies; this does not hold
for object-oriented persistency. From a very stringent point of view,
object-oriented persistency can even be seen as a step back towards
hierarchical databases, which have been wide-spread in the sixties and
have been the motivation for Codd to develop the concept of relational
databases \cite{bib:Codd-79}. The practical drawbacks
\marginlabel{Relational databases are now quite mature.}have more or
less to do with the fact that usable relational databases have now
been on the market for at least 15 years, so they are now quite mature
and have sophisticated query optimization techniques, good transaction
behavior etc.\footnote{It took quite long for the market since Codd's
  original publication on relational databases in 1969 to provide a
  `real' relational database.} These are also features which should be
found in an object-oriented database, but naturally it takes some time
to code and test these features.
\begin{figure}\centerline{%
\begin{tabular}{|c|c|}
\hline
\textbf{Object-oriented}
        & \textbf{Relational Database}\\
\hline\hline
Class
        & Relation\\
Instance
        & Tuple \\
Slot
        & Attribute\\
\hline
\end{tabular}}%
\caption{Mapping between o-o and relational database concepts}%
\label{tab:MapObjectsToRelations}%
\end{figure}

\section[The approach taken by PLOB]{The approach taken by \protect\plob}

Start working on \plobwoexcl, the aim was and is to implement a
database with certain features, some more external features from
viewing towards \plobwoexcl's API, some more internal features for the
overall design. Some of them are borrowed from \emph{The
  Object-Oriented Database System Manifesto}\ 
\cite{bib:Atkinson-et-al-92}.

\iconpar{\psfig{figure=\Path/schampus.eps,height=2.666666\baselineskip}}%
At first, persistency should be an
\hyperlink{link:Orthogonal}{orthogonal}\ property
\marginlabel{Persistency should be orthogonal and transparent.}w.r.t.\ 
LISP's type system; in practice, this means that each LISP and \clos\ 
object should have the possibility of becoming persistent, independent
of its actual type.  Persistency should be as
\hyperlink{link:Transparent}{transparent}\ as possible, meaning that
users should not need to learn a new programming language for handling
persistent objects, but rather integrate persistency into their
applications with minimal or no changes, covering both new and legacy
systems. Transparency in a wider sense also lead to integrate some of
the transient LISP environment into \plobwoexcl, \marginlabel{Features
  of LISP should be integrated; database features should be
  added.}like persistent packages and persistent symbols, and
implementing persistency by
\hyperlink{link:Reachability}{reachability}.  Other important features
are \emph{database features}, like transactions, locking, queries (at
least to some extent), \emph{efficiency} and \emph{performance}. All
of these features have been taken into consideration or have been
implemented in \plobwoexcl.

Looking towards the internal features, \marginlabel{Explicite
  representation of types and classes as objects in \plobwoexcl.}it
was decided to represent types and classes of persistent objects
explicit, similar the way LISP does it; this simplifies the usage of
persistent objects, since there is no necessity for a mapping between
different paradigms as shown in
figure~\ref{tab:MapObjectsToRelations}. Since LISP offers a lot of
useful and efficiently represented numeric and non-numeric types,
\marginlabel{Demand for type completeness.}it was decided to represent
all of those types in \plobwoexcl, too. \plob\ distinguishes
internally between representations \marginlabel{Efficient and
  appropiate representation of persistent objects.}for instances of
\hyperlink{link:Immediate}{immediates}\footnote{Immediates will be
  defined in section \Nameref{sec:DefinitionOfObjects}.}, built in
classes, structure classes and `true' \clos\ classes
(figure~\ref{fig:ObjectTaxonomie}),
\begin{figure}[htbp]
\centerline{\psfig{figure=\Path/objtax.eps}}
\caption{Taxonomy of object representations}%
\label{fig:ObjectTaxonomie}
\end{figure}%
all of them except the immediates extended by some database-specific
slots. The principle of self-describing structures was obeyed, which
may be characterized by two axioms:
\begin{enumerate}
\item Each object has a class.
\item Classes are represented by objects.
\end{enumerate}
Exactly as in LISP, structure classes have some restrictions compared
to \clos\ classes, the most important one that
\hyperlink{link:SchemaEvolution}{schema evolution}\footnote{Schema
  evolution will be defined in section
  \Nameref{sec:DefinitionOfSchemaEvolution}.}\ is supported for
instances of structure classes only if either the number of slots did
not increase or a structure class' instance undergoing a schema
evolution is allowed to change its persistent identity. For instances
of persistent \clos\ classes, full schema evolution without loss of
identity is supported.

The experience with integrated systems in the past showed that
integration is most of the time useful when doing everyday work. When
it comes to certain `special tasks' (for example, for performance or
efficiency reasons), \marginlabel{Working `directly' on persistent
  objects.}it is sometimes necessary to abandon integration and to do
it on her or his own. When using a system which prevents bypassing
integration, these tasks are hard to solve. A consequence for \plob\ 
is that the transparent interface should and can be used, but it is
also possible to work directly on low-level representations of
persistent objects. This is important for LISP objects which are out
of control for \plob\ and are modified by side effects, for example,
lists which are modified destructively.

\section{Related documentation}

This document describes how to use \plobwoexcl's API to make objects
persistent, and how the system itself is administrated.  The more
low-level documentation accompaning this user's guide can be found in
\cite{bib:PLOB-manual}, which describes in detail each constant,
variable, macro, class, function, generic function and method
implemented in \plob\ and exported from package \lisp{PLOB}.
\cite{bib:PLOB-internals} describes the internal program constructs
not exported from package \lisp{PLOB}.  \cite{bib:Kirschke-95b}\ is an
abstract giving a course overview on \plob\ and is a supplement to
this introduction.

The fundamental principles of \plob\ are explained in my
(german-written) computer science master's thesis
\cite{bib:Kirschke-94a}.  If you are interested in making a reference
manual by extracting documentation from LISP source files see
\cite{bib:LispDoc-Manual}.

More background information on (relational) databases can be found in
\cite{bib:ONeil-94}. A vast amount of transaction techniques for
databases is explained in \cite{bib:Gray-et-al-93}.

\section{Notion used}

\lisp{Typewriter}\ \lisp{type}\ is used for programming examples and
LISP code. \marginlabel{The margin notes summarize a section of text
  or give some other background information.} Code which is important
in the current context is emphasized by
\underline{\lisp{underlining}}. Using a listener prompt
\lisp{\listener} in front of a code example means that the example can
be typed in directly into a listener. This is a
\emph{\lt{}non-terminal\gt}, meant to be replaced by some concrete
object. Repetitions are indicated by curly braces, so
\lisp{\{}\,\emph{\lt{}repeat me\gt}\,\lisp{\}} means 0 to \emph{n}
occurencies of the non-terminal \emph{\lt{}repeat me\gt}.

\refpar A paragraph labeled \textbf{References} gives references to
other sections or documents.

\section{Terms used}%
\label{sec:TermsUsed}

This section defines some of the terms used within this document, some
having to do with object-oriented programming in common, others having
to do with databases.

\subsection{Objects and classes}%
\label{sec:DefinitionOfObjects}

\subsubsection{Object}

An \emph{object} is the abstraction of a real-world entity; it owns an
\emph{identity} and has a \emph{state} and a \emph{behavior}.

\subsubsection{Identity}

Identity is a property which makes an object unique among the set of
all objects in consideration; by its identity it can be distinguished
from all other objects.

\subsubsection{Value}

A \hypertarget{link:Value}{value}\ is an object whose identity can be
completely defined by its state, that means, two values are identical
if and only if they have the same state. In this context, a value is
further restricted to be an instance of a LISP's built in class, like
a numbers or a character. With `real' (non-value) objects, their
identity is independent from their current state, that means, two
non-value objects having the same state are equal, but not necessarily
identical.  With a \emph{representation of identity} an object can be
referenced.

\subsubsection{Immediate}

An \hypertarget{link:Immediate}{Immediate}\ is a value object whose
state itself is represented directly in a representation of identity,
because its state needs less memory than an identity representation.
It differs from a value only by its special form of representation.
Candidate classes for an immediate representation are for example
fixnums, that means, integer numbers with their absolute value $<
2^{n}$, with $n <$ the number of bits used for an identity
representation.

\subsubsection{Class}

\begin{fortune}[4cm]%
More than \lisp{this}
\from{Roxy Music}
\end{fortune}%

The structure of an object's state and its behavior is defined by its
\emph{class}. The structure of an object's state is an aggregation of
\emph{slots}; a \emph{slot state} is the partial state of an object
represented in a single slot. To be accordant with \clos' terminology,
a slot state is called also a \emph{slot value}.  The behavior of an
object is made up by its \emph{methods}; each single method defines a
part of the object's behavior. Classes are represented by \clsmo[s],
slots by \sltmo[s], and methods are represented by \mtdmo[s].

\subsection{Database terms}

\subsubsection{Reachability}

\hypertarget{link:Reachability}{Persistency by reachability} means
that all objects which are reachable from a designated persistent root
object will remain persistent. Objects not being reachable from this
root object will be deleted with the next garbage collection. This
closely resembles the concept of \emph{indefinite extent} as defined
in \stcite{43}.  Further details will be given in section
\Nameref{sec:ObtainingReachability}.

\subsubsection{Orthogonal and transparent persistency}

Persistency is \hypertarget{link:Orthogonal}{orthogonal}\ if it is a
base property of an object which is independent from other object
properties, like inheriting from a certain class. Persistency is
\hypertarget{link:Transparent}{transparent}\ if there are no or few
changes necessary to link persistency into an application's code.

\subsubsection{Transaction}

A transaction is a database state changing operation which fulfills
the following \acid\ properties
\cite[\citepage{166}]{bib:Gray-et-al-93}:
\begin{quote}\begin{description}
\item[Atomicity] A transaction's changes to the state are atomic:
  either all happen or none happen. These changes include database
  changes, messages, and actions on transducers.
\item[Consistency] A transaction is a correct transformation of
  state. The actions taken as a group do not violate any of the
  integrity constraints associated with the state. This requires the
  transaction be a correct program.
\item[{\hypertarget{link:Isolation}{Isolation}}] Even though
  transactions execute concurrently, it appears to each transaction,
  T, that others executed either before T or after T, but not both.
\item[Durability] Once a transaction completes successfully (commits),
  its changes to the state survive failures.
\end{description}\end{quote}
Figure~\ref{fig:TransactionStateChange}\ %
\begin{figure}[htbp]
\centerline{\psfig{figure=\Path/trtime.eps}}
\caption{State change at ending or aborting a transaction}%
\label{fig:TransactionStateChange}
\end{figure}%
shows what happens during a transaction: At $t_n$, a transaction is
started which lasts until $t_{n+1}$. At $t_{n+1}$, the transaction is
either ended (comitted) or aborted (for example, if an error occurs
during the transaction being active). The database states $z_n$ and
$z_{n+1}$ are defined as being consistent from the application's point
of view before $t_n$ or after $t_{n+1}$. In the interval
$[t_n,t_{n+1}]$, the database may become temporary inconsistent, which
is shielded to other concurrent transactions by isolation.

\subsubsection{Index and query}

An \emph{index} is an associative memory mapping a key to a value.  In
a database, an index is usually bound to one or more slots of a class,
with imposing a 1-dimensional key ordering when using more than one
slot.  It maps a slot value (as an index key) to the object having
that slot value (as the value associated to the index' key).  This
association is maintained more or less automatically by the database
system itself.  In general, the keys of the index are ordered, for
better performance on selecting objects falling into a range of slot
values.  As will be shown in \Nameref{sec:SlotIndexAndReachability},
an index will also ensure reachability for a persistent object.

A \emph{query} selects objects out of the set of all objects of a
class or aggregates information on a [sub]set of these objects; the
condition used for discrimination is usually some specific slot value
or a range of slot values. So, an index is used to speed up a query.

For example, let us assume that a class \class{person}\ maintains an
index on its slot \lisp{soc-sec-\#}. This index will maintain the
mapping of \lisp{soc-sec-\#}s to the \class{person}\ instance bound to
that \lisp{soc-sec-\#}:
\begin{CompactCode}
\listener{}(make-instance 'person :soc-sec-# 2000)\marginnumber{\smalloi}
#\lt{}instance person soc-sec-#=2000 short-objid=50326730\gt
\listener{}(make-instance 'person :soc-sec-# 3000)\marginnumber{\smalloii}
#\lt{}instance person soc-sec-#=3000 short-objid=50326732\gt
\listener{}(p-select 'person :where 'soc-sec-# := 2000)\marginnumber{\smalloiii}
#\lt{}instance person soc-sec-#=2000 short-objid=50326730\gt
\end{CompactCode}

The mappings are established at \oi\ and \oii, and the index is
accessed with the query at \oiii.

\subsubsection{Schema evolution}%
\label{sec:DefinitionOfSchemaEvolution}

A \hypertarget{link:SchemaEvolution}{schema evolution}\ means changing
the definition (`schema' in database terminology) of a persistent
class stored in a database and what should happen to the instances
whose class has been redefined. In general, there are two ways to
handle this, either a \emph{destructive} or \emph{non-destructive}
proceeding.\marginlabel{Schema evolution can be done destructive or
  non-destructive.} In the destructive approach, a mapping from the
instance' structure as given by the `old' definition to the instance'
structure as given by the `new' definition is established, and the
instance slots are rearranged permanently to match the `new'
definition. With the non-destructive approach, the mapping is done
too, but the instance is not written back into the database, but used
only to pretend the `new' structure to the upper application levels
requesting the object's state.

%%% Local Variables: 
%%% mode: latex
%%% TeX-master: "userg"
%%% End: 
