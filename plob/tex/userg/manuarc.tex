% This is -*-LaTeX-*-
%
% userg.tex
% HK 26.5.94
% Rewritten 1997/09

\chapter[Architecture]%
{Architecture of \protect\plob}

For a detailed description of \protect\plob's architecture see
\cite{bib:Kirschke-99}.

In its first version up to September 1996, \plob\ was a single-user
system prepared for multi-user capabilities. At end of 1996, the
system was redesigned into a client/server architecture.
\begin{description}

\item[Client] The client is the LISP process using persistent
  objects. A client is connected to no or one server. At the moment,
  \lwcl\ and \allegrocl\ processes are supported as clients.

\item[Server] The server is a \unix\ process serving requests from the
  clients which connected itselves to the server. A server can handle
  many clients.

\end{description}

Figure~\ref{sec:PLOBwRPCs} shows 
\begin{figure}[htbp]
%% \centerline{\psfig{figure=\Path/plobrpce.eps}}
\centerline{\psfig{figure=\Path/plobrpce.pdf}}
\caption{\protect\plob\ layers}%
\label{sec:PLOBwRPCs}
\end{figure}%
the layers of \plob\ as actually implemented.

Based upon an evaluation of a few persistent memory systems done by
\cite{bib:Mueller-91}, it was decided to use the \postore\ library
(Persistent Object Store) as low level store.  The layers 2--5 have
been implemented by the author, layers 2 and 3 in ANSI-C and layer 4
and 5 in \cl. The communication between layers 2 and 3 is done by RPCs
(Remote Procedure Calls), an approved \unix\ standard communication
tool.

\section[POSTORE layer]%
{Layer 1:\ \protect\postore\ layer}

\postore\ is the persistent memory used in Napier88, a persistent
programming language developed at the University of St.\ Andrews.
This layer offers a persistent heap, a concept very similar to a
transient heap, but the `records' allocated from a persistent heap are
persistent. Each \postore\ vector is subdivided into a first part
interpreted as having references to other \postore\ vectors and a
second part which is not interpreted by \postore\ any further; the
second part is used to store the state of value-like objects, that
means, objects not referencing other objects. This is used by
\postore\ to implement a garbage collection based on reachability.

The \postore\ has no usable transaction processing in the sense
of a database. The locking offered by \postore\ is a simple semaphore
locking on a per-object basis.

\section[C-PLOB server layer]%
{Layer 2:\ C-\protect\plob\ server layer}

This layer adds a whole bunch of functionality to the system:
Sessions, object locking and transactions, persistent B-Trees.
Layer 1 and 2 make up the \plob\ server process.

\section[C-PLOB client layer]%
{Layer 3:\ C-\protect\plob\ client layer}

The task of this layer is to interface between the previous and the
next layer. Besides this, it contains the code for connecting the
client with the server process, implements some caching and
pre-allocation of persistent objects etc.  A part of the functionality
of layer 2 is used directly on the client side by layer 3 to reduce
the number of RPC calls to the server, for example, the print
representation for persistent immediates is computed on the client
side. Layer 3 up to layer 5 make up the \plob\ client process.

\section[PLOB layer]%
{Layer 4:\ \protect\plob\ layer}

This layer implements an interface to the application layer for
convenient usage of persistency. Its functionality is documented in
this user's guide.

\section[Application layer]%
{Layer 5:\ Application layer}

The application layer uses the functionality offered by the previous
layer: Persistent packages, persistent symbols, sessions, associative
access by using persistent btrees.

\section[Porting PLOB]{Porting \protect\plob}

This section describes how to port \plob\ to a new, not-yet-supported
operating system and to a new, not-yet-supported LISP system.


\subsection{New Operating System}

This section assumes that the port of the LISP system itself is
already done and running stable. The \plob-specific tasks are
described in the following sections.

\subsubsection{Skills needed}

The port to a new operating system involves mainly a successfull
compilation of the low-level \postore\ library. This means that the
tasks to be done have to do more or less almost nothing with LISP
coding, but with operating-system specific programming in form of
shell scripts, makefiles, and C system programming.  The following
skills are needed for a port to a new operating system:

\begin{itemize}

\item Good knowledge of \texttt{make}\ and \texttt{make}\ scripts

\item Good knowledge of shell programming, esp.\ Bourne shell

\item Good knowledge in C, and the specialities of the C preprocessor.

\item Good knowledge about the virtual memory functions
  (\texttt{mmap}) and signal handling capabilities (\texttt{signal},
  \texttt{sigaction}) of the target operating system.

\end{itemize}

\subsubsection{Step 1: Check out from CVS}

Check out \plob's sources from
\begin{quotation}
\texttt{:pserver:anonymous@cvs..sourceforge.net:/cvsroot/plob}
\end{quotation}
This will create a subdirectory \texttt{plob} containing the
checked-out files.

\subsubsection{Step 2: Select an identifier}

Select an identifier for the operating system, e.g. \texttt{win32}.
The identifier has to obey the rules for a single C token and has to
be in lower case, i.e.\ the syntax is:
\begin{quotation}
<opsys> ::= [a-z]\{[a-z0-9]\}*
\end{quotation}

\subsubsection{Step 3: Build the \protect\postore\ library}

In the next step, the \postore\ library will be adapted to the target
operating system. To do this, a high skill level in system-near
programming in C in necessary, esp.\ on the virtual memory functions of
the target operating system.  The sources of the \postore\ library are
located in directory \texttt{plob/postore}. This library is a 3rd
party product used as a persistent heap. Only some general rules can
be given here, since the differences between the operating systems
might be quite large.

First step in compiling the \postore\ library is to create the
operating system specific configuration file in
\texttt{plob/postore/<opsys>\us{}Makefile}. Use one of the exisiting
\texttt{<opsys>\us{}Makefile}s as base, e.g.\ the Linux file in
\texttt{plob/postore/i586\us{}Makefile}.

Next step in compiling the \postore\ library is to create the
compiler specific configuration file in
\texttt{plob/postore/\$\{CC\}\us{}Makefile}\ with \texttt{CC} being
set in the operating system specific configuration file. Use one of
the exisiting \texttt{plob/postore/\$\{CC\}\us{}Makefile}s as base,
e.g.\ the gcc file in \texttt{plob/postore/gcc\us{}Makefile}.

Now, edit \texttt{plob/postore/sstore/svmstore.c} to meet the target
operating system. The source code contains a lot of operating-system
specific code steered by preprocessor defines. The biggest challenge
is to find a suitable virtual memory mapping function \texttt{mmap} in
the target system, and to check for the various flags to be passed
into \texttt{mmap}. Another challenge is to set up the signal handling
(\texttt{sigaction}, \texttt{signal}) for passing page faults to the
page fault handling procedure \texttt{page\us{}fault}. Besides
\texttt{plob/postore/sstore/svmstore.c}, the persistent heap code
itself is contained in \texttt{plob/postore/sheap/sheap.c}; this
should need no special adaption for a specific operating system.

The library and its supporting programs can be build by calling
\texttt{make} in directory \texttt{plob/postore}. When the build is
successfull, try calling
\texttt{plob/postore/release/<opsys>bin/sstoreformat}. It takes a
directory name as argument and tries to initialize that directory as a
stable store directory. If this work, the \postore\ library port was
successfull, otherwise some debugging is necessary. An unsuccessull
compilation is indicated by error messages like `unexpected page
fault', in this case the \texttt{page\us{}fault} procedure and signal
handling in \texttt{svmstore.c} needs some more refinement.

\subsubsection{Step 4: Check RPC capabilities}

The target operating system should support standard SUN RPCs (RPC:
RFC1057, XDR: RFC1014), this is e.g.\ the case for almost all *nix
variants (this mechanism is also sometimes called ONC). If the
operating system does not support standard SUN RPCs, subdirectory
\texttt{plob/oncrpc-1.12} contains an implementation of these RPCs and
could be adapted accordingly. Standard SUN RPC support can be detected
by calling \texttt{rpcgen} in a shell. If that program is found, the
probability of having SUN RPCs on the target machine is quite high.

\subsubsection{Step 5: Adapt plob/conf/make.rules.in}

File \texttt{plob/conf/make.rules.in} contains operating-system
specific rules for \texttt{make}. For the target operating system, the
file suffixes should be checked (see comment `File name suffixes:').
The operating system and compiler specific settings have to be checked
(see comment `OS/compiler-specific settings:'). Add the <opsys>
identifier to file \texttt{plob/conf/make.rules.in}\ to the line
starting with \texttt{ALLOPSYS=}.

\subsubsection{Step 6: Generate Makefiles}

In a shell with current directory \texttt{plob/src}, call \texttt{./mk
  makefiles}. This will (re-)build all makefiles for the target
operating system. If this fails, the rules in file
\texttt{plob/conf/make.rules.in} should be checked. Re-iterate this
step until the makefiles have been build successfully.

\subsubsection{Step 7: Generate binary directories}

In a shell with current directory \texttt{plob/src}, call \texttt{./mk
  initial}. This will create all target system specific subdirectories
for placing binary, target system specific code. It should be
sufficient to do this call only once for the target system.

Copy in the \postore\ library into the \plob\ binary directory.  This
has to be done manually by copying
\texttt{plob/postore/release/<opsys>lib/libpostore.a} to
\texttt{plob/lib/<opsys>/libpostore.a}.

\subsubsection{Step 8: Building \protect\plob}

In a shell with current directory \texttt{plob/src}, call
\texttt{./mk}.  This will create all target system specific binaries.
Some of the source codes contain preprocessor directives for adapting
the \texttt{\#include}s to the different operating systems, this might
need some adjustment. File \texttt{splobadmin.c} also contains some
signal handling code.

\subsubsection{Step 9: Adapt plob/src/lisp/defsystem-plob.lisp}

The last step is to extend \texttt{plob/src/lisp/defsystem-plob.lisp}
by the new supported operating system, mainly to allow defsystem to
map to the correct foreign function libraries to be loaded. See the
comments in \texttt{plob/src/lisp/defsystem-plob.lisp} for further
details.

\subsubsection{Not supported operating systems}

A port to HP-UX is not possible due to a mismatch between the virtual
memory functions needed by \plob's low-level \postore\ library and the
actual implementation of these functions in HP-UX.

\subsection{New LISP System}

This section assumes that the port of the LISP system itself is
already done and running stable. The \plob-specific tasks are
described in the following sections. Another assumption is that
\plob's binaries for layers 3 and below (figure~\ref{sec:PLOBwRPCs})
have been succesfully build already (maybe for another LISP system).
The following skills are needed for a port to a new LISP system:

\begin{itemize}

\item Moderate knowledge of \texttt{make}\ and \texttt{make}\ scripts

\item Moderate knowledge of shell programming, esp.\ Bourne shell.

\item Excellent knowledge of the foreign function interface of the
  target LISP system.

\item Excellent knowledge of the Metaobject Protocol as implemented in
  the target LISP system.

\item Experience and inspiration on debugging self-reflective system
  like \clos\ for debugging \plob's MOP binding.

\end{itemize}

\subsubsection{Step 1: Check out from CVS}

Check out \plob's sources from
\begin{quotation}
\texttt{:pserver:anonymous@cvs.sourceforge.net:/cvsroot/plob}
\end{quotation}
This will create a subdirectory \texttt{plob} containing the
checked-out files.

\subsubsection{Step 2: Select an identifier}

Select an identifier for the target LISP system, e.g. \texttt{cmucl}.
This identifier has to be in lower case. It should contain no suffix
number, since the version number of the LISP system will be added
afterwards.  The identifier has to obey the rules for a single C
token, i.e.\ the syntax is:
\begin{quotation}
<lispsys> ::= \{[a-z]\{[a-z0-9]\}*\}+[a-z]
\end{quotation}
In the following text these conventions are used:
\begin{description}

\item[<Lispsys>] means <lispsys> with its first letter capitalized

\item[<LISPSYS>] means <lispsys> with all of its letters capitalized

\end{description}

\subsubsection{Step 3: Adapt plob/conf/make.rules.in}

File \texttt{plob/conf/make.rules.in} contains target LISP system
specific rules for generating the foreign function interface source
code.  Look at the MkAllegro*Files and MkLispWorks*Files and generate
the corresponding Mk<Lispsys>*Files macros.

\subsubsection{Step 4: Adapt plob/bin/c2lisp.h}

Towards the end of file \texttt{plob/bin/c2lisp.h}, add the target
LISP system specific include file as indicated there (in this example,
for version 3 and 4 of the target LISP system):
\begin{CompactCode}
#if defined(LISPWORKS3)
#include	<c2lispworks3.h>
#elif defined(LISPWORKS4)
#include	<c2lispworks4.h>
#elif defined(ALLEGRO4)
#include	<c2allegro4.h>
#elif defined(ALLEGRO5)
#include	<c2allegro5.h>
#elif defined(ALLEGRO6)
#include	<c2allegro6.h>
#elif defined(ALLEGRO7)
#include	<c2allegro7.h>
\emph{#elif defined(<LISPSYS>3)}
\emph{#include	<c2<lispsys>3.h}
\emph{#elif defined(<LISPSYS>4)}
\emph{#include	<c2<lispsys>4.h>}
#elif defined(another_lisp_system)
#include	<c2another_lisp_system.h>
#else
#error Missing target LISP system.
#endif
\end{CompactCode}

\subsubsection{Step 5: Create plob/bin/c2<lispsys>.h}

Create the file containing the foreign function code generator macros;
base this file on one of the existing \texttt{plob/bin/c2allegro*.h}
or \texttt{plob/bin/c2lispworks*.h} files.

\subsubsection{Step 6: Adapt plob/src/include/makefile.in}

In \texttt{plob/src/include/makefile.in}, the rules for creating the
LISP specific directory containing the foreign function interface code
have to be added.

Extend rule with target \texttt{initial} with a dependency on the
target LISP system directories; extend the existing rule:
\begin{CompactCode}
drule ( initial, \verb|\|
        allegro dash directories lispworks dash directories \verb|\|
        \emph{<lispsys> dash directories}, \verb|\|
        noActions )
\end{CompactCode}

Add a new rule for creating the version specific target LISP system
directories (in this example, for version 3 and 4 of the target LISP
system):
\begin{CompactCode}
rule ( <lispsys> dash directories, \verb|\|
       <lispsys>3 <lispsys>4, \verb|\|
       noActions )
\end{CompactCode}

Add the rule for creating the directories:
\begin{CompactCode}
rule ( <lispsys>3 <lispsys>4, \verb|\|
       noDependencies, \verb|\|
       mkdir ruleTarget )
\end{CompactCode}

Add a preprocessor macro for the new set of foreign function interface
source code files, e.g.
\begin{CompactCode}
/* ----------------------------------------------------------------------
| LISP foreign language interface files: <lispsys>
 ---------------------------------------------------------------------- */
#define	<Lispsys>CodeTarget <lispsys> dash code
\end{CompactCode}

Extend the rule with target \texttt{lisp} with a dependency on the
target LISP foreign function code files; extend the existing rule:
\begin{CompactCode}
rule ( lisp, \verb|\|
       HarlequinCodeTarget AllegroCodeTarget \emph{<Lispsys>CodeTarget}, \verb|\|
       noActions )
\end{CompactCode}

Behind that rule, add the file generating macros defined in
\texttt{plob/conf/make.rules.in} (in this example, for version 3 and 4
of the target LISP system):
\begin{CompactCode}
Mk<Lispsys>3aFiles
Mk<Lispsys>3bFiles
Mk<Lispsys>3cFiles
Mk<Lispsys>3dFiles

Mk<Lispsys>4aFiles
Mk<Lispsys>4bFiles
Mk<Lispsys>4cFiles
Mk<Lispsys>4dFiles
\end{CompactCode}

Add the rules for building the target LISP system foreign function
interface source code (in this example, for version 3 and 4 of the
target LISP system):
\begin{CompactCode}
<LISPSYS>3FILES=\verb|\|
PlobFiles(<lispsys>3 slash,dot lisp)
<LISPSYS>4FILES=\verb|\|
PlobFiles(<lispsys>4 slash,dot lisp)
<LISPSYS>FILES=$(<LISPSYS>3FILES) $(<LISPSYS>4FILES)
rule ( <lispsys> dash code, \verb|\|
       $(<LISPSYS>FILES), \verb|\|
       noActions )
\end{CompactCode}

\subsubsection{Step 7: (Re-)generate foreign function interface sources}

In a shell with current directory \texttt{plob/src/include}, call
\texttt{./mk <lispsys>-code}. This will (re-)generate the foreign function
interface source code for the target LISP systems.

\subsubsection{Step 8: Adapt plob/src/lisp/defsystem-plob.lisp}

Adapt the definitions of parameters
\texttt{+plob-include-lisp-directory+} and
\texttt{+plob-target-lisp-directory+} to capture the new target LISP
system. Load \texttt{defsystem-plob.lisp} and check if the logical
pathname transformations work as expected (e.g.\ by looking at
constant \texttt{+plob-members+}).

\subsubsection{Step 9: Extend plob/src/lisp/ff-mapping.lisp}

The challenging task is to extend
\texttt{plob/src/lisp/ff-mapping.lisp} by the macro definitions suited
for the target LISP system, namely the macros
\texttt{define-foreign-function} and \texttt{define-foreign-callable}.
Macro \texttt{define-foreign-function} has to expand into a definition
which describes to the target LISP system a foreign function to be
called by the target LISP system.  Macro
\texttt{define-foreign-callable} has to expand into a definition which
describes to the target LISP system a foreign function which will call
a LISP function of the target LISP system, e.g. for error callbacks.

\subsubsection{Step 10: System specific code}

LISP system specific code is also to be found in
\texttt{plob-sysdep.lisp} and has to be adapted to the target LISP
system.

\subsubsection{Step 11: Metaobject Protocol}

\plob\ uses the Metaobject Protocol generally as described in
\cite{bib:AMOP} and more specifically as implemented in the target
LISP system (be assured, that \emph{makes} a difference). Both \lwcl\
and \allegrocl\ are relatively close to the definition given in
\cite{bib:AMOP}. Most of the problems with the MOP binding stem from
the non-compatible parts of the implementation (which are often not
documented), often for performance reasons, and from problems about
run-time stability of a self-reflective system as \clos\ actually is.

As a consequence, only some rules of thumb can be given to check the
MOP binding. Assumingly, a new target LISP system might exhibit a
behavior similar to that of one of the currently supported systems, so
it makes sense to look for the conditional statements especially in
files \texttt{plob/src/lisp/plob-clos-slot-value.lisp},
\texttt{plob/src/lisp/plob-clos-descr.lisp} and
\texttt{plob/src/lisp/plob-metaclass-*.lisp}, and read the comments
in these files.

\subsection{New Operating and LISP System}

When considering doing a port both to a new operating system and a new
LISP system, my recommendation is to start with the port of the LISP
system running on an operating system currently supported by \plob,
and do the operating system specific port afterwards. The reason is
that the operating system specific port involves operating system
specific low-level functions (esp.\ virtual memory functions); testing
a new operating system port becomes a lot easier when the tests can be
done from a running application level.

\clearpage\thispagestyle{plain}\cleardoublepage%

%%% Local Variables: 
%%% mode: latex
%%% TeX-master: "userg"
%%% End: 
