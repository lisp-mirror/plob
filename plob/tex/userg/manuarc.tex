% This is -*-LaTeX-*-
%
% userg.tex
% HK 26.5.94
% Rewritten 1997/09

\chapter[Architecture]%
{Architecture of \protect\plob}

For a detailed description of \protect\plob's architecture see
\cite{bib:Kirschke-99}.

In its first version up to September 1996, \plob\ was a single-user
system prepared for multi-user capabilities. At end of 1996, the
system was redesigned into a client/server architecture.
\begin{description}

\item[Client] The client is the LISP process using persistent
  objects. A client is connected to no or one server. At the moment,
  \lwcl\ and \allegrocl\ processes are supported as clients.

\item[Server] The server is a \unix\ process serving requests from the
  clients which connected itselves to the server. A server can handle
  many clients.

\end{description}

Figure~\ref{sec:PLOBwRPCs} shows 
\begin{figure}[htbp]
\centerline{\psfig{figure=\Path/plobrpce.eps}}
\caption{\protect\plob\ layers}%
\label{sec:PLOBwRPCs}
\end{figure}%
the layers of \plob\ as implemented in the actual version.

Based upon an evaluation of a few persistent memory systems done by
\cite{bib:Mueller-91}, it was decided to use the \postore\ library
(Persistent Object Store) as low level store.  The layers 2--5 have
been implemented by the author, layers 2 and 3 in ANSI-C and layer 4
and 5 in \cl. The communication between layers 2 and 3 is done by RPCs
(Remote Procedure Calls), an approved \unix\ standard communication
tool.

\section[POSTORE layer]%
{Layer 1:\ \protect\postore\ layer}

\postore\ is the persistent memory used in Napier88, a persistent
programming language developed at the University of St.\ Andrews.
This layer offers a persistent heap, a concept very similar to a
transient heap, but the `records' allocated from a persistent heap are
persistent. Each \postore\ vector is subdivided into a first part
interpreted as having references to other \postore\ vectors and a
second part which is not interpreted by \postore\ any further; the
second part is used to store the state of value-like objects, that
means, objects not referencing other objects. This is used by
\postore\ to implement a garbage collection based on reachability.

The \postore\ has no usable transaction processing in the sense
of a database. The locking offered by \postore\ is a simple semaphore
locking on a per-object basis.

\section[C-PLOB server layer]%
{Layer 2:\ C-\protect\plob\ server layer}

This layer adds a whole bunch of functionality to the system:
Sessions, object locking and transactions, persistent B-Trees.
Layer 1 and 2 make up the \plob\ server process.

\section[C-PLOB client layer]%
{Layer 3:\ C-\protect\plob\ client layer}

The task of this layer is to interface between the previous and the
next layer. Besides this, it contains the code for connecting the
client with the server process, implements some caching and
pre-allocation of persistent objects etc.  A part of the functionality
of layer 2 is used directly on the client side by layer 3 to reduce
the number of RPC calls to the server, for example, the print
representation for persistent immediates is computed on the client
side. Layer 3 up to layer 5 make up the \plob\ client process.

\section[PLOB layer]%
{Layer 4:\ \protect\plob\ layer}

This layer implements an interface to the application layer for
convenient usage of persistency. Its functionality is documented in
this user's guide.

\section[Application layer]%
{Layer 5:\ Application layer}

The application layer uses the functionality offered by the previous
layer: Persistent packages, persistent symbols, sessions, associative
access by using persistent btrees.

\clearpage\thispagestyle{plain}\cleardoublepage%

%%% Local Variables: 
%%% mode: latex
%%% TeX-master: "userg"
%%% End: 
