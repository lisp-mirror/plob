% This is -*-LaTeX-*-
% trailer.tex
% 1998/01/30 HK

\def\NoOfSides{twoside}%
\def\WideOption{12pt}%
\def\ClassName{article}
\def\thistitle{version \thisversion\ available}
\def\thispartnumber{}
%
% 1998/01/15 HK
%
\ifx\NoOfSides\undefined%
\def\NoOfSides{oneside}%
\fi%
\ifx\WideOption\undefined%
\def\WideOption{narrow}%
\fi%
\ifx\ClassName\undefined
\def\ClassName{pretty}%
\fi%
%
\ifx\Path\undefined%
\def\Path{.}%
\fi%
%
\def\OneSideToken{oneside}
\def\TwoSideToken{twoside}
\def\PrettyToken{pretty}
%
\documentclass[a4paper,\NoOfSides,\WideOption]{\ClassName}%
%
% This is -*-LaTeX-*-
%
 



































% This file was generated from plobversion.h;
% changes done here will be lost!


 








 
 























 







































 





































   \def\thisversion{2.09}            
   \def\thisday{22}            
   \def\thismonth{May}            
   \def\thisyear{2000}            

   \def\thisauthor{Heiko Kirschke}            
   \def\thisemail{Heiko.Kirschke@acm.org}            
   \def\thisurl{www.lisp.de/software/plob/}            

 







 





\author{\thisauthor}%
%
\def\thissubject{Persistent LISP Objects %
\thisversion\space of \thisday.\space\thismonth\space\thisyear}%
%
\def\thiskeywords{persistency,database,persistent objects,LISP,CLOS,MOP}%
%
%\date{6.\ September 1994}%
%\date{30.\ October 1996}%
%\date{18.\ March 1997}%
%\date{14.\ April 1997}%
%\date{9.\ July 1997}%
%\date{5.\ August 1997}%
%\date{27.\ January 1998}%
%\date{6.\ February 1998}%
%\date{9.\ March 1998}%
%\date{12.\ March 1998}% 2.04
%
\def\thisdate{\thisday.\ \thismonth\ \thisyear}%
\date{\thisdate}%
\def\thiswww{http://\thisurl}
%
%
%
\usepackage[english]{babel}
\usepackage{epsfig}
\usepackage{deflist}
\usepackage{avb}
\usepackage{longtable}
\ifx\ClassName\PrettyToken%
\def\usepackagelispdoc{\usepackage{lispdoc}}%
\else%
\let\usepackagelispdoc\relax%
\fi%
\usepackagelispdoc%
\let\usepackagelispdoc\undefined%
\usepackage{crossref}%
\usepackage{dipldefs}
\usepackage{iconpar}
\usepackage{readaux}
\usepackage{cdnamed}
\usepackage{plobbib}
%\usepackage{think}
\usepackage[nothink]{think}
%
\ifx\NoOfSides\TwoSideToken%
%
\def\usepackagetimes{\usepackage{timestt}}%
%
\def\usepackagehyperref{\usepackage[breaklinks,draft]{hyperref}}
%
\let\BetterSloppy\relax
%
\else%
%
\def\usepackagetimes{\usepackage{times}}%
%
\def\usepackagehyperref{\usepackage[%
ps2pdf,%
%pdftex,%
%% breaklinks does not work with ps2pdf:
%breaklinks,%
pdftitle={PLOB \thistitle},%
pdfauthor={\thisauthor},%
pdfsubject={\thissubject},%
pdfkeywords={\thiskeywords}
]{hyperref}}
%
\let\BetterSloppy\sloppy
%
\fi%
%
\usepackagetimes%
\let\usepackagetimes\undefined%
\usepackagehyperref%
\let\usepackagehyperref\undefined%
%
\defattr{objid}{\sl}%
\defattr{typetag}{\sl}%
\defattr{Typetag}{\sl}%
\defattr{TypeTag}{\sl}%
%
\def\figurefontsize{small}%
%
\providecommand\td{\ensuremath{\sim}}% ASCII-Tilde
%

\addtolength\textwidth      {20mm}%
\addtolength\evensidemargin {-20mm}%
\addtolength\textheight     {16.10072mm}%

\begin{document}

\pagestyle{empty}
\thispagestyle{empty}

\title{\ifx\avb\undefined\relax\else\avb\fi%
Persistent LISP Objects\\
\plob\ \thistitle}
\author{\thisauthor\\%
\normalsize Beim Alten Sch\"{u}tzenhof 4 $\bullet$ %
D 22083 Hamburg}%
\maketitle
\pagestyle{empty}
\thispagestyle{empty}

\noindent\plob\ version \thisversion, the general purpose,
object-oriented database for LISP and \clos, will be released in
February 1998; check out
\texttt{{\small\url{http://www.lisp.de/software/plob/}}}\ for details.
It is available for Solaris 2.x, IRIX 6.x and Linux kernel version
2.x. Supported LISP systems are \allegrocl\ 4.3\ and \lwcl\ 3.2.
\plob\ is distributed as free software.
Some new features:
\begin{deflist}[Documentation]

\item[Documentation] The documentation has been revised completely; it
  is offered in PostScript and PDF format, the PDF files being
  hyperlinked.

  A completely new User's Guide is available now, explaining the
  rationale behind \plob, with many example code, hints for
  programming with persistent data etc.

\item[API] The LISP API has been made more stringent.

\item[New classes] New persistent utility classes have been added, for
  example, a class for persistent s-expressions.

\item[Example code] The example code showing much of \plob's
  capabilities has been revised.

\item[Platforms] \plob\ will now run stable on Linux. A port to DEC
  Alpha is in progress.

\end{deflist}

\noindent Attached find an excerpt from \plobwoexcl's User's Guide
  explaining the design of \plobwoexcl.

\cleardoublepage
\end{document}
