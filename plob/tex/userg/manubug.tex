
\chapter{Release Notes}

This section contains the release notes and describes the known bugs
of \plob. Further features can be found in \cite{bib:Kirschke-94a}.

\section{Release Notes}%
\label{sec:ReleaseNotes}

This section describes the changes done between the different releases
of \plob

\subsection{Release 2.11}

Released May 1, 2005.

\begin{itemize}
  
\item Support for  \lwcl\ 4.4 and \allegrocl\ 7 has been
  added.

\item Hot backup possible by server suspending and resume, improved
  \texttt{plobdadmin}\ tool with support for hot backup, additional
  shell escape command

\item Added regular expressions for comparing persistent strings and
  filtering on Btree iterators

\item Added example code in \texttt{plob-dupkeys.lisp}\ for handling
  of duplicate keys in btrees

\item \texttt{plobdadmin} is now statically linked

\item The LISP sources are no longer divided by the different LISP
  systems

\item \lwcl\ fasl files are now written to an extra directory

\end{itemize}

\subsubsection{Improved compare functionality}

The compare functionality used e.g.\ for searching in Btrees has been
improved. Now, keys composed of lists and vectors can be used as long
as their structure is equal.

\subsubsection{Bugfix on transaction rollback}

The transaction log files have been written to wrong when exceeding
the internally used block size of 64K, this has been fixed. All
operating system versions are affected by this bug and its fix.

\subsubsection{Shutdown handling for DLLs on Windows}

The client DLLs on Windows suffered from an incorrect shutdown
handling which resulted in \plob's state being logically terminated
although still being in use. This has been fixed.

\subsubsection{\protect\allegrocl\ 7, \protect\lwcl\ 4.4}

The latest LISP versions are now supported.

\subsubsection{Virtual memory management on Windows}

Although the memory management functions of Unix and Windows are quite
similar, there are subtle differences between these. One difference
leading to an instable behavior of \plob\ on Windows is about what is
considered as free memory by the operating system. To my
understanding, Unix uses a continuous memory region between a minimum
and a maximum address per process.  All memory above the maximum
address is considered as free memory. On Unix, this free memory is
used by \plob\ to map the database file into memory. On Windows, there
is no free memory in this sense, i.e. there is no designated maximum
address. Instead, all memory is administered in blocks, which can be
free or allocated. Before version 2.11, \plob\ used the first of these
free blocks returned by the file mapping call to Windows; especially,
this could be a rather small block spanning only a very small memory
region. This means that when the memory needed became larger, the
block might be used up. In this case, \plob\ signalled errors of the
kind `cannot mmap memory'. In version 2.11, \plob\ now looks for a
free memory block spanning at least 384 MB and uses this block.

This bug and the fix adheres only to the Windows version of \plob, the
other versions are not affected by this behavior.

As a consequence, the overall database size of \plob\ has been reduced
from 1~GB to 384~MB, since the possibility of finding a free 1~GB
memory region on Windows is quite low. Unfortunately, older versions
of Windows databases cannot be opened by version 2.11 any more.

\subsubsection{Testing}

This release of \plob\ was tested on the following platforms.

\begin{tabular}{|l|l|l|l|l|l|}
\hline
\textbf{OS}
        & \textbf{LISP version}
                & \textbf{Date}
                        & \textbf{Tester}
                                & \textbf{Remarks}\\
\hline\hline
Windows
        & \allegrocl\ 7
                & 2005-04
                        & Kirschke
                                & plob-example\\
\hline
Windows
        & \lwcl\ 4.3
                & 2005-04
                        & Kirschke
                                & plob-example\\
\hline
Windows
        & \lwcl\ 4.4
                & 2005-05
                        & Kirschke
                                & plob-example\\
Linux
        & \allegrocl\ 7
                & 2005-04
                        & Kirschke
                                & plob-example\\
\hline
Linux
        & \lwcl\ 4.4
                & 2005-05
                        & Kirschke
                                & plob-example\\
\hline
Solaris
        & \allegrocl\ 7
                & 2005-05
                        & Kirschke
                                & plob-example\\
\hline
\end{tabular}

\subsubsection{Known bugs}

Multithreading access from a single \lwcl\ process on Windows does not
work, it raises runtime exceptions.

\subsection{Release 2.10}

Released May 1, 2002.

\begin{itemize}
  
\item Support for Windows XP, \lwcl\ 4.2 and \allegrocl\ 6.2 has been
  added.
  
\item Further support for IRIX will be on request. Contact
  \href{mailto:\thisemail}{me}\ if you are interested in an IRIX port.
  
\item The sources are now maintained within CVS. Currently, I try to
  set up a CVS archive at \href{http:\thisproject}{SourceForge}\ and a
  possibility for anonymous ftp.

\end{itemize}

\subsubsection{Allegro Common LISP 6.0}

2001-02-06: Release 2.10 supports Allegro Common LISP 6.0.

\subsection{Release 2.09}

Released May 22, 2000.

\subsubsection{Multiple servers}

The code for running multiple servers on a single host has been fixed.
Prior to 2.09, the RPC version number was used to discriminate between
different databases. The ONC RPC documentation says that only for
adjacent version numbers the portmap daemon ensures a correct mapping
to the registered server. Since \plob\ did not obey this rule, the
success of connecting to a database not being the master database
was more or less given by chance.

\subsubsection{LispWorks Common LISP on Linux}

2000-05-22: \lwcl\ 4.1 on Linux is now supported. Although not tested,
\lwcl\ 4.1 on Solaris should work, too. \lwcl\ 4.1 on Linux and
Solaris have the restriction that strings cannot exceed the length of
\lisp{plob::+max-ff-string-length+}.

\subsection{Release 2.08}

Released December 1, 1998.

\subsubsection{64 bit proof}

1998-11-05: The code is now 64 bit proof; the btree interface and code
had to be changed a lot, since this code was written with the
assumption that \texttt{sizeof ( int )}\ \texttt{==}\ \texttt{sizeof (
  void* )}; this has been fixed. A pointer can now have any size.

\subsubsection{Bignums for ACL 5.0}

1998-11-19: Bignums are now handled correctly for ACL 5.0.

\subsubsection{Rewritten plobdadmin utility}

1998-11-25: The \lisp{plobdadmin}\ has been rewritten in C. Its source
code can be found in \lisp{plob-\thisversion/}\lisp{src/}\lisp{admin/}.

\subsubsection{Heterogenous architectures for client and server}

The client/server code has been fixed; arguments are now marshalled
correctly for all client/server combinations of heterogenous
architectures. The `values' section of a postore vector is now
transferred along with a type tag, so that marshalling can select the
right procedure.

\subsection{Release 2.04}

\subsubsection{Parameter *verbose*}

1998-02-06: Setting \fcite{*verbose*}\ to \lispnil\ or 0 will continue
all non-fatal \lisp{cerror}s without any user intervention.

\subsubsection{Windows/NT support}

1998-02-26: The server \lisp{plobd.exe}\ is running under Microsoft
Windows/NT 4.0.  1998-04-15: \plobwoexcl's client code for \lwcl\ 
4.0.1 is running under Windows/NT 4.0. Fixed final release on
1998-05-05.

\subsubsection{Btrees}

1998-03-10: \Fcite{+plob-min-marker+}\ and \fcite{+plob-max-marker+}\ 
refer now to the first and last existing object in a btree, and no
longer to the object `before' or `behind' the first or last
existing object, as it was before. The page size of (empty) btrees can
now be changed.  1998-05-05: Fixed code for inspecting btrees in
\lwcl.  1998-06-25: BTrees can now be iterated in ascending and
descending order. The intervals for a search request can now be
specified as open and closed intervals.

\subsubsection{Built in types representation}

1998-03-10: Type tags are represented as first class objects by
instances of class \textbf{built-in-class}.

\subsubsection{Multithreading}

1998-04-22: The RPC client code for Solaris and Windows/NT is now
multithreading safe. For IRIX and Linux, no multithreading RPC code
can be generated by \lisp{rpcgen}.

\subsubsection{Storing and loading of metaobjects}

1998-05-05: For storing and loading of metaobjects, the
\lisp{*root-persistent-heap*}\ is now used instead of the
\lisp{*default-persistent-heap*}. This decouples transaction handling
for metaobjects from the transaction handling for
non-metaobjects. And, it uses the \lisp{*root-persistent-heap*}\ which
was used before only during the bootstrap.

1998-11-19: This idea was not very good, since it raised lock
conflicts for symbols used for classes and in the user's code.  For
storing and loading of metaobjects, now the
\lisp{*default-persistent-heap*}\ is used again.

\subsection{Release 2.02}

\subsubsection{Documentation}

1997-08 -- 1998-01: Wrote completely new user's guide.

1997-12-01: The manual has been split into two documents, one
documenting the external API (file \url{eref.pdf}) and the other one
documenting the internal entities (file \url{iref.pdf}).

1998-01: All documentation is now available in Portable Document
Format (PDF), too, for improved cross-referencing and text searching.

\subsubsection{Macro with-transaction}

1998-01-28: \Fcite{with-transaction}\ has now an optional argument,
which passes the database where the transaction should be started. For
the moment, always use an empty optional argument:
\begin{IndentedCompactCode}
(with-transaction ()
  \comment{\lt{}forms\gt})
\end{IndentedCompactCode}

\subsubsection{Multiple servers on one machine}

1998-02-04: More than one database can now be served on a single
machine.

Each database directory contains a file \lisp{version.inf}\ with a RPC
version number (1 for the `main server', \gt\ 1 for each database
directory). Each open request from a client to a database is directed
to the `main server' (or any other server running), which looks up the
RPC version number from the RPC version number file, start up a daemon
(if non is running), and return the RPC version number to the client.
This RPC version number is in turn used by the client to create the
connection to the daemon serving the database.  A new started server
process will terminate itself after the last client has disconnected.

\subsubsection{Administration}

1998-02-04: Moved code from script \lisp{plobdadmin}\ into the
server's C code; databases can now be created by calling
\fcite{p-create-database}.

\subsubsection{URL-based database naming}

1998-02-06: The baroq splitting of a database' name into three
components of transport protocol (former parameter
\textbf{*default-server-transport*}), host (former parameter
\textbf{*default-server-host*}) and directory (former parameter
\textbf{*default-server-directory*}) has been removed. Instead, a
database is now named by a single URL (Universal Resource Locator),
see \fcite{url}\ for details. The name of the current database is
found in \fcite{*database-url*}, whereas the default database is in
\fcite{*default-database-url*}; the effective database opened is
formed by merging these both URLs.

\subsection{Release 2.00 and 2.01}

\subsubsection{Macro character for persistent symbols changed}

1997-02: In the first version the macro character \lisp{\#l}\ (hash
sign followed by the letter `l') was used for adressing persistent
symbols.  At the port from \lw\ to \allegro, it was found that the
macro character \lisp{\#l}\ is already occupied by \allegro\ for
reading logical pathnames, so the \lisp{\#!}\ (hash sign followed by
an exclamation mark) macro character is now used for adressing
persistent symbols.

\subsubsection[On-demand loading for slots of CLOS objects]{On-demand
  loading for slots of persistent \protect\clos\ objects}

1997-08: The slots of classes declared with a
\lisp{:metaclass}\hspace{\lispblank}\lisp{persistent-metaclass}\ are
now loaded on-demand.

\subsubsection[Direct `p-...' functions]{Direct \protect\lisp{p-\ldots}\
  functions}

1997-10: The functions for directly working on persistent objects have
been fixed to modify destructively a transient representation, too, if
it is found in the cache.

\subsubsection{Parameter *default-setf-depth*}

1997-10: The parameter \textbf{*default-setf-depth*} was removed;
\fcite{*default-depth*}\ is now used as default depth for both storing
and loading of objects.

\subsubsection[Example code]{Example code in file
  \protect\lisp{plob-example.lisp}}

Fixed 1997-09: The example showing usage of transactions did not work,
since the \lisp{:extent}\ of the example \fcite{person}\ was changed
from \lisp{:persistent}\ to \lisp{:cached-write-through}; so, an
object reload is necessary after a transaction's abort.

\subsubsection{Initialization of transient slots}

Fixed 1997-11: Slots declared \lisp{:transient}\ are now initialized
to their \lisp{:initform}.

\subsubsection{Structure accessors}

Fixed 1997-12-01: In \allegrocl, the structure constructors and
structure slot readers are now retrieved correctly from \allegro's
internal structure descriptions.

\subsubsection{Read-only lock conflict}

Fixed 1997-12-15: Added missing conflict between write-intent and
read-only locks.

\section{Known bugs and limitations}

This section contains known bugs still left to be fixed for the next
release. When one of the following bugs has been fixed in a next
release, it will either be removed completely or be moved to section
\Nameref{sec:ReleaseNotes}.

\subsection{Crash on transfer of large persistent vectors}

There is one bug not associated directly with \plobwoexcl, but with
one of the RPC layers of \unix: When a RPC call requests to transfer
very big data blocks, the server or client may crash because it can't
allocate enough memory. The affected RPC layer is the xdr-layer, doing
the data conversion between the client and the server process. The
xdr-layer is coded (by the provider of the used \unix\ system) rather
straight-forward: It allocates as much memory as it needs and does
\emph{no} blockwise transfer for big data blocks. IMHO, doing a `good'
blockwise transfer would also increase the fault tolerance of the
connection between client and server; a feature which should be
implemented by the RPC layer and \emph{not} the RPC-using application
layer.

In a few words, the consequences for \plob\ are that the server or
client may crash if and only if a very large persistent vector
(`vector' in the sense of \cl) is transferred between client and
server. The concrete length coupled with `very large' depends on the
amount of memory which is available to the xdr-layer, so it is
difficult to name a concrete length here. All which can be said is
that it should be possible to transfer vectors with at least some
thousand elements; for vectors with some tenthousand elements, this
may not be possible.

Since vectors are used in the internal representation of arrays,
this bug may also occure when transferring large arrays. In persistent
structure and \clos\ objects, persistent vectors are used for
representing the instance data contained in their slots too (one
element per slot), but I don't think that there will be persistent
structure or \clos\ objects containing thousands of slots.

\plob\ is already prepared to do a blockwise transfer for
large vectors, but up to now (March 4th, 1997) I had no time to
complete the code for the blockwise transfer.

\subsection{Transferring of large, complicated graph-like structures}

The code for transferring the state of objects between
transient and persistent memory is a recursive-descent algorithm on
each of an object's slots. For large, complicated graph-like
structures this will grow the LISP stack very much; more worse, the
time needed for one object transfer increases with the LISP stack
size, too. `Large' in this sense means some hundred objects in the
transitive closure of an instance to store or load.

It would be a good idea to make something like a `store-plan' or a
`load-plan' for the transfer of all object states referenced by the
transitive closure of an instance to be stored or loaded, for example,
transfer the states of referenced objects before the states of
referencing objects.

\subsection{Cache littering LISP memory}%
\label{sec:CacheLitteringMemory}

Since \plob\ uses a cache internally in the LISP code, this cache may
become very large after some storing or loading of persistent objects.
The cache can be cleared by a call to \fcite{clear-cache}, but this
will also break the bindings of some transient LISP objects to their
persistent identity.

The correct solution to this problem would be using weak pointers in
the hash tables implementing the cache. Unfortunately, \lwcl\ has no
weak pointers at all. \allegrocl\ 4.3 has weak pointers on vectors,
\allegrocl\ 5.0 will support hash tables with weak pointers.

\subsection{Changes to database not propagated to each client's cache}

When one client changes the state of a persistent object, this change
isn't propagated to other clients holding the object's state too in
their cache. The new object's state will be seen by the other
clients on a next reload of the object (this can be forced by
evaluating \lisp{(clear-cache)}\ and \lisp{(load-object
\emph<objid>)}\ in each of the client's listener which should see the
new state).

A solution would be to add a lock mode \lisp{cached}\ to the already
existing lock modes \lisp{read}, \lisp{read-only}, \lisp{write}\ and
the \lisp{\ldots-intent}\ modes. So, when the object's state changes,
each of the clients holding the object in its cache could be notified
of the change.

There would be a siginificant amount of overhead on maintaining
consistency between the \lisp{cache}-locks represented in the
persistent heap on the server side and the cache represented in the
client's LISP process.

Idea: Do the communication between client and server in an own thread
started at least by the client. So, a client waiting for input will do
this in its own thread and won't block the `main' client (LISP)
process.

\subsection[Missing `use-package' feature]{Missing
  \protect\lisp{use-package}\ feature}%
\label{sec:MissingUsePackage}

A \lisp{use-package}\ feature for persistent packages would be
necessary to resolve ambiguities among the names of persistent classes
used across more than one LISP system (for details see remarks at
\fcite{p-find-class}). At the moment, there is no support for a
\lisp{use-package}\ feature for persistent packages.

\subsection[Slot option :extent :persistent]{Slot option
  \protect\lisp{:extent}\protect\lisp{\ % 
    }\protect\lisp{:persistent}}

Limitation for \allegro: The extent \lisp{:persistent}\ will also
allocate transient memory for the slot.  This is a limitation of
\allegro's MOP, since it allows only for \lisp{:instance}\ and
\lisp{:class}\ slot allocations.

For \lw, \lisp{:extent :persistent}\ slots are not represented in
transient memory, since \plob\ patches the class metaobjects of \lw\ 
to have the desired effect.

\subsection[slot-value for classes with :metaclass
:persistent-metaclass]{slot-value for classes with
  \protect\lisp{:metaclass}\protect\lisp{\ %
  }\protect\lisp{persistent-metaclass}}

\plob\ will only gain control over a slot's state for classes with a
class option \lisp{:metaclass persistent-\lb{}metaclass}. Only for
those classes, \fcite{slot-value}\ and \fcite{(setf slot-value)}\ call
their \fcite{slot-value-using-class}\ and \fcite{(setf
  slot-value-using-class)}\ counterparts with methods specialized in
\plobwoexcl.

\subsection{Solaris 2.6}

Use either the newest Solaris' \lisp{cc}\ or \lisp{gcc}\ version 2.8.0
for compiling under Solaris 2.6, \lisp{gcc}\ version 2.7.2 does not
work under Solaris 2.6 at all.

\subsection{Schema evolution}

Schema evolution still seems to be a bit buggy; the LISP code for
schema evolution should be translated to C and moved to the server's
side.

\subsection{Missing compare methods}%
\label{sec:MissingCompareMethods}

The compare methods for ratios and bignums are still missing. For
complex numbers, 2-dimensional indexes are missing.

\subsection{LispWorks 3.2.0}

\plobwoexcl's LISP code will not compile under LispWorks 3.2.0, since
some defsystem features and the foreign function interface have been
changed by Harlequin with 3.2.2. An upgrade to LispWorks 3.2.2 or
4.0.1 is necessary.

\subsection{LispWorks 4.0.1}

With \lwcl\ 4.0.1 for Windows/NT, bit vectors cannot be stored.

\subsection{Function subtypep in ACL 4.3.0}

In Allegro CL 4.3.0, function subtypep may return incorrect results or
raise an error instead of returning a result. If the stack backtrace
in the debugger tells that the error is within function subtypep,
an upgrade to Allegro CL 4.3.1 is necessary or the appropriate
patches should be downloaded from Franz' ftp site.

\section{To do and future features}

This section contains things still left to be done for the next
release. When one of the following topics has been implemented in a
next release, it will be moved to section \Nameref{sec:ReleaseNotes}.

\subsection{Views onto persistent objects}%
\label{sec:Views}

Write function [GS]etSlotValues (replacement for
SH\us{}read\us{}indices(), SH\us{}write\us{}indices()) which takes the
\objid\ of an object and the \objid\ of an `expected class'; coerce
internal physical representation to/from class expected. With
GetSlotValues, a list of added, transient slots should be returned,
too; these slots are then initialized by LISP by a call to
shared-initialize.

In the server, differences between two versions of a class should be
represented by a kind of `Delta-object', which can transform from the
`old' to the `new' physical representation. This would also be useful
for establishing views onto objects. Views are considered as `weak
classes', allowing no durable (but only transient) schema evolution
onto its instances.

On the LISP side, add a slot \lisp{descr}\ to
\fcite{persistent-clos-object}, containing the class description the
object actually has. This class description stored within each
instance can be used to detect if the object should undergo a schema
evolution.  Problem for loose-bounded classes: Where should the
information about the class description go? Perhaps into the cache;
entries into the cache for loose bound \clos\ objects should be a
\lisp{defstruct}\ mapping a transient object onto its objid and actual
class description.

Create bultin class ViewSet, which contains instances belonging to a
single view with equal slot values. Could also be used within btrees
to represent the set of objects belonging to a single, non-unique key.

\subsection{Multidimensional indexes}

Add multidimensional indexes, for example x-trees
\cite{bib:Berchtold-et-al-96,bib:Beckmann-et-al-90}.

\subsection{Multi-threaded server}

With moderate effort, the server could be made multithreaded. This
could be used to reactivate the suspend- and wakeup-algorithms
implemented in \plob\ 1.0 used for clients waiting on locks to be
granted. Also, integrating some event support would be nice.

\subsection{Name spaces}

Add more name spaces, for example a hierarchical name space, similar
to a file system. There is alr

\subsection{Better representation}

Introduce explicit or better representations for databases (these are
not represented at all) and transactions (these are represented as
transaction ids embedded into a session).

%%% Local Variables: 
%%% buffer-file-coding-system: raw-text-unix
%%% mode: latex
%%% TeX-master: "userg"
%%% End: 
