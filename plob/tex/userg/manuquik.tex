% This is -*-LaTeX-*-
%
% manuqik.tex
% 1997/12/04 Heiko Kirschke

\chapter[Quick tour through PLOB]%
{Quick tour through \protect\plob}

This chapter shows in a few example statements how to work with \plob\ 
to make objects persistent. Each of the concepts shown here is
explained in detail in one of the following chapters of the manual;
references to these chapters are given at the end of each section. The
examples assume that \plob\ has been installed as described in
\cite{bib:PLOB-install}. The example code itself is in
\lisp{plob-\thisversion/}\lisp{src/}\lisp{example/}\lisp{plob-quick-tour.lisp}

\section[Starting LISP and loading PLOB]%
{Starting LISP and loading \protect\plob}%
\label{sec:QuickTourStart}

At first, \allegrocl\ is started and the system itself is loaded:
\begin{CompactCode}
/home/kirschke>cd plob-\thisversion/src/allegro
/home/kirschke/plob-\thisversion/src/allegro>cl
Allegro CL 4.3 [Linux/X86; R1] (8/5/97 16:30)
\OmitUnimportant
\listener{}(load "defsystem-plob")
; Fast loading ./defsystem-plob.fasl
T
\listener{}(load-plob)
; Loading system: "PLOB!".
;   Loading product for module: "PLOB:SOURCE;plob-defpackage".
\OmitUnimportant
; Fast loading PLOB:SOURCE;plob-sexpr.fasl
;    (/home/kirschke/plob-\thisversion/src/allegro/plob-sexpr.fasl)
T
\end{CompactCode}

Omissions of unimportant system messages are marked by
`\OmitUnimportant'.

\refpar Section \Nameref{sec:StartClient}; section \fcite{bootstrap
  ...}.

\section{Opening the database}

Next, the database session to \plob\ is opened explicitly by a call to
\fcite{open-my-session}; this step could also be omitted, \plob\ will
open the database when it is needed the first time. The next example
code assumes that the lisp root is already formatted; this is done
automatically the very first time a database is opened.
\begin{CompactCode}
\listener{}(open-my-session)
;;;; Bootstrap   : Opening tcp://nthost/database\marginnumber{\smalloi}
;;; Info from server at executing client:SH_short_read_root:
;;; =======================================================
;;; PLOB! daemon version \thisversion\space{}database version \thisversion\space{}- WIN32
;;; =======================================================
;;; Copyright (C) 1994-\thisyear Heiko Kirschke \thisemail
;;; Part of server code (C) University of St. Andrews
;;; PLOB! comes with ABSOLUTELY NO WARRANTY; for details, look into the
;;; user's guide `plob/ps/userg.ps' provided within the distribution.
;;;; Bootstrap(1): Loaded PLOB::STRUCTURE-DESCRIPTION, 11/13 slots.
\OmitUnimportant
;;;; Bootstrap(4): Loaded PLOB::METHOD-DESCRIPTION, 5/6 slots.
;;;; LISP root formatted at 1997/11/21 16:56 by ALLEGRO
#<heap kirschke@nthost `Initial Lisp Listener' short-objid=50327650>\marginnumber{\smalloii}
\end{CompactCode}

This bootstrap is always done the first time a database is opened; the
bootstrap loads some \mo[s] from the persistent storage into the LISP
image. The first line of output \oi\ shows the database opened; it
uses TCP/IP as transmission protocol (\lisp{tcp:}), the server host is
\lisp{nthost}\ (a remote Windows/NT PC) and the database opened is
located in subdirectory \lisp{database}\ relative to the server's
installation directory. The call returned an object representing the
session to the database \oii; this object is stored in
\fcite{*default-persistent-heap*}\ and will be used as default session
for all database operations.

\refpar Section \Nameref{sec:SessionManagement}

\section[Showing other sessions]%
{Showing other sessions active on the database}

Now let's look who is on the database:
\begin{CompactCode}
\listener{}(show-sessions)
  Sessions on tcp://nthost/plob
* #<heap kirschke@nthost `Initial Lisp Listener'\marginnumber{\smalloi}
         transaction=514 short-objid=50327650>
. #<heap kirschke@nthost `Metaheap' short-objid=50329172>\marginnumber{\smalloii}
\end{CompactCode}

There are 2 sessions started from the current LISP process; the first
one \oi\ is the actual session (marked by a `\lisp{*}'), the other one
\oii\ is a session used by \plob\ internally for storing and loading
of metaobjects.

\refpar Section \Nameref{sec:SessionManagement}

\section{Defining a persistent class}%
\label{sec:QuickTourDefclass}

The most simple way to make \clos\ classes and their instances
persistent is by adding the class option \lisp{(:metaclass
  persistent-metaclass)}:
\begin{CompactCode}
\listener{}(defclass person ()
              ((name :initarg :name :initform nil :accessor person-name)
               (soc-sec-# :initarg :soc-sec-#
                          :accessor person-soc-sec-#
                          :index (btree :test equal)))
              (:metaclass persistent-metaclass))
#<PERSISTENT-METACLASS PERSON>
\end{CompactCode}

This example uses the tight-binding interface of \plob\ for persistent
objects. As will be shown later, persistency is available too for
classes whose definition cannot or should not be changed by adding the
class option \lisp{:metaclass}\ \lisp{persistent-metaclass}. Those
classes use the loose binding interface of \plob, which differs from
the tight-binding interface by the lack of some functionality.

The class itself is not yet persistent; this will be done the first
time an instance of the class is created and stored.

\section{Binding to a persistent symbol}

\begin{CompactCode}
\listener{}(setf #!*cleese*\marginnumber{\smalloi}
              (make-instance 'person :name "Cleese" :soc-sec-# 2000))
;;;;; Stored #<instance class-description person 1.00\marginnumber{\smalloii}
;;;;;                   slots=2/3 short-objid=50327268>.
#<instance person soc-sec-#=2000 short-objid=50326603>\marginnumber{\smalloiii}
\end{CompactCode}

The \cl\ macro reader construct
\lisp{\#!\textrm{\emph{\lt{}symbol\gt}}}\ at \oi\ introduces a
persistent symbol with the same name and package like the transient
\emph{\lt{}symbol\gt}. Persistent symbols are used as `entry points'
into the database, to reference persistent objects by a name. An
object bound to a persistent symbol is always reachable. Therefore,
the symbol's value is persistent, too, and will never be removed by a
garbage collection. The persistent symbol is a symbol on its own; the
only relation to its transient companion is that they share the same
name and package name.

The output \oii\ shows that a class description for class
\class{person}\ is stored in the database; the name of the class is
\lisp{person}, its version number is \lisp{1.00}; it has 3 slots, 2
are persistent slots and 1 is a transient slot (the last one is added
silently by \plob\ at establishing the \clsmo\ representing class
\class{person}). For the slot named \lisp{soc-sec-\#}, an index will
be maintained automatically which maps a value to the instance having
that slot value.

The call shown here \oiii\ returned the now persistent object which
will be bound to the persistent symbol \lisp{*cleese*}. For slots with
a declared index, the slot name and its value will be shown in the
print representation.

\refpar Section \Nameref{sec:TightBinding};
section \Nameref{sec:SlotIndex};
section \Nameref{sec:PersistentSymbol}.

\section{Making persistent instances}

Now, some more persistent instances are created:
\begin{CompactCode}
\listener{}(make-instance 'person :name "Palin" :soc-sec-# 3000)
#<instance person soc-sec-#=3000 short-objid=50326605>
\listener{}(make-instance 'person :name "Gilliam" :soc-sec-# 4000)
#<instance person soc-sec-#=4000 short-objid=50326607>
\end{CompactCode}

Besides using persistent symbols, this example shows the second way to
ensure reachability for a persistent object. Indexes are defined as
being reachable, so the index declared on slot \lisp{soc-sec-\#}\ will
ensure reachability for all instances of class \class{person},
although not every individual \class{person}\ instance may be bound to
a persistent symbol.

\refpar Section \Nameref{sec:SlotIndexAndReachability}.

\section{Doing simple queries}

Get the instance of class \class{person}\ whose slot value of slot
\lisp{soc-sec-\#}\ equals to 2000:
\begin{CompactCode}
\listener{}(p-select 'person :where 'soc-sec-# := 2000)
#<instance person soc-sec-#=2000 short-objid=50326603>
\listener{}(person-name *)
"Cleese"
\end{CompactCode}

This is the person named \lisp{"Cleese"}\ which has been stored above
as the very first instance of class \textbf{person}. The query
accessed the index declared for slot \lisp{soc-sec-\#}.

\refpar Section \Nameref{sec:IndexQuery}.

\section{Defining a print method}

Since the \objid\ is rather poor for identifying the person instance,
a print method will be defined:
\begin{CompactCode}
\listener{}(defmethod print-object ((obj person) stream)
               (print-unreadable-object (obj stream :type t :identity t)
                 (format stream "~A (short-objid ~A)"
                         (if (slot-boundp obj 'name) (person-name obj))
                         (persistent-object-objid obj))))
\end{CompactCode}

Now, the output of the simple query will look somewhat better:
\begin{CompactCode}
\listener{}(p-select 'person :where 'soc-sec-# := 2000)
#<PERSON Cleese (short-objid=50326603) @ #xa064f02>
\end{CompactCode}

Unfortunately, the method itself cannot be stored in the database,
since \plob\ does not support storing of binary function code; section
\Nameref{sec:BinaryCode} explaines the workarounds which are
implemented in \plob\ to cope with function code.

\section{Leaving LISP}

When leaving LISP, all open sessions to the database are closed
automatically. A close of all sessions can be forced by calling
\fcite{close-heap}.
\begin{CompactCode}
\listener{}:exit
cplobheap.c(196): fnDeinitializeHeapModule:
  PLOB! is trying to close 2 pending session(s), please wait ... done!
\end{CompactCode}

\refpar Section \Nameref{sec:FinishClient}

\section[Restarting the system]%
{Restarting the system; reload of classes}

\ResetListener After a restart of the LISP system and a reload of
\plob\ (as described in section \Nameref{sec:QuickTourStart}), loading
the value of a persistent symbol will re-establish the class of the
object found in the value cell of the persistent symbol into the
current LISP image:
\begin{CompactCode}
/home/kirschke/plob-\thisversion/src/allegro>cl
\OmitUnimportant
\listener{}(load "defsystem-plob")
\OmitUnimportant
\listener{}(load-plob)
\OmitUnimportant
\listener{}#!*cleese*
;;;; Bootstrap   : Opening tcp://nthost/database\marginnumber{\smalloi}
\OmitUnimportant
;;;; LISP root formatted at 1997/11/21 16:56 by ALLEGRO
;;;;; Compiling #<instance class-description person 1.00\marginnumber{\smalloii}
;;;;;                      slots=2/3 short-objid=50327268>
#<instance person soc-sec-#=2000 short-objid=50326603>\marginnumber{\smalloiii}
\end{CompactCode}

The output \oi\ shows that the database is opened implicit at loading
the persistent symbol, and that a class description has been compiled
into the current LISP image \oii. This compilation will re-establish
the class as it has been specified within its original definition as
given with the \lisp{defclass}\ statement in section
\Nameref{sec:QuickTourDefclass}; all superclasses, slots, slot
initargs, slot initforms, slot accessors etc.\ will be re-established,
too. The return value \oiii\ is the value of the persistent symbol
\lisp{*cleese*}:
\begin{CompactCode}
\listener{}(person-name *)
"Cleese"
\end{CompactCode}

\section{Doing a schema evolution}

Adding, removing or changing a slot of a class definition will result
in a modification of the persisten class description:
\begin{CompactCode}
\listener{}(defclass person ()
              ((name :initarg :name :initform nil :accessor person-name)
               (soc-sec-# :initarg :soc-sec-#
                          :accessor person-soc-sec-#
                          :index (btree :test equal))
               \comment{;; Adding slot age:}
               (age :initarg :age :initform nil :accessor person-age))
              (:metaclass persistent-metaclass))
#<PERSISTENT-METACLASS PERSON>
\end{CompactCode}

Next time an instance of class \textbf{person} is loaded or stored,
the new class description will be stored within the database, and the
instance loaded or stored will be updated to this new class
description. In the following example, a reload of the top-level
state of the object bound to the persistent symbol \lisp{*cleese*}\ is
done to enforce the new class description being stored to the
database:
\begin{CompactCode}
\listener{}#!(*cleese* :flat)\marginnumber{\smalloi}
Error: The persistent definition for class
PERSON does not match its transient counterpart.
       Reason: total number of slots; tr.: 4 / pe.: 3\marginnumber{\smalloii}
  [condition type: SIMPLE-ERROR]

Restart actions (select using :continue):
 0: Store the transient definition to the persistent store.
 1: Replace the transient definition by its persistent counterpart.
 2: Ignore definition mismatch (might result in LISP runtime errors).
 3: Show the transient class definition.
 4: Show the persistent class definition.
[1] \listener{}:cont
;;;;; Updated #<instance class-description person 1.01
;;;;;                    slots=3/4 short-objid=50327096>.
#<instance person soc-sec-#=2000 short-objid=50326603>
\end{CompactCode}

The evaluation \oi\ forces a reload of the top-level state of the
object bound to the persistent symbol (this is done by the
\lisp{:flat}\ argument given to the persistent symbol reader). An error
is raised \oii\ which shows the reason for the schema evolution
(the number of slots has been increased to 4 slots).

\refpar Section \Nameref{sec:SchemaEvolution}.

\section{Using transactions}

Each read and write access to the database is guarded by a transaction
(and locking). Transactions are used for ensuring consistent changes
to the database: Either all or none of the changes done to objects
states during a transaction are stored in the database. They can be
used explicit by embedding the code which should do such a consistent
change into \fcite{with-transaction}:
\begin{CompactCode}
\listener{}(with-transaction ()
              (setf (person-age #!*cleese*) 52)
              \comment{;; \lt{}some other code in between\gt}
              (setf (person-soc-sec-# #!*cleese*) 2001))
\end{CompactCode}

This call will ensure that either both slots are changed or none of
them: Executing the whole \lisp{(with-transaction ...)}\ block will
result in the changes written to the database. Each jump out of the
\lisp{(with-transaction ...)}\ block will result in a transaction's
abort, which will reset the slot states to their value before the
block was entered. This jump out-of-the-block happens for example when
\emph{\lt{}some other code in between\gt} raises a runtime error.
\footnote{For now, a transaction's abort will only affect the
  persistent representation of the object; its transient
  representation remains unchanged and may therefore contain some new
  state.}

\refpar Section \Nameref{sec:Transaction}.

\section{Defining and using btrees}

Persistent btrees are used internally for many purposes (for example,
for maintaining the persistent package and symbol tables, and for
indexes), so it was decided to make them available to the user API.
Btrees are similar to hash tables, with the difference that their keys
are sorted. Similar to hash tables, they come with two (not three)
test modes \lisp{eq}\ and \lisp{equal}. A test mode of \lisp{eq}\ 
means that the btree's keys are sorted by identity, whereas a test
mode of \lisp{equal}\ will sort the keys accroding to their state.
\begin{CompactCode}
\listener{}(setf #!*btree* (make-btree :test 'equal))
#<btree equal 0/0 short-objid=50327317>
\listener{}(setf (getbtree "Cleese" #!*btree*) #!*cleese*)
#<instance person soc-sec-#=2000 short-objid=50326603>
\listener{}(setf (getbtree "myself" #!*btree*) "Heiko Kirschke")
"Heiko Kirschke"
\listener{}(p-apropos-btree #!*btree*)
"Clesse", data: #<instance person short-objid=50326603>
"myself", data: #<simple-string `Heiko Kirschke' short-objid=50327186>
\end{CompactCode}

\refpar Section \Nameref{sec:PersistentBTree}.

\section{Server log file}

The server reports important actions, errors and information messages
to a log file containing plain text. The name of the file is
\lisp{messages.log}. This file is located on the server host in the
database directory directly below the database root directory
specified at \plobwoexcl's installation. So, when encountering any
problems with running \plob, a look into the server's log file might
help.  A typical excerpt looks like this:
\begin{CompactCode}
1998/01/07 10:41:18 kirschke@nthost\marginnumber{\smalloi}
       splob.c(462): fnServerInitializePlob:
       Started ./plobd\marginnumber{\smalloii}
1998/01/07 10:41:26 #<heap kirschke@nthost `Metaheap'
                                           short-objid=50328853>
       splob.c(1926): fnSHopen (#'c-sh-open):
       Client kirschke@nthost, now 1 client(s):\marginnumber{\smalloiii}
       Opening stable heap `plob' succeeded.
       Reason: Administrator login
1998/01/07 10:41:46 #<heap kirschke@nthost `Metaheap'
                                           short-objid=50328853>
       splob.c(1190): fnSHclose (#'c-sh-close):
       Closed stable heap `plob',\marginnumber{\smalloiv}
       now 0 client(s).
       Garbage collection resulted in 4317 - 1524 = 2793 objects.
\end{CompactCode}

At \oi, each message shows the date and time it was generated and the
current client which was responsible for the raised message. The
source code location and C function which raised the message is shown,
too, and the message itself \oii. All physical database
openings are reported \oiii; when the last client disconnects, a
garbage collection is triggered, which writes the number of deleted
objecs into the log file \oiv.

%\begin{CompactCode}
%\listener{}
%\end{CompactCode}

%%% Local Variables: 
%%% mode: latex
%%% TeX-master: "userg"
%%% End: 
