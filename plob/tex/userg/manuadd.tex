% This is -*-LaTeX-*-
%
% userg.tex
% HK 26.5.94
% Rewritten 1997/09

\chapter{Database functionality}

This chapter describes concepts, functions and classes of \plob\ whose
transient counterparts are either not found in \cite{bib:CLtLII}\ at
all because they represent certain database features not needed by a
transient LISP system or whose meaning in \plob\ has been adopted for
the needs in a persistent system.

\section{Databases, sessions, transactions and locks}

This section describes some terms and design issues of \plob\ 
concerning databases, sessions, transactions and locks from a
programmer's point of view, and the relations between them.

\subsection{Database}

A database contains persistent objects. At the moment (1998/08/27), a
\plob\ client can have only one database open at a
time.\footnote{A single \plob\ server can handle any (reasonable)
  number of clients.} It is a global resource shared among all LISP
lightweight processes running in a LISP image. A database is opened
the very first time when the first `session' is established.

\subsubsection{Multiple Databases}

Working with multiple databases would mean to open the first database,
do something with it, close the first database, open the second
database, etc. To be more concrete, for true working with multiple
databases, the processing would look like:
\begin{IndentedCompactCode}
(progn

  \comment{;; Open the session; since the database is currently closed,}
  \comment{;; this implies opening the database referenced by the passed URL:}
  (open-my-session "tcp://localhost/database1")
  \comment{;; Do something:}
  (do-something-with-database-1)
  \comment{;; This closes all sessions and the database:}
  (close-heap)

  \comment{;; Open the session; since the database is currently closed,}
  \comment{;; this implies opening the database referenced by the passed URL:}
  (open-my-session "tcp://localhost/database2")
  \comment{;; Do something:}
  (do-something-with-database-2)
  \comment{;; This closes all sessions and the database:}
  (close-heap))
\end{IndentedCompactCode}

For historical reasons (because of its global character), the
representation of a database is a bit unclear/smeared in \plobwoexcl.
     
\subsection{Session}%
\label{sec:SessionManagement}

A session is used as an `access organizer' or `access helper' to a
database. A session organizes access to the persistent objects
contained in the database by ensuring that each access is within an
active transaction, that a read access to a persistent object in the
database is done within a read lock and a write access within a write
lock. By default, each LISP lightweight process (thread) gets its own
session when the thread does its first access to a persistent
object. All sessions of a running LISP image are bound to the same
underlying database.

The session associated to the current thread is contained in
\fcite{*default-persistent-heap*}. There is one global session which
is used at the bootstrap and for storing metaobject information; this
one is contained in \fcite{*root-persistent-heap*}.

Sessions are represented as persistent objects of
\fcite{persistent-heap}.

\refpar \Fcite{open-my-session}\ and \fcite{*database-url*},
\fcite{*default-database-url*}; \fcite{close-my-session},
\fcite{show-sessions}, \fcite{p-sessions}, \fcite{with-session}.

\subsection{Transaction}%
\label{sec:Transaction}

A transaction is some state-changing operation to the underlying
database. The transactions implemented in \plob\ are two phased
transactions. This transaction model ensures that each single
transaction has isolated access to persistent objects; essentially, it
means that a transaction is always divided into two phases, the first
phase where locks are established and the second phase which starts
with the first lock being released. From the user's side, the second
phase is never visible, since it is embedded into the transaction's
commit processing.

Figure~\ref{fig:TransactionSTDiagram} shows
\begin{figure}[htbp]
%% \centerline{\psfig{figure=\Path/sttrans.eps}}
\centerline{\psfig{figure=\Path/sttrans.pdf}}
\caption{S/T diagram of a transaction}%
\label{fig:TransactionSTDiagram}
\end{figure}%
the possible states of a transaction; only in the active state, a
transaction is allowed to change the database's state.  According to
the isolation table given in \cite[\citepage{399}]{bib:Gray-et-al-93},
\plob\ implements an isolation of 3$^{\circ}$. Since starting and
stopping a transaction might have side effects in the highest
application layer, transactions are always started and ended by the
highest application layer.

A transaction could also be seen as a mean to ensure that the database
is in a consistent state before the transaction has been started and
after it has been finished. Here, consistency is defined in terms
purely of the application programmer's point of view, and involves no
other constraints.

For now, a transaction is bound exactly to one session. An active
transaction in a session is for example shown in its print output;
with no active transaction on the default session, the value of
\fcite{*default-persistent-heap*}\ prints like:
\begin{CompactCode}
\listener{}*default-persistent-heap*
#<heap kirschke@localhost `Initial Lisp Listener' short-objid=133987850>
\end{CompactCode}

Within an active transaction, it prints like:
\begin{CompactCode}
\listener{}(begin-transaction *default-persistent-heap*)
179048
\listener{}*default-persistent-heap*
#<heap kirschke@localhost `Initial Lisp Listener' transaction=179048
                                                  short-objid=133987850>
\listener{}(in-transaction-p *default-persistent-heap*)
179048
\listener{}(end-transaction *default-persistent-heap*)
179048
\listener{}*default-persistent-heap*
#<heap kirschke@localhost `Initial Lisp Listener' short-objid=133987850>
\listener{}(in-transaction-p *default-persistent-heap*)
NIL
\end{CompactCode}

Since transactions are bound to exactly one session, a transaction is
not represented explicit but implicit by a session containing a
transaction id.

\refpar \Fcite{with-transaction}, \fcite{begin-transaction},
\fcite{end-transaction}, \fcite{cancel-transaction}.

\subsubsection{Aborting a transaction}%
\label{sec:TransactionAbort}

At the moment, aborting a transaction will clear the cache to enforce
a reload of the persistent objects into transient representations,
with all of its consequences (section \Nameref{sec:IdAndCache}).
Future versions of \plob\ will support a more elaborate handling of
transaction aborts.

\subsection{Lock}%
\label{sec:Locking}

A lock ensures that there is no conflicting access to a persistent
object. In conjunction with transactions, locking is an implementation
of the concept of \hyperlink{link:Isolation}{isolation}.  The locking
mechanism implemented is hierarchical or tree locking, a specialized
version of predicate locking \cite[\citepage{406}]{bib:Gray-et-al-93}.
Locking is possible on the levels \textbf{store} (lock all persistent
objects), \textbf{vector} (lock one persistent object) and
\textbf{element} (lock one slot of a persistent object) with the lock
modes \textbf{read-only}, \textbf{read} or \textbf{write}. The locking
protocol is a pessimistic one: An object must be locked before its
state can be accessed. All locking is done implicit during accessing
the state of a persistent object.

All locks must be set within an active transaction; they are bound to
the transaction. When ending the transaction (either committing or
aborting), all locks set in the scope of the transaction are removed.
Before any access can be done to a persistent object, a read or write
lock must be granted to the modifying transaction. If a lock cannot be
granted, a lock conflict occurs. A lock conflict is given by the
following rules:

\begin{enumerate}

\item Conflicts between locks set by different transactions are given
  by the conflict matrix shown in table~\ref{tab:lockconfl}. In
  essence, this matrix says that read locks are shared among a locked
  object, whereas a write lock requires no other lock set onto the
  object.

\item A lock which should be set by a transaction never conflicts with
  other locks already set by the same transaction. For example, if a
  transaction requested a write lock for a persistent object and the
  same transaction requests later a read lock, these locks do not
  conflict.

\end{enumerate}

\begin{figure}[htbp]%
\centering%
\def\coliw{3em}%
\def\pbox#1{\parbox{\coliw}{\centering#1}}%
\def\vvbox{\pbox{\checked}}%
\def\XXbox{\pbox{\crossed}}%
\begin{\figurefontsize}%
\begin{tabular}{%
|c|p{\coliw}|p{\coliw}|p{\coliw}|p{\coliw}|p{\coliw}|p{\coliw}|}
\cline{2-7}
\multicolumn{1}{c|}{\parbox{8em}{%
 \hspace*{\fill}Already granted$\rightarrow$\\[\smallskipamount]%
 Requested$\downarrow$\hspace*{\fill}}}
 &\pbox{\tabularheader{RO-Intent}}
  &\pbox{\tabularheader{Read-Only}}
   &\pbox{\tabularheader{Read-Intent}}
    &\pbox{\tabularheader{Read}}
     &\pbox{\tabularheader{Write-Intent}}
      &\pbox{\tabularheader{Write}}\\ \hline
%
%        ROI     RO      RI      R       WI      W
\tabularheader{Read-Only-Intent}
        &\vvbox &\vvbox &\vvbox &\vvbox &\vvbox &\vvbox\\ \hline
\tabularheader{Read-Only}
        &\vvbox &\vvbox &\vvbox &\vvbox &\vvbox &\vvbox\\ \hline
\tabularheader{Read-Intent}
        &\vvbox &\vvbox &\vvbox &\vvbox &\vvbox &\XXbox\\ \hline
\tabularheader{Read}
        &\vvbox &\vvbox &\vvbox &\vvbox &\XXbox &\XXbox\\ \hline
\tabularheader{Write-Intent}
        &\vvbox &\vvbox &\vvbox &\XXbox &\vvbox &\XXbox\\ \hline
\tabularheader{Write}
        &\vvbox &\XXbox &\XXbox &\XXbox &\XXbox &\XXbox\\ \hline
%
\multicolumn{7}{r}{%
\checked\ $\equiv$\ Compatible locks\quad
\crossed\ $\equiv$\ Conflicting locks}
\end{tabular}%
\end{\figurefontsize}%

\let\figurename\tablename%
\caption{Conflictmatrix for locks}%
\label{tab:lockconfl}%
\end{figure}%

A conflict is `resolved' by the client requesting the conflicting lock
to wait until either the other lock involved in the conflict is
removed (for example, because another transaction holding this lock
has terminated) or a timeout is encountered (this indicates a maybe
not solvable conflict, like a deadlock, and will raise an error).

The functions \fcite*{write-lock-store}\ and \fcite*{read-lock-store}\ 
can be used to place a write or read lock onto the whole database.
This is usefull when the whole database should be written or read, see
section \Nameref{sec:LotsOfData} for an example.

Locks are represented as persistent objects. They will almost never be
seen directly by the database programmer, since most of the locking is
done automatically.

\refpar \Fcite{write-lock-store}, \fcite{read-lock-store}; section
\fcite{locking ...}.

\section{Making objects reachable}%
\label{sec:ObtainingReachability}

In figure~\ref{fig:ReachableObjects},
\begin{figure}[htbp]
%% \centerline{\psfig{figure=\Path/gctrpe.eps}}
\centerline{\psfig{figure=\Path/gctrpe.pdf}}
\caption{Reachable objects}%
\label{fig:ReachableObjects}
\end{figure}%
objects \oi\ -- \oiv\ are reachable from the designated persistent
root object and will not be garbage collected.  Objects \ov\ and \ovi\ 
are only reachable from a transient variable and are not reachable
from the persistent root,\marginlabel{References from transient
  variables to persistent objects are not regarded by \plobwoexcl.}
since the persistent system does not know anything about references
from transient objects to persistent objects. In consequence, objects
\ov\ and \ovi\ would be garbage-collected. The problem with references
from (transient) variables to persistent objects is solved two-fold in
\plobwoexcl:
\begin{enumerate}

\item A garbage collection is triggered\marginlabel{Offline garbage
    collection.} if and only if there are no references from transient
  to persistent objects. This is the case when there is no (transient)
  client connected to the persistent object system.  To be more
  specific, a garbage collection is triggered in the server after the
  last client has disconnected itself.

\item \plob\ offers an interface to persistent
  symbols,\marginlabel{Persistent symbols with persistent values.}
  which are defined as being reachable, both the symbol itself and its
  value.  Another advantage is that an object bound to a persistent
  symbol can be located in the persistent object system by the
  symbol's name.

\end{enumerate}
Additionally, declaring an index on a persistent class using the tight
binding will ensure reachability for the class' instances (section
\Nameref{sec:SlotIndexAndReachability}).

\section{Persistent symbols}%
\label{sec:PersistentSymbol}

As with transient symbols, persistent symbols serve for naming
persistent objects.\footnote{Named objects are found in ODMG
  \cite[\citepage{242}, \textbf{bind()} method]{bib:ODMG-2}, too, but
  ODMG does not cope with symbols, only with simple names.} Persistent
symbols always have a global scope, there is no mean to establish a
local scope for a persistent symbol. For convenience, the construct
\lisp{\#!}\emph{\lt{}symbol\gt}\ is established by \plob\ to refer to
the persistent \emph{\lt{}symbol\gt} having the same name and package
as the transient \emph{\lt{}symbol\gt}.
\begin{IndentedCompactCode}
\listener{}(setf #!*foo* 3)\marginnumber{\smalloi}
3
\listener{}#!*foo*
3
\listener{}*foo*\marginnumber{\smalloii}
Error: Attempt to take the value of the unbound variable `*FOO*'.
  [condition type: UNBOUND-VARIABLE]
\OmitUnimportant
\listener{}#!*bar*\marginnumber{\smalloiii}
Error: Cannot locate a persistent symbol named *BAR*.
  [condition type: SIMPLE-ERROR]
\OmitUnimportant
\listener{}(p-intern '*bar*)\marginnumber{\smalloiv}
50327204        \comment{;; This is the numeric objid of the created symbol}
NIL             \comment{;; NIL means it is a new created persistent symbol}
\listener{}#!*bar*\marginnumber{\smallov}
#<unbound-marker=0x104>
\end{IndentedCompactCode}

The value of a persistent symbol can be set with \fcite{setf}\ \oi.
The value of the transient symbol will neither be set
nor be changed \oii; only the name and package of the transient symbol
are used to reference the persistent symbol's name and package.
Opposed to transient symbols, a read access to a non-existing
persistent symbol will not result in the persistent symbol being
automatically inserted into the database's symbol table, but will
raise an error \oiii.  This is done to prevent the database from
getting littered with a lot of automatic inserted and maybe otherwise
unused symbols.  Creating a new, unbound persistent symbol can be done
by calling \fcite{p-intern}\ \oiv. Referencing an unbound persistent
symbol will not raise an error, but will return a special marker
instance representing an unbound persistent object \ov.

The complete reference chain to a persistent symbol is:
\begin{quote}
  Persistent root $\rightarrow$ package table
  $\stackrel{*}{\rightarrow}$ package $\rightarrow$ list of internal
  symbols $\stackrel{*}{\rightarrow}$ symbol
\end{quote}

\refpar[for persistent symbols]
\begin{RefList}\itemsep0pt

  \Allocator \Fcite{p-allocate-symbol}

  \Creator \Fcite{p-setq}

  \TypePredicate \Fcite{p-symbolp}

  \ObjectStore \Fcite{(setf p-symbol)}

  \ObjectLoad \Fcite{p-symbol}

  \SlotReader \Fcite{p-symbol-function}, \fcite{p-symbol-name},
  \fcite{p-symbol-package}, \fcite{p-symbol-plist},
  \fcite{p-symbol-value}

  \SlotWriter \Fcite{(setf p-symbol-function)}, \fcite{(setf
    p-symbol-plist)}, \fcite{(setf p-symbol-value)},
  \fcite{p-fmakunbound}, \fcite{p-makunbound}.

  Once established, the name of a persistent symbol can not be
  changed, so there is no writer for a symbol's name.

  \Information \Fcite{p-boundp}, \fcite{p-fboundp}

\item[Administration] \Fcite{p-find-symbol}, \fcite{p-intern},
  \fcite{p-unintern}

\end{RefList}

\section{Persistent packages}

Persistent packages serve as containers for persistent symbols; all
persistent packages are reachable and are therefore never garbage
collected. At the moment, all persistent symbols belonging to a
persistent package are internal persistent symbols.  If the same
database is used by different LISP systems (for example, by \lwcl\ and
\allegrocl\ at the same time), there is a need for exporting symbols
and a persistent use-package list, see \fcite{p-find-class}\ and
section \Nameref{sec:MissingUsePackage}.

\refpar \Fcite{p-find-package}, \fcite{p-delete-package},
\fcite{p-package-internals}, \fcite{p-apropos-packages},
\fcite{p-package}, \fcite{p-package-name},
\fcite{p-package-externals}.

\section{Markers}%
\label{sec:Markers}

\plob\ has an addtional class \class{marker}.  Its nature and semantic
shows some resemblance to \cl's condition classes, with the difference
that it does not have an associated error handling protocol.  Markers
are used for representing certain conditions, which are not
necessarily error conditions.
\begin{description}

\item[Unbound slot] The marker object contained in
  \fcite{+plob-unbound-marker+}\ is used for slots which are not bound
  to a value: A slot which references such a marker is considered as
  being unbound.

\item[Unstorable object] If \plob\ encounteres objects it can not
  store into the database, the instance in
  \fcite{+plob-unstorable-object-marker+}\ is stored instead of the
  unstorable object itself. Since \plob\ is almost type-complete, this
  marker is used very rare, mainly when attempting to store binary
  function code.

\item[Extremal value] The marker object contained in
  \fcite{+plob-min-marker+}/\fcite{+plob-max-marker+}\ represent a
  generic object with its state being equal to the state of the
  minimum/maximum object in a set. For a set of persistent numeric
  values, they serve as a representation for the minimum/maximum
  number contained in the set. For a set with persistent strings, they
  represent the first/last string in this [ASCII-]ordered set.

  For example, let's take the \lisp{*c*}\ btree from
  \Nameref{sec:HierarchicalKeys}. The data associated to the
  minimum/maximum key stored in btree \lisp{*c*}\ can be retrieved by:
  \begin{CompactCode}
\listener{}(getbtree +plob-min-marker+ *c*)
"string (1 1)."
\listener{}(getbtree +plob-max-marker+ *c*)
"string (4d0 3)."
  \end{CompactCode}
  The corresponding key values themselves can be requested by a call
  to \fcite{btree-minkey}/\fcite{btree-maxkey}:
  \begin{CompactCode}
\listener{}(btree-minkey *c*)
#(1 1)
\listener{}(btree-maxkey *c*)
#(4d0 3)
  \end{CompactCode}

\end{description}
Often, those conditions are represented by LISP implementators by
using singular objects, like a string or a symbol for each marker
instance.  The concept of a marker class represents those conditions
explicit by instances of a disjunctive class. Markers are
represented as immediates, and so do not take up much space.

\refpar \Fcite{+plob-min-marker+}, \fcite{+plob-max-marker+},
\fcite{+plob-unbound-marker+},
\fcite{+plob-unstorable-object-marker+}.

\section{Persistent btrees}%
\label{sec:PersistentBTree}

A btree is similar to the well-known hash table: It associates a value
to a key. The difference to a hash table is that the keys are
1-dimensional sorted. Two kinds of sortings can be used:
\begin{description}

\item[By identity] This will sort the keys according to their
  \objid[s]. Since each persistent object has an \objid, all
  persistent objects regardless of their class can be used as key
  objects in a single btree.

\item[By state] This will sort the keys according to their state. This
  implies for key objects to be inserted, that they must be able to
  compare their state with the state of all keys which are already
  stored in the btree. For the most important persistent classes used
  with btrees, compare methods have been implemented.

\end{description}
The kind of sorting is selected with the \keyarg{test}\ argument
passed to \fcite{make-btree}. A \keyarg{test}\ argument of \lisp{eq}\ 
will sort the keys by identity, and a \keyarg{test}\ argument of
\lisp{equal}\ will sort the keys by state.

\refpar \Fcite{persistent-btree}\ and \fcite{make-btree},
\fcite{getbtree}, \fcite{clrbtree}, \fcite{rembtree},
\fcite{btree-count}, \fcite{btree-size},
\fcite{btree-test}, \fcite{p-apropos-btree}.

\subsection{Btrees sorted by state}

For an \lisp{equal}\ btree, the type(s) of keys it can contain is
fixed with the first object inserted into the btree.
\begin{IndentedCompactCode}
\listener{}(setf *b* (make-btree :test 'equal))
#<btree equal 0/0=0*678 short-objid=50327519>
\listener{}(setf (getbtree 1 *b*) "string 1.")\marginnumber{\smalloi}
"string 1."
\listener{}(setf (getbtree 2.0 *b*) "string 2.0.")\marginnumber{\smalloii}
"string 2.0."
\listener{}(setf (getbtree "three" *b*) "string three.")\marginnumber{\smalloiii}
Error: From server at executing client:SH_btree_insert_by_string:
       splobbtree.c(1153): fnKeyCmp:
       Illegal search key "three" for
       object #<btree equal 2/678=1*678 short-objid=50327519>:
       compare failed with
       object 1.
  [condition type: POSTORE-ERROR]
\end{IndentedCompactCode}

For example, if the first object inserted into an \lisp{equal}\ btree
is a number \oi, only numbers can be inserted afterwards \oii; trying
to insert for example a string will raise an error \oiii.

\subsubsection{Hierarchical keys}%
\label{sec:HierarchicalKeys}

Using a non-atomic object as a key in an \lisp{equal}\ btree has a
special meaning: \plob\ will try to insert the key according to an
element-wise comparision to the slots of the keys already found in the
btree. This way, a non-atomic key object is a compound key, with its
first slot being the primary key, the second slot being the secondary
key and so on:
\begin{IndentedCompactCode}
\listener{}(setf *c* (make-btree :test 'equal))
#<btree equal 0/0=0*678 short-objid=50327520>
\listener{}(setf (getbtree #(1 1) *c*) "string (1 1).")
"string (1 1)."
\listener{}(setf (getbtree #(1 2) *c*) "string (1 2).")
"string (1 2)."
\listener{}(setf (getbtree #(1 3.0) *c*) "string (1 3.0).")
"string (1 3.0)."
\listener{}(setf (getbtree #(4d0 3) *c*) "string (4d0 3).")
"string (4d0 3)."
\listener{}(p-apropos-btree *c*)
#(1 1), data: #<simple-string `string (1 1).' short-objid=50327507>
#(1 2), data: #<simple-string `string (1 2).' short-objid=50327504>
#(1 3.0), data: #<simple-string `string (1 3.0).' short-objid=50327502>
#(4.0d0 3), data: #<simple-string `string (4d0 3).' short-objid=50327369>
\end{IndentedCompactCode}

\subsection{Mapping btrees}

\Fcite{mapbtree}\ can be used to map all elements of a btree.
\begin{IndentedCompactCode}
\listener{}(mapbtree #'(lambda (key data)\marginnumber{\smalloi}
                          (format t "key ~A, data ~A~%" key data) t)\marginnumber{\smalloii}
                       *b*)
key 1, data string 1.
key 2.0, data string 2.0.
2\marginnumber{\smalloiii}
\listener{}(mapbtree #'(lambda (key data)
                          (format t "key ~A, data ~A~%" key data) t)
                       *b* :descending t)\marginnumber{\smalloiv}
key 2.0, data string 2.0.
key 1, data string 1.
2\marginnumber{\smalloiii}
\listener{}(mapbtree #'(lambda (key data)\marginnumber{\smallov}
                          (format t "key ~A, data ~A~%" key data) t)
                       *b* :>= 2)
key 2.0, data string 2.0.
1\marginnumber{\smalloiii}
\end{IndentedCompactCode}

Since the keys are ordered, the map function is called by default with
the keys in ascending order \oi.  It is important that the map
function returns a \nonnil\ value if the mapping should continue with
the next element \oii. The call to \fcite{mapbtree}\ returns the
number of mapped btree elements \oiii.  If the \keyarg{:descending}\ 
argument is passed with a \nonnil\ value, the key interval is iterated
in descending order \oiv.  It is possible to restrict the mapping to
an interval of key values \ov. The lower key to map can be passed with
a \keyarg{:>}\ or \keyarg{:>=}\ argument, the upper key to map with a
\keyarg{:<}\ or \keyarg{:<=}\ argument.

It is allowed to delete the key currently passed to the map function
from the btree.

\refpar \Fcite{mapbtree}

\subsection{Cursors on btrees}

An access interface similar to cursors in SQL is available for btrees.
Instances of \fcite{persistent-btree-mapper}\ can be used for
accessing a btree by positioning. For example, create a btree with
some elements in it.
\begin{IndentedCompactCode}
\listener{}(setf *btree* (make-btree :test 'equal))
#<btree equal 0/0=0*678 short-objid=50327518>
\listener{}(dotimes (i 1000) (setf (getbtree i *btree*)
                                   (format nil "String ~A" i)))
\end{IndentedCompactCode}

Now, the btree is accessed by an instance of
\fcite{persistent-btree-mapper}.
\begin{IndentedCompactCode}
\listener{}(setf *mapper* (make-btree-mapper *btree*))\marginnumber{\smalloi}
\listener{}(btree-mapper-seek *mapper* 1)\marginnumber{\smalloii}
1
0
"String 0"
\listener{}(btree-mapper-seek *mapper* 1)\marginnumber{\smalloii}
1
1
"String 1"
\listener{}(btree-mapper-seek *mapper* -1 :end)\marginnumber{\smalloiii}
1
998
"String 998"
\end{IndentedCompactCode}
A mapper is established by calling the constructor of the mapper with
the btree as its argument which should be iterated \oi; further
optional arguments can specify a search interval for the mapper. With
a call to \fcite{btree-mapper-seek}, the next or previous element is
returned. Positioning can be done relative to the start or end of the
current search interval \smalloiii.

\refpar \Fcite{make-btree-mapper}, \fcite{btree-mapper-search},
\fcite{btree-mapper-seek}, \fcite{(setf btree-mapper-seek)}

\subsection{Setting the page size}

The internal used page size of a btree can be set either by passing
the key argument \keyarg{pagesize}\ along with the number of objects
each btree page should hold, or by calling \fcite{(setf
  btree-pagesize)}. For very small or very big btrees, a smaller
or larger page size than the default page size should be used. To
be more exact, the page size' optimum is fulfilled for
\begin{displaymath}\begin{array}{rcl}
%
n^{n} &\approx& o\\[\smallskipamount]
%
n &\equiv& \textrm{page size}\\
o &\equiv& \textrm{total number of objects in btree}
\end{array}\end{displaymath}

\refpar \Fcite{make-btree}, \fcite{(setf btree-pagesize)},
\fcite{btree-pagesize}.

\subsection{Requesting extremal values}

The data object associated to the minimum or maximum key object
contained in a btree can be requested by passing the special marker
objects \fcite{+plob-min-marker+}\ and \fcite*{+plob-max-marker+}\ to
\fcite{getbtree}. This way, it is also possible to modify the data
objects associated to the minimum or maximum key by passing the
marker object to \fcite*{(setf getbtree)}.  The minimum or maximum
key object itself is returned by a call to \fcite{btree-minkey}\ 
or \fcite{btree-maxkey}.

\refpar \Fcite{+plob-min-marker+}, \fcite{+plob-max-marker+};
\Fcite{btree-minkey}, \fcite{btree-maxkey}.

\subsection{Inspecting btrees}

In \lwcl, a btree's contents can be viewed with the built in \lw\
inspector. Simply start an inspector on a btree object to look at its
contents. To see the structure of the btree correctly, in the
inspector's `View' menu the entry `Unsorted' should be selected.

\subsubsection[Inspecting the index associated to a slot]{Example:
  Inspecting the index associated to a slot} 

For example, store some thousand persons by evaluating
\lisp{(store-n-random-people 2000)}\ (found in
\lisp{plob-\thisversion/}\lisp{src/}\lisp{example/}\lisp{plob-example.lisp}.
Evaluate \lisp{(p-find-class 'person)}, this will return as first
value the \clsdo\ of class \class{person}. Inspect this first value.
Next, inspect the vector labelled \lisp{PLOB::P-EFFECTIVE-SLOTS}.
Next, inspect the instance marked as being the
effective-slot-description for slot \textbf{soc-sec-\#}. In this \sltdo,
the index defined for slot \textbf{soc-sec-\#}\ of class \class{person}\ 
can be found within the line labelled \lisp{PLOB::P-INDEX}. Inspect
this instance; this will show the btree which is used for representing
the index declared on slot \textbf{soc-sec-\#}\ within class
\class{person}.

\section{Regular expressions}%
\label{sec:RegEx}

Regular expressions are builtin objects in \plob\ and can be used as a
filter in iterating on btrees.

\begin{CompactCode}
\listener{}(setf *o* (p-make-regex "Heik[eo]"))
#<regex `Heik[eo]' short-objid=50328180>
\listener{}(setf *c* (make-btree :test 'equal))
#<btree equal 0/0=0*678 short-objid=50328176>
\listener{}(setf (getbtree "Heiko" *c*) "Kirschke")
"Kirschke"
\listener{}(setf (getbtree "Heike" *c*) "Pflugradt")
"Pflugradt"
\listener{}(setf (getbtree "Nicola" *c*) "Kirschke")
"Kirschke"
\listener{}(mapbtree #'(lambda (key data)
                          (format t "key ~A, data ~A~%" key data) t)
                     *c* :filter *o*)
key Heike, data Pflugradt
key Heiko, data Kirschke
2
\end{CompactCode}

The regular expressions can only be used in iterating queries, not in
searching queries.  The reason is that there is no direct search
criteria implemented for regular expression on btress, instead a
brute-force search is done, applying the regular expression as a
filter on all keys read from the btree. For this reason, setting the
minimum and maximum search keys is important, since this could reduce
search time.

\plob\ uses the regex code of Henry Spencer as e.g. also used in
cygwin and FreeBSD. Regular expressions are used only for matching,
i.e.\ the code for locating subexpressions has not been interfaced to
LISP. In the following sections the manpages of regex have been added
and adopted to its usage in \plob

\subsection{regex(3)}%
\label{sec:manre3}

\subsubsection{NAME}

\Fcite{p-make-regex}, \fcite{p-compile-regex}\ - regular-expression library

\subsubsection{SYNOPSIS}

%% #include <sys/types.h>
%% #include <regex.h>

%%     int regcomp(regex_t *preg, const char *pattern, int cflags); 
%%     int regexec(const regex_t *preg, const char *string, size_t nmatch, regmatch_t pmatch[], int eflags); 
%%     size_t regerror(int errcode, const regex_t *preg, char *errbuf, size_t errbuf_size); 
%%     void regfree(regex_t *preg);

\begin{CompactCode}
p-make-regex
  pattern
  &key
  ;; regcomp() flags
  (basic nil) (extended t) (icase nil) (nosub t) (newline nil)
  (nospec nil) (notmatching nil)
  ;; regexec() flags
  (notbol nil) (noteol nil)
  (trace nil) (large nil) (backref nil)
  (p-heap *default-persistent-heap*)

p-compile-regex
  objid
  &optional (p-heap *default-persistent-heap*)
\end{CompactCode}
 
\subsubsection{DESCRIPTION}

These routines implement POSIX 1003.2 regular expressions (``RE''s);
see \Nameref{sec:manre7}. 

\Fcite{p-make-regex}\ creates a persistent regular expression,
\fcite{p-compile-regex}\ compiles the regular expression.

%% Regcomp compiles an RE written as a string
%% into an internal form, regexec matches that internal form against a
%% string and reports results, regerror transforms error codes from
%% either into human-readable messages, and regfree frees any
%% dynamically-allocated storage used by the internal form of an RE.

%% The header <regex.h> declares two structure types, regex_t and
%% regmatch_t, the former for compiled internal forms and the latter for
%% match reporting. It also declares the four functions, a type regoff_t,
%% and a number of constants with names starting with ``REG_''.

%% Regcomp compiles the regular expression contained in the pattern
%% string, subject to the flags in cflags, and places the results in the
%% regex\verb|_|t structure pointed to by preg. Cflags is the bitwise OR of zero
%% or more of the following flags:

At compiling, the following flags passed in at a call to
\fcite{p-make-regex} are used:

\begin{description}

\item[\keyarg{basic}] %% REG_BASIC
  This is a synonym for 0, provided as a counterpart to \keyarg{extended}
  to improve readability. This is an extension, compatible with but
  not specified by POSIX 1003.2, and should be used with caution in
  software intended to be portable to other systems.
\item[\keyarg{extended}] %% REG_EXTENDED
  Compile modern (``extended'') REs, rather than the obsolete
  (``basic'') REs that are the default.
\item[\keyarg{icase}] %% REG_ICASE
  Compile for matching that ignores upper/lower case distinctions. See
  \Nameref{sec:manre7}.
\item[\keyarg{nosub}] %% REG_NOSUB
  Compile for matching that need only report success or failure, not
  what was matched.
\item[\keyarg{newline}] %% REG_NEWLINE
  Compile for newline-sensitive matching. By default, newline is a
  completely ordinary character with no special meaning in either REs
  or strings. With this flag, `[\verb|^|' bracket expressions and `.'
  never match newline, a `\verb|^|' anchor matches the null string
  after any newline in the string in addition to its normal function,
  and the `\verb|$|' anchor matches the null string before any newline
  in the string in addition to its normal function.
\item[\keyarg{nospec}] %% REG_NOSPEC
  Compile with recognition of all special characters turned off. All
  characters are thus considered ordinary, so the ``RE'' is a literal
  string. This is an extension, compatible with but not specified by
  POSIX 1003.2, and should be used with caution in software intended
  to be portable to other systems. \keyarg{extended}\ and
  \keyarg{nospec}\ may not be used in the same call to regcomp.
\item[\keyarg{notmatching}]
  Signal a match for strings not matching the pattern.

%% REG_PEND

%%   The regular expression ends, not at the first NUL, but just before
%%   the character pointed to by the re_endp member of the structure
%%   pointed to by preg. The re_endp member is of type const char *. This
%%   flag permits inclusion of NULs in the RE; they are considered
%%   ordinary characters. This is an extension, compatible with but not
%%   specified by POSIX 1003.2, and should be used with caution in
%%   software intended to be portable to other systems.

\end{description}

%% When successful, regcomp returns 0 and fills in the structure pointed
%% to by preg. One member of that structure (other than re\verb|_|endp)
%% is publicized: re\verb|_|nsub, of type size\verb|_|t, contains the
%% number of parenthesized subexpressions within the RE (except that the
%% value of this member is undefined if the \keyarg{nosub} flag was used).
%% If regcomp fails, it returns a non-zero error code; see
%% \Nameref{sec:mre3diag}.

The regular expression can be used directly as a first argument
to \fcite{p-compare}. At matching, the following flags passed in
to the call to \fcite{p-make-regex}\ are used:

%% Regexec matches the compiled RE pointed to by preg against the string,
%% subject to the flags in eflags, and reports results using nmatch,
%% pmatch, and the returned value. The RE must have been compiled by a
%% previous invocation of regcomp. The compiled form is not altered
%% during execution of regexec, so a single compiled RE can be used
%% simultaneously by multiple threads.

%% By default, the NUL-terminated string pointed to by string is
%% considered to be the text of an entire line, with the NUL indicating
%% the end of the line. (That is, any other end-of-line marker is
%% considered to have been removed and replaced by the NUL.) The eflags
%% argument is the bitwise OR of zero or more of the following flags:

\begin{description}

\item[\keyarg{notbol}] %% REG_NOTBOL
  The first character of the string is not the beginning of a line, so
  the `\verb|^|' anchor should not match before it. This does not
  affect the behavior of newlines under \keyarg{newline}.

\item[\keyarg{noteol}] %% REG_NOTEOL
  The NUL terminating the string does not end a line, so the
  `\verb|$|' anchor should not match before it. This does not affect
  the behavior of newlines under \keyarg{newline}.

%% \item[startend] %% REG_STARTEND
%%   The string is considered to start at string + pmatch[0].rm_so and to
%%   have a terminating NUL located at string + pmatch[0].rm_eo (there
%%   need not actually be a NUL at that location), regardless of the
%%   value of nmatch. See below for the definition of pmatch and nmatch.
%%   This is an extension, compatible with but not specified by POSIX
%%   1003.2, and should be used with caution in software intended to be
%%   portable to other systems. Note that a non-zero rm_so does not imply
%%   REG_NOTBOL; REG_STARTEND affects only the location of the string,
%%   not how it is matched.

\item[\keyarg{trace}]
  Tracing of execution.

\item[\keyarg{large}]
  Force large representation.

\item[\keyarg{backref}]
  Force use of backref code.

\end{description}

See \Nameref{sec:manre7} for a discussion of what is matched in
situations where an RE or a portion thereof could match any of several
substrings of string.

See \fcite{p-compare}\ for a desrciption of its return values.

%% Normally, regexec returns 0 for success and the non-zero code
%% \keyarg{nomatch} for failure. Other non-zero error codes may be returned in
%% exceptional situations; see \Nameref{sec:mre3diag}.

%% If \keyarg{nosub} was specified in the compilation of the RE, or if nmatch
%% is 0, regexec ignores the pmatch argument (but see below for the case
%% where \keyarg{startend} is specified). Otherwise, pmatch points to an array
%% of nmatch structures of type regmatch_t. Such a structure has at least
%% the members rm_so and rm_eo, both of type regoff_t (a signed
%% arithmetic type at least as large as an off_t and a ssize_t),
%% containing respectively the offset of the first character of a
%% substring and the offset of the first character after the end of the
%% substring. Offsets are measured from the beginning of the string
%% argument given to regexec. An empty substring is denoted by equal
%% offsets, both indicating the character following the empty substring.

%% The 0th member of the pmatch array is filled in to indicate what
%% substring of string was matched by the entire RE. Remaining members
%% report what substring was matched by parenthesized subexpressions
%% within the RE; member i reports subexpression i, with subexpressions
%% counted (starting at 1) by the order of their opening parentheses in
%% the RE, left to right. Unused entries in the array-corresponding
%% either to subexpressions that did not participate in the match at all,
%% or to subexpressions that do not exist in the RE (that is, i >
%% preg->re_nsub)-have both rm_so and rm_eo set to -1. If a subexpression
%% participated in the match several times, the reported substring is the
%% last one it matched. (Note, as an example in particular, that when the
%% RE `(b*)+' matches `bbb', the parenthesized subexpression matches the
%% three `b's and then an infinite number of empty strings following the
%% last `b', so the reported substring is one of the empties.)

%% If REG_STARTEND is specified, pmatch must point to at least one
%% regmatch_t (even if nmatch is 0 or REG_NOSUB was specified), to hold
%% the input offsets for REG_STARTEND. Use for output is still entirely
%% controlled by nmatch; if nmatch is 0 or REG_NOSUB was specified, the
%% value of pmatch[0] will not be changed by a successful regexec.

%% Regerror maps a non-zero errcode from either regcomp or regexec to a
%% human-readable, printable message. If preg is non-NULL, the error code
%% should have arisen from use of the regex\verb|_|t pointed to by preg, and if
%% the error code came from regcomp, it should have been the result from
%% the most recent regcomp using that regex\verb|_|t. (Regerror may be able to
%% supply a more detailed message using information from the regex\verb|_|t.)
%% Regerror places the NUL-terminated message into the buffer pointed to
%% by errbuf, limiting the length (including the NUL) to at most
%% errbuf\verb|_|size bytes. If the whole message won't fit, as much of it as
%% will fit before the terminating NUL is supplied. In any case, the
%% returned value is the size of buffer needed to hold the whole message
%% (including terminating NUL). If errbuf\verb|_|size is 0, errbuf is ignored
%% but the return value is still correct.

%% If the errcode given to regerror is first ORed with REG_ITOA, the
%% ``message'' that results is the printable name of the error code, e.g.
%% ``REG_NOMATCH'', rather than an explanation thereof. If errcode is
%% REG_ATOI, then preg shall be non-NULL and the re_endp member of the
%% structure it points to must point to the printable name of an error
%% code; in this case, the result in errbuf is the decimal digits of the
%% numeric value of the error code (0 if the name is not recognized).
%% REG_ITOA and REG_ATOI are intended primarily as debugging facilities;
%% they are extensions, compatible with but not specified by POSIX
%% 1003.2, and should be used with caution in software intended to be
%% portable to other systems. Be warned also that they are considered
%% experimental and changes are possible.

%% Regfree frees any dynamically-allocated storage associated with the
%% compiled RE pointed to by preg. The remaining regex\verb|_|t is no
%% longer a valid compiled RE and the effect of supplying it to regexec
%% or regerror is undefined.

None of these functions references global variables except for tables
of constants; all are safe for use from multiple threads if the
arguments are safe.

\subsubsection{IMPLEMENTATION CHOICES}

There are a number of decisions that 1003.2 leaves up to the
implementor, either by explicitly saying ``undefined'' or by virtue of
them being forbidden by the RE grammar. This implementation treats
them as follows.

See \Nameref{sec:manre7} for a discussion of the definition of
case-independent matching.

There is no particular limit on the length of REs, except insofar as
memory is limited. Memory usage is approximately linear in RE size,
and largely insensitive to RE complexity, except for bounded
repetitions. See \Nameref{sec:mre3bugs} for one short RE using them
that will run almost any system out of memory.

A backslashed character other than one specifically given a magic
meaning by 1003.2 (such magic meanings occur only in obsolete
[``basic''] REs) is taken as an ordinary character.

Any unmatched [ is a \texttt{+regex-ebrack+} error.

Equivalence classes cannot begin or end bracket-expression ranges. The
endpoint of one range cannot begin another.

RE\verb|_|DUP\verb|_|MAX, the limit on repetition counts in bounded
repetitions, is 255.

A repetition operator (?, *, +, or bounds) cannot follow another
repetition operator. A repetition operator cannot begin an expression
or subexpression or follow `\verb|^|' or `|'.

`|' cannot appear first or last in a (sub)expression or after another
`|', i.e. an operand of `|' cannot be an empty subexpression. An empty
parenthesized subexpression, `()', is legal and matches an empty
(sub)string. An empty string is not a legal RE.

A `\verb|{|' followed by a digit is considered the beginning of bounds
for a bounded repetition, which must then follow the syntax for
bounds. A `\verb|{|' not followed by a digit is considered an
ordinary character.

`\verb|^|' and `\verb|$|' beginning and ending subexpressions in
obsolete (``basic'') REs are anchors, not ordinary characters.

\subsubsection{SEE ALSO}

grep(1), \Nameref{sec:manre7}

POSIX 1003.2, sections 2.8 (Regular Expression Notation) and B.5 (C
Binding for Regular Expression Matching).

\subsubsection{DIAGNOSTICS}%
\label{sec:mre3diag}

Non-zero error codes from regcomp and regexec include the following:

\begin{tabular}{ll}
\texttt{+regex-nomatch+} & regexec() failed to match\\
\texttt{+regex-badpat+} &  invalid regular expression\\
\texttt{+regex-ecollate+} & invalid collating element\\
\texttt{+regex-ectype+} &  invalid character class\\
\texttt{+regex-eescape+} & \ applied to unescapable character\\
\texttt{+regex-esubreg+} & invalid backreference number\\
\texttt{+regex-ebrack+} & brackets [ ] not balanced\\
\texttt{+regex-eparen+} & parentheses ( ) not balanced\\
\texttt{+regex-ebrace+} & braces \verb|{| \verb|}| not balanced\\
\texttt{+regex-badbr+} & invalid repetition count(s) in \verb|{| \verb|}|\\
\texttt{+regex-erange+} & invalid character range in [ ]\\
\texttt{+regex-espace+} & ran out of memory\\
\texttt{+regex-badrpt+} & ?, *, or + operand invalid\\
\texttt{+regex-empty+} & empty (sub)expression\\
\texttt{+regex-assert+} &``can't happen''-you found a bug\\
\texttt{+regex-invarg+} & invalid argument, e.g. negative-length string
\end{tabular}
 
\subsubsection{HISTORY}

Written by Henry Spencer, formerly \url{mailto:henry@zoo.toronto.edu}.

\subsubsection{BUGS}%
\label{sec:mre3bugs}

This is an alpha release with known defects. Please report problems.

There is one known functionality bug. The implementation of
internationalization is incomplete: the locale is always assumed to be
the default one of 1003.2, and only the collating elements etc. of
that locale are available.

The back-reference code is subtle and doubts linger about its
correctness in complex cases.

Regexec performance is poor. This will improve with later releases.
Nmatch exceeding 0 is expensive; nmatch exceeding 1 is worse. Regexec
is largely insensitive to RE complexity except that back references
are massively expensive. RE length does matter; in particular, there
is a strong speed bonus for keeping RE length under about 30
characters, with most special characters counting roughly double.

Regcomp implements bounded repetitions by macro expansion, which is
costly in time and space if counts are large or bounded repetitions
are nested. An RE like, say,
`((((a\verb|{|1,100\verb|}|)\verb|{|1,100\verb|}|)\verb|{|1,100\verb|}|)\verb|{|1,100\verb|}|)\verb|{|1,100\verb|}|'
will (eventually) run almost any existing machine out of swap space.

There are suspected problems with response to obscure error
conditions. Notably, certain kinds of internal overflow, produced only
by truly enormous REs or by multiply nested bounded repetitions, are
probably not handled well.

Due to a mistake in 1003.2, things like `a)b' are legal REs because
`)' is a special character only in the presence of a previous
unmatched `('. This can't be fixed until the spec is fixed.

The standard's definition of back references is vague. For example,
does `a\verb|\|(\verb|\|(b\verb|\|)*\verb|\|2\verb|\|)*d' match
`abbbd'? Until the standard is clarified, behavior in such cases
should not be relied on.

The implementation of word-boundary matching is a bit of a kludge, and
bugs may lurk in combinations of word-boundary matching and anchoring.

\subsection{regex(7)}%
\label{sec:manre7}

\subsubsection{NAME}

regex - POSIX 1003.2 regular expressions  

\subsubsection{DESCRIPTION}

Regular expressions (``RE''s), as defined in POSIX 1003.2, come in two
forms: modern REs (roughly those of egrep; 1003.2 calls these
``extended'' REs) and obsolete REs (roughly those of ed; 1003.2
``basic'' REs). Obsolete REs mostly exist for backward compatibility
in some old programs; they will be discussed at the end. 1003.2 leaves
some aspects of RE syntax and semantics open; `+' marks decisions on
these aspects that may not be fully portable to other 1003.2
implementations.

A (modern) RE is one+ or more non-empty+ branches, separated by `|'.
It matches anything that matches one of the branches.

A branch is one+ or more pieces, concatenated. It matches a match for
the first, followed by a match for the second, etc.

A piece is an atom possibly followed by a single+ `*', `+', `?', or
bound. An atom followed by `*' matches a sequence of 0 or more matches
of the atom. An atom followed by `+' matches a sequence of 1 or more
matches of the atom. An atom followed by `?' matches a sequence of 0
or 1 matches of the atom.

A bound is `\verb|{|' followed by an unsigned decimal integer,
possibly followed by `,' possibly followed by another unsigned
decimal integer, always followed by `\verb|}|'. The integers must
lie between 0 and RE\verb|_|DUP\verb|_|MAX (255+) inclusive, and if
there are two of them, the first may not exceed the second. An atom
followed by a bound containing one integer i and no comma matches a
sequence of exactly i matches of the atom. An atom followed by a bound
containing one integer i and a comma matches a sequence of i or more
matches of the atom. An atom followed by a bound containing two
integers i and j matches a sequence of i through j (inclusive) matches
of the atom.

An atom is a regular expression enclosed in `()' (matching a match for
the regular expression), an empty set of `()' (matching the null
string)+, a bracket expression (see below), `.' (matching any single
character), `\verb|^|' (matching the null string at the beginning of a
line), `\verb|$|' (matching the null string at the end of a line), a
`\verb|\|' followed by one of the characters
`\verb|^|.[\verb|$|()|*+?\verb|{|\verb|\|' (matching that character
taken as an ordinary character), a `\verb|\|' followed by any other
character+ (matching that character taken as an ordinary character,
as if the `\verb|\|' had not been present+), or a single character
with no other significance (matching that character). A `\verb|{|'
followed by a character other than a digit is an ordinary
character, not the beginning of a bound+. It is illegal to end an
RE with `\verb|\|'.

A bracket expression is a list of characters enclosed in `[]'. It
normally matches any single character from the list (but see below).
If the list begins with `\verb|^|', it matches any single character
(but see below) not from the rest of the list. If two characters in
the list are separated by `-', this is shorthand for the full range of
characters between those two (inclusive) in the collating sequence,
e.g. `[0-9]' in ASCII matches any decimal digit. It is illegal+ for
two ranges to share an endpoint, e.g.  `a-c-e'. Ranges are very
collating-sequence-dependent, and portable programs should avoid
relying on them.

To include a literal `]' in the list, make it the first character
(following a possible `\verb|^|'). To include a literal `-', make it
the first or last character, or the second endpoint of a range. To use
a literal `-' as the first endpoint of a range, enclose it in `[.' and
`.]' to make it a collating element (see below). With the exception of
these and some combinations using `[' (see next paragraphs), all other
special characters, including `\verb|\|', lose their special significance
within a bracket expression.

Within a bracket expression, a collating element (a character, a
multi-character sequence that collates as if it were a single
character, or a collating-sequence name for either) enclosed in `[.'
and `.]' stands for the sequence of characters of that collating
element. The sequence is a single element of the bracket expression's
list. A bracket expression containing a multi-character collating
element can thus match more than one character, e.g. if the collating
sequence includes a `ch' collating element, then the RE `[[.ch.]]*c'
matches the first five characters of `chchcc'.

Within a bracket expression, a collating element enclosed in `[=' and
`=]' is an equivalence class, standing for the sequences of characters
of all collating elements equivalent to that one, including itself.
(If there are no other equivalent collating elements, the treatment is
as if the enclosing delimiters were `[.' and `.]'.) For example, if o
and are the members of an equivalence class, then `[[=o=]]', `[[==]]',
and `[o]' are all synonymous. An equivalence class may not+ be an
endpoint of a range.

Within a bracket expression, the name of a character class enclosed in
`[:' and `:]' stands for the list of all characters belonging to that
class. Standard character class names are:

\begin{itemize}

\item alnum digitpunct
\item alpha graphspace
\item blank lowerupper
\item cntrl printxdigit

\end{itemize}

These stand for the character classes defined in ctype(3). A locale
may provide others. A character class may not be used as an endpoint
of a range.

There are two special cases+ of bracket expressions: the bracket
expressions `[[:$<$:]]' and `[[:$>$:]]' match the null string at the
beginning and end of a word respectively. A word is defined as a
sequence of word characters which is neither preceded nor followed by
word characters. A word character is an alnum character (as defined by
ctype(3)) or an underscore. This is an extension, compatible with but
not specified by POSIX 1003.2, and should be used with caution in
software intended to be portable to other systems.

In the event that an RE could match more than one substring of a given
string, the RE matches the one starting earliest in the string. If the
RE could match more than one substring starting at that point, it
matches the longest. Subexpressions also match the longest possible
substrings, subject to the constraint that the whole match be as long
as possible, with subexpressions starting earlier in the RE taking
priority over ones starting later. Note that higher-level
subexpressions thus take priority over their lower-level component
subexpressions.

Match lengths are measured in characters, not collating elements. A
null string is considered longer than no match at all. For example,
`bb*' matches the three middle characters of `abbbc',
`(wee|week)(knights|nights)' matches all ten characters of
`weeknights', when `(.*).*' is matched against `abc' the parenthesized
subexpression matches all three characters, and when `(a*)*' is
matched against `bc' both the whole RE and the parenthesized
subexpression match the null string.

If case-independent matching is specified, the effect is much as if
all case distinctions had vanished from the alphabet. When an
alphabetic that exists in multiple cases appears as an ordinary
character outside a bracket expression, it is effectively transformed
into a bracket expression containing both cases, e.g. `x' becomes
`[xX]'. When it appears inside a bracket expression, all case
counterparts of it are added to the bracket expression, so that (e.g.)
`[x]' becomes `[xX]' and `[\verb|^|x]' becomes `[\verb|^|xX]'.

No particular limit is imposed on the length of REs+. Programs
intended to be portable should not employ REs longer than 256 bytes,
as an implementation can refuse to accept such REs and remain
POSIX-compliant.

Obsolete (``basic'') regular expressions differ in several respects.
`|', `+', and `?' are ordinary characters and there is no equivalent
for their functionality. The delimiters for bounds are `\{' and `\}',
with `{' and `}' by themselves ordinary characters. The parentheses
for nested subexpressions are `\(' and `\)', with `(' and `)' by
themselves ordinary characters. `\verb|^|' is an ordinary character
except at the beginning of the RE or+ the beginning of a parenthesized
subexpression, `\verb|$|' is an ordinary character except at the end
of the RE or+ the end of a parenthesized subexpression, and `*' is an
ordinary character if it appears at the beginning of the RE or the
beginning of a parenthesized subexpression (after a possible leading
`\verb|^|'). Finally, there is one new type of atom, a back reference:
`\verb|\|' followed by a non-zero decimal digit d matches the same sequence
of characters matched by the dth parenthesized subexpression
(numbering subexpressions by the positions of their opening
parentheses, left to right), so that (e.g.)
`\verb|\|([bc]\verb|\|)\verb|\|1' matches `bb' or `cc' but not `bc'.

\subsubsection{SEE ALSO}

\Nameref{sec:manre3}

POSIX 1003.2, section 2.8 (Regular Expression Notation).  

\subsubsection{HISTORY}

Written by Henry Spencer, based on the 1003.2 spec.  

\subsubsection{BUGS}

Having two kinds of REs is a botch.

The current 1003.2 spec says that `)' is an ordinary character in the
absence of an unmatched `('; this was an unintentional result of a
wording error, and change is likely. Avoid relying on it.

Back references are a dreadful botch, posing major problems for
efficient implementations. They are also somewhat vaguely defined
(does `a\verb|\|(\verb|\|(b\verb|\|)*\verb|\|2\verb|\|)*d' match
`abbbd'?). Avoid using them.

1003.2's specification of case-independent matching is vague. The
``one case implies all cases'' definition given above is current
consensus among implementors as to the right interpretation.

The syntax for word boundaries is incredibly ugly. 

\section{Class management}

A \clsmo\ is stored into the database as a \clsdo\ the first time an
instance of the class is to be stored to the database. Vice versa, if
a persistent instance is loaded from the database, its class
definition is compiled into the transient LISP image.  A \clsdo\ is a
poor man's version of a \clsmo: In contrary to the elegant definition
of \mo[s]\ based on their behavior as given in \cite{bib:AMOP}, it is
defined in terms of its structure.  This restriction was necessary
since \plob\ can not store methods.

Conflicts are resolved at the moment that at storing a class a
definition mismatch will result in the class being stored anew, and a
conflict at loading a class will be solved by overwriting the
transient definition by the persistent one. For future directions, see
section \Nameref{sec:Views}.

Not all slots of a \clsmo\ are stored; some slots are omitted, since
otherwise the transitive closure of the \clsmo\ would contain more or
less all classes of the current LISP image. For example, the slot
containing all subclasses of a class is omitted, since this would
store all classes of the current LISP image to the database (when
referencing the class' superclasses).  Since \plob\ can not store
binary function code, no methods are stored, too.  System created
methods, like the ones created by the \lisp{:accessor},
\lisp{:reader}\ or \lisp{:writer}\ slot options, are not stored as
methods, but will be generated when the class is compiled into the
current LISP image.

If a class is deleted from the class table by a call to
\fcite{p-delete-class}, it will be reinserted if an instance of the
class is referenced.

\refpar \Fcite{p-find-class}, \fcite{p-delete-class},
\fcite{p-class-of}, \fcite{p-apropos-classes}.

\section{Schema evolution}%
\label{sec:SchemaEvolution}

The default schema evolution for all structure classes is specified by
the value of \fcite{*default-structure-schema-evolution*}; for \clos\ 
classes, the value of \fcite{*default-clos-schema-evolution*}\ is
used.

\refpar \Fcite{(setf schema-evolution)},
\fcite{*default-structure-schema-evolution*},
\fcite{*default-clos-schema-evolution*}.

\section{Other useful functions}

\Fcite{p-apropos}\ will print out all persistent symbols matching with
their name the passed regular expression:
\begin{IndentedCompactCode}
\listener{}(p-apropos "person")
common-lisp-user::*last-name->person-list*,
                  value: #<btree equal 9/678 short-objid=50329169>
common-lisp-user::person
common-lisp-user::person-age (defined)
common-lisp-user::person-first-name (defined)
common-lisp-user::person-last-name (defined)
common-lisp-user::person-name (defined)
common-lisp-user::person-occupation (defined)
common-lisp-user::person-sex (defined)
common-lisp-user::person-soc-sec-# (defined)
\listener{}(p-apropos "[os].*lock")
plob::one-lock
plob::sum-lock
\end{IndentedCompactCode}

This is the output after file
\lisp{plob-\thisversion/}\lisp{}src/\lisp{example/}\lisp{plob-example.lisp}\ 
has been compiled and loaded and some random persons have been
generated (see \fcite{random-person}). A \lisp{(defined)} indicates
that a function binding has been done to the persistent symbol as
explained in section \Nameref{sec:BinaryCode}.

Functions \fcite*{p-statistics}\ and \fcite*{p-configuration}\ will
return some low level database informations.

Calling \fcite{p-stabilise}\ will force a flush of all pending file
operations in the server.

\refpar \Fcite{p-apropos}, \fcite{p-statistics}, \fcite{p-configuration},
\fcite{p-stabilise}.

\section{Performance}%
\label{sec:Performance}%
%
\begin{fortune}[7cm]%
Don't let the sun go down on me.
\from{Elton John}\\[\medskipamount]
\emph{[\plob\ 1.0 was developed on a SUN SPARCstation.]}
\end{fortune}%

\newlength{\codew}\setlength{\codew}{0.3\textwidth}%
\newlength{\perfw}\setlength{\perfw}{0.35\textwidth}%
\newlength{\timew}\setlength{\timew}{8\lispblank}%
\newcounter{perfcount}%
\newlength\codeblank%
\settowidth\codeblank{\texttt{\CompactCodeSize\ }}%
\def\timeboxi#1{\parbox[t]{\timew}{%
\hspace*{\fill}#1}\smallskip}%
\def\timeboxii#1#2{\parbox[t]{\timew}{%
S+\hspace*{\fill}#1\\%
S-\hspace*{\fill}#2}\smallskip}%
%%%
\noindent{}The equipment used for performance testing was a 200 MHz
Pentium-II PC with 64 MB RAM, standard EIDE Harddisk, Windows/NT 4.0
build 1381 service pack 3. The LISP system used was LispWorks Common
LISP 4.0.1.  The checked \plob\ version was 2.07. The numbers in the
very right column are operations per second. Those numbers marked `S+'
are with a server running on \lisp{localhost}\ (that means, on the
same machine as the LispWorks client using the server); the numbers
marked by `S-' are for the serverless version.  In the performance
measurements, care was taken that all operations involved a real
database access, that means, no cache was used for the storing or
loading operations.

\subsection{Sequences}

This section contains performance measurements for sequences, namely
strings, lists and vectors.

\begin{longtable}[c]{|r|p{\codew}|p{\perfw}|p{\timew}|}
%%%
  \hline%
  \textbf{Step} & \textbf{Code} & \textbf{Description} &
  \textbf{Op./s}\\
  \hline\hline\endhead
%%%
  \refstepcounter{perfcount}\theperfcount\label{step:step1} &
%%%
  \parbox[t]{\codew}{\begin{tt}\CompactCodeSize%
      (store-object\\
      \hspace*{2\codeblank}(make-string 32\\
      \hspace*{4\codeblank}:initial-element\\
      \hspace*{4\codeblank}\#\bslash{}Space))\end{tt}}\smallskip &
%%%
  \parbox[t]{\perfw}{%
    Store strings with a length of 32 characters.}\smallskip &
%%%
  \timeboxii{182}{401}\\
%%%
%%%
  \hline%
  \refstepcounter{perfcount}\theperfcount\label{step:step2} &
%%%
  \parbox[t]{\codew}{\begin{tt}\CompactCodeSize%
      (load-object\\
      \hspace*{2\codeblank}\emph{\lt{}objid\gt})\end{tt}}\smallskip &
%%%
  \parbox[t]{\perfw}{%
    Load strings with a length of 32 characters. \emph{\lt{}objid\gt}
    references one of the persistent strings as generated in the last
    performance measurement step.}\smallskip &
%%%
  \timeboxii{92}{467}\\
%%%
%%%
  \hline%
  \refstepcounter{perfcount}\theperfcount\label{step:step3} &
%%%
  \parbox[t]{\codew}{\begin{tt}\CompactCodeSize%
      (store-object\\
      \hspace*{2\codeblank}(make-list\\
      \hspace*{4\codeblank}\emph{\lt{}length\gt}\\
      \hspace*{4\codeblank}:initial-element\\
      \hspace*{4\codeblank}\emph{\lt{}fixnum\gt}))\end{tt}}\smallskip &
%%%
  \parbox[t]{\perfw}{%
    Store a linear list with \emph{\lt{}length\gt} elements, containing
    only immediates; Op./s is the number of stored list elements per
    second.}\smallskip &
%%%
  \timeboxii{560}{1666}\\
%%%
%%%
  \hline%
  \refstepcounter{perfcount}\theperfcount\label{step:step4} &
%%%
  \parbox[t]{\codew}{\begin{tt}\CompactCodeSize%
      (load-object\\
      \hspace*{2\codeblank}\emph{\lt{}objid\gt})\end{tt}}\smallskip &
%%%
  \parbox[t]{\perfw}{%
    Load a linear list; \emph{\lt{}objid\gt} references a persistent
    linear list as generated in step \ref{step:step1}. Op./s is the
    number of loaded list elements per second.}\smallskip &
%%%
  \timeboxii{414}{1428}\\
%%%
%%%
  \hline%
  \refstepcounter{perfcount}\theperfcount\label{step:step5} &
%%%
  \parbox[t]{\codew}{\begin{tt}\CompactCodeSize%
      (store-object\\
      \hspace*{2\codeblank}(make-array\\
      \hspace*{4\codeblank}\emph{\lt{}length\gt}\\
      \hspace*{4\codeblank}:initial-element\\
      \hspace*{4\codeblank}\emph{\lt{}float\gt}))\end{tt}}\smallskip &
%%%
  \parbox[t]{\perfw}{%
    Store a simple vector with \emph{\lt{}length\gt} elements,
    containing non-immediate \emph{\lt{}floats\gt}; Op./s is the number
    of stored vector elements per second.}\smallskip &
%%%
  \timeboxii{600}{4761} \\
%%%
%%%
  \hline%
  \refstepcounter{perfcount}\theperfcount\label{step:step6} &
%%%
  \parbox[t]{\codew}{\begin{tt}\CompactCodeSize%
      (load-object\\
      \hspace*{2\codeblank}\emph{\lt{}objid\gt})\end{tt}}\smallskip &
%%%
  \parbox[t]{\perfw}{%
    Load a simple vector; \emph{\lt{}objid\gt} references a persistent
    simple vector as generated in step \ref{step:step5}. Op./s is
    the number of loaded vector elements per second.}\smallskip &
%%%
  \timeboxii{2100}{2857}\\
%%%
%%%
  \hline%
  \refstepcounter{perfcount}\theperfcount\label{step:step7}&
%%%
  \parbox[t]{\codew}{\begin{tt}\CompactCodeSize%
      (store-object\\
      \hspace*{2\codeblank}(make-array\\
      \hspace*{4\codeblank}\emph{\lt{}length\gt}\\
      \hspace*{4\codeblank}:initial-element\\
      \hspace*{4\codeblank}\emph{\lt{}fixnum\gt}))\end{tt}}\smallskip &
%%%
  \parbox[t]{\perfw}{%
    Store a simple vector with \emph{\lt{}length\gt} elements,
    containing only immediates; Op./s is
    the number of stored vector elements per second.}\smallskip &
%%%
  \timeboxii{14000}{12500}\\
%%%
%%%
  \hline%
  \refstepcounter{perfcount}\theperfcount\label{step:step8}&
%%%
  \parbox[t]{\codew}{\begin{tt}\CompactCodeSize%
      (load-object\\
      \hspace*{2\codeblank}\emph{\lt{}objid\gt})\end{tt}}\smallskip &
%%%
  \parbox[t]{\perfw}{%
    Load a simple vector; \emph{\lt{}objid\gt} references a persistent
    simple vector as generated in step \ref{step:step7}. Op./s is
    the number of loaded vector elements per second.}\smallskip &
%%%
  \timeboxii{20895}{33333}\\
%%%
%%%
  \hline
%%%
\end{longtable}

Step \ref{step:step7} is the only performance test step where the
server mode outperforms the serverless mode. The only explanation I
have is that this performance test step has been done with a rather
big array containing 100000 elements, and that the blockwise (and
implicit concurrent) transfer of the server mode does a better job
here than the serverless mode, which does no blockwise or concurrent
transfer at all.

\subsection{CLOS instances}

Following CLOS class has been used for testing tight-bounded
persistency (section \Nameref{sec:TightBinding}):
\begin{IndentedCompactCode}
(defclass p-example-clos-class ()\\
  ((slot-3 :initform nil :extent :transient)\\
   (slot-5 :initform nil :extent :cached)\\
   (slot-7 :initform nil :extent :cached-write-through))\\
  (:metaclass persistent-metaclass))
\end{IndentedCompactCode}

An \lisp{:extent}\ of \lisp{:transient}\ means that the slot is
not represented in the database at all.  An \lisp{:extent}\ of
\lisp{:cached}\ means that the slot is represented in the database,
but will only be updated when explicitely requesting so.  An
\lisp{:extent}\ of \lisp{:cached-write-through}\ will update the
slot in the database each time it is written to; this is the default
extent used by \plob\ (section \Nameref{sec:AdditionalSlotOptions}).

\begin{longtable}[c]{|r|p{\codew}|p{\perfw}|p{\timew}|}
%%%
  \hline
  \textbf{Step} & \textbf{Code} & \textbf{Description} &
  \textbf{Op./s} \\ 
  \hline\hline\endhead
%%%
  \refstepcounter{perfcount}\theperfcount\label{step:step9}&
%%%
  \parbox[t]{\codew}{\begin{tt}\CompactCodeSize%
      (make-instance\\
      \hspace*{2\codeblank}'p-example-clos-\\
      \hspace*{3\codeblank}class)\end{tt}}\smallskip &
%%%
  \parbox[t]{\perfw}{%
    Allocation of a persistent instance and initialization of their slots
    according to their \lisp{:initform}.}\smallskip &
%%%
  \timeboxii{165}{285}\\
%%%
%%%
  \hline%
  \refstepcounter{perfcount}\theperfcount\label{step:step10} &
%%%
  \parbox[t]{\codew}{\begin{tt}\CompactCodeSize%
      (load-object\\
      \hspace*{2\codeblank}\emph{\lt{}objid\gt})\end{tt}}\smallskip &
%%%
  \parbox[t]{\perfw}{%
    Load of a persistent instance; \emph{\lt{}objid\gt} references a
    persistent instance as generated in step
    \ref{step:step9}.}\smallskip &
%%%
  \timeboxii{240}{344}\\
%%%
%%%
  \hline%
  \refstepcounter{perfcount}\theperfcount\label{step:step11}&
%%%
  \parbox[t]{\codew}{\begin{tt}\CompactCodeSize%
      (setf (slot-value
      \hspace*{7\codeblank}\emph{\lt{}object\gt}\\
      \hspace*{7\codeblank}'slot-3)\\
      \hspace*{2\codeblank}\emph{\lt{}fixnum\gt})\end{tt}}\smallskip &
%%%
  \parbox[t]{\perfw}{%
    Set a \lisp{:transient}\ slot of a tight-bounded persistent class
    (implies no storing of the slot's value to the database).
    \emph{\lt{}object\gt} references a persistent instance as generated
    in step \ref{step:step9}.}\smallskip &
%%%
  \timeboxi{21000}\\
%%%
%%%
  \hline%
  \refstepcounter{perfcount}\theperfcount\label{step:step12}&
%%%
  \parbox[t]{\codew}{\begin{tt}\CompactCodeSize%
      (slot-value \emph{\lt{}object\gt}\\
      \hspace*{2\codeblank}'slot-3)\end{tt}}\smallskip &
%%%
  \parbox[t]{\perfw}{%
    Load the value of a \lisp{:transient}\ slot of a tight-bounded
    persistent class (implies no loading of the slot's value from the
    database).  \emph{\lt{}object\gt} references a persistent instance
    as generated in step \ref{step:step9}.}\smallskip &
%%%
  \timeboxi{29400}\\
%%%
%%%
  \hline%
  \refstepcounter{perfcount}\theperfcount\label{step:step13} &
%%%
  \parbox[t]{\codew}{\begin{tt}\CompactCodeSize%
      (setf (slot-value\\
      \hspace*{7\codeblank}\emph{\lt{}object\gt}\\
      \hspace*{7\codeblank}'slot-5)\\
      \hspace*{2\codeblank}\emph{\lt{}fixnum\gt})\end{tt}}\smallskip &
%%%
  \parbox[t]{\perfw}{%
    Set a \lisp{:cached}\ slot of a tight-bounded persistent class
    (implies no storing of the slot's value to the database).
    \emph{\lt{}object\gt} references a persistent instance as generated
    in step \ref{step:step9}.}\smallskip &
%%%
  \timeboxi{22000}\\
%%%
%%%
  \hline%
  \refstepcounter{perfcount}\theperfcount\label{step:step14} &
%%%
  \parbox[t]{\codew}{\begin{tt}\CompactCodeSize%
    (slot-value \emph{\lt{}object\gt}\\
    \hspace*{2\codeblank}'slot-5)\end{tt}}\smallskip &
%%%
  \parbox[t]{\perfw}{%
    Load the value of a \lisp{:cached}\ slot of a tight-bounded
    persistent class (implies no loading of the slot's value from the
    database).  \emph{\lt{}object\gt} references a persistent instance
    as generated in step \ref{step:step9}.}\smallskip &
%%%
  \timeboxi{31300}\\
%%%
%%%
  \hline%
  \refstepcounter{perfcount}\theperfcount\label{step:step15}&
%%%
  \parbox[t]{\codew}{\begin{tt}\CompactCodeSize%
      (setf (slot-value\\
      \hspace*{7\codeblank}\emph{\lt{}object\gt}\\
      \hspace*{7\codeblank}'slot-7)\\
      \hspace*{2\codeblank}\emph{\lt{}fixnum\gt})\end{tt}}\smallskip &
%%%
  \parbox[t]{\perfw}{%
    Set a \lisp{:cached-write-through}\ slot of a tight-bounded
    persistent class to an immediate fixnum.  \emph{\lt{}object\gt}
    references a persistent instance as generated in step
    \ref{step:step9}.}\smallskip &
%%%
  \timeboxii{384}{1315}\\
%%%
%%%
  \hline%
  \refstepcounter{perfcount}\theperfcount\label{step:step16}&
%%%
  \parbox[t]{\codew}{\begin{tt}\CompactCodeSize%
      (slot-value \emph{\lt{}object\gt}\\
      \hspace*{2\codeblank}'slot-7)\end{tt}}\smallskip &
%%%
  \parbox[t]{\perfw}{%
    Load the value of a \lisp{:cached-write-through}\ slot of a
    tight-bounded persistent class. The value loaded is an immediate
    fixnum.  \emph{\lt{}object\gt} references a persistent instance as
    generated in step \ref{step:step9}.}\smallskip &
%%%
  \timeboxi{31300}\\
%%%
%%%
  \hline
%%%
\end{longtable}

For steps \ref{step:step11}--\ref{step:step14} and \ref{step:step16}
there are no performance differences between the serverless and server
mode, because no data transfer between the client and the database is
involved (for these steps, the LISP layer is never left).

\subsection{Database-specific functions}

\begin{longtable}[c]{|r|p{\codew}|p{\perfw}|p{\timew}|}
%%%
  \hline
  \textbf{Step} & \textbf{Code} & \textbf{Description} &
  \textbf{Op./s} \\ 
  \hline\hline\endhead
%%%
  \refstepcounter{perfcount}\theperfcount\label{step:step17}&
%%%
  \parbox[t]{\codew}{\begin{tt}\CompactCodeSize%
      (p-allocate-cons)\end{tt}}\smallskip &
%%%
  \parbox[t]{\perfw}{%
    Allocation of an empty persistent cons cell.}\smallskip &
%%%
  \timeboxii{1639}{9090}\\
%%%
%%%
  \hline%
  \refstepcounter{perfcount}\theperfcount\label{step:step18} &
%%%
  \parbox[t]{\codew}{\begin{tt}\CompactCodeSize%
      (with-transaction ()\\
      \hspace*{2\codeblank}nil)\end{tt}}\smallskip &
%%%
  \parbox[t]{\perfw}{%
    Empty transaction.}\smallskip &
%%%
  \timeboxii{770}{5000}\\
%%%
%%%
  \hline%
  \refstepcounter{perfcount}\theperfcount\label{step:step19} &
%%%
  \parbox[t]{\codew}{\begin{tt}\CompactCodeSize%
    (with-transaction ()\\
    \hspace*{2\codeblank}(write-lock-store)\\
    \hspace*{2\codeblank}nil)\end{tt}}\smallskip &
%%%
  \parbox[t]{\perfw}{%
    Empty transaction locking the whole database (this performance is
    comparable to the performance of single-object level
    locking).}\smallskip &
%%%
  \timeboxii{476}{1818}\\
%%%
%%%
  \hline
%%%
\end{longtable}

\subsection{Different usage of transactions and locking}

In this section, the performance is shown for the single operation of
step \ref{step:step9}, but with different usage of transactions
and object locking.  The default behavior of \plob\ as shown in
step \ref{step:step9} involves one transaction for each call to
\fcite{make-instance}\ and the (implicit) usage of some locks at
storing the object. Using transactions and locks on bigger `chunks' of
objects improves performance for a single user significantly, but will
make the `granularity' of object access more coarse, making concurrent
access slower.

\begin{longtable}[c]{|r|p{\codew}|p{\perfw}|p{\timew}|}
%%%
  \hline%
  \textbf{Step} & \textbf{Code} & \textbf{Description} &
  \textbf{Op./s} \\
  \hline\hline\endhead
%%%
  \ref{step:step9} &
  \parbox[t]{\codew}{\begin{tt}\CompactCodeSize%
      (make-instance\\
      \hspace*{2\codeblank}'p-example-clos-\\
      \hspace*{3\codeblank}class)\end{tt}}\smallskip &
%%%
  \parbox[t]{\perfw}{%
    Allocation of a persistent instance and initialization of their
    slots according to their \lisp{:initform}\ (this is step
    \ref{step:step9} repeated here for completeness).}\smallskip &
%%%
  \timeboxii{165}{285}\\
%%%
%%%
  \hline%
  \refstepcounter{perfcount}\theperfcount\label{step:step20}&
%%%
 \parbox[t]{\codew}{\begin{tt}\CompactCodeSize%
     (with-transaction ()\\
     \hspace*{2\codeblank}(dotimes (i 1000)\\
     \hspace*{4\codeblank}(make-instance\\
     \hspace*{6\codeblank}'p-example-clos-\\
     \hspace*{7\codeblank}class)))\end{tt}}\smallskip &
%%%
  \parbox[t]{\perfw}{%
    1000 instances of step \ref{step:step9} generated in a single
    transaction; Op./s is the number of stored instances per
    second.}\smallskip &
%%%
 \timeboxii{175}{322}\\
%%%
%%%
 \hline%
 \refstepcounter{perfcount}\theperfcount\label{step:step21} &
%%%
 \parbox[t]{\codew}{\begin{tt}\CompactCodeSize%
     (with-transaction ()\\
     \hspace*{2\codeblank}(write-lock-store)\\
     \hspace*{2\codeblank}(dotimes (i 1000)\\
     \hspace*{4\codeblank}(make-instance\\
     \hspace*{6\codeblank}'p-example-clos-\\
     \hspace*{7\codeblank}class)))\end{tt}}\smallskip &
%%%
  \parbox[t]{\perfw}{%
    1000 instances of step \ref{step:step9} generated in a single
    transaction with an exclusive write lock set onto the whole
    database; Op./s is the number of stored instances per
    second.}\smallskip &
%%%
 \timeboxii{270}{357}\\
%%%
%%%
  \hline
%%%
\end{longtable}

\subsection{File space allocation}

This section explains how much file space is allocated by \plob\ for a
persistent object. All persistent objects are stored in a
(memory-mapped) file; this is the file named \lisp{stablestore}\ in
each database directory.  \hyperlink{link:Immediate}{Immediates}\ use
up no file space at all.

\subsubsection[Non-CLOS objects]{Non-\protect\clos\ objects}

Non-immediates, non-\clos\ objects and \lisp{defstruct}\ objects have
an overhead of 6 words $\equiv$ 24 bytes. The total file space
allocated for such an object is the sum of the overhead plus 1 word
$\equiv$ 4 bytes per `slot'. For example, a cons cell has 2 slots
(\lisp{car}\ and \lisp{cdr}) and will take up a total of 6 + 2 = 8
words $\equiv$ 24 + 8 = 32 bytes.

\subsubsection[CLOS objects]{\protect\clos\ objects}

For \clos\ instances, the overhead is 14 words $\equiv$ 56 bytes plus
1 word $\equiv$ 4 bytes per slot. For example, a \clos\ instance with
2 slots will take up a total of 14 + 2 = 16 words $\equiv$ 56 + 8 = 64
bytes.

\section{Technical data}%
\label{sec:TechnicalData}

%
\begin{deflist}[XXXXXXXXXXXXXXX]

\item[Architecture] Client/server

\item[Operating systems] Solaris 2.x, Linux kernel version 2.x,
  Windows XP. IRIX 6.x is  supported on request.

\item[LISP systems] \href{http://www.lispworks.com}{\lwcl}\ 4.4,
  \href{http://www.franz.coml}{\allegrocl}\ 7

\item[Databases per Server] Any reasonable number of open databases,
  depending on installed memory.

\item[Databases per Client] 1

\item[Multi user access] Multiple clients can connect to one server
  process.

\item[Object space size] Maximum of 384 MB per database. This is a
  technical limit of the low-level persistent storage used for the
  server.

\item[Access control] Login deny/allow on per-machine basis

\item[Lock protocol] Hierarchical locking.

\item[Lock levels] Whole storage, object, slot.

\item[Lock modes] Read-only, read, write; read-only intent, read
  intent, write intent

\item[Transactions] Flat transactions. Concurrent transactions are
  supported, both between different clients and between different
  database sessions of a single client.

\item[Access protocol] Two phased transactions and locking of the
  persistent object at object state transfer between client and
  server.

\item[Indexes supported] Btrees on slots of persistent \clos\ 
  instances are maintained automatically; direct usage of btrees is
  possible.

\item[LISP types] All types defined in \cite{bib:CLtLII} are
  supported, with the exception of binary function code.

\item[Name spaces] Persistent packages containing persistent symbols.

\end{deflist}


%%% Local Variables: 
%%% mode: latex
%%% TeX-master: "userg"
%%% End: 
