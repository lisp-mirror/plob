% This is -*-LaTeX-*-
%
% userg.tex
% HK 26.5.94
% Rewritten 1997/09

\chapter{Database functionality}

This chapter describes concepts, functions and classes of \plob\ whose
transient counterparts are either not found in \cite{bib:CLtLII}\ at
all because they represent certain database features not needed by a
transient LISP system or whose meaning in \plob\ has been adopted for
the needs in a persistent system.

\section{Databases, sessions, transactions and locks}

This section describes some terms and design issues of \plob\ 
concerning databases, sessions, transactions and locks from a
programmer's point of view, and the relations between them.

\subsection{Database}

A database contains persistent objects. At the moment (1998/08/27), a
\plob\ client can have only one database open at a
time.\footnote{Please note that a single \plob\ server can handle any
  number of clients.} It is a global resource shared among all LISP
lightweight processes running in a LISP image. A database is opened
the very first time when the first `session' is established.

\subsubsection{Multiple Databases}

Working with multiple databases would mean to open the first database,
do something with it, close the first database, open the second
database, etc. To be more concrete, for true working with multiple
databases, the processing would look like:
\begin{IndentedCompactCode}
(progn

  \comment{;; Open the session; since the database is currently closed,}
  \comment{;; this implies opening the database referenced by the passed URL:}
  (open-my-session "tcp://localhost/database1")
  \comment{;; Do something:}
  (do-something-with-database-1)
  \comment{;; This closes all sessions and the database:}
  (close-heap)

  \comment{;; Open the session; since the database is currently closed,}
  \comment{;; this implies opening the database referenced by the passed URL:}
  (open-my-session "tcp://localhost/database2")
  \comment{;; Do something:}
  (do-something-with-database-2)
  \comment{;; This closes all sessions and the database:}
  (close-heap))
\end{IndentedCompactCode}

For historical reasons (because of its global character), the
representation of a database is a bit unclear/smeared in \plobwoexcl.
     
\subsection{Session}%
\label{sec:SessionManagement}

A session is used as an `access organizer' or `access helper' to a
database. A session organizes access to the persistent objects
contained in the database by ensuring that each access is within an
active transaction, that a read access to a persistent object in the
database is done within a read lock and a write access within a write
lock. By default, each LISP lightweight process (thread) gets its own
session when the thread does its first access to a persistent
object. All sessions of a running LISP image are bound to the same
underlying database.

The session associated to the current thread is contained in
\fcite{*default-persistent-heap*}. There is one global session which
is used at the bootstrap and for storing metaobject information; this
one is contained in \fcite{*root-persistent-heap*}.

Sessions are represented as persistent objects of
\fcite{persistent-heap}.

\refpar \Fcite{open-my-session}\ and \fcite{*database-url*},
\fcite{*default-database-url*}; \fcite{close-my-session},
\fcite{show-sessions}, \fcite{p-sessions}, \fcite{with-session}.

\subsection{Transaction}%
\label{sec:Transaction}

A transaction is some state-changing operation to the underlying
database. The transactions implemented in \plob\ are two phased
transactions. This transaction model ensures that each single
transaction has isolated access to persistent objects; essentially, it
means that a transaction is always divided into two phases, the first
phase where locks are established and the second phase which starts
with the first lock being released. From the user's side, the second
phase is never visible, since it is embedded into the transaction's
commit processing.

Figure~\ref{fig:TransactionSTDiagram} shows
\begin{figure}[htbp]
\centerline{\psfig{figure=\Path/sttrans.eps}}
\caption{S/T diagram of a transaction}%
\label{fig:TransactionSTDiagram}
\end{figure}%
the possible states of a transaction; only in the active state, a
transaction is allowed to change the database's state.  According to
the isolation table given in \cite[\citepage{399}]{bib:Gray-et-al-93},
\plob\ implements an isolation of 3$^{\circ}$. Since starting and
stopping a transaction might have side effects in the highest
application layer, transactions are always started and ended by the
highest application layer.

A transaction could also be seen as a mean to ensure that the database
is in a consistent state before the transaction has been started and
after it has been finished. Here, consistency is defined in terms
purely of the application programmer's point of view, and involves no
other constraints.

For now, a transaction is bound exactly to one session. An active
transaction in a session is for example shown in its print output;
with no active transaction on the default session, the value of
\fcite{*default-persistent-heap*}\ prints like:
\begin{CompactCode}
\listener{}*default-persistent-heap*
#<heap kirschke@localhost `Initial Lisp Listener' short-objid=133987850>
\end{CompactCode}

Within an active transaction, it prints like:
\begin{CompactCode}
\listener{}(begin-transaction *default-persistent-heap*)
179048
\listener{}*default-persistent-heap*
#<heap kirschke@localhost `Initial Lisp Listener' transaction=179048
                                                  short-objid=133987850>
\listener{}(in-transaction-p *default-persistent-heap*)
179048
\listener{}(end-transaction *default-persistent-heap*)
179048
\listener{}*default-persistent-heap*
#<heap kirschke@localhost `Initial Lisp Listener' short-objid=133987850>
\listener{}(in-transaction-p *default-persistent-heap*)
NIL
\end{CompactCode}

Since transactions are bound to exactly one session, a transaction is
not represented explicit but implicit by a session containing a
transaction id.

\refpar \Fcite{with-transaction}, \fcite{begin-transaction},
\fcite{end-transaction}, \fcite{cancel-transaction}.

\subsubsection{Aborting a transaction}%
\label{sec:TransactionAbort}

At the moment, aborting a transaction will clear the cache to enforce
a reload of the persistent objects into transient representations,
with all of its consequences (section \Nameref{sec:IdAndCache}).
Future versions of \plob\ will support a more elaborate handling of
transaction aborts.

\subsection{Lock}%
\label{sec:Locking}

A lock ensures that there is no conflicting access to a persistent
object. In conjunction with transactions, locking is an implementation
of the concept of \hyperlink{link:Isolation}{isolation}.  The locking
mechanism implemented is hierarchical or tree locking, a specialized
version of predicate locking \cite[\citepage{406}]{bib:Gray-et-al-93}.
Locking is possible on the levels \textbf{store} (lock all persistent
objects), \textbf{vector} (lock one persistent object) and
\textbf{element} (lock one slot of a persistent object) with the lock
modes \textbf{read-only}, \textbf{read} or \textbf{write}. The locking
protocol is a pessimistic one: An object must be locked before its
state can be accessed. All locking is done implicit during accessing
the state of a persistent object.

All locks must be set within an active transaction; they are bound to
the transaction. When ending the transaction (either committing or
aborting), all locks set in the scope of the transaction are removed.
Before any access can be done to a persistent object, a read or write
lock must be granted to the modifying transaction. If a lock cannot be
granted, a lock conflict occurs. A lock conflict is given by the
following rules:

\begin{enumerate}

\item Conflicts between locks set by different transactions are given
  by the conflict matrix shown in table~\ref{tab:lockconfl}. In
  essence, this matrix says that read locks are shared among a locked
  object, whereas a write lock requires no other lock set onto the
  object.

\item A lock which should be set by a transaction never conflicts with
  other locks already set by the same transaction. For example, if a
  transaction requested a write lock for a persistent object and the
  same transaction requests later a read lock, these locks do not
  conflict.

\end{enumerate}

\begin{figure}[htbp]%
\centering%
\def\coliw{3em}%
\def\pbox#1{\parbox{\coliw}{\centering#1}}%
\def\vvbox{\pbox{\checked}}%
\def\XXbox{\pbox{\crossed}}%
\begin{\figurefontsize}%
\begin{tabular}{%
|c|p{\coliw}|p{\coliw}|p{\coliw}|p{\coliw}|p{\coliw}|p{\coliw}|}
\cline{2-7}
\multicolumn{1}{c|}{\parbox{8em}{%
 \hspace*{\fill}Already granted$\rightarrow$\\[\smallskipamount]%
 Requested$\downarrow$\hspace*{\fill}}}
 &\pbox{\tabularheader{RO-Intent}}
  &\pbox{\tabularheader{Read-Only}}
   &\pbox{\tabularheader{Read-Intent}}
    &\pbox{\tabularheader{Read}}
     &\pbox{\tabularheader{Write-Intent}}
      &\pbox{\tabularheader{Write}}\\ \hline
%
%        ROI     RO      RI      R       WI      W
\tabularheader{Read-Only-Intent}
        &\vvbox &\vvbox &\vvbox &\vvbox &\vvbox &\vvbox\\ \hline
\tabularheader{Read-Only}
        &\vvbox &\vvbox &\vvbox &\vvbox &\vvbox &\vvbox\\ \hline
\tabularheader{Read-Intent}
        &\vvbox &\vvbox &\vvbox &\vvbox &\vvbox &\XXbox\\ \hline
\tabularheader{Read}
        &\vvbox &\vvbox &\vvbox &\vvbox &\XXbox &\XXbox\\ \hline
\tabularheader{Write-Intent}
        &\vvbox &\vvbox &\vvbox &\XXbox &\vvbox &\XXbox\\ \hline
\tabularheader{Write}
        &\vvbox &\XXbox &\XXbox &\XXbox &\XXbox &\XXbox\\ \hline
%
\multicolumn{7}{r}{%
\checked\ $\equiv$\ Compatible locks\quad
\crossed\ $\equiv$\ Conflicting locks}
\end{tabular}%
\end{\figurefontsize}%

\let\figurename\tablename%
\caption{Conflictmatrix for locks}%
\label{tab:lockconfl}%
\end{figure}%

A conflict is `resolved' by the client requesting the conflicting lock
to wait until either the other lock involved in the conflict is
removed (for example, because another transaction holding this lock
has terminated) or a timeout is encountered (this indicates a maybe
not solvable conflict, like a deadlock, and will raise an error).

The functions \fcite*{write-lock-store}\ and \fcite*{read-lock-store}\ 
can be used to place a write or read lock onto the whole database.
This is usefull when the whole database should be written or read, see
section \Nameref{sec:LotsOfData} for an example.

Locks are represented as persistent objects. They will almost never be
seen directly by the database programmer, since most of the locking is
done automatically.

\refpar \Fcite{write-lock-store}, \fcite{read-lock-store}; section
\fcite{locking ...}.

\section{Making objects reachable}%
\label{sec:ObtainingReachability}

In figure~\ref{fig:ReachableObjects},
\begin{figure}[htbp]
\centerline{\psfig{figure=\Path/gctrpe.eps}}
\caption{Reachable objects}%
\label{fig:ReachableObjects}
\end{figure}%
objects \oi\ -- \oiv\ are reachable from the designated persistent
root object and will not be garbage collected.  Objects \ov\ and \ovi\ 
are only reachable from a transient variable and are not reachable
from the persistent root,\marginlabel{References from transient
  variables to persistent objects are not regarded by \plobwoexcl.}
since the persistent system does not know anything about references
from transient objects to persistent objects. In consequence, objects
\ov\ and \ovi\ would be garbage-collected. The problem with references
from (transient) variables to persistent objects is solved two-fold in
\plobwoexcl:
\begin{enumerate}

\item A garbage collection is triggered\marginlabel{Offline garbage
    collection.} if and only if there are no references from transient
  to persistent objects. This is the case when there is no (transient)
  client connected to the persistent object system.  To be more
  specific, a garbage collection is triggered in the server after the
  last client has disconnected itself.

\item \plob\ offers an interface to persistent
  symbols,\marginlabel{Persistent symbols with persistent values.}
  which are defined as being reachable, both the symbol itself and its
  value.  Another advantage is that an object bound to a persistent
  symbol can be located in the persistent object system by the
  symbol's name.

\end{enumerate}
Additionally, declaring an index on a persistent class using the tight
binding will ensure reachability for the class' instances (section
\Nameref{sec:SlotIndexAndReachability}).

\section{Persistent symbols}%
\label{sec:PersistentSymbol}

As with transient symbols, persistent symbols serve for naming
persistent objects.\footnote{Named objects are found in ODMG
  \cite[\citepage{242}, \textbf{bind()} method]{bib:ODMG-2}, too, but
  ODMG does not cope with symbols, only with simple names.} Persistent
symbols always have a global scope, there is no mean to establish a
local scope for a persistent symbol. For convenience, the construct
\lisp{\#!}\emph{\lt{}symbol\gt}\ is established by \plob\ to refer to
the persistent \emph{\lt{}symbol\gt} having the same name and package
as the transient \emph{\lt{}symbol\gt}.
\begin{IndentedCompactCode}
\listener{}(setf #!*foo* 3)\marginnumber{\smalloi}
3
\listener{}#!*foo*
3
\listener{}*foo*\marginnumber{\smalloii}
Error: Attempt to take the value of the unbound variable `*FOO*'.
  [condition type: UNBOUND-VARIABLE]
\OmitUnimportant
\listener{}#!*bar*\marginnumber{\smalloiii}
Error: Cannot locate a persistent symbol named *BAR*.
  [condition type: SIMPLE-ERROR]
\OmitUnimportant
\listener{}(p-intern '*bar*)\marginnumber{\smalloiv}
50327204        \comment{;; This is the numeric objid of the created symbol}
NIL             \comment{;; NIL means it is a new created persistent symbol}
\listener{}#!*bar*\marginnumber{\smallov}
#<unbound-marker=0x104>
\end{IndentedCompactCode}

The value of a persistent symbol can be set with \fcite{setf}\ \oi.
The value of the transient symbol will neither be set
nor be changed \oii; only the name and package of the transient symbol
are used to reference the persistent symbol's name and package.
Opposed to transient symbols, a read access to a non-existing
persistent symbol will not result in the persistent symbol being
automatically inserted into the database's symbol table, but will
raise an error \oiii.  This is done to prevent the database from
getting littered with a lot of automatic inserted and maybe otherwise
unused symbols.  Creating a new, unbound persistent symbol can be done
by calling \fcite{p-intern}\ \oiv. Referencing an unbound persistent
symbol will not raise an error, but will return a special marker
instance representing an unbound persistent object \ov.

The complete reference chain to a persistent symbol is:
\begin{quote}
  Persistent root $\rightarrow$ package table
  $\stackrel{*}{\rightarrow}$ package $\rightarrow$ list of internal
  symbols $\stackrel{*}{\rightarrow}$ symbol
\end{quote}

\refpar[for persistent symbols]
\begin{RefList}\itemsep0pt

  \Allocator \Fcite{p-allocate-symbol}

  \Creator \Fcite{p-setq}

  \TypePredicate \Fcite{p-symbolp}

  \ObjectStore \Fcite{(setf p-symbol)}

  \ObjectLoad \Fcite{p-symbol}

  \SlotReader \Fcite{p-symbol-function}, \fcite{p-symbol-name},
  \fcite{p-symbol-package}, \fcite{p-symbol-plist},
  \fcite{p-symbol-value}

  \SlotWriter \Fcite{(setf p-symbol-function)}, \fcite{(setf
    p-symbol-plist)}, \fcite{(setf p-symbol-value)},
  \fcite{p-fmakunbound}, \fcite{p-makunbound}.

  Once established, the name of a persistent symbol can not be
  changed, so there is no writer for a symbol's name.

  \Information \Fcite{p-boundp}, \fcite{p-fboundp}

\item[Administration] \Fcite{p-find-symbol}, \fcite{p-intern},
  \fcite{p-unintern}

\end{RefList}

\section{Persistent packages}

Persistent packages serve as containers for persistent symbols; all
persistent packages are reachable and are therefore never garbage
collected. At the moment, all persistent symbols belonging to a
persistent package are internal persistent symbols.  If the same
database is used by different LISP systems (for example, by \lwcl\ and
\allegrocl\ at the same time), there is a need for exporting symbols
and a persistent use-package list, see \fcite{p-find-class}\ and
section \Nameref{sec:MissingUsePackage}.

\refpar \Fcite{p-find-package}, \fcite{p-delete-package},
\fcite{p-package-internals}, \fcite{p-apropos-packages},
\fcite{p-package}, \fcite{p-package-name},
\fcite{p-package-externals}.

\section{Markers}%
\label{sec:Markers}

\plob\ has an addtional class \class{marker}.  Its nature and semantic
shows some resemblance to \cl's condition classes, with the difference
that it does not have an associated error handling protocol.  Markers
are used for representing certain conditions, which are not
necessarily error conditions.
\begin{description}

\item[Unbound slot] The marker object contained in
  \fcite{+plob-unbound-marker+}\ is used for slots which are not bound
  to a value: A slot which references such a marker is considered as
  being unbound.

\item[Unstorable object] If \plob\ encounteres objects it can not
  store into the database, the instance in
  \fcite{+plob-unstorable-object-marker+}\ is stored instead of the
  unstorable object itself. Since \plob\ is almost type-complete, this
  marker is used very rare, mainly when attempting to store binary
  function code.

\item[Extremal value] The marker object contained in
  \fcite{+plob-min-marker+}/\fcite{+plob-max-marker+}\ represent a
  generic object with its state being equal to the state of the
  minimum/maximum object in a set. For a set of persistent numeric
  values, they serve as a representation for the minimum/maximum
  number contained in the set. For a set with persistent strings, they
  represent the first/last string in this [ASCII-]ordered set.

  For example, let's take the \lisp{*c*}\ btree from
  \Nameref{sec:HierarchicalKeys}. The data associated to the
  minimum/maximum key stored in btree \lisp{*c*}\ can be retrieved by:
  \begin{CompactCode}
\listener{}(getbtree +plob-min-marker+ *c*)
"string (1 1)."
\listener{}(getbtree +plob-max-marker+ *c*)
"string (4d0 3)."
  \end{CompactCode}
  The corresponding key values themselves can be requested by a call
  to \fcite{btree-minkey}/\fcite{btree-maxkey}:
  \begin{CompactCode}
\listener{}(btree-minkey *c*)
#(1 1)
\listener{}(btree-maxkey *c*)
#(4d0 3)
  \end{CompactCode}

\end{description}
Often, those conditions are represented by LISP implementators by
using singular objects, like a string or a symbol for each marker
instance.  The concept of a marker class represents those conditions
explicit by instances of a disjunctive class. Markers are
represented as immediates, and so do not take up much space.

\refpar \Fcite{+plob-min-marker+}, \fcite{+plob-max-marker+},
\fcite{+plob-unbound-marker+},
\fcite{+plob-unstorable-object-marker+}.

\section{Persistent btrees}%
\label{sec:PersistentBTree}

A btree is similar to the well-known hash table: It associates a value
to a key. The difference to a hash table is that the keys are
1-dimensional sorted. Two kinds of sortings can be used:
\begin{description}

\item[By identity] This will sort the keys according to their
  \objid[s]. Since each persistent object has an \objid, all
  persistent objects regardless of their class can be used as key
  objects in a single btree.

\item[By state] This will sort the keys according to their state. This
  implies for key objects to be inserted, that they must be able to
  compare their state with the state of all keys which are already
  stored in the btree. For the most important persistent classes used
  with btrees, compare methods have been implemented.

\end{description}
The kind of sorting is selected with the \keyarg{test}\ argument
passed to \fcite{make-btree}. A \keyarg{test}\ argument of \lisp{eq}\ 
will sort the keys by identity, and a \keyarg{test}\ argument of
\lisp{equal}\ will sort the keys by state.

\refpar \Fcite{persistent-btree}\ and \fcite{make-btree},
\fcite{getbtree}, \fcite{clrbtree}, \fcite{rembtree},
\fcite{btree-count}, \fcite{btree-size},
\fcite{btree-test}, \fcite{p-apropos-btree}.

\subsection{Btrees sorted by state}

For an \lisp{equal}\ btree, the type(s) of keys it can contain is
fixed with the first object inserted into the btree.
\begin{IndentedCompactCode}
\listener{}(setf *b* (make-btree :test 'equal))
#<btree equal 0/0=0*678 short-objid=50327519>
\listener{}(setf (getbtree 1 *b*) "string 1.")\marginnumber{\smalloi}
"string 1."
\listener{}(setf (getbtree 2.0 *b*) "string 2.0.")\marginnumber{\smalloii}
"string 2.0."
\listener{}(setf (getbtree "three" *b*) "string three.")\marginnumber{\smalloiii}
Error: From server at executing client:SH_btree_insert_by_string:
       splobbtree.c(1153): fnKeyCmp:
       Illegal search key "three" for
       object #<btree equal 2/678=1*678 short-objid=50327519>:
       compare failed with
       object 1.
  [condition type: POSTORE-ERROR]
\end{IndentedCompactCode}

For example, if the first object inserted into an \lisp{equal}\ btree
is a number \oi, only numbers can be inserted afterwards \oii; trying
to insert for example a string will raise an error \oiii.

\subsubsection{Hierarchical keys}%
\label{sec:HierarchicalKeys}

Using a non-atomic object as a key in an \lisp{equal}\ btree has a
special meaning: \plob\ will try to insert the key according to an
element-wise comparision to the slots of the keys already found in the
btree. This way, a non-atomic key object is a compound key, with its
first slot being the primary key, the second slot being the secondary
key and so on:
\begin{IndentedCompactCode}
\listener{}(setf *c* (make-btree :test 'equal))
#<btree equal 0/0=0*678 short-objid=50327520>
\listener{}(setf (getbtree #(1 1) *c*) "string (1 1).")
"string (1 1)."
\listener{}(setf (getbtree #(1 2) *c*) "string (1 2).")
"string (1 2)."
\listener{}(setf (getbtree #(1 3.0) *c*) "string (1 3.0).")
"string (1 3.0)."
\listener{}(setf (getbtree #(4d0 3) *c*) "string (4d0 3).")
"string (4d0 3)."
\listener{}(p-apropos-btree *c*)
#(1 1), data: #<simple-string `string (1 1).' short-objid=50327507>
#(1 2), data: #<simple-string `string (1 2).' short-objid=50327504>
#(1 3.0), data: #<simple-string `string (1 3.0).' short-objid=50327502>
#(4.0d0 3), data: #<simple-string `string (4d0 3).' short-objid=50327369>
\end{IndentedCompactCode}

\subsection{Mapping btrees}

\Fcite{mapbtree}\ can be used to map all elements of a btree.
\begin{IndentedCompactCode}
\listener{}(mapbtree #'(lambda (key data)\marginnumber{\smalloi}
                          (format t "key ~A, data ~A~%" key data) t)\marginnumber{\smalloii}
                       *b*)
key 1, data string 1.
key 2.0, data string 2.0.
2\marginnumber{\smalloiii}
\listener{}(mapbtree #'(lambda (key data)
                          (format t "key ~A, data ~A~%" key data) t)
                       *b* :descending t)\marginnumber{\smalloiv}
key 2.0, data string 2.0.
key 1, data string 1.
2\marginnumber{\smalloiii}
\listener{}(mapbtree #'(lambda (key data)\marginnumber{\smallov}
                          (format t "key ~A, data ~A~%" key data) t)
                       *b* :>= 2)
key 2.0, data string 2.0.
1\marginnumber{\smalloiii}
\end{IndentedCompactCode}

Since the keys are ordered, the map function is called by default with
the keys in ascending order \oi.  It is important that the map
function returns a \nonnil\ value if the mapping should continue with
the next element \oii. The call to \fcite{mapbtree}\ returns the
number of mapped btree elements \oiii.  If the \keyarg{:descending}\ 
argument is passed with a \nonnil\ value, the key interval is iterated
in descending order \oiv.  It is possible to restrict the mapping to
an interval of key values \ov. The lower key to map can be passed with
a \keyarg{:>}\ or \keyarg{:>=}\ argument, the upper key to map with a
\keyarg{:<}\ or \keyarg{:<=}\ argument.

It is allowed to delete the key currently passed to the map function
from the btree.

\refpar \Fcite{mapbtree}

\subsection{Cursors on btrees}

An access interface similar to cursors in SQL is available for btrees.
Instances of \fcite{persistent-btree-mapper}\ can be used for
accessing a btree by positioning. For example, create a btree with
some elements in it.
\begin{IndentedCompactCode}
\listener{}(setf *btree* (make-btree :test 'equal))
#<btree equal 0/0=0*678 short-objid=50327518>
\listener{}(dotimes (i 1000) (setf (getbtree i *btree*)
                                   (format nil "String ~A" i)))
\end{IndentedCompactCode}

Now, the btree is accessed by an instance of
\fcite{persistent-btree-mapper}.
\begin{IndentedCompactCode}
\listener{}(setf *mapper* (make-btree-mapper *btree*))\marginnumber{\smalloi}
\listener{}(btree-mapper-seek *mapper* 1)\marginnumber{\smalloii}
1
0
"String 0"
\listener{}(btree-mapper-seek *mapper* 1)\marginnumber{\smalloii}
1
1
"String 1"
\listener{}(btree-mapper-seek *mapper* -1 :end)\marginnumber{\smalloiii}
1
998
"String 998"
\end{IndentedCompactCode}
A mapper is established by calling the constructor of the mapper with
the btree as its argument which should be iterated \oi; further
optional arguments can specify a search interval for the mapper. With
a call to \fcite{btree-mapper-seek}, the next or previous element is
returned. Positioning can be done relative to the start or end of the
current search interval \smalloiii.

\refpar \Fcite{make-btree-mapper}, \fcite{btree-mapper-search},
\fcite{btree-mapper-seek}, \fcite{(setf btree-mapper-seek)}

\subsection{Setting the page size}

The internal used page size of a btree can be set either by passing
the key argument \keyarg{pagesize}\ along with the number of objects
each btree page should hold, or by calling \fcite{(setf
  btree-pagesize)}. For very small or very big btrees, a smaller
or larger page size than the default page size should be used. To
be more exact, the page size' optimum is fulfilled for
\begin{displaymath}\begin{array}{rcl}
%
n^{n} &\approx& o\\[\smallskipamount]
%
n &\equiv& \textrm{page size}\\
o &\equiv& \textrm{total number of objects in btree}
\end{array}\end{displaymath}

\refpar \Fcite{make-btree}, \fcite{(setf btree-pagesize)},
\fcite{btree-pagesize}.

\subsection{Requesting extremal values}

The data object associated to the minimum or maximum key object
contained in a btree can be requested by passing the special marker
objects \fcite{+plob-min-marker+}\ and \fcite*{+plob-max-marker+}\ to
\fcite{getbtree}. This way, it is also possible to modify the data
objects associated to the minimum or maximum key by passing the
marker object to \fcite*{(setf getbtree)}.  The minimum or maximum
key object itself is returned by a call to \fcite{btree-minkey}\ 
or \fcite{btree-maxkey}.

\refpar \Fcite{+plob-min-marker+}, \fcite{+plob-max-marker+};
\Fcite{btree-minkey}, \fcite{btree-maxkey}.

\subsection{Inspecting btrees}

In \lwcl, a btree's contents can be viewed with the built in \lw\
inspector. Simply start an inspector on a btree object to look at its
contents. To see the structure of the btree correctly, in the
inspector's `View' menu the entry `Unsorted' should be selected.

\subsubsection[Inspecting the index associated to a slot]{Example:
  Inspecting the index associated to a slot} 

For example, store some thousand persons by evaluating
\lisp{(store-n-random-people 2000)}\ (found in
\lisp{plob-\thisversion/}\lisp{src/}\lisp{example/}\lisp{plob-example.lisp}.
Evaluate \lisp{(p-find-class 'person)}, this will return as first
value the \clsdo\ of class \class{person}. Inspect this first value.
Next, inspect the vector labelled \lisp{PLOB::P-EFFECTIVE-SLOTS}.
Next, inspect the instance marked as being the
effective-slot-description for slot \textbf{soc-sec-\#}. In this \sltdo,
the index defined for slot \textbf{soc-sec-\#}\ of class \class{person}\ 
can be found within the line labelled \lisp{PLOB::P-INDEX}. Inspect
this instance; this will show the btree which is used for representing
the index declared on slot \textbf{soc-sec-\#}\ within class
\class{person}.

\section{Class management}

A \clsmo\ is stored into the database as a \clsdo\ the first time an
instance of the class is to be stored to the database. Vice versa, if
a persistent instance is loaded from the database, its class
definition is compiled into the transient LISP image.  A \clsdo\ is a
poor man's version of a \clsmo: In contrary to the elegant definition
of \mo[s]\ based on their behavior as given in \cite{bib:AMOP}, it is
defined in terms of its structure.  This restriction was necessary
since \plob\ can not store methods.

Conflicts are resolved at the moment that at storing a class a
definition mismatch will result in the class being stored anew, and a
conflict at loading a class will be solved by overwriting the
transient definition by the persistent one. For future directions, see
section \Nameref{sec:Views}.

Not all slots of a \clsmo\ are stored; some slots are omitted, since
otherwise the transitive closure of the \clsmo\ would contain more or
less all classes of the current LISP image. For example, the slot
containing all subclasses of a class is omitted, since this would
store all classes of the current LISP image to the database (when
referencing the class' superclasses).  Since \plob\ can not store
binary function code, no methods are stored, too.  System created
methods, like the ones created by the \lisp{:accessor},
\lisp{:reader}\ or \lisp{:writer}\ slot options, are not stored as
methods, but will be generated when the class is compiled into the
current LISP image.

If a class is deleted from the class table by a call to
\fcite{p-delete-class}, it will be reinserted if an instance of the
class is referenced.

\refpar \Fcite{p-find-class}, \fcite{p-delete-class},
\fcite{p-class-of}, \fcite{p-apropos-classes}.

\section{Schema evolution}%
\label{sec:SchemaEvolution}

The default schema evolution for all structure classes is specified by
the value of \fcite{*default-structure-schema-evolution*}; for \clos\ 
classes, the value of \fcite{*default-clos-schema-evolution*}\ is
used.

\refpar \Fcite{(setf schema-evolution)},
\fcite{*default-structure-schema-evolution*},
\fcite{*default-clos-schema-evolution*}.

\section{Other useful functions}

\Fcite{p-apropos}\ will print out all persistent symbols containing
the passed string in their name:
\begin{IndentedCompactCode}
\listener{}(p-apropos "person")
common-lisp-user::*last-name->person-list*,
                  value: #<btree equal 9/678 short-objid=50329169>
common-lisp-user::person
common-lisp-user::person-age (defined)
common-lisp-user::person-first-name (defined)
common-lisp-user::person-last-name (defined)
common-lisp-user::person-name (defined)
common-lisp-user::person-occupation (defined)
common-lisp-user::person-sex (defined)
common-lisp-user::person-soc-sec-# (defined)
\end{IndentedCompactCode}

This is the output after file
\lisp{plob-\thisversion/}\lisp{}src/\lisp{example/}\lisp{plob-example.lisp}\ 
has been compiled and loaded and some random persons have been
generated (see \fcite{random-person}). A \lisp{(defined)} indicates
that a function binding has been done to the persistent symbol as
explained in section \Nameref{sec:BinaryCode}.

Functions \fcite*{p-statistics}\ and \fcite*{p-configuration}\ will
return some low level database informations.

Calling \fcite{p-stabilise}\ will force a flush of all pending file
operations in the server.

\refpar \Fcite{p-apropos}, \fcite{p-statistics}, \fcite{p-configuration},
\fcite{p-stabilise}.

\section{Performance}%
\label{sec:Performance}%
%
\begin{fortune}[7cm]%
Don't let the sun go down on me.
\from{Elton John}\\[\medskipamount]
\emph{[\plob\ 1.0 was developed on a SUN SPARCstation.]}
\end{fortune}%

\newlength{\codew}\setlength{\codew}{0.3\textwidth}%
\newlength{\perfw}\setlength{\perfw}{0.35\textwidth}%
\newlength{\timew}\setlength{\timew}{8\lispblank}%
\newcounter{perfcount}%
\newlength\codeblank%
\settowidth\codeblank{\texttt{\CompactCodeSize\ }}%
\def\timeboxi#1{\parbox[t]{\timew}{%
\hspace*{\fill}#1}\smallskip}%
\def\timeboxii#1#2{\parbox[t]{\timew}{%
S+\hspace*{\fill}#1\\%
S-\hspace*{\fill}#2}\smallskip}%
%%%
\noindent{}The equipment used for performance testing was a 200 MHz
Pentium-II PC with 64 MB RAM, standard EIDE Harddisk, Windows/NT 4.0
build 1381 service pack 3. The LISP system used was LispWorks Common
LISP 4.0.1.  The checked \plob\ version was 2.07. The numbers in the
very right column are operations per second. Those numbers marked `S+'
are with a server running on \lisp{localhost}\ (that means, on the
same machine as the LispWorks client using the server); the numbers
marked by `S-' are for the serverless version.  In the performance
measurements, care was taken that all operations involved a real
database access, that means, no cache was used for the storing or
loading operations.

\subsection{Sequences}

This section contains performance measurements for sequences, namely
strings, lists and vectors.

\begin{longtable}[c]{|r|p{\codew}|p{\perfw}|p{\timew}|}
%%%
  \hline%
  \textbf{Step} & \textbf{Code} & \textbf{Description} &
  \textbf{Op./s}\\
  \hline\hline\endhead
%%%
  \refstepcounter{perfcount}\theperfcount\label{step:step1} &
%%%
  \parbox[t]{\codew}{\begin{tt}\CompactCodeSize%
      (store-object\\
      \hspace*{2\codeblank}(make-string 32\\
      \hspace*{4\codeblank}:initial-element\\
      \hspace*{4\codeblank}\#\bslash{}Space))\end{tt}}\smallskip &
%%%
  \parbox[t]{\perfw}{%
    Store strings with a length of 32 characters.}\smallskip &
%%%
  \timeboxii{182}{401}\\
%%%
%%%
  \hline%
  \refstepcounter{perfcount}\theperfcount\label{step:step2} &
%%%
  \parbox[t]{\codew}{\begin{tt}\CompactCodeSize%
      (load-object\\
      \hspace*{2\codeblank}\emph{\lt{}objid\gt})\end{tt}}\smallskip &
%%%
  \parbox[t]{\perfw}{%
    Load strings with a length of 32 characters. \emph{\lt{}objid\gt}
    references one of the persistent strings as generated in the last
    performance measurement step.}\smallskip &
%%%
  \timeboxii{92}{467}\\
%%%
%%%
  \hline%
  \refstepcounter{perfcount}\theperfcount\label{step:step3} &
%%%
  \parbox[t]{\codew}{\begin{tt}\CompactCodeSize%
      (store-object\\
      \hspace*{2\codeblank}(make-list\\
      \hspace*{4\codeblank}\emph{\lt{}length\gt}\\
      \hspace*{4\codeblank}:initial-element\\
      \hspace*{4\codeblank}\emph{\lt{}fixnum\gt}))\end{tt}}\smallskip &
%%%
  \parbox[t]{\perfw}{%
    Store a linear list with \emph{\lt{}length\gt} elements, containing
    only immediates; Op./s is the number of stored list elements per
    second.}\smallskip &
%%%
  \timeboxii{560}{1666}\\
%%%
%%%
  \hline%
  \refstepcounter{perfcount}\theperfcount\label{step:step4} &
%%%
  \parbox[t]{\codew}{\begin{tt}\CompactCodeSize%
      (load-object\\
      \hspace*{2\codeblank}\emph{\lt{}objid\gt})\end{tt}}\smallskip &
%%%
  \parbox[t]{\perfw}{%
    Load a linear list; \emph{\lt{}objid\gt} references a persistent
    linear list as generated in step \ref{step:step1}. Op./s is the
    number of loaded list elements per second.}\smallskip &
%%%
  \timeboxii{414}{1428}\\
%%%
%%%
  \hline%
  \refstepcounter{perfcount}\theperfcount\label{step:step5} &
%%%
  \parbox[t]{\codew}{\begin{tt}\CompactCodeSize%
      (store-object\\
      \hspace*{2\codeblank}(make-array\\
      \hspace*{4\codeblank}\emph{\lt{}length\gt}\\
      \hspace*{4\codeblank}:initial-element\\
      \hspace*{4\codeblank}\emph{\lt{}float\gt}))\end{tt}}\smallskip &
%%%
  \parbox[t]{\perfw}{%
    Store a simple vector with \emph{\lt{}length\gt} elements,
    containing non-immediate \emph{\lt{}floats\gt}; Op./s is the number
    of stored vector elements per second.}\smallskip &
%%%
  \timeboxii{600}{4761} \\
%%%
%%%
  \hline%
  \refstepcounter{perfcount}\theperfcount\label{step:step6} &
%%%
  \parbox[t]{\codew}{\begin{tt}\CompactCodeSize%
      (load-object\\
      \hspace*{2\codeblank}\emph{\lt{}objid\gt})\end{tt}}\smallskip &
%%%
  \parbox[t]{\perfw}{%
    Load a simple vector; \emph{\lt{}objid\gt} references a persistent
    simple vector as generated in step \ref{step:step5}. Op./s is
    the number of loaded vector elements per second.}\smallskip &
%%%
  \timeboxii{2100}{2857}\\
%%%
%%%
  \hline%
  \refstepcounter{perfcount}\theperfcount\label{step:step7}&
%%%
  \parbox[t]{\codew}{\begin{tt}\CompactCodeSize%
      (store-object\\
      \hspace*{2\codeblank}(make-array\\
      \hspace*{4\codeblank}\emph{\lt{}length\gt}\\
      \hspace*{4\codeblank}:initial-element\\
      \hspace*{4\codeblank}\emph{\lt{}fixnum\gt}))\end{tt}}\smallskip &
%%%
  \parbox[t]{\perfw}{%
    Store a simple vector with \emph{\lt{}length\gt} elements,
    containing only immediates; Op./s is
    the number of stored vector elements per second.}\smallskip &
%%%
  \timeboxii{14000}{12500}\\
%%%
%%%
  \hline%
  \refstepcounter{perfcount}\theperfcount\label{step:step8}&
%%%
  \parbox[t]{\codew}{\begin{tt}\CompactCodeSize%
      (load-object\\
      \hspace*{2\codeblank}\emph{\lt{}objid\gt})\end{tt}}\smallskip &
%%%
  \parbox[t]{\perfw}{%
    Load a simple vector; \emph{\lt{}objid\gt} references a persistent
    simple vector as generated in step \ref{step:step7}. Op./s is
    the number of loaded vector elements per second.}\smallskip &
%%%
  \timeboxii{20895}{33333}\\
%%%
%%%
  \hline
%%%
\end{longtable}

Step \ref{step:step7} is the only performance test step where the
server mode outperforms the serverless mode. The only explanation I
have is that this performance test step has been done with a rather
big array containing 100000 elements, and that the blockwise (and
implicit concurrent) transfer of the server mode does a better job
here than the serverless mode, which does no blockwise or concurrent
transfer at all.

\subsection{CLOS instances}

Following CLOS class has been used for testing tight-bounded
persistency (section \Nameref{sec:TightBinding}):
\begin{IndentedCompactCode}
(defclass p-example-clos-class ()\\
  ((slot-3 :initform nil :extent :transient)\\
   (slot-5 :initform nil :extent :cached)\\
   (slot-7 :initform nil :extent :cached-write-through))\\
  (:metaclass persistent-metaclass))
\end{IndentedCompactCode}

An \lisp{:extent}\ of \lisp{:transient}\ means that the slot is
not represented in the database at all.  An \lisp{:extent}\ of
\lisp{:cached}\ means that the slot is represented in the database,
but will only be updated when explicitely requesting so.  An
\lisp{:extent}\ of \lisp{:cached-write-through}\ will update the
slot in the database each time it is written to; this is the default
extent used by \plob\ (section \Nameref{sec:AdditionalSlotOptions}).

\begin{longtable}[c]{|r|p{\codew}|p{\perfw}|p{\timew}|}
%%%
  \hline
  \textbf{Step} & \textbf{Code} & \textbf{Description} &
  \textbf{Op./s} \\ 
  \hline\hline\endhead
%%%
  \refstepcounter{perfcount}\theperfcount\label{step:step9}&
%%%
  \parbox[t]{\codew}{\begin{tt}\CompactCodeSize%
      (make-instance\\
      \hspace*{2\codeblank}'p-example-clos-\\
      \hspace*{3\codeblank}class)\end{tt}}\smallskip &
%%%
  \parbox[t]{\perfw}{%
    Allocation of a persistent instance and initialization of their slots
    according to their \lisp{:initform}.}\smallskip &
%%%
  \timeboxii{165}{285}\\
%%%
%%%
  \hline%
  \refstepcounter{perfcount}\theperfcount\label{step:step10} &
%%%
  \parbox[t]{\codew}{\begin{tt}\CompactCodeSize%
      (load-object\\
      \hspace*{2\codeblank}\emph{\lt{}objid\gt})\end{tt}}\smallskip &
%%%
  \parbox[t]{\perfw}{%
    Load of a persistent instance; \emph{\lt{}objid\gt} references a
    persistent instance as generated in step
    \ref{step:step9}.}\smallskip &
%%%
  \timeboxii{240}{344}\\
%%%
%%%
  \hline%
  \refstepcounter{perfcount}\theperfcount\label{step:step11}&
%%%
  \parbox[t]{\codew}{\begin{tt}\CompactCodeSize%
      (setf (slot-value
      \hspace*{7\codeblank}\emph{\lt{}object\gt}\\
      \hspace*{7\codeblank}'slot-3)\\
      \hspace*{2\codeblank}\emph{\lt{}fixnum\gt})\end{tt}}\smallskip &
%%%
  \parbox[t]{\perfw}{%
    Set a \lisp{:transient}\ slot of a tight-bounded persistent class
    (implies no storing of the slot's value to the database).
    \emph{\lt{}object\gt} references a persistent instance as generated
    in step \ref{step:step9}.}\smallskip &
%%%
  \timeboxi{21000}\\
%%%
%%%
  \hline%
  \refstepcounter{perfcount}\theperfcount\label{step:step12}&
%%%
  \parbox[t]{\codew}{\begin{tt}\CompactCodeSize%
      (slot-value \emph{\lt{}object\gt}\\
      \hspace*{2\codeblank}'slot-3)\end{tt}}\smallskip &
%%%
  \parbox[t]{\perfw}{%
    Load the value of a \lisp{:transient}\ slot of a tight-bounded
    persistent class (implies no loading of the slot's value from the
    database).  \emph{\lt{}object\gt} references a persistent instance
    as generated in step \ref{step:step9}.}\smallskip &
%%%
  \timeboxi{29400}\\
%%%
%%%
  \hline%
  \refstepcounter{perfcount}\theperfcount\label{step:step13} &
%%%
  \parbox[t]{\codew}{\begin{tt}\CompactCodeSize%
      (setf (slot-value\\
      \hspace*{7\codeblank}\emph{\lt{}object\gt}\\
      \hspace*{7\codeblank}'slot-5)\\
      \hspace*{2\codeblank}\emph{\lt{}fixnum\gt})\end{tt}}\smallskip &
%%%
  \parbox[t]{\perfw}{%
    Set a \lisp{:cached}\ slot of a tight-bounded persistent class
    (implies no storing of the slot's value to the database).
    \emph{\lt{}object\gt} references a persistent instance as generated
    in step \ref{step:step9}.}\smallskip &
%%%
  \timeboxi{22000}\\
%%%
%%%
  \hline%
  \refstepcounter{perfcount}\theperfcount\label{step:step14} &
%%%
  \parbox[t]{\codew}{\begin{tt}\CompactCodeSize%
    (slot-value \emph{\lt{}object\gt}\\
    \hspace*{2\codeblank}'slot-5)\end{tt}}\smallskip &
%%%
  \parbox[t]{\perfw}{%
    Load the value of a \lisp{:cached}\ slot of a tight-bounded
    persistent class (implies no loading of the slot's value from the
    database).  \emph{\lt{}object\gt} references a persistent instance
    as generated in step \ref{step:step9}.}\smallskip &
%%%
  \timeboxi{31300}\\
%%%
%%%
  \hline%
  \refstepcounter{perfcount}\theperfcount\label{step:step15}&
%%%
  \parbox[t]{\codew}{\begin{tt}\CompactCodeSize%
      (setf (slot-value\\
      \hspace*{7\codeblank}\emph{\lt{}object\gt}\\
      \hspace*{7\codeblank}'slot-7)\\
      \hspace*{2\codeblank}\emph{\lt{}fixnum\gt})\end{tt}}\smallskip &
%%%
  \parbox[t]{\perfw}{%
    Set a \lisp{:cached-write-through}\ slot of a tight-bounded
    persistent class to an immediate fixnum.  \emph{\lt{}object\gt}
    references a persistent instance as generated in step
    \ref{step:step9}.}\smallskip &
%%%
  \timeboxii{384}{1315}\\
%%%
%%%
  \hline%
  \refstepcounter{perfcount}\theperfcount\label{step:step16}&
%%%
  \parbox[t]{\codew}{\begin{tt}\CompactCodeSize%
      (slot-value \emph{\lt{}object\gt}\\
      \hspace*{2\codeblank}'slot-7)\end{tt}}\smallskip &
%%%
  \parbox[t]{\perfw}{%
    Load the value of a \lisp{:cached-write-through}\ slot of a
    tight-bounded persistent class. The value loaded is an immediate
    fixnum.  \emph{\lt{}object\gt} references a persistent instance as
    generated in step \ref{step:step9}.}\smallskip &
%%%
  \timeboxi{31300}\\
%%%
%%%
  \hline
%%%
\end{longtable}

For steps \ref{step:step11}--\ref{step:step14} and \ref{step:step16}
there are no performance differences between the serverless and server
mode, because no data transfer between the client and the database is
involved (for these steps, the LISP layer is never left).

\subsection{Database-specific functions}

\begin{longtable}[c]{|r|p{\codew}|p{\perfw}|p{\timew}|}
%%%
  \hline
  \textbf{Step} & \textbf{Code} & \textbf{Description} &
  \textbf{Op./s} \\ 
  \hline\hline\endhead
%%%
  \refstepcounter{perfcount}\theperfcount\label{step:step17}&
%%%
  \parbox[t]{\codew}{\begin{tt}\CompactCodeSize%
      (p-allocate-cons)\end{tt}}\smallskip &
%%%
  \parbox[t]{\perfw}{%
    Allocation of an empty persistent cons cell.}\smallskip &
%%%
  \timeboxii{1639}{9090}\\
%%%
%%%
  \hline%
  \refstepcounter{perfcount}\theperfcount\label{step:step18} &
%%%
  \parbox[t]{\codew}{\begin{tt}\CompactCodeSize%
      (with-transaction ()\\
      \hspace*{2\codeblank}nil)\end{tt}}\smallskip &
%%%
  \parbox[t]{\perfw}{%
    Empty transaction.}\smallskip &
%%%
  \timeboxii{770}{5000}\\
%%%
%%%
  \hline%
  \refstepcounter{perfcount}\theperfcount\label{step:step19} &
%%%
  \parbox[t]{\codew}{\begin{tt}\CompactCodeSize%
    (with-transaction ()\\
    \hspace*{2\codeblank}(write-lock-store)\\
    \hspace*{2\codeblank}nil)\end{tt}}\smallskip &
%%%
  \parbox[t]{\perfw}{%
    Empty transaction locking the whole database (this performance is
    comparable to the performance of single-object level
    locking).}\smallskip &
%%%
  \timeboxii{476}{1818}\\
%%%
%%%
  \hline
%%%
\end{longtable}

\subsection{Different usage of transactions and locking}

In this section, the performance is shown for the single operation of
step \ref{step:step9}, but with different usage of transactions
and object locking.  The default behavior of \plob\ as shown in
step \ref{step:step9} involves one transaction for each call to
\fcite{make-instance}\ and the (implicit) usage of some locks at
storing the object. Using transactions and locks on bigger `chunks' of
objects improves performance for a single user significantly, but will
make the `granularity' of object access more coarse, making concurrent
access slower.

\begin{longtable}[c]{|r|p{\codew}|p{\perfw}|p{\timew}|}
%%%
  \hline%
  \textbf{Step} & \textbf{Code} & \textbf{Description} &
  \textbf{Op./s} \\
  \hline\hline\endhead
%%%
  \ref{step:step9} &
  \parbox[t]{\codew}{\begin{tt}\CompactCodeSize%
      (make-instance\\
      \hspace*{2\codeblank}'p-example-clos-\\
      \hspace*{3\codeblank}class)\end{tt}}\smallskip &
%%%
  \parbox[t]{\perfw}{%
    Allocation of a persistent instance and initialization of their
    slots according to their \lisp{:initform}\ (this is step
    \ref{step:step9} repeated here for completeness).}\smallskip &
%%%
  \timeboxii{165}{285}\\
%%%
%%%
  \hline%
  \refstepcounter{perfcount}\theperfcount\label{step:step20}&
%%%
 \parbox[t]{\codew}{\begin{tt}\CompactCodeSize%
     (with-transaction ()\\
     \hspace*{2\codeblank}(dotimes (i 1000)\\
     \hspace*{4\codeblank}(make-instance\\
     \hspace*{6\codeblank}'p-example-clos-\\
     \hspace*{7\codeblank}class)))\end{tt}}\smallskip &
%%%
  \parbox[t]{\perfw}{%
    1000 instances of step \ref{step:step9} generated in a single
    transaction; Op./s is the number of stored instances per
    second.}\smallskip &
%%%
 \timeboxii{175}{322}\\
%%%
%%%
 \hline%
 \refstepcounter{perfcount}\theperfcount\label{step:step21} &
%%%
 \parbox[t]{\codew}{\begin{tt}\CompactCodeSize%
     (with-transaction ()\\
     \hspace*{2\codeblank}(write-lock-store)\\
     \hspace*{2\codeblank}(dotimes (i 1000)\\
     \hspace*{4\codeblank}(make-instance\\
     \hspace*{6\codeblank}'p-example-clos-\\
     \hspace*{7\codeblank}class)))\end{tt}}\smallskip &
%%%
  \parbox[t]{\perfw}{%
    1000 instances of step \ref{step:step9} generated in a single
    transaction with an exclusive write lock set onto the whole
    database; Op./s is the number of stored instances per
    second.}\smallskip &
%%%
 \timeboxii{270}{357}\\
%%%
%%%
  \hline
%%%
\end{longtable}

\subsection{File space allocation}

This section explains how much file space is allocated by \plob\ for a
persistent object. All persistent objects are stored in a
(memory-mapped) file; this is the file named \lisp{stablestore}\ in
each database directory.  \hyperlink{link:Immediate}{Immediates}\ use
up no file space at all.

\subsubsection[Non-CLOS objects]{Non-\protect\clos\ objects}

Non-immediates, non-\clos\ objects and \lisp{defstruct}\ objects have
an overhead of 6 words $\equiv$ 24 bytes. The total file space
allocated for such an object is the sum of the overhead plus 1 word
$\equiv$ 4 bytes per `slot'. For example, a cons cell has 2 slots
(\lisp{car}\ and \lisp{cdr}) and will take up a total of 6 + 2 = 8
words $\equiv$ 24 + 8 = 32 bytes.

\subsubsection[CLOS objects]{\protect\clos\ objects}

For \clos\ instances, the overhead is 14 words $\equiv$ 56 bytes plus
1 word $\equiv$ 4 bytes per slot. For example, a \clos\ instance with
2 slots will take up a total of 14 + 2 = 16 words $\equiv$ 56 + 8 = 64
bytes.

\section{Technical data}%
\label{sec:TechnicalData}

%
\begin{deflist}[XXXXXXXXXXXXXXX]

\item[Architecture] Client/server

\item[Operating systems] Solaris 2.x, Linux kernel version 2.x,
  Windows XP. IRIX 6.x is  supported on request.

\item[LISP systems] \href{http://www.lispworks.com}{\lwcl}\ 4.4,
  \href{http://www.franz.coml}{\allegrocl}\ 7

\item[Databases per Server] Any reasonable number of open databases,
  depending on installed memory.

\item[Databases per Client] 1

\item[Multi user access] Multiple clients can connect to one server
  process.

\item[Object space size] Maximum of 384 MB per database. This is a
  technical limit of the low-level persistent storage used for the
  server.

\item[Access control] Login deny/allow on per-machine basis

\item[Lock protocol] Hierarchical locking.

\item[Lock levels] Whole storage, object, slot.

\item[Lock modes] Read-only, read, write; read-only intent, read
  intent, write intent

\item[Transactions] Flat transactions. Concurrent transactions are
  supported, both between different clients and between different
  database sessions of a single client.

\item[Access protocol] Two phased transactions and locking of the
  persistent object at object state transfer between client and
  server.

\item[Indexes supported] Btrees on slots of persistent \clos\ 
  instances are maintained automatically; direct usage of btrees is
  possible.

\item[LISP types] All types defined in \cite{bib:CLtLII} are
  supported, with the exception of binary function code.

\item[Name spaces] Persistent packages containing persistent symbols.

\end{deflist}


%%% Local Variables: 
%%% mode: latex
%%% TeX-master: "userg"
%%% End: 
