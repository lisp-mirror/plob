% This is -*-LaTeX-*-
%
% manrefg.tex
% HK 27.4.94
%
% This is -*-LaTeX-*-
%

\title{Persistent LISP Objects\\
\plob\\[\bigskipamount]
\thistitle\\[\bigskipamount]}

\maketitle

\clearpage\thispagestyle{empty}

\vspace*{\fill}%
\noindent\plob\ Copyright \copyright\ 1994--\thisyear\ \thisauthor. All
rights reserved.

\vspace*{0.5cm}%
\noindent{}See \fcite{WWW and email addresses}\ for contact addresses
to the author.

\vspace*{0.5cm}%
\noindent{}\plob\ uses the \postore\ library (Persistent Object Store)
by permission of the Persistent Programming Research Group at St.\ 
Andrews University.

\vspace*{0.5cm}%
\noindent{}This product includes software developed by the University of
California, Berkeley and its contributors.

\vspace*{0.5cm}%
\noindent{}Part number PLOB--\thispartnumber--\thisversion. This
document covers \plob\ version \thisversion\ released at \thisdate.

\setcounter{page}{2}\clearpage

%%% Local Variables: 
%%% mode: latex
%%% TeX-master: "manual"
%%% End: 


\pagenumbering{roman}\pagestyle{headings}
\ifx\mytocdepth\undefined\relax\else%
\settocdepth{\mytocdepth}%
\fi%
\ifx\notableofcontents\undefined%
\tableofcontents\clearpage%
\thispagestyle{plain}\cleardoublepage%
\listoffigures%
\fi%
\clearpage\thispagestyle{plain}\cleardoublepage%

\pagenumbering{arabic}%

\BetterSloppy

\chapter{Introduction}

This chapter describes all constants, variables, macros, classes,
functions, generic functions and methods of
\plob\ifx\thistitle\undefined\relax\else{}'{}s \thistitle\fi\ in
alphabetical order. The style of presentation and the notation used in
this chapter are `borrowed' and extended from chapter~6 of
\cite{bib:AMOP}; see \cite[\citepage{163}]{bib:AMOP} for details
concerning the notation used in this chapter for [generic] functions
and methods.  Here are the extensions and changes introducted to the
style of chapter~6 of \cite{bib:AMOP}:
\begin{itemize}
\item Constants, variables, macros and classes are documented too.
\item Since this chapter does not only describe the external interface
  but also all internal entities, there is an addition to each section
  header stating whether the entity is {\it external} or {\it
    internal}.  The internal entities are described here to document
  the \plob\ system itself.
\item Added to each section is a subsection named `{\sc See Also}'
  with references to other entities; the references are ordered
  according to their importance.
\item References are given by an identifier naming the kind of entity,
  the symbol naming the entity and a page resp.\ bibliographic
  reference in square brackets, e.g.: \Fcite{make-btree}.
\item If a form has a {\bf (setf)} equivalent, the {\bf(setf)} form is
  placed right behind the non-{\bf(setf)} form, e.g.\ the description
  of \fcite{(setf btree-test)}\ is placed behind the description of
  \fcite{btree-test}.
\end{itemize}

\section{Naming conventions}

The following naming conventions are often obeyed for function resp.\ 
argument names; the subsections are ordered from low-level (i.e.\ more
C-like) to high-level (i.e.\ more \cl-like) name prefixes.

\subsection{Prefix `c-sh-'}

This is the prefix used by \plob\ for the \cl\ functions defined by
the foreign function interface of \lw. The `c-sh-'\footnote{`c-sh-' is
  an abbrevation for `C level of \sh-'.} functions are the `direct'
interface functions to the C level; their $\lambda$-lists are very
C-like. The arguments taken by these functions are of simple types.

These functions are generated from the C header files whose
names match {\tt c-plob*.h} by C preprocessor macro expansion; they
are not documented here.

\subsection{Prefix `sh-'}

Each function with a `c-sh-' prefix has a documented corresponding
function with a `sh-'\footnote{`sh-' is an abbrevation for `\sh-'.}
prefix; they make up a more \cl-like interface, e.g.\ they use {\opt}
and {\key{}} parameters or transform between programming constructs
which are more C-like and \cl\ programming constructs, like mapping
argument values from symbols to C level \lisp{enum} constants and vice
versa or composing \cl\ types from obtained values of the C level.

Although already offering a suitable \cl\ interface, their
functionality w.r.t.\ \cl\ is still relatively simple, i.e.\ `they do
nothing more than transforming arguments to resp.\ return values from
the level below'.

\subsection{Prefix `p-'}

The meaning of the `p-'\footnote{`p-' is an abbrevation for
  `persistent-'.} prefix depends on its context:
\begin{description}
\item[As a function name:] The functions named with a `p-' prefix are
  the high-level interface to \plob; many of them are exported.
  Functions whose names are composed of the prefix `p-' with a name
  specified in \cite{bib:CLtLII} do quite the same, but the `p-'
  function works on a persistent object, e.g.\ the \fcite{p-car}\ 
  returns the car of a persistent cons cell.
\item[As a variable name:] The value of variables whose names start
  with `p-' is a persistent object; this object is perhaps not
  represented at all in transient memory. The \objid\ of such an
  object is a handle passed down to the C level to reference it.
\end{description}

\subsection{Prefix `t-'}

Prefix 't-'\footnote{`t-' is an abbrevation for `transient-'.} is used
for names having to do with pure-transient objects. This prefix is
used for function and argument names at places where it is necessary
to distinguish exactly betweeen the transient and persistent
representation of an object.

\section{Standard function arguments}

Some function arguments are specified for a lot of \plob\ functions;
these are mostly the {\opt} arguments \funarg{p-heap}\ and
\funarg{depth}. Because they are used so often and their meaning is
always the same, they are explained here and not at every function
which specifies them.

\subsection[Function argument p-heap]%
{Standard function argument \protect\funarg{p-heap}}

For the {\opt} \funarg{p-heap}\ argument with a default value of {\bf
  *default-persistent-heap*} see the description of
\fcite{*default-persistent-heap*}\ and of \fcite{persistent-heap}.

\subsection[Function argument depth]%
{Standard function argument \protect\funarg{depth}}

When the function using the \funarg{depth}\ argument is a storing
function, i.e.\ a function which transfers a transient object into the
persistent memory, like {\bf store-object}, {\bf t-object-to-p-objid}
or {\bf (setf sub-object)}, the meaning of the \funarg{depth} argument
is as specified for the \fcite{store-object}.

When the function using the \funarg{depth}\ argument is a loading
function, i.e.\ a function which transfers a persistent object into
its transient representation, like {\bf load-object}, {\bf
  p-objid-to-t-object} or {\bf sub-object}, the meaning of the
\funarg{depth} argument is as specified for the \fcite{load-object}.

\sloppy\numberingoff%
\cleardoublepage%
\chapter[Reference Guide]{\protect\thistitle\ Reference Guide}%
