%       This is -*-LaTeX-*-
%
% diagdefs.tex
% 21.09.1993 Heiko Kirschke, Fachbereich Informatik, Uni Hamburg
% e-mail: kirschke@informatik.uni-hamburg.de
%
% Enthaelt diverse Definitionen fuer Diagramme
%
%
% Masse
%
% Rand um Objekte in Diagrammen:
\newlength{\stdomargin}%
\setlength{\stdomargin}{0.6em}%
%
% Breite eines Textkastens:
\newlength{\ucrtxtw}%
\setlength{\ucrtxtw}{4em}                       % sehr schmal
\newlength{\smltxtw}%
\setlength{\smltxtw}{8em}                       % schmal
\newlength{\stdtxtw}%
\setlength{\stdtxtw}{10em}                      % normal
\newlength{\bigtxtw}%
\setlength{\bigtxtw}{12em}                      % breit
%
% Masse fuer strid*.dgr:
%
% Spaltenbreiten
\let\cstiw\stdtxtw%                     % Breite erste Spalte
\def\cstiiw{5.5em}%                     % Breite zweite Spalte
\let\cstiiiw\smltxtw%                   % Breite dritte Spalte
%
\newlength{\ocstiw}\setlength{\ocstiw}{\cstiw}%
\addtolength{\ocstiw}{3\stdomargin}%
\newlength{\ocstiiw}\setlength{\ocstiiw}{\cstiiw}%
\addtolength{\ocstiiw}{3\stdomargin}%
\newlength{\ocstiiiw}\setlength{\ocstiiiw}{\cstiiiw}%
\addtolength{\ocstiiiw}{3\stdomargin}%
%
% 'Sonderzeichen'
%
\def\mkdasharrow{%
\ifx\dasharrow\undefined%
\newsavebox{\dashabox}\sbox{\dashabox}{%
\spreaddiagramcolumns{-1pc}%
\begin{diagram}\xdashed[0,1]|>{\tip}&\end{diagram}}%
\def\dasharrow{\usebox{\dashabox}}%
\fi}%
%
\def\mkddotarrow{%
\ifx\ddotarrow\undefined%
\newsavebox{\ddotabox}\sbox{\ddotabox}{%
\spreaddiagramcolumns{-1pc}%
\begin{diagram}\xdotted[0,1]|<{\rotate\tip}|>{\tip}&\end{diagram}}%
\def\ddotarrow{\usebox{\ddotabox}}%
\fi}%
%
\def\mkdotdarrow{%
\ifx\dotdarrow\undefined%
\newsavebox{\dotdabox}\sbox{\dotdabox}{%
\spreaddiagramcolumns{-1pc}%
\begin{diagram}\xdotted[0,1]|>{\tip}&\end{diagram}}%
\def\dotdarrow{\usebox{\dotdabox}}%
\fi}%
%
\def\mklinearrow{%
\ifx\linearrow\undefined%
\newsavebox{\lineabox}\sbox{\lineabox}{%
\spreaddiagramcolumns{-1pc}%
\begin{diagram}\xto[0,1]&\end{diagram}}%
\def\linearrow{\usebox{\lineabox}}%
\fi}%
%
\def\stolu{%
\xto~{`[0,0]+<-\cL,0ex>[0,0]+<-\cL,\cH>+<8pt,0ex>}%
`l+<0em,8pt>`[0,0]+<0em,8pt>+<0em,10pt>}%
%
\newcommand{\nonull}{\raisebox{0ex}[0ex][0ex]{\onull}}%
\newcommand{\noi}{\raisebox{0ex}[0ex][0ex]{\oi}}%
\newcommand{\noii}{\raisebox{0ex}[0ex][0ex]{\oii}}%
\newcommand{\noiii}{\raisebox{0ex}[0ex][0ex]{\oiii}}%
\newcommand{\noiv}{\raisebox{0ex}[0ex][0ex]{\oiv}}%
\newcommand{\nov}{\raisebox{0ex}[0ex][0ex]{\ov}}%
\newcommand{\novi}{\raisebox{0ex}[0ex][0ex]{\ovi}}%
\newcommand{\novii}{\raisebox{0ex}[0ex][0ex]{\ovii}}%
\newcommand{\noiix}{\raisebox{0ex}[0ex][0ex]{\oiix}}%
\newcommand{\noix}{\raisebox{0ex}[0ex][0ex]{\oix}}%
%
\let\subclassof\xline%
\let\stip\tip%
\let\instanceof\xdouble%
\let\itip\Tip%
%
% Legenden
%
\def\mkhleg{%                           % Box mit Kl.hier.-Legende
\spreaddiagramrows{-2pc}%
\spreaddiagramcolumns{-2pc}%
\spreaddiagramcolumns{2pt}%
\footnotesize\begin{diagram}%
\text{\textrm{$\uparrow$ Subklasse von}}
 & \text{\textrm{$\Uparrow$ Instanz von}} \\%
\dgmc{\textrm{\mc}} & \dgc{\textrm{\cls}} & \dgi{\textrm{Instanz}}%
\end{diagram}}%
%
\def\mkmleg{%                           % Box mit Kl.hier.-Legende
\spreaddiagramrows{-2pc}%
%\spreaddiagramrows{4pt}%
\spreaddiagramcolumns{-2pc}%
%\spreaddiagramcolumns{2pt}%
\footnotesize\begin{diagram}%
\dgc{\textrm{\std-\cls}}\\%
\text{\textrm{$\uparrow$ Subklasse von}}\\%
\dgmc{\textrm{\dec}}%
\end{diagram}}%
%
\def\mknleg{%                           % Box mit Kl.hier.-Legende
\spreaddiagramrows{-2pc}%
%\spreaddiagramrows{4pt}%
\spreaddiagramcolumns{-2pc}%
%\spreaddiagramcolumns{2pt}%
\footnotesize\begin{diagram}%
\dgc{\textrm{\std-}\\{}\textrm{\mc}}\\%
\text{\textrm{$\uparrow$ Subklasse von}}\\%
\dgmc{\textrm{\Spc}\\{}\textrm{\mc}}%
\end{diagram}}%
%
\def\mksleg{%                           % Box mit Kl.hier.-Legende
\spreaddiagramrows{-2pc}%
%\spreaddiagramrows{4pt}%
\spreaddiagramcolumns{-2pc}%
%\spreaddiagramcolumns{2pt}%
\footnotesize\begin{diagram}%
\dgc{\textrm{Struktur-\cls}}\\%
\text{\textrm{$\uparrow$ Subklasse von}}\\%
\dgmc{\textrm{\dec}}%
\end{diagram}}%
%
% diagram: Metaobjekt-Klasse
\def\dgmc#1{\Framed<5pt>\grow{\Text{#1}}}%
% diagram: Klasse
\def\dgc#1{\framed<5pt>\grow{\Text{#1}}}%
% diagram: Instanz
\def\dgi#1{\framed\grow{\Text{#1}}}%
%
\def\mkcleg{%                   % Box mit Klassen-Legende
\spreaddiagramrows{-2pc}%
\spreaddiagramcolumns{-2pc}%
\footnotesize\begin{diagram}%
\text{\textrm{Instanz\ }}
                &\framed<5pt>{\parbox{6em}{\centering\smallskip%
                        Klassenname\\[-\smallskipamount]
                        \makebox[6em]{\hrulefill}\\[\smallskipamount]
                        \slt[s]\smallskip}}
\end{diagram}}%
%
\def\mktleg{%                   % Box mit Tupel-Legende
\spreaddiagramrows{-2pc}%
\spreaddiagramcolumns{-2pc}%
\footnotesize\begin{diagram}%
\text{\textrm{Tupel\ }}
                &\framed<5pt>{\parbox{12em}{\centering\smallskip%
                        Name der Tupel-Relation\\[-\smallskipamount]
                        \makebox[10em]{\hrulefill}\\[\smallskipamount]
                        Identifizierendes Attribut\\[-\smallskipamount]
                        \makebox[11em]{\dotfill}\\[\smallskipamount]
                        Sonstige Attribute\smallskip}}
\end{diagram}}%
%

%%% Local Variables: 
%%% mode: latex
%%% TeX-master: "~/plob/tex/diplom/main"
%%% End: 
