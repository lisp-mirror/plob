% -*-LaTeX-*-
\ifx\Path\undefined%
\def\Path{.}%
\fi%
\ifx\labeltabextent\undefined\def\labeltabextent{tab:extent}\fi
\begin{figure}[htbp]\centering%
\begin{small}%
\def\coli{25ex}%
\def\colii{0.95\textwidth}%
\begin{tabular}{|p{\coli}ccl|}
\hline
\textbf{Value of slot}
        & \multicolumn{2}{|c|}{\textbf{Represent.}}&\\
\textbf{option \lisp{:extent}}
        & \multicolumn{1}{|c|}{\textbf{Tr.}}
        & \multicolumn{1}{|c|}{\textbf{Pe.}}
        & \textbf{Explanation}\\

\hline\hline
{\lisp{:transient}} & Yes & No &\\
\multicolumn{4}{|p{\colii}|}{%
The slot will only be represented in transient memory but will not be
represented in persistent memory. This slot is `ignored' by
\plob\ w.r.t.\ persistency.}\\

\hline
\lisp{:cached} & Yes & Yes &\\
\multicolumn{4}{|p{\colii}|}{%
The slot will be represented both in transient and persistent memory.
Slot reading is done from the transient slot's state.
Slot writing only affects the transient slot's state;
no changes of the transient representation will be propagated
automatically to persistent memory. The slot's state will only be
promoted to persistent memory by a call to \fcite{store-object}\ on
the \clos\ instance the slot belongs to.}\\

\hline
\lisp{:cached-write-through} & Yes & Yes &\\
\multicolumn{4}{|p{\colii}|}{%
The slot will be represented in transient and persistent memory.
Slot reading is done from the transient slot's state.
Changes of the transient representation will be propagated
automatically and immediately to persistent memory.}\\

\hline
\lisp{:persistent} & No & Yes &\\
\multicolumn{4}{|p{\colii}|}{%
The slot will only be represented in persistent memory but will not be
represented in transient memory. Reading is done directly from the
persistent memory. For writing, this slot extent implies that changes
to the slot's state will be propagated automatically to persistent
memory. Use of this extent is depreciated, since it is very resource
consuming.}\\

\hline
\lisp{:object} & Yes & Yes &\\
\multicolumn{4}{|p{\colii}|}{%
This extent indicates that a direct persistent representation should
be used for the slot's state: The slot will not contain a transient
representation of a persistent object, but a forward object
referencing the persistent object.}\\

\hline
\lisp{:objid} & Yes & Yes &\\
\multicolumn{4}{|p{\colii}|}{%
This extent indicates that a direct persistent representation should
be used for the slot's state: The slot will not contain a
transient representation of a persistent object, but a numeric \objid\
referencing the persistent object.}\\

\hline
\end{tabular}\end{small}%
\caption{Slot extent, transient and persistent object representation}%
\label{\labeltabextent}
\end{figure}%

\ifx\labelfigextent\undefined\def\labelfigextent{fig:extent}\fi
\vspace*{\bigskipamount}\noindent Figure~\ref{\labelfigextent} shows
how the slot extent affects the slot representation and promoting of
the slot's state between the transient and persistent representation
of an object; the structure and class definitions of this example are:
\begin{CompactCode}
(defstruct persistent-example-structure slot-1 slot-2)
(setf (class-extent (find-class 'persistent-example-structure))
      :cached-demand-load)

(defclass persistent-example-class ()
  ((slot-3  :extent :transient)
   (slot-4  :extent :transient)
   (slot-5  :extent :cached)
   (slot-6  :extent :cached)
   (slot-7  :extent :cached-write-through)
   (slot-8  :extent :cached-write-through)
   (slot-9  :extent :persistent)
   (slot-10 :extent :persistent))
  (:metaclass persistent-metaclass))
\end{CompactCode}
\begin{figure}[htbp]
\centerline{\psfig{figure=\Path/../manual/manext.eps}}
\caption{Transient object, persistent object and slot extent}%
\label{\labelfigextent}
\end{figure}%
