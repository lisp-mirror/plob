% This is -*-LaTeX-*-
%
% manual.tex
% HK 27.4.94
%
%
% Switches
%
% Fuer Vorab-Drucke \nodiagram, \notableofcontents ein- und \nothink
% ausschalten:
%       \def\notableofcontents{}% Um Inhaltsverzeichnis auszuschalten
%       %\def\nothink{}         % Um die Kommentare sichtbar zu machen
% Fuer druckfertige Version \nodiagram, \notableofcontents aus- und \nothink
% einschalten:
%       %\def\notableofcontents{}% Um Inhaltsverzeichnis einzuschalten
%       \def\nothink{}          % Um die Kommentare verschwinden zu lassen
%
%\def\notableofcontents{}       % Unterdrueckt Inhaltsverzeichnis
%\nofiles%
%
% One-sided manual:
%\documentclass[a4paper,wide]{pretty}%
% Two-sided manual:
%
% This is -*-LaTeX-*-
%
 



































% This file was generated from plobversion.h;
% changes done here will be lost!


 








 
 























 







































 





































   \def\thisversion{2.09}            
   \def\thisday{22}            
   \def\thismonth{May}            
   \def\thisyear{2000}            

   \def\thisauthor{Heiko Kirschke}            
   \def\thisemail{Heiko.Kirschke@acm.org}            
   \def\thisurl{www.lisp.de/software/plob/}            

 







 





\author{\thisauthor}%
%
\def\thissubject{Persistent LISP Objects %
\thisversion\space of \thisday.\space\thismonth\space\thisyear}%
%
\def\thiskeywords{persistency,database,persistent objects,LISP,CLOS,MOP}%
%
%\date{6.\ September 1994}%
%\date{30.\ October 1996}%
%\date{18.\ March 1997}%
%\date{14.\ April 1997}%
%\date{9.\ July 1997}%
%\date{5.\ August 1997}%
%\date{27.\ January 1998}%
%\date{6.\ February 1998}%
%\date{9.\ March 1998}%
%\date{12.\ March 1998}% 2.04
%
\def\thisdate{\thisday.\ \thismonth\ \thisyear}%
\date{\thisdate}%
\def\thiswww{http://\thisurl}
%
%
\def\WideOption{wide}
%
% 1998/01/15 HK
%
\ifx\NoOfSides\undefined%
\def\NoOfSides{oneside}%
\fi%
\ifx\WideOption\undefined%
\def\WideOption{narrow}%
\fi%
\ifx\ClassName\undefined
\def\ClassName{pretty}%
\fi%
%
\ifx\Path\undefined%
\def\Path{.}%
\fi%
%
\def\OneSideToken{oneside}
\def\TwoSideToken{twoside}
\def\PrettyToken{pretty}
%
\documentclass[a4paper,\NoOfSides,\WideOption]{\ClassName}%
%
% This is -*-LaTeX-*-
%
 



































% This file was generated from plobversion.h;
% changes done here will be lost!


 








 
 























 







































 





































   \def\thisversion{2.09}            
   \def\thisday{22}            
   \def\thismonth{May}            
   \def\thisyear{2000}            

   \def\thisauthor{Heiko Kirschke}            
   \def\thisemail{Heiko.Kirschke@acm.org}            
   \def\thisurl{www.lisp.de/software/plob/}            

 







 





\author{\thisauthor}%
%
\def\thissubject{Persistent LISP Objects %
\thisversion\space of \thisday.\space\thismonth\space\thisyear}%
%
\def\thiskeywords{persistency,database,persistent objects,LISP,CLOS,MOP}%
%
%\date{6.\ September 1994}%
%\date{30.\ October 1996}%
%\date{18.\ March 1997}%
%\date{14.\ April 1997}%
%\date{9.\ July 1997}%
%\date{5.\ August 1997}%
%\date{27.\ January 1998}%
%\date{6.\ February 1998}%
%\date{9.\ March 1998}%
%\date{12.\ March 1998}% 2.04
%
\def\thisdate{\thisday.\ \thismonth\ \thisyear}%
\date{\thisdate}%
\def\thiswww{http://\thisurl}
%
%
%
\usepackage[english]{babel}
\usepackage{epsfig}
\usepackage{deflist}
\usepackage{avb}
\usepackage{longtable}
\ifx\ClassName\PrettyToken%
\def\usepackagelispdoc{\usepackage{lispdoc}}%
\else%
\let\usepackagelispdoc\relax%
\fi%
\usepackagelispdoc%
\let\usepackagelispdoc\undefined%
\usepackage{crossref}%
\usepackage{dipldefs}
\usepackage{iconpar}
\usepackage{readaux}
\usepackage{cdnamed}
\usepackage{plobbib}
%\usepackage{think}
\usepackage[nothink]{think}
%
\ifx\NoOfSides\TwoSideToken%
%
\def\usepackagetimes{\usepackage{timestt}}%
%
\def\usepackagehyperref{\usepackage[breaklinks,draft]{hyperref}}
%
\let\BetterSloppy\relax
%
\else%
%
\def\usepackagetimes{\usepackage{times}}%
%
\def\usepackagehyperref{\usepackage[%
ps2pdf,%
%pdftex,%
%% breaklinks does not work with ps2pdf:
%breaklinks,%
pdftitle={PLOB \thistitle},%
pdfauthor={\thisauthor},%
pdfsubject={\thissubject},%
pdfkeywords={\thiskeywords}
]{hyperref}}
%
\let\BetterSloppy\sloppy
%
\fi%
%
\usepackagetimes%
\let\usepackagetimes\undefined%
\usepackagehyperref%
\let\usepackagehyperref\undefined%
%
\defattr{objid}{\sl}%
\defattr{typetag}{\sl}%
\defattr{Typetag}{\sl}%
\defattr{TypeTag}{\sl}%
%
\def\figurefontsize{small}%
%
\providecommand\td{\ensuremath{\sim}}% ASCII-Tilde
%
%
\readaux{\Path/../userg/\NoOfSides/userg}{bib:PLOB-UsersGuide}{userg.pdf}
\readaux{\Path/../userg/\NoOfSides/manudis}{bib:PLOB-UsersGuide}{userg.pdf}
%
% Tiefe des Inhaltsverzeichnis; wird von report.sty standardmaessig
% auf 2 gesetzt:
%\def\mytocdepth{99}
%
% Commands
%
\newcommand{\idxadmfn}%                 % Index admin. function
{Index administration function}

\newcommand{\isabtree}[1]%              % The ... is a BTree.
{\isa{#1}{a persistent BTree}}
%
\newcommand{\isacbtree}[1]%             % The ... is a cached BTree.
{\isa{#1}{a cached persistent BTree}}
%
\newcommand{\isaidxtable}[1]%           % The ... is an index table.
{\isa{#1}%
     {a (persistent) object which maintains an index for a slot,
      i.e.\ which maps the value of a slot to the persistent
      \clos\ instance with that value in the slot.}}
%
\newcommand{\isakey}[2]%                % The ... is BTree key.
{\isa{#1}%
     {a \cl\ \obj\ whose type is
      compatible with all types of all keys stored in
      the persistent BTree #2\ so far and which obeys the
      \plob\ restrictions on \cl\ types for keys
      of persistent BTrees}}

\newcommand{\basecls}[1]{%
\par{}Class {\bf #1} is a \plob\ base class; this imposes following
restrictions on changes to its definition:
\begin{enumerate}

\item The persistent memory layout of this class is defined in the C
level; adding or removing slots must be propagated to the
constants definitions for the \sh\ vector indices
for this class in the C header files and a recompilation of the
C files must be started.

\item After a class change, the \sh\ {\sl must} be
reformatted by calling \fcite{format-plob-root}.
 
\end{enumerate}}%

\newcommand{\horrible}{{\sc {\it h}\/o{\bf r}ri{\underline b}l{\sf e}}}

\newcommand{\retoldmode}[2]{%
Returns the previous lock information held by #1\ for the persistent
object referenced by #2. This is a bitwise-or of the constant values of
{\bf *lock-level-\ldots{}*}
(table~\ref{tab:lock-levels}%
%% 1998/01/07: This makes problems with hyperref:
%% \fcitepage{\pageref{tab:lock-levels}}%
)
with {\bf *lock-mode-\ldots{}*}%
(table~\ref{tab:lock-modes}
%% 1998/01/07: This makes problems with hyperref:
%% \fcitepage{\pageref{tab:lock-modes}}
)
.}

\newcommand{\lb}{\linebreak[0]}                 % Linebreaks in \tt
