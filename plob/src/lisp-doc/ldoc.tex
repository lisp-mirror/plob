% This is -*-LaTeX-*-
%
% ldoc.tex
% HK 16.6.94
%
\chapter{Introduction}
%
The package \lisp{lisp-doc}\ can be used for generating semi-automated
documentation from LISP source code. Its way of processing could best
be described by the term `Inverse Literate Programming'. To begin with
the non-inverse approach, Literate Programming means generating both
program documentation and source code from one `meta-source' file; a
good example for literate programming is WEB. A WEB file is broken
down by a `compilation' process into a \TeX\ file containing the
program's documentation and a Pascal or C file containing the
program's source code (among providing the poor Pascal programmers
with the benefits of a textual preprocessor). Alas, \lisp{lisp-doc}\
works a little bit the other way round: It takes a LISP source code
and generates a \TeX\ documentation for it. I developed this strange
looking way of processing `the other way round' w.r.t.\ Literate
Programming because my diploma thesis
\cite{bib:ki94}\ \cite{bib:ki95}\ supervisor raised the request to
make a good-looking manual for my diploma thesis at the very end of
programming around 500 KBytes of LISP code; so I thought it would be a
good idea to use the nice syntax, structuring and its regular
treatment of documentating items inherent to the LISP language to
generate a manual right from the source code itself, especially
because LISP programmers are encouraged to document their `products'
directly in the source code. This means too that \lisp{lisp-doc}\ is
`only' a spin-off product of my diploma thesis and will only be
supported by me if I have the time to do so. It was thought by me of
being a use-once product, but meanwhile it is used by some people in
the Computer Science Department at the University of Hamburg. To
benefit from \lisp{lisp-doc}, follow these steps:
\begin{enumerate}
%
\item Write your LISP code almost as usual, but bear in mind that all
documentation strings will be processed later by \TeX.\footnote{Only
documentation strings are processed, comments are not.} This has the
advantage that all your documentation can contain really any \TeX\
commands and that the output will look very pretty, if you know how to
use \TeX. The last meant constraint is also the disadvantage for
non-\TeX-expirienced programmers. Hints on how the documentation
should look like are given in
section \ref{sec:doingdoc}\ (p.\ \pageref{sec:doingdoc}).
%
\item Compile \& load \lisp{lisp-doc}\ as usual into your LISP system;
let it scan your source code as explained in section
\ref{sec:gentex}\ (p.\ \pageref{sec:gentex}).
%
\item Include the generated \TeX\ file in a main \TeX\ file as shown
in the following example file.
%
\item Call \LaTeX\ with the name of the main file.
%
\item Print and distribute the hopefully pretty looking manual.
%
\end{enumerate}
%
\par{}This is the documentation about the \lisp{lisp-doc}\ module
itself; the next chapter was generated from its own documentation
strings. Use this \LaTeX\ file as an example how to include the
documented \TeX\ source into a `surrounding' manual.
%
\par{}A minimal necessary `main' \LaTeX\ file working on a file
generated from \lisp{lisp-doc}\ looks like this:
%
\begin{quote}\begin{verbatim}
\documentclass{report}
\usepackage[english]{babel}%    Otherwise lispdoc will fail ...
\usepackage{lispdoc,crossref}
%
\begin{document}
%
\tableofcontents%    The table of contents is useful as an index too
\clearpage
\sloppy%             \sloppy lets the output look a little bit better
\numberingoff%       Switch section numbering off
\include{...}%       \include here the \TeX\ file generated by lisp-doc
\numberingon%        Switch section numbering on again
\fussy
%
\end{document}
\end{verbatim}\end{quote}
%
\par{}The class file \lisp{lispdoc.cls}\ was adpated to use NFSS, but
care was taken that it should also be working without NFSS. You can
use the original \lisp{cmr}\ or the \ps\ fonts; add the
\lisp{times}\ package in front of the \lisp{lispdoc}\ packages to get
\ps\ fonts.
%
\section{Generating the \protect\TeX\ source from LISP documentation}
\label{sec:gentex}
%
For this, use \fcite{scan-files}; see its documentation.
The text in the next chapter of this documentation was generated by
calling the \fcite{scan-myself}.
%
\par{}Please note that \TeX\ commands embedded into the documentation
strings mut be preceeded by {\sl two} backslashes since the LISP
reader handles the backslash character as a quoting character;
the scanning process of \lisp{lisp-doc}\ `collapses' each pair of two
backslashes into one backslash. Look into file
\lisp{lisp-doc}\ itself for examples.
%
\section{Generated \protect\TeX\ source}
%
The style of presentation and the notation are `borrowed' and extended
from chapter~6 of \cite{bib:amop91}; see
\cite[\citepage{163}]{bib:amop91} for details concerning the notation
used for [generic] functions and methods.
Here are the extensions and changes introducted to the style of
chapter~6 of \cite{bib:amop91}:
\begin{itemize}
%
\item Constants, variables, macros and classes are documented too.
%
\item Since not only the external interface but also all internal
entities are described, there is an addition to each section
header if the entity is {\it external} or {\it internal}. The internal
entities are described to document the system itself.
%
\item Added to each section should be a subsection named `{\sc See
Also}' with references to other entities; the references should be
ordered according to their importance.
%
\item References are done by using the \verb|\fcite{}| macro; they
generate an identifier naming the kind of entity,
the symbol naming the entity and a page resp.\ bibliographic reference
in square brackets, e.g.: \verb|\Fcite{scan-files}| gives
`\Fcite{scan-files}'.
%
\item If a form has a {\bf (setf)} equivalent, the {\bf(setf)} form is
placed right behind the non-{\bf(setf)} form.
%
\end{itemize}
%
\section{Documenting source code}
\label{sec:doingdoc}
%
The \TeX\ code is read directly from the documentation strings which
may be given to many LISP top level expressions; it should be built up
as follows:
%
\begin{quote}
\newlength{\blankwidth}\settowidth{\blankwidth}{\tt\ }
\newcommand{\blank}{\hspace*{\blankwidth}}
\newlength{\restwidth}\setlength{\restwidth}{\textwidth}%
\addtolength{\restwidth}{-\leftmargin}%
\newcommand{\ibox}[1]{{%
\addtolength{\restwidth}{-\leftmargin}%
\parbox{\restwidth}{#1}}}
\newcommand{\dbs}{\bslash\bslash}
\begin{tabbing}
\addtolength{\restwidth}{-\leftmargin}%
\verb|(d|\=\verb|efun sample-function (| {\it\lt{}argument 1\gt\/} \ldots\
{\it\lt{}argument n\gt\/} \verb|)|\\
  \>\verb|"|\pushtabs\\
\verb|\|\=\verb|\Argumentslabel|\\
\>\verb|\\isa|\=\verb|{\\funarg{| {\it\lt{}argument 1\gt\/} \verb|}}|
 for keyword arguments, use \verb|\\keyarg| instead of \verb|\\funarg|\\
\>            \>\verb|{| Text explaining {\it\lt{}argument 1\gt},
 e.g.\ \verb|a string}|\\
\>\ldots\\
\>\verb|\\isa{\\funarg{| {\it\lt{}argument n\gt\/} \verb|}}|\\
\>            \>\verb|{| Text explaining {\it\lt{}argument n\gt\/},
 e.g.\ \verb|a symbol}|\\
\\
\verb|\\Valueslabel|\\[\smallskipamount]
\>\ibox{Explain here the values returned by the function, e.g.\ %
{\tt Returns always \dbs{}lispt.} Omit the {\tt\dbs Valueslabel} section if
the function returns no values at all; if it returns the value of one
of its arguments, use {\tt\dbs retarg\{\dbs funarg\{} {\it\lt{}returned
argument\gt\/}~{\tt\}\}}.}\\
\\
\verb|\\Purposelabel|\\[\smallskipamount]
\>\ibox{Explain here what the documented item does, e.g.\ {\tt This
is a sample function showing how to document functions.}
References to function arguments should be enclosed in
{\tt\dbs funarg\{} {\it\lt{}argument\gt\/}~{\tt\}} resp.\ {\tt\dbs
keyarg\{} {\it\lt{}argument\gt\/}~{\tt\}}; this will emphasize them.}\\
\\
\verb|\\Seealsolabel|\\[\smallskipamount]
\>\ibox{Give here references ordered by their importance to other
documented entities. The {\tt\dbs Fcite} macro capitilizes the first
letter:}\\[\smallskipamount]
\>\verb|\\Fcite{|\ldots\verb|}; \\fcite{|\ldots\verb|}."|\\
\poptabs\>\ldots\\
\>\verb|t)|
\end{tabbing}\end{quote}
%
This is an example of how to document a function. Actually, the
documentation of almost all top-level expressions of the form
\lisp{(def\ldots)}\ will be put into the reference manual.
To suppress a top-level expression from being put into the reference
manual, put a \lisp{\#-:Lisp-Doc}\ expression in front of it; for
top-level expressions which are only for being put into the reference
manual but are not intended to be compiled, use a \lisp{\#+:Lisp-Doc}\
expression. The above text is shown as it would be found in a LISP
documentation string, e.g.\ all backslashes are `doubled'. Variables,
constants, methods and slots should be documented only by a short text
without any of the labels specified.
%
\par{}A table of contents is very useful and should be generated in
the main \LaTeX\ file; since each documented item is handled as a
\LaTeX\ \lisp{section}, the table of contents serves as a quite useful
index too.
%
\section{Cross-referencing}
%
Cross-referencing is done with the \verb|\flabel{}{}{}{}|\ and
\verb|\fcite{}|\ resp.\ \verb|\Fcite{}|\ macros.
The name `fcite' means \textbf{f}unction \textbf{cite}, i.e.\ it
was created by me to reference function names easily.
%
\par{}The \verb|\flabel{}{}{}{}|\ macro defines an item which should be
referenced in the text by the \verb|\fcite{}|\ macro. You can think of
using the \verb|\flabel{}{}{}{}| macro is very similar to using the
standard \LaTeX\ \verb|\label{}|\ macro. The difference is that
references set with a \verb|\label{}|\ macro expand at their usage
with \verb|\ref{}|\ to the number defined by the enclosing environment
of the \verb|\label{}|\ macro, i.e.\ a \verb|\label{}|\ macro placed
right behind a \verb|\section{}|\ macro expands at its usage with
\verb|\ref{}|\ to the section number of the section in which the
\verb|\label{}|\ macro was placed. With \verb|\flabel{}{}{}{}|\ you are
free to define another text which should appear at referencing instead
of the number defined by the enclosing environment. The
\verb|\flabel{}{}{}{}|\ macro takes 4 arguments:
\begin{enumerate}
%
\item A text describing the 2nd argument. Predefined are some
shorthands which must be used to make the \verb|\fcite{}|\ and
\verb|\Fcite{}|\ macros explained below work as expected:\\
\begin{tabular}{| l | l || l | l |}
\hline
\tabularheader{Shorthand}
& \tabularheader{expands to}
& \tabularheader{Shorthand}
& \tabularheader{expands to}\\
\hline\hline
  \verb|\crfchapter| & \crfchapter
& \verb|\crfsection| & \crfsection\\
\hline
  \verb|\cls|   & \cls
& \verb|\clsmc| & \clsmc\\
\hline
  \verb|\clsmo| & \clsmo
& \verb|\const| & \const\\
\hline
  \verb|\fn|    & \fn
& \verb|\gfn|   & \gfn\\
\hline
  \verb|\mac|   & \mac
& \verb|\mc|    & \mc\\
\hline
  \verb|\mo|    & \mo
& \verb|\mtd|   & \mtd\\
\hline
  \verb|\mtdmc| & \mtdmc
&  \verb|\mtdmo| & \mtdmo\\
\hline
  \verb|\obj|   & \obj
& \verb|\slt|   & \slt\\
\hline
  \verb|\sltmc| & \sltmc
& \verb|\sltmo| & \sltmo\\
\hline
  \verb|\spfrm| & \spfrm
&  \verb|\var|   & \var\\
\hline
\end{tabular}\\
Please note that you must place a \verb|\protect|\ macro right ahead
the shorthands described above.
%
\item The label name. Normally this is the name of a function, macro
or something similar.
%
\item The text which should appear when the label is
referenced. Normally this is a \verb|\cite{}|\ reference.
%
\item The text the label refers to.
%
\end{enumerate}
%
All calls to \verb|\flabel{}{}{}|\ are placed normally into the
document's preamble and define references to external items, i.e.\
the standard LISP items described in \cite{bib:st90}.
%
\par{}Example: Assuming that the label of the bibliographic reference
to \cite{bib:st90}\ is \texttt{bib:st90}, the call
\begin{quotation}
\verb|\flabel{\protect\var}{*print-case*}{\protect\cite[p. 560]{bib:st90}}|
\end{quotation}
is referenced in the text by\footnote{Actually,
\texttt{\bslash{}fcite}\ should better be named
\texttt{\bslash{}fref}.}
\begin{quotation}
\verb|\fcite{*print-case*}|
\end{quotation}
and will expand to the text `\fcite{*print-case*}'. The \verb|\Fcite{}|\
macro capitalizes the first letter of the referenced text, i.e.
\begin{quotation}
\verb|\Fcite{*print-case*}|
\end{quotation}
expands to `\Fcite{*print-case*}'. The \verb|\Fcite{}|\ macro works
only as expcted if the above shorthands are used as the first argument
to \verb|\flabel{}{}{}|.
%
\par{}For more examples see file
\lisp{plob/\lb{}tex/\lb{}inputs/\lb{}dipldefs.\lb{}sty}\ 
on using \verb|\flabel{}{}{}| and file
\lisp{plob/\lb{}tex/\lb{}manual/\lb{}plobrefg.\lb{}tex}\ for examples
on using \verb|\fcite{}|.
%
\section{Further ideas}
%
Perhaps it would be a good idea to select a similar approach for
generating HTML code from LISP source code, so that the friends of WWW
can make an online-available program documentation.
%
\section{Known bugs}
%
This package works for now only with \textsc{LispWorks}\ \cl. Set the
global \fcite{*print-case*}\ to \lisp{:downcase}\ to print all
identifier names in lower case; otherwise they will appear in upper
case.
%
\par{}When using the \texttt{[twoside]}\ class option, the page
headings on the left pages are not as expected: They contain the name
of the last documented item on the left page and not the first
one. This is not a
\lisp{lisp-doc}\ bug, but a good known \LaTeX\ `feature' which occurs
when section names instead of chapter names are used in page headers
of left pages (each documented item forms a \LaTeX\
\texttt{\bslash{}section}). To my opinion, using a chapter name in the page
header makes it redundant, because the headers of all left pages of
the reference manual would contain something like `Reference Guide',
which is bad for locating a documented item in the reference manual.
